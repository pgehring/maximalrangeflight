\documentclass[aspectratio=169]{beamer}
\mode<presentation>
{
  \usetheme{myulm}
  \setbeamercovered{transparent}
  \setbeamertemplate{navigation symbols}{} % no navigation bar
  \setbeamersize{sidebar width left=1.17cm}

  \setbeamercolor{block body alerted}{bg=alerted text.fg!10}
  \setbeamercolor{block title alerted}{bg=alerted text.fg!20}
  \setbeamercolor{block body}{bg=structure!10}
  \setbeamercolor{block title}{bg=structure!20}
  \setbeamercolor{block body example}{bg=green!10}
  \setbeamercolor{block title example}{bg=green!20}
}

\usepackage[ngerman]{babel}
\usepackage[utf8]{inputenc}
\usepackage{amsmath,amssymb,amsfonts}
\usepackage{times}
\usepackage{graphicx}
\usepackage{fancyvrb}
\usepackage{array}
\usepackage{colortbl}
\usepackage[framemethod=TikZ]{mdframed}



% Anfang der Titelfolie
% Anpassung von: Titel, Untertitel, Autor, Datum und Institut
\newcommand{\fakultaet}{Mathematik und Wirtschaftswissenschaften}
\newcommand{\institut}{Institut für numerische Mathematik}
\newcommand{\Heiko}{Heiko Karus}
\newcommand{\Philipp}{Philipp Gehring}
\newcommand{\Felix}{Felix Götz}

\title{\textbf{Optimalsteuerungsproblem  \\ \textit{Maximal Range Flight}}}
\subtitle{Projektpräsentation -- Numerik 4}
\author{\Heiko,\ \Philipp,\ \Felix}
\newcommand{\presdatum}{\today} % alternativ zu \today: Eingabe eines festen Datums
\institute{\institut, \fakultaet\\}
%Ende der Titelfolie

% Anfang der Kopfzeile der Folien
% Anpassung von: Zwischentitel, Leitthema oder Name
% Das Datum wird oben geändert: unter \presdatum{}!

\newcommand{\zwischentitel}{Projektpräsentation}
\newcommand{\leitthema}{\insertsection}
% Ende der Kopfzeile

% Anfang der Folien
\begin{document}
\hspace*{-1.49cm}
\frame[plain]{\titlepage}

% Das Inhaltsverzeichnis
\hspace*{-0.7cm}
\begin{frame}
  \frametitle{Gliederung}
  \tableofcontents
\end{frame}


% 1. Folie
\section{Problemformulierung}
\begin{frame}
  \frametitle{Maximal Range Flight}
  \vspace{1em}
  \begin{itemize}
    \item Maximieren der Reichweite eines Steigfluges
  \end{itemize}
\end{frame}


\begin{frame}
  \frametitle{Modellskizze}
  \vspace{1em}
  \includegraphics[scale=.75]{images/Flugzeug.pdf}
\end{frame}


\begin{frame}
  % \frametitle{Optimalsteuerungsproblem}
  \begin{block}{Optimalsteuerungsproblem - Maximal-Range-Flight}
    \tiny    
    Für das Optimalsteuerungsproblem ergibt sich mit dem Zustandsvektor
    $X(t) = (h(t),\gamma(t),x(t),v(t))^T$
    und der Steuerfunktion
    $U(t) = (T(t),C_L(t))^T$
    das Problem
    \begin{align*}
        \min_{U} F(X,U) &:= g(X(t_f)) + \int_{t_0}^{t_f} f_0(X(t),U(t)) dt = -(x(t_f) - x_0) & & \\\
        \text{unter} \hspace{10mm} \dot{X}(t) &= f(X(t),U(t)) = (\dot{h}(t),\dot{\gamma}(t),\dot{x}(t),\dot{v}(t))^T  & & \\\
        &= 
        \begin{pmatrix}
            v(t) \sin(\gamma(t)) \\ 
            \dfrac{L(v(t),h(t),C_L(t)) - W \cos(\gamma(t))}{mv(t)} \\ 
            v(t) \cos(\gamma(t)) \\ 
            \dfrac{T(t) - D(v(t),h(t),C_L(t)) - W \sin(\gamma(t))}{m}
        \end{pmatrix} & & \\\
        (h,\gamma,x,v)(t_0) &= (h_0,\gamma_0,x_0,v_0) & & \\\
        (h,\gamma)(t_f) &= (h_f,\gamma_f) & & \\\
        q(v(t),h(t)) &\leq q_{\max} & & \forall t \in [t_0,t_f]\\\
        U(t) &= (T(t),C_L(t))^T \in \mathcal{U} = \left[ 
        \begin{matrix}
            [T_{\min},T_{\max}] \\ 
            [C_{L, \min},C_{L, \max}]
        \end{matrix} 
        \right] & & \forall t \in [t_0,t_f]
    \end{align*}
  \end{block}
\end{frame}


% 2. Folie
\section{Lösung mit direktem Verfahren}
\begin{frame}
  % TODO: section title aus
  \frametitle{Vollständige Diskretisierung}
\vspace{-2.6cm}
  \begin{itemize}
    \item Das ist eine M\"{o}glichkeit f\"{u}r eine Aufz\"{a}hlung
    \item Aufz\"{a}hlung 2
    \item Aufz\"{a}hlung 3
  \end{itemize}
\end{frame}

% 3. Folie
\section{Lösung mit indirekten Verfahren}
\begin{frame}
  \frametitle{Minimumprinzip nach Pontryagin}
\vspace{-2.6cm}
  \begin{itemize}
    \item Das ist eine M\"{o}glichkeit f\"{u}r eine Aufz\"{a}hlung
    \item Aufz\"{a}hlung 2
    \item Aufz\"{a}hlung 3
  \end{itemize}
\end{frame}


\end{document}