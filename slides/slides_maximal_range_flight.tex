\documentclass[aspectratio=169]{beamer}
\mode<presentation>
{
  \usetheme{myulm}
  \setbeamercovered{transparent}
  \setbeamertemplate{navigation symbols}{} % no navigation bar
  \setbeamersize{sidebar width left=0.17cm}
  
  
  \setbeamertemplate{itemize/enumerate subbody begin}{  \scriptsize}
  
  \setbeamertemplate{blocks}[rounded][shadow=false]

  \setbeamercolor{block body alerted}{bg=alerted text.fg!10}
  \setbeamercolor{block title alerted}{bg=alerted text.fg!20}
  \setbeamercolor{block body}{bg=structure!10}
  \setbeamercolor{block title}{bg=structure!20}
  \setbeamercolor{block body example}{bg=green!10}
  \setbeamercolor{block title example}{bg=green!20}
}

\usepackage[ngerman]{babel}
\usepackage[utf8]{inputenc}
\usepackage{amsmath,amssymb,amsfonts,amsthm}
\usepackage{times}
\usepackage{graphicx}
\usepackage{fancyvrb}
\usepackage{array}
\usepackage{colortbl}
\usepackage[framemethod=TikZ]{mdframed}

\usepackage{tabularx} 
\usepackage{booktabs}

% Mengen Buchstaben:
\newcommand{\E}{\mathbb{E}}
\newcommand{\Q}{\mathbb{Q}}
\newcommand{\R}{\mathbb{R}}
\newcommand{\N}{\mathbb{N}}

% Anfang der Titelfolie
% Anpassung von: Titel, Untertitel, Autor, Datum und Institut
\newcommand{\fakultaet}{Mathematik und Wirtschaftswissenschaften}
\newcommand{\institut}{Institut für numerische Mathematik}
\newcommand{\Heiko}{Heiko Karus}
\newcommand{\Philipp}{Philipp Gehring}
\newcommand{\Felix}{Felix Götz}

\title{\textbf{Optimalsteuerungsproblem  \\ \textit{Maximal Range Flight}}}
\subtitle{Projektpräsentation -- Numerik 4}
\author{\Heiko,\ \Philipp,\ \Felix}
\newcommand{\presdatum}{\today} % alternativ zu \today: Eingabe eines festen Datums
\institute{\institut, \fakultaet\\}
%Ende der Titelfolie

% Anfang der Kopfzeile der Folien
% Anpassung von: Zwischentitel, Leitthema oder Name
% Das Datum wird oben geändert: unter \presdatum{}!

\newcommand{\zwischentitel}{Projektpräsentation}
\newcommand{\leitthema}{\insertsection}
% Ende der Kopfzeile

% Anfang der Folien
\begin{document}


\hspace*{-0.6cm}
\setcounter{framenumber}{-1}
\frame[plain]{\titlepage}



% Das Inhaltsverzeichnis
\section*{Inhaltsverzeichnis}
\begin{frame}
  \frametitle{Gliederung}
  \tableofcontents
\end{frame}


% 1. Folie
\section{Einleitung}
\begin{frame}
  \frametitle{Maximal Range Flight}
  \vspace{1em}
  \begin{itemize}
    \item Einfache Lösung für den Horizontalflug: Maximierung des Verhältnisses $\sqrt{C_{L}/C_{D}}$ der Auftriebs- und Widerstandskoeffizienten
    \item Erweiterung auf den Steigflug, zweidimensionaler Flug in der $x$-$h$-Ebene, Steuerung von Auftriebsbeiwert $C_L(t)$ und Schub $T(t)$ 
    \item Ziel: Von gegebener Anfangsposition zur Reisehöhe $h_f = 10668\ m$ und ein Anstellwinkel $\gamma_f = 0\ ^{\circ}$ in einer Flugzeit von $1800 \ s$ zu steuern, wobei die zurückgelegte Strecke maximal wird
    \item Steuerbeschränkungen für Schub und Auftriebsbeiwert und maximaler Staudruck
  \end{itemize}
\end{frame}

\section{Problemformulierung}
\begin{frame}
  \frametitle{Freikörperbild Airbus A380-800}
  \vspace{1em}
   \begin{center}
  	\includegraphics[scale=.75]{images/Flugzeug.pdf}
   \end{center}
\end{frame}


\begin{frame}
  % \frametitle{Optimalsteuerungsproblem}
  \begin{center}
  \begin{block}{Optimalsteuerungsproblem - Maximal Range Flight}
    \scriptsize   
    Für das Optimalsteuerungsproblem ergibt sich mit dem Zustandsvektor
    $X(t) = (h(t),\gamma(t),x(t),v(t))^T$
    und der Steuerfunktion
    $U(t) = (T(t),C_L(t))^T$
    das Problem
    \begin{align*}
        \min_{U} F(X,U) &:= g(X(t_f)) + \int_{t_0}^{t_f} f_0(X(t),U(t)) dt = -(x(t_f) - x_0) & & \\\
        \text{unter} \hspace{20mm} \dot{X}(t) &= f(X(t),U(t)) =     
         \begin{pmatrix}
         \dot{h}(t)  \\ 
         \dot{\gamma}(t)  \\ 
         \dot{x}(t)  \\ 
         \dot{v}(t)   \\ 
	 \end{pmatrix} 
        = 
        \begin{pmatrix}
            v(t) \sin(\gamma(t)) \\ 
            \dfrac{L(v(t),h(t),C_L(t)) - W \cos(\gamma(t))}{mv(t)} \\ 
            v(t) \cos(\gamma(t)) \\ 
            \dfrac{T(t) - D(v(t),h(t),C_L(t)) - W \sin(\gamma(t))}{m}
        \end{pmatrix} & & \\\
        (h,\gamma,x,v)(t_0) &= (h_0,\gamma_0,x_0,v_0) & & \\\
        (h,\gamma)(t_f) &= (h_f,\gamma_f) & & \\\
        q(v(t),h(t)) &\leq q_{\max}  \forall t \in [t_0,t_f]\\\
        U(t) &= (T(t),C_L(t))^T \in \mathcal{U} = \left[ 
        \begin{matrix}
            [T_{\min},T_{\max}] \\ 
            [C_{L, \min},C_{L, \max}]
        \end{matrix} 
        \right]  \forall t \in [t_0,t_f]
    \end{align*}
  \end{block}
  \end{center}
\end{frame}



% \begin{frame}
%   \frametitle{Autonomes Mayer-Problem???}
% \vspace{-2.6cm}
%   \begin{itemize}
%     \item ????
%   \end{itemize}
% \end{frame}
\section{Numerische Untersuchungen}
\begin{frame}
  \frametitle{Numerische Lösungsverfahren gewöhnlicher Differentialgleichungen}
  % \vspace{-2.6cm}
  \begin{center}
    \includegraphics[scale=0.75]{images/methods_plot_d_x.pdf}%
    \includegraphics[scale=0.75]{images/methods_plot_d_v.pdf}\\
  \end{center}
    
  \newcolumntype{Z}{>{\raggedright\let\newline\\\arraybackslash\hspace{0pt}}X}
  \begin{table}[htbp]
      \scriptsize
      \centering
      \begin{tabularx}{.6\paperwidth}{Zccc}
          \toprule
          \textbf{Algorithmus}        & \textbf{Laufzeit} & \textbf{Laufzeitdifferenz } \\
                                      & \textbf{in \text{$\mathbf{s}$}} & \textbf{in \text{$\mathbf{s}$}} \\
          \midrule
          explizites Euler Verfahren  &   0,00708 &   0 \\
          implizites Euler Verfahren  &   0,0185  &   0,0115\\
          \textit{MATLAB} \texttt{ode23s}      &   0,0914  &   0,0843 \\
          \textit{MATLAB} \texttt{ode45}       &   0,529   &   0,522 \\
          RADAU-2A Verfahren         &   1,36    &   1,35 \\
          \bottomrule
      \end{tabularx}
  \end{table}
\end{frame}






\section{Lösung mit direkten Verfahren}
\begin{frame}
  \begin{block}{Direkte Formulierung des Optimierungsproblems}
    \scriptsize
    Endlichdimensionales, nichtlineares Optimierungsproblem der Form
    \begin{align*}
        \min_{z \in S} F(z) &:= \varphi(X_0,X_{N-1}) & & \\
        \text{unter} \hspace{10mm} G(z) &\leq 0_{n_s \cdot N} & &  \text{(Ungleichungsnebenbedingungen)} \\
        H(z) &= 0_{n_X \cdot (N-1) + n_{\psi}} & & \text{(Gleichungsnebenbedingungen)}\\
        z &\in S := \R^{n_X \cdot N} \times [U_{\min},U_{\max}]^{N} & &  \text{(Zulässige Menge)}
    \end{align*}
    mit der Optimierungsvariable
    \begin{equation*}
        z := (X_0,X_1,...,X_{N-1},U_0,U_1,...,U_{N-1})^T
    \end{equation*}
    und den Nebenbedingungsfunktionen 
    \begin{equation*}
        G(z) := 
        \begin{pmatrix}
            s(t_0,X_0,U_0) \\ 
            \vdots \\ 
            s(t_{N-1},X_{N-1},U_{U-1})
        \end{pmatrix}
        \ \text{und}\ H(z) := 
        \begin{pmatrix}
            X_0 + h_0 f(t_0,X_0,U_0) - X_1 \\ 
            \vdots \\ 
            X_{N-2} + h_{N-2} f(t_{N-2},X_{N-2},U_{U-2}) - X_{N-1} \\
            \psi(X_0,X_N)
        \end{pmatrix} 
    \end{equation*}
    \end{block}
\end{frame}

\begin{frame}
  	\frametitle{Lösen des Optimierungsproblems}
  	\begin{itemize}
  		\item Programmcode: Parameterabhängig und in Klassen strukturiert, für die Wiederverwendbarkeit.
  		\item \textit{MATLAB}-Methode \texttt{fmincon}:
  			\begin{itemize}
  				\item[\(\rightarrow\)] Benötigt: Nebenbedingungsfunktionen und Startwert.
      			\item[\(\rightarrow\)] Lösen mit SQP-Verfahren oder Innere-Punkte-Verfahren.
      			\item[\(\rightarrow\)] Ausgabe: Lösung und Abbruchkriterium.
    		\end{itemize}
    	\item Abbruchkriterium: Auskunft über die Qualität der berechneten Lösung.
  	\end{itemize}
\end{frame}

\begin{frame}
	\frametitle{Versuch mit gegebenen Parametern und SQP-Verfahren}
	Abbruchkriterium: -2 (Kein zulässiger Punkt gefunden)
  	\begin{center}
  		\includegraphics[scale=.6]{images/test_0_1.pdf}
  	\end{center}
\end{frame}

\begin{frame}
	\frametitle{Versuch mit gegebenen Parametern und Innere-Punkte-Verfahren}
  	Abbruchkriterium: -2 (Kein zulässiger Punkt gefunden)
  	\begin{center}
  		\includegraphics[scale=.6]{images/test_0_2.pdf}
  	\end{center}
\end{frame}

\begin{frame}
  	\frametitle{Vergleich von SQP- und Innere-Punkte-Verfahren}
  	\begin{center}
  		\includegraphics[scale=0.6]{images/Vergleich_SQP_IP.pdf}
  	\end{center}
  	\begin{itemize}
    	\item SQP-Verfahren konvergiert schneller und erreicht kleineren Zielfunktionswert.
    	\item Auswahl des SQP-Verfahrens für nachfolgende Untersuchungen.
  	\end{itemize}
\end{frame}

\begin{frame}
  	\begin{center}
 	\begin{block}{Angepasstes Optimalsteuerungsproblem - Maximal Range Flight}
    \scriptsize   
    Für das Optimalsteuerungsproblem ergibt sich mit dem Zustandsvektor $\tilde{X}(s) = (h(s),\gamma(s),x(s),v(s),t_f)^T$ und der Steuerfunktion $\tilde{U}(s) = (T(s),C_L(s))^T$ das Problem:
	\begin{align*}
		\min_{U} F(\tilde{X},\tilde{U}) &:= t_f \int^{\tilde{t}_f = 1}_{t_0 = 0} f_0(s \cdot t_f,\tilde{X}(s),\tilde{U}(s)) ds = t_f \\\
		\text{unter} \hspace{10mm} \dot{\tilde{X}}(t) &= t_f f(s \cdot t_f,\tilde{X}(s),\tilde{U}(s)) = \begin{pmatrix}
		t_f v(t) \sin(\gamma(t)) \\ 
		t_f \left( \frac{F \alpha e^{-\beta h(t)} v(t) C_L(t)}{2m} - \frac{g \cos(\gamma(t))}{v(t)} \right) \\ 
		t_f v(t) \cos(\gamma(t)) \\ 
		t_f \left( \frac{T(t)}{m} - \frac{(C_{D_0} + k C_L^2(t)) F \alpha e^{-\beta h(t)} v^2(t)}{2m} - g \sin(\gamma(t)) \right) \\
0
		\end{pmatrix} \\\
		(h,\gamma,x,v)(t_0) &= (h_0,\gamma_0,x_0,v_0) \\\
		(h,\gamma)(\tilde{t}_f) &= (h_f,\gamma_f) \\\
		q(v(s),h(s)) &\leq q_{\max}  \ \ \ \ \forall s \in [t_0,\tilde{t}_f]\\\
		U(s) &= (T(s),C_L(s))^T \in \mathcal{U} \ \ \ \ \forall s \in [t_0,\tilde{t}_f]
	\end{align*}
  	\end{block}
  	\end{center}
\end{frame}

\begin{frame}
	\frametitle{Versuch: Optimierung der Endzeit $t_f$}
  	\begin{center}
  		\includegraphics[scale=.6]{images/test_0.pdf}
  	\end{center}
  	\begin{itemize}
    	\item Abbruchkriterium: 1 (Optimum gefunden)
    	\item Optimalsteuerungsproblem für gegebenen Zeitraum $0 - 1800 \ s$ lösbar.
  	\end{itemize}
\end{frame}


\begin{frame}
  \begin{center}
    \includegraphics[scale=.75]{images/test_1_1.pdf}
  \end{center}
\end{frame}

\begin{frame}
 \begin{center}
  \includegraphics[scale=.75]{images/test_3_2.pdf}
 \end{center}
\end{frame}

\begin{frame}
 \begin{center}
  \includegraphics[scale=.75]{images/test_4_2.pdf}
 \end{center}
\end{frame}


% 3. Folie
\section{Lösung mit indirekten Verfahren}
\begin{frame}
  \frametitle{Minimumprinzip von Pontryagin}
   \begin{center}
  \scriptsize
  \begin{block}{Hamilton-Funktion}  
          \vspace{-5mm}
 \begin{align*} 
        H(X(t),U(t),\lambda(t)) &= \lambda_0 f_0(X(t),U(t)) + \lambda(t)^T f(X(t),U(t)) \\\
        &= \lambda(t)^T f(X(t),U(t)) \\\
        &= \lambda_1(t) \dot{h}(t) + \lambda_2(t) \dot{\gamma}(t) + \lambda_3(t) \dot{x}(t) + \lambda_4(t) \dot{v}(t) \\\
        &= \sin(\gamma(t)) v(t) \lambda_1 \\\
        &\hspace{7mm} + \dfrac{F \alpha e^{-\beta h(t)} C_L(t) v(t) \lambda_2(t)}{2m} - \dfrac{g \cos(\gamma(t)) \lambda_2(t)}{v(t)} \\\
        &\hspace{7mm} + \cos(\gamma(t)) v(t) \lambda_3(t) \\\
        &\hspace{7mm} + \dfrac{T(t) \lambda_4(t)}{m} - \dfrac{(C_{D_0} + k C_L^2(t)) F \alpha e^{-\beta h(t)} v^2(t) \lambda_4(t)}{2m} \\\
        &\hspace{7mm} - g \sin(\gamma(t)) \lambda_4(t)
\end{align*}
\end{block}
 \end{center}
\end{frame}




\begin{frame}
  \begin{block}{Minimumsbedingung}  
  \scriptsize
  Es gilt an allen Stetigkeitsstellen $t \in [t_0,t_f]$ von $u^{\ast}(t)$ die Minimumbedingung \[H(X^{\ast}(t),U^{\ast}(t),\lambda(t)) = \min_{U(t) \in \mathcal{U}} H(X^{\ast}(t),U(t),\lambda(t))\] Ableiten nach der Steuerfunktion $U(t)$ ergibt für den unbeschränkten Fall der Steuerung die Minimumbedingung
    \[\dfrac{\partial}{\partial U} H(X^{\ast}(t),U^{\ast}(t),\lambda(t)) = \begin{pmatrix}
    \dfrac{\lambda_4(t)}{m} \\ 
    - \dfrac{k F \alpha e^{-\beta h(t)} v^2(t) \lambda_4(t) C_L(t)}{m} + \dfrac{F \alpha e^{-\beta h(t)} v(t) \lambda_2(t)}{2m}
    \end{pmatrix}^T \stackrel{!}{=} 0\]
    und 
    \[\dfrac{\partial^2}{\partial U^2} H(X^{\ast}(t),U^{\ast}(t),\lambda(t)) = \begin{pmatrix}
    0 & - \dfrac{k F \alpha e^{-\beta h(t)} v^2(t) \lambda_4(t)}{m} 
    \end{pmatrix} \stackrel{!}{\geq} 0\] wobei \[\sigma(x(t),\lambda(t)) := H_u(X^{\ast}(t),U^{\ast}(t),\lambda(t))\] 
  \end{block}
\end{frame}

\begin{frame}
  \frametitle{Minimumprinzip von Pontryagin}
  \begin{block}{Synthesesteuerung}  
  \scriptsize
  Für die Betrachtung mit den Beschränkungen der Steuerfunktion ergibt sich für den Schub
\[T(t) = \left\lbrace \begin{array}{ll}
T_{\min} & ,\text{falls } \lambda_4 > 0  \\ 
\text{beliebig} \in [T_{\min},T_{\max}] & ,\text{falls } \lambda_4 = 0  \\ 
T_{\max} & ,\text{falls } \lambda_4 < 0
\end{array} \right.\]
Für die Bestimmung der Steuerfunktion des Auftriebsbeiwerts lässt sich die Hamilton-Funktion verkürzt mit den Termen $K_1(t)$ und $K_2(t)$ schreiben.
    \[\begin{split}
        \tilde{H}(X^{\ast}(t),C_L(t),\lambda(t)) &= \dfrac{F \alpha e^{-\beta h^{\ast}(t)} v^{\ast}(t) \lambda_2(t)}{2m} \cdot C_L(t) - \dfrac{k F \alpha e^{-\beta h^{\ast}(t)}  v^{\ast 2}(t) \lambda_4(t)}{2m} \cdot C_L^2(t) \\\
        &= K_1(t) C_L(t) - K_2(t) C_L^2(t)
    \end{split}\]
      \end{block}
\end{frame}

\begin{frame}
  \frametitle{Minimumprinzip von Pontryagin}
  \begin{block}{Synthesesteuerung}  
  \scriptsize
    Soll nun $\tilde{H}(X^{\ast}(t),C_L(t),\lambda(t))$ minimal werden, so müssen die Folgenden Bedinungen überprüft werden:
    \scriptsize
    \begin{enumerate}
        \item[1.)] $\mathbf{K_1(t) < 0 \wedge K_2(t)} < 0$:
          \scriptsize
        \begin{enumerate}
            \item[1.1.)] Für die Ableitungen von $\tilde{H}$ ergeben sich
            \[\begin{split}
            \frac{d \tilde{H}}{d C_L} = - 2 K_2(t) C_L(t) + K_1(t) &\stackrel{!}{=} 0 \\\
            \frac{d^2 \tilde{H}}{d C_L^2} = - 2 K_2(t) &\geq 0 \Rightarrow \text{Minimum}
            \end{split}\]
            So folgt für die Steuerfunktion $C_L(t) = \dfrac{K_1(t)}{2 K_2(t)} > 0$.
            %
            \item[1.2.)] Falls $C_L(t) = \dfrac{K_1(t)}{2 K_2(t)} > C_{L, \max}$ so gilt  $C_L(t) = C_{L, \max}$.
        \end{enumerate}
        %
        \end{enumerate}
      \end{block}
\end{frame}

\begin{frame}
  \frametitle{Minimumprinzip von Pontryagin}
  \begin{block}{Synthesesteuerung}  
  \scriptsize
  Damit folgt aus allen Fallunterscheidungen die Steuerfunktion des Auftriebsbeiwerts
\[C_L(t) = \left\lbrace 
\begin{array}{ll}
C_{L, \min} & ,\text{falls Bedingung } 2,3,6 \text{ oder } 9.2 \text{ gilt} \\ 
\text{beliebig} \in [C_{L, \min},C_{L, \max}] & ,\text{falls Bedingung } 5 \text{ gilt} \\ 
\dfrac{K_1(t)}{2 K_2(t)} & ,\text{falls Bedingung } 1.1 \text{ gilt} \\ 
C_{L, \max} & ,\text{falls Bedingung } 1.2,4,7,8 \text{ oder } 9.1 \text{ gilt} \\
C_L(t) = C_{L,\max} \vee C_{L,\min} & ,\text{falls Bedingung } 9.3 \text{ gilt} \\ 
\end{array} 
\right.\]
      \end{block}
\end{frame}




\begin{frame}
  \begin{block}{Adjungierte DGL}  
  \scriptsize
Leitet man nach dem Zustandsvektor $X(t)$ ab, also $H_{X}(X^{\ast}(t),U^{\ast}(t),\lambda(t))$ so erhält man 
        \[\begin{split}
            \dfrac{\partial}{\partial h} H &= - \dfrac{\alpha \beta F e^{-\beta h(t)} C_L(t) v(t) \lambda_2(t)}{2m} + \dfrac{(C_{D_0}+k C_L^2(t)) \alpha \beta F e^{-\beta h(t)} v^2(t) \lambda_4(t)}{2m} \\\
            \dfrac{\partial}{\partial \gamma} H &= \cos(\gamma(t)) v(t) \lambda_1(t) + \dfrac{g \sin(\gamma(t)) \lambda_2(t)}{v(t)} - \sin(\gamma(t)) v(t) \lambda_3(t) - \cos(\gamma(t)) g \lambda_4(t) \\\
            \dfrac{\partial}{\partial x} H &= 0 \\\
            \dfrac{\partial}{\partial v} H &= \sin(\gamma(t)) \lambda_1(t) + \left( \dfrac{F \alpha e^{-\beta h(t)} C_L(t)}{2m} + \dfrac{g \cos(\gamma(t))}{v^2(t)} \right) \lambda_2(t) \\\
            &\hspace{7mm} + \cos(\gamma(t)) \lambda_3(t) - \dfrac{(C_{D_0} + k C_L^2(t)) F \alpha e^{-\beta h(t)} v(t) \lambda_4(t)}{m}
        \end{split}\]
        wobei gilt 
        \[\dot{\lambda}(t)^T = - \dfrac{\partial}{\partial X} H = -H_{X} = \left( -\dfrac{\partial}{\partial h} H, -\dfrac{\partial}{\partial \gamma} H, -\dfrac{\partial}{\partial x} H, -\dfrac{\partial}{\partial v} H \right)\]
  \end{block}
\end{frame}


\begin{frame}
\begin{block}{Transversalitätsbedingung} 
  \scriptsize
Im Endzeitpunkt $t_f$ gilt die Transversalitätsbedingung mit dem Vektor $\nu \in \R^{n_{\psi}}$ mit $(\lambda_0,\lambda(t),\nu) \neq 0$ für alle $t \in [t_0,t_f]$
        \[\begin{split}
            \lambda(t_f)^T &= \lambda_0 g_X(X^{\ast}(t_f)) + \nu^T \psi_X(X^{\ast}(t_f)) \\\
            &= \lambda_0 
            \begin{pmatrix}
            0 & 0 & -1 & 0
            \end{pmatrix}  
            + \nu^T 
            \begin{pmatrix}
            1 & 0 & 0 & 0 \\
            0 & 1 & 0 & 0 
            \end{pmatrix}  \\\
            &= \begin{pmatrix}
            \nu_1 & \nu_2 & -\lambda_0 & 0 
            \end{pmatrix}
        \end{split}\]
  \end{block}
  \begin{block}{Konstanz} 
  \scriptsize
 Für autonome Systeme gilt \[H(X^{\ast}(t),U^{\ast}(t), \lambda(t)) = const \ \in [t_0,t_f]\]
   \end{block}
\end{frame}


\begin{frame}
\begin{block}{Zweipunkt-Randwertproblem} 
  \scriptsize
Bestehend aus den beiden Funktionen $G(t,Z(t),U(t))$ und $R(Z(t_0),Z(t_f)) = 0_{n_Z}$ für $n_Z = 8$. Für $G(t,Z(t),U(t))$ ergibt sich
\begin{align*} 
    \dot{Z}(t) = G(t,Z(t),U(t)) = 
    \begin{pmatrix}
        \dot{h}(t),\dot{\gamma}(t),\dot{x}(t),\dot{v}(t),\dot{\lambda}_1(t),\dot{\lambda}_2(t),\dot{\lambda}_3(t),\dot{\lambda}_4(t)
    \end{pmatrix}^T
\end{align*}
und für die Ableitung die Matrix
\begin{align*} 
    \dfrac{\partial G(t,Z(t),U(t))}{\partial Z} = 
    \begin{pmatrix}
        0 & J_G^{(1,2)} & 0 & J_G^{(1,4)} & 0 & 0 & 0 & 0 \\ 
        J_G^{(2,1)} & J_G^{(2,2)} & 0 & J_G^{(2,4)} & 0 & 0 & 0 & 0 \\ 
        0 & J_G^{(3,2)} & 0 & J_G^{(3,4)} & 0 & 0 & 0 & 0 \\ 
        J_G^{(4,1)} & J_G^{(4,2)} & 0 & J_G^{(4,4)} & 0 & 0 & 0 & 0 \\
        J_G^{(5,1)} & 0 & 0 & J_G^{(5,4)} & 0 & J_G^{(5,6)} & 0 & J_G^{(5,8)} \\
        0 & J_G^{(6,2)} & 0 & J_G^{(6,4)} & J_G^{(6,5)} & J_G^{(6,6)} & J_G^{(6,7)} & J_G^{(6,8)} \\
        0 & 0 & 0 & 0 & 0 & 0 & 0 & 0 \\
        J_G^{(8,1)} & J_G^{(8,2)} & 0 & J_G^{(8,4)} & J_G^{(8,5)} & J_G^{(8,6)} & J_G^{(8,7)} & J_G^{(8,8)}
    \end{pmatrix}
\end{align*}
   \end{block}
\end{frame}


\begin{frame}
\begin{block}{Zweipunkt-Randwertproblem } 
  \scriptsize
Für $R(Z(t_0),Z(t_f)) = 0_{n_Z}$ müssen zunächst die Endbedingungen mit
\begin{align*}
X_i(t_f) &= c_i & & (i=1,...,n_{\psi}=2) \\
\lambda_i(t_f) &= \lambda_0 g_{X_i}(X^{\ast}(t_f)) & &(i=n_{\psi}+1,...,n_X=4)
\end{align*}
aus
\begin{align*}
c = \begin{pmatrix} h_f  \gamma_f \end{pmatrix} \text{und} \  g_{X}(X^{\ast}(t_f)) = \begin{pmatrix} 0  0 -1 0 \end{pmatrix}
\end{align*}
gebildet werden. Es ergibt sich dann
\begin{align*}
R(Z(t_0),Z(t_f)) = 0_{n_Z} = \begin{pmatrix}
h(t_0) - h_0 \\ 
\gamma(t_0) - \gamma_0 \\
x(t_0) - x_0 \\ 
v(t_0) - v_0 \\ 
h(t_f) - h_f \\ 
\gamma(t_f) - \gamma_f \\
\lambda_3(t_f) + \lambda_0 \\ 
\lambda_4(t_f) - 0
\end{pmatrix}
\end{align*}
   \end{block}
\end{frame}


\begin{frame}
\begin{block}{Zweipunkt-Randwertproblem } 
  \scriptsize
\begin{align*}
\dfrac{d R(Z(t_0),Z(t_f))}{d Z(t_0)} = \begin{pmatrix}
1 & 0 & 0 & 0 & 0 & 0 & 0 & 0 \\ 
0 & 1 & 0 & 0 & 0 & 0 & 0 & 0 \\ 
0 & 0 & 1 & 0 & 0 & 0 & 0 & 0 \\ 
0 & 0 & 0 & 1 & 0 & 0 & 0 & 0 \\
0 & 0 & 0 & 0 & 0 & 0 & 0 & 0 \\
0 & 0 & 0 & 0 & 0 & 0 & 0 & 0 \\
0 & 0 & 0 & 0 & 0 & 0 & 0 & 0 \\
0 & 0 & 0 & 0 & 0 & 0 & 0 & 0
\end{pmatrix}
\end{align*}
\begin{align*}
\dfrac{d R(Z(t_0),Z(t_f))}{d Z(t_f)} = \begin{pmatrix}
0 & 0 & 0 & 0 & 0 & 0 & 0 & 0 \\ 
0 & 0 & 0 & 0 & 0 & 0 & 0 & 0 \\ 
0 & 0 & 0 & 0 & 0 & 0 & 0 & 0 \\ 
0 & 0 & 0 & 0 & 0 & 0 & 0 & 0 \\
1 & 0 & 0 & 0 & 0 & 0 & 0 & 0 \\
0 & 1 & 0 & 0 & 0 & 0 & 0 & 0 \\
0 & 0 & 0 & 0 & 0 & 0 & 1 & 0 \\
0 & 0 & 0 & 0 & 0 & 0 & 0 & 1
\end{pmatrix}
\end{align*}
   \end{block}
\end{frame}




\begin{frame}
  % TODO: section title aus
  \frametitle{Schwierigkeiten}

  \begin{itemize}
    \item Lösung mittels Einfach- und Mehrfachschießverfahren scheiterte an numerischen Schwierigkeiten
    \item Bestimmung der Nullstellen von $R(Z(t_0),Z(t_f))$ nicht möglich durch singuläre Jacobimatrix in Newton-Verfahren
    \item Versuch das unterbestimmte Gleichungssystem mittels Gauss-Seidel- Vefahren zu lösen gelang ebenfalls nicht
    \item Folglich konnten keine Ergebnisse mit den Schießverfahren berechnet werden
  \end{itemize}
\end{frame}

% Letzte Folie
\section{Fazit und Ausblick}
\begin{frame}
  % TODO: section title aus
  \frametitle{Fazit und Ausblick}
    \vspace{1em}
\small
  \begin{itemize}
    \item \textbf{Optimalsteuerungsproblem}: Steigflug von Anfangspunkt zu einer Reiseflughöhe in fester Zeit, wobei zurückgelegte Strecke maximal
     \item Lösen mit \textbf{direktem Verfahren} unter Verwendung von explizitem Euler- und SQP-Verfahren
      \item Umformulieren mit Hilfe des \textbf{Minimumsprinzips von Pontryagin} in ein Zweipunkt-Randwertproblem 
       \item Lösen mittels Einfach- und Mehrfachschießverfahren scheiterte an \textbf{numerischen Schwierigkeiten}
        \item Numerische Untersuchung ergab, dass sich die direkten Verfahren eignen, das gegebene Optimalsteuerungsproblem zu lösen.  	 \item Betrachtung des Optimalsteuerungsproblems mit anderen Methoden, wie dem Homotopieverfahren, wurde offen gelassen

  \end{itemize}
\end{frame}

\begin{frame}[plain]{}
\end{frame}


% Backup Folien
\section*{Zusatz}
\begin{frame}
  \frametitle{Minimumprinzip von Pontryagin}
    \tiny
  \begin{block}{Synthesesteuerung}  
      \begin{enumerate}
          \item[2.)] $\mathbf{K_1(t) = 0 \wedge K_2(t)} < 0$: Für die Ableitungen von $\tilde{H}$ ergeben sich
        \[\begin{split}
        \frac{d \tilde{H}}{d C_L} = - 2 K_2(t) C_L(t) &\stackrel{!}{=} 0 \\\
        \frac{d^2 \tilde{H}}{d C_L^2} = - 2 K_2(t) &\geq 0 \Rightarrow \text{Minimum}
        \end{split}\]
        So folgt für die Steuerfunktion $C_L(t) = 0 = C_{L, \min}$.
        %
        \item[3.)] $\mathbf{K_1(t) > 0 \wedge K_2(t)} < 0$: Für die Ableitungen von $\tilde{H}$ ergeben sich
        \[\begin{split}
        \frac{d \tilde{H}}{d C_L} = - 2 K_2(t) C_L(t) + K_1(t) &\stackrel{!}{=} 0 \\\
        \frac{d^2 \tilde{H}}{d C_L^2} = - 2 K_2(t) &\geq 0 \Rightarrow \text{Minimum}
        \end{split}\]
        So folgt für die Steuerfunktion $C_L(t) = \dfrac{K_1(t)}{2 K_2(t)} < 0$. Es gilt also $C_L(t) = C_{L, \min}$.
        \end{enumerate}
      \end{block}
\end{frame}


\begin{frame}
  \frametitle{Minimumprinzip von Pontryagin}
    \tiny
  \begin{block}{Synthesesteuerung}  
      \begin{enumerate}
        \item[4.)] $\mathbf{K_1(t) < 0 \wedge K_2(t)} = 0$: Für die Ableitungen von $\tilde{H}$ ergeben sich
        \[\begin{split}
        \frac{d \tilde{H}}{d C_L} = K_1(t) &\stackrel{!}{=} 0 \\\
        \frac{d^2 \tilde{H}}{d C_L^2} = 0 &\geq 0 \Rightarrow \text{Minimum}
        \end{split}\]
        So folgt für die Steuerfunktion $C_L(t) = C_{L, \max}$.
        %
        \item[5.)] $\mathbf{K_1(t) = 0 \wedge K_2(t)} = 0$: Daraus folgt für die Steuerfunktion $C_L(t) = \text{beliebig} \in [C_{L, \min},C_{L, \max}]$.
                \item[6.)] $\mathbf{K_1(t) > 0 \wedge K_2(t)} = 0$: Für die Ableitungen von $\tilde{H}$ ergeben sich
        \[\begin{split}
        \frac{d \tilde{H}}{d C_L} = K_1(t) &\stackrel{!}{=} 0 \\\
        \frac{d^2 \tilde{H}}{d C_L^2} = 0 &\geq 0 \Rightarrow \text{Minimum}
        \end{split}\]
        So folgt für die Steuerfunktion $C_L(t) = C_{L, \min}$.
        \end{enumerate}
      \end{block}
\end{frame}


\begin{frame}
  \frametitle{Minimumprinzip von Pontryagin}
    \tiny
  \begin{block}{Synthesesteuerung}  
      \begin{enumerate}
   \item[7.)] $\mathbf{K_1(t) < 0 \wedge K_2(t)} > 0$: Für die Ableitungen von $\tilde{H}$ ergeben sich
        \[\begin{split}
        \frac{d \tilde{H}}{d C_L} = - 2 K_2(t) C_L(t) + K_1(t) &\stackrel{!}{=} 0 \\\
        \frac{d^2 \tilde{H}}{d C_L^2} = - 2 K_2(t) &\leq 0 \Rightarrow \text{Maximum}
        \end{split}\]
        So folgt für die Steuerfunktion $C_L(t) = \dfrac{K_1(t)}{2 K_2(t)} < 0$. Es gilt also $C_L(t) = C_{L, \max}$.
        %
        \item[8.)] $\mathbf{K_1(t) = 0 \wedge K_2(t)} > 0$: Für die Ableitungen von $\tilde{H}$ ergeben sich
        \[\begin{split}
        \frac{d \tilde{H}}{d C_L} = - 2 K_2(t) C_L(t) &\stackrel{!}{=} 0 \\\
        \frac{d^2 \tilde{H}}{d C_L^2} = - 2 K_2(t) &\leq 0 \Rightarrow \text{Maximum}
        \end{split}\]
        So folgt für die Steuerfunktion $C_L(t) = 0$. Es gilt also $C_L(t) = C_{L, \max}$.
        \end{enumerate}
      \end{block}
\end{frame}

\begin{frame}
  \frametitle{Minimumprinzip von Pontryagin}
    \tiny
  \begin{block}{Synthesesteuerung}  
      \begin{enumerate}
\item[9.)] $\mathbf{K_1(t) > 0 \wedge K_2(t)} > 0$: Für die Ableitungen von $\tilde{H}$ ergeben sich
\[\begin{split}
\frac{d \tilde{H}}{d C_L} = - 2 K_2(t) C_L(t) + K_1(t) &\stackrel{!}{=} 0 \\\
\frac{d^2 \tilde{H}}{d C_L^2} = - 2 K_2(t) &\leq 0 \Rightarrow \text{Maximum}
\end{split}\]
So folgt für die Steuerfunktion $C_L(t) = \dfrac{K_1(t)}{2 K_2(t)} > 0$. Für das Minimum bestimme man nun die Nullstellen von $\tilde{H}$:
        \end{enumerate}
      \end{block}
\end{frame}


  \setbeamertemplate{itemize/enumerate subbody begin}{  \tiny}
\begin{frame}
  \frametitle{Minimumprinzip von Pontryagin}
    \tiny
  \begin{block}{Synthesesteuerung}  
        \begin{enumerate}
\item[9.1.)] Für die Bestimmung der Nullstellen von $\tilde{H}$ gilt
\[\begin{split}
\tilde{H} \stackrel{!}{=} 0 &= - K_2(t) C_L^2(t) + K_1(t)C_L(t) \\\
&= (- K_2(t) C_L(t) + K_1(t)) C_L(t) \\\
0 &= K_1(t) - K_2(t)C_L(t) \\\
C_L(t) &= \dfrac{K_1(t)}{K_2(t)}
\end{split}\]
So gilt $\tilde{H} = 0$ bei $C_L(t) = 0$ und $C_L(t) = \dfrac{K_1(t)}{K_2(t)}$ Falls $C_L(t) = \dfrac{K_1(t)}{K_2(t)} < C_{L,\max}$, so gilt also $C_L(t) = C_{L, \max}$.
%
\item[9.2.)] Falls $C_L(t) = \dfrac{K_1(t)}{K_2(t)} > C_{L,\max}$, so gilt $C_L(t) = C_{L,\min}$.
\item[9.3.)] Falls $C_L(t) = \dfrac{K_1(t)}{K_2(t)} = C_{L,\max}$, so gilt $C_L(t) = C_{L,\max} \vee C_{L,\min}$.
    \end{enumerate}
      \end{block}
\end{frame}

\end{document}