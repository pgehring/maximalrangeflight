\documentclass[aspectratio=169]{beamer}
\mode<presentation>
{
  \usetheme{myulm}
  \setbeamercovered{transparent}
  \setbeamertemplate{navigation symbols}{} % no navigation bar
  \setbeamersize{sidebar width left=0.17cm}
  
  \setbeamertemplate{blocks}[rounded][shadow=false]

  \setbeamercolor{block body alerted}{bg=alerted text.fg!10}
  \setbeamercolor{block title alerted}{bg=alerted text.fg!20}
  \setbeamercolor{block body}{bg=structure!10}
  \setbeamercolor{block title}{bg=structure!20}
  \setbeamercolor{block body example}{bg=green!10}
  \setbeamercolor{block title example}{bg=green!20}
}

\usepackage[ngerman]{babel}
\usepackage[utf8]{inputenc}
\usepackage{amsmath,amssymb,amsfonts}
\usepackage{times}
\usepackage{graphicx}
\usepackage{fancyvrb}
\usepackage{array}
\usepackage{colortbl}
\usepackage[framemethod=TikZ]{mdframed}

\usepackage{amssymb, amsmath, amsthm} % Pakete für Mathe-Umgebungen und -Symbole

% Mengen Buchstaben:
\newcommand{\E}{\mathbb{E}}
\newcommand{\Q}{\mathbb{Q}}
\newcommand{\R}{\mathbb{R}}
\newcommand{\N}{\mathbb{N}}

% Anfang der Titelfolie
% Anpassung von: Titel, Untertitel, Autor, Datum und Institut
\newcommand{\fakultaet}{Mathematik und Wirtschaftswissenschaften}
\newcommand{\institut}{Institut für numerische Mathematik}
\newcommand{\Heiko}{Heiko Karus}
\newcommand{\Philipp}{Philipp Gehring}
\newcommand{\Felix}{Felix Götz}

\title{\textbf{Optimalsteuerungsproblem  \\ \textit{Maximal Range Flight}}}
\subtitle{Projektpräsentation -- Numerik 4}
\author{\Heiko,\ \Philipp,\ \Felix}
\newcommand{\presdatum}{\today} % alternativ zu \today: Eingabe eines festen Datums
\institute{\institut, \fakultaet\\}
%Ende der Titelfolie

% Anfang der Kopfzeile der Folien
% Anpassung von: Zwischentitel, Leitthema oder Name
% Das Datum wird oben geändert: unter \presdatum{}!

\newcommand{\zwischentitel}{Projektpräsentation}
\newcommand{\leitthema}{\insertsection}
% Ende der Kopfzeile

% Anfang der Folien
\begin{document}
\hspace*{-0.6cm}
\setcounter{framenumber}{-1}
\frame[plain]{\titlepage}



% Das Inhaltsverzeichnis
\section{Inhaltsverzeichnis}
\begin{frame}
  \frametitle{Gliederung}
  \tableofcontents
\end{frame}


% 1. Folie
\section{Einleitung}
\begin{frame}
  \frametitle{Maximal Range Flight}
  \vspace{1em}
  \begin{itemize}
    \item Berechnung von Flugtrajektorien zur Reichweitenmaximierung haben eine lange Historie
    \item Einfache Lösung für den Horizontalflug: Maximierung des Verhältnisses $\sqrt{C_{L}/C_{D}}$ der Auftriebs- und Widerstandskoeffizienten
    \item Erweiterung auf den Steigflug, zweidimensionaler Flug in der $x$-$h$-Ebene, Steuerung von Auftriebsbeiwert $C_L(t)$ und Schub $T(t)$ 
    \item  Ziel: Von gegebener Anfangsposition zur Reisehöhe $h_f = 10668\ m$ und ein Anstellwinkel $\gamma_f = 0\ ^{\circ}$ in einer Flugzeit von $1800 \ s$ zu steuern, wobei die zurückgelegte Strecke maximal wird
    \item Steuerbeschränkungen für Schub und Auftriebsbeiwert und maximaler Staudruck
  \end{itemize}
\end{frame}

\section{Problemformulierung}
\begin{frame}
  \frametitle{Modellskizze}
  \vspace{1em}
   \begin{center}
  	\includegraphics[scale=.75]{images/Flugzeug.pdf}
   \end{center}
\end{frame}


\begin{frame}
  % \frametitle{Optimalsteuerungsproblem}
  \begin{center}
  \begin{block}{Optimalsteuerungsproblem - Maximal Range Flight}
    \scriptsize   
    Für das Optimalsteuerungsproblem ergibt sich mit dem Zustandsvektor
    $X(t) = (h(t),\gamma(t),x(t),v(t))^T$
    und der Steuerfunktion
    $U(t) = (T(t),C_L(t))^T$
    das Problem
    \begin{align*}
        \min_{U} F(X,U) &:= g(X(t_f)) + \int_{t_0}^{t_f} f_0(X(t),U(t)) dt = -(x(t_f) - x_0) & & \\\
        \text{unter} \hspace{20mm} \dot{X}(t) &= f(X(t),U(t)) =     
         \begin{pmatrix}
         \dot{h}(t)  \\ 
         \dot{\gamma}(t)  \\ 
         \dot{x}(t)  \\ 
         \dot{v}(t)   \\ 
	 \end{pmatrix} 
        = 
        \begin{pmatrix}
            v(t) \sin(\gamma(t)) \\ 
            \dfrac{L(v(t),h(t),C_L(t)) - W \cos(\gamma(t))}{mv(t)} \\ 
            v(t) \cos(\gamma(t)) \\ 
            \dfrac{T(t) - D(v(t),h(t),C_L(t)) - W \sin(\gamma(t))}{m}
        \end{pmatrix} & & \\\
        (h,\gamma,x,v)(t_0) &= (h_0,\gamma_0,x_0,v_0) & & \\\
        (h,\gamma)(t_f) &= (h_f,\gamma_f) & & \\\
        q(v(t),h(t)) &\leq q_{\max}  \forall t \in [t_0,t_f]\\\
        U(t) &= (T(t),C_L(t))^T \in \mathcal{U} = \left[ 
        \begin{matrix}
            [T_{\min},T_{\max}] \\ 
            [C_{L, \min},C_{L, \max}]
        \end{matrix} 
        \right]  \forall t \in [t_0,t_f]
    \end{align*}
  \end{block}
  \end{center}
\end{frame}



\begin{frame}
  % TODO: section title aus
  \frametitle{Autonomes Mayer-Problem???}
\vspace{-2.6cm}
  \begin{itemize}
    \item ????
  \end{itemize}
\end{frame}


% 2. Folie
\section{Lösung mit direktem Verfahren}
\begin{frame}
  % TODO: section title aus
  \frametitle{Vollständige Diskretisierung}
\vspace{-2.6cm}
  \begin{itemize}
    \item Das ist eine M\"{o}glichkeit f\"{u}r eine Aufz\"{a}hlung
    \item Aufz\"{a}hlung 2
    \item Aufz\"{a}hlung 3
  \end{itemize}
\end{frame}

\begin{frame}
  % TODO: section title aus
  \frametitle{Auswahl Lösungsverfahren}
\vspace{-2.6cm}
  \begin{itemize}
    \item ????
  \end{itemize}
\end{frame}

\begin{frame}
  % TODO: section title aus
  \frametitle{Vergleich von Optimierungs-Verfahren}
\vspace{-2.6cm}
  \begin{itemize}
    \item ????
  \end{itemize}
\end{frame}


\begin{frame}
  % TODO: section title aus
  \frametitle{Ergebnisse}
\vspace{-2.6cm}
  \begin{itemize}
    \item ????
  \end{itemize}
\end{frame}

\begin{frame}
  % TODO: section title aus
  \frametitle{Ergebnisse}
\vspace{-2.6cm}
  \begin{itemize}
    \item ????
  \end{itemize}
\end{frame}

\begin{frame}
  % TODO: section title aus
  \frametitle{Ergebnisse}
\vspace{-2.6cm}
  \begin{itemize}
    \item ????
  \end{itemize}
\end{frame}

% 3. Folie
\section{Lösung mit indirekten Verfahren}
\begin{frame}
  \frametitle{Minimumprinzip nach Pontryagin}
  \scriptsize
  \begin{block}{Hamilton-Funktion}  
 \begin{align*} 
        H(X(t),U(t),\lambda(t)) &= \lambda_0 f_0(X(t),U(t)) + \lambda(t)^T f(X(t),U(t)) \\\
        &= \lambda(t)^T f(X(t),U(t)) \\\
        &= \lambda_1(t) \dot{h}(t) + \lambda_2(t) \dot{\gamma}(t) + \lambda_3(t) \dot{x}(t) + \lambda_4(t) \dot{v}(t) \\\
        &= \sin(\gamma(t)) v(t) \lambda_1 \\\
        &\hspace{7mm} + \dfrac{F \alpha e^{-\beta h(t)} C_L(t) v(t) \lambda_2(t)}{2m} - \dfrac{g \cos(\gamma(t)) \lambda_2(t)}{v(t)} \\\
        &\hspace{7mm} + \cos(\gamma(t)) v(t) \lambda_3(t) \\\
        &\hspace{7mm} + \dfrac{T(t) \lambda_4(t)}{m} - \dfrac{(C_{D_0} + k C_L^2(t)) F \alpha e^{-\beta h(t)} v^2(t) \lambda_4(t)}{2m} \\\
        &\hspace{7mm} - g \sin(\gamma(t)) \lambda_4(t)
\end{align*}
\end{block}
\end{frame}




\begin{frame}
  \begin{block}{Minimumsbedingung}  
  \scriptsize
  Es gilt an allen Stetigkeitsstellen $t \in [t_0,t_f]$ von $u^{\ast}(t)$ die Minimumbedingung \[H(X^{\ast}(t),U^{\ast}(t),\lambda(t)) = \min_{U(t) \in \mathcal{U}} H(X^{\ast}(t),U(t),\lambda(t))\] Ableiten nach der Steuerfunktion $U(t)$ ergibt für den unbeschränkten Fall der Steuerung die Minimumbedingung
    \[\dfrac{\partial}{\partial U} H(X^{\ast}(t),U^{\ast}(t),\lambda(t)) = \begin{pmatrix}
    \dfrac{\lambda_4(t)}{m} \\ 
    - \dfrac{k F \alpha e^{-\beta h(t)} v^2(t) \lambda_4(t) C_L(t)}{m} + \dfrac{F \alpha e^{-\beta h(t)} v(t) \lambda_2(t)}{2m}
    \end{pmatrix}^T \stackrel{!}{=} 0\]
    und 
    \[\dfrac{\partial^2}{\partial U^2} H(X^{\ast}(t),U^{\ast}(t),\lambda(t)) = \begin{pmatrix}
    0 & - \dfrac{k F \alpha e^{-\beta h(t)} v^2(t) \lambda_4(t)}{m} 
    \end{pmatrix} \stackrel{!}{\geq} 0\] wobei \[\sigma(x(t),\lambda(t)) := H_u(X^{\ast}(t),U^{\ast}(t),\lambda(t))\] Schaltfunktion genannt wird. 
  \end{block}
\end{frame}

\begin{frame}
  \frametitle{Minimumprinzip nach Pontryagin}
  \begin{block}{Synthesesteuerung}  
  \scriptsize
  Für die Betrachtung mit den Beschränkungen der Steuerfunktion ergibt sich für den Schub
\[T(t) = \left\lbrace \begin{array}{ll}
T_{\min} & ,\text{falls } \lambda_4 > 0  \\ 
\text{beliebig} \in [T_{\min},T_{\max}] & ,\text{falls } \lambda_4 = 0  \\ 
T_{\max} & ,\text{falls } \lambda_4 < 0
\end{array} \right.\]
Für die Bestimmung der Steuerfunktion des Auftriebsbeiwerts lässt sich die Hamilton-Funktion verkürzt mit den Termen $K_1(t)$ und $K_2(t)$ schreiben.
    \[\begin{split}
        \tilde{H}(X^{\ast}(t),C_L(t),\lambda(t)) &= \dfrac{F \alpha e^{-\beta h^{\ast}(t)} v^{\ast}(t) \lambda_2(t)}{2m} \cdot C_L(t) - \dfrac{k F \alpha e^{-\beta h^{\ast}(t)}  v^{\ast 2}(t) \lambda_4(t)}{2m} \cdot C_L^2(t) \\\
        &= K_1(t) C_L(t) - K_2(t) C_L^2(t)
    \end{split}\]
      \end{block}
\end{frame}

\begin{frame}
  % TODO: section title aus
  \frametitle{Fallunterscheidung?????}
\vspace{-2.6cm}
  \begin{itemize}
    \item ????
  \end{itemize}
\end{frame}


\begin{frame}
  \begin{block}{Adjungierte DGL}  
  \scriptsize
Leitet man nach dem Zustandsvektor $X(t)$ ab, also $H_{X}(X^{\ast}(t),U^{\ast}(t),\lambda(t))$ so erhält man 
        \[\begin{split}
            \dfrac{\partial}{\partial h} H &= - \dfrac{\alpha \beta F e^{-\beta h(t)} C_L(t) v(t) \lambda_2(t)}{2m} + \dfrac{(C_{D_0}+k C_L^2(t)) \alpha \beta F e^{-\beta h(t)} v^2(t) \lambda_4(t)}{2m} \\\
            \dfrac{\partial}{\partial \gamma} H &= \cos(\gamma(t)) v(t) \lambda_1(t) + \dfrac{g \sin(\gamma(t)) \lambda_2(t)}{v(t)} - \sin(\gamma(t)) v(t) \lambda_3(t) - \cos(\gamma(t)) g \lambda_4(t) \\\
            \dfrac{\partial}{\partial x} H &= 0 \\\
            \dfrac{\partial}{\partial v} H &= \sin(\gamma(t)) \lambda_1(t) + \left( \dfrac{F \alpha e^{-\beta h(t)} C_L(t)}{2m} + \dfrac{g \cos(\gamma(t))}{v^2(t)} \right) \lambda_2(t) \\\
            &\hspace{7mm} + \cos(\gamma(t)) \lambda_3(t) - \dfrac{(C_{D_0} + k C_L^2(t)) F \alpha e^{-\beta h(t)} v(t) \lambda_4(t)}{m}
        \end{split}\]
        wobei gilt 
        \[\dot{\lambda}(t)^T = - \dfrac{\partial}{\partial X} H = -H_{X} = \left( -\dfrac{\partial}{\partial h} H, -\dfrac{\partial}{\partial \gamma} H, -\dfrac{\partial}{\partial x} H, -\dfrac{\partial}{\partial v} H \right)\]
  \end{block}
\end{frame}


\begin{frame}
\begin{block}{Transversalitätsbedingung} 
  \scriptsize
Im Endzeitpunkt $t_f$ gilt die Transversalitätsbedingung mit dem Vektor $\nu \in \R^{n_{\psi}}$ mit $(\lambda_0,\lambda(t),\nu) \neq 0$ für alle $t \in [t_0,t_f]$
        \[\begin{split}
            \lambda(t_f)^T &= \lambda_0 g_X(X^{\ast}(t_f)) + \nu^T \psi_X(X^{\ast}(t_f)) \\\
            &= \lambda_0 
            \begin{pmatrix}
            0 & 0 & -1 & 0
            \end{pmatrix}  
            + \nu^T 
            \begin{pmatrix}
            1 & 0 & 0 & 0 \\
            0 & 1 & 0 & 0 
            \end{pmatrix}  \\\
            &= \begin{pmatrix}
            \nu_1 & \nu_2 & -\lambda_0 & 0 
            \end{pmatrix}
        \end{split}\]
  \end{block}
  \begin{block}{Konstanz} 
  \scriptsize
 Für autonome Systeme gilt \[H(X^{\ast}(t),U^{\ast}(t), \lambda(t)) = const \ \in [t_0,t_f]\]
   \end{block}
\end{frame}


\begin{frame}
\begin{block}{Zweipunkt-Randwertproblem } 
  \scriptsize

   \end{block}
\end{frame}

\begin{frame}
  % TODO: section title aus
  \frametitle{Schwierigkeiten}
\vspace{-2.6cm}
  \begin{itemize}
    \item ????
  \end{itemize}
\end{frame}

\begin{frame}
  % TODO: section title aus
  \frametitle{Ergebnisse}
\vspace{-2.6cm}
  \begin{itemize}
    \item ????
  \end{itemize}
\end{frame}


% 2. Folie
\section{Fazit und Ausblick}
\begin{frame}
  % TODO: section title aus
  \frametitle{Fazit und Ausblick}
\vspace{-2.6cm}
  \begin{itemize}
    \item Ergebnisse 
    \item Schwierigkeiten
    \item Möglichkeiten
  \end{itemize}
\end{frame}


\end{document}