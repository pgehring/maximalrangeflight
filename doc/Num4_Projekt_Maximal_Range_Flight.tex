%==============================================================================
%------------------------------------------------------------------------------
% Notizen, Bemerkungen, Quellen
%------------------------------------------------------------------------------
% Bemerkungen:
% 	- Kompilieren mit PDFLaTeX
%	- 

% Notizen:
%	- 

% Quellen:
%	-
%==============================================================================






%==============================================================================
%------------------------------------------------------------------------------
% Dokument Einstellungen
%------------------------------------------------------------------------------
\NeedsTeXFormat{LaTeX2e}
\documentclass[
a4paper,
11pt,
headsepline,           % Linie zw. Kopfzeile und Text
oneside,               % einseitig
numbers=noenddot,	   % keine Punkte nach den letzten Ziffern in Überschriften
bibliography=totoc,    % LV in IV
%DIV=15,                % Satzspiegel auf 15er Raster, schmalere Ränder   
%BCOR15mm              % Bindekorrektur
%,draft
]{scrbook}
\KOMAoptions{DIV=last} % Neuberechnung Satzspiegel nach Laden von Paket helvet
%==============================================================================






%==============================================================================
%------------------------------------------------------------------------------
% Seitenlayout
%------------------------------------------------------------------------------
\pagestyle{headings}
\usepackage{blindtext}

% für Texte in deutscher Sprache
\usepackage[ngerman]{babel}
\usepackage[utf8]{inputenc}
\usepackage[T1]{fontenc}

% Helvetica als Standard-Dokumentschrift
\usepackage[scaled]{helvet}
\renewcommand{\familydefault}{\sfdefault} 

\usepackage{geometry}
\usepackage{layout}

\newcounter{SeitenzahlSpeicher}

%------------------------------------------------------------------------------
% Grafiken
%------------------------------------------------------------------------------
\usepackage{graphicx}
\usepackage{float} % um figure/Bild/Abbildung an bestimmte Stelle mit [H]
\usepackage{framed}

%------------------------------------------------------------------------------
% Literaturverzeichnis mit BibLKaTeX
%------------------------------------------------------------------------------
\usepackage[babel,german=quotes]{csquotes}
%\usepackage[backend=bibtex8]{biblatex}
\usepackage[backend=biber,bibencoding=utf8,style=numeric]{biblatex}
%\bibliography{bibliography}
\addbibresource{bibliography.bib}

%------------------------------------------------------------------------------
% Für Tabellen mit fester Gesamtbreite und variabler Spaltenbreite
%------------------------------------------------------------------------------
\usepackage{tabularx} 

%------------------------------------------------------------------------------
% Besondere Schriftauszeichnungen
%------------------------------------------------------------------------------
\usepackage{url}              % \url{http://...} in Schreibmaschinenschrift
\usepackage{color}            % zum Setzen farbigen Textes
\usepackage[dvipsnames]{xcolor}

%------------------------------------------------------------------------------
% Pakete für Mathe-Umgebungen, -Symbole und Neudefinition von Kommandos
%------------------------------------------------------------------------------
\usepackage{amssymb, amsmath, amsthm} % Pakete für Mathe-Umgebungen und -Symbole

% Mengen Buchstaben:
\newcommand{\E}{\mathbb{E}}
\newcommand{\Q}{\mathbb{Q}}
\newcommand{\R}{\mathbb{R}}
\newcommand{\N}{\mathbb{N}}

%------------------------------------------------------------------------------
% Mathematische Umgebungen (Satz, Definition, Beweis ...)
%------------------------------------------------------------------------------
% Quellen:
%	Umgebungen: 	https://www.youtube.com/watch?v=_KFr29O7Jrc
%					https://de.overleaf.com/learn/latex/Environments
%	Zähler: https://www.youtube.com/watch?v=_NjyOVC54aQ

\usepackage{ifthen}

% Zähler für die Nummerierung der Umgebungen und dessen Referenzen
% 	Zu jedem Zähler umgnr gehört der Ausgabebefehl \theumgnr, der u.a. von 
% 	\ref aufgerufen wird (siehe Quelle).
\newcounter{umgnr}[chapter]
\renewcommand{\theumgnr}{\arabic{chapter}.\arabic{umgnr}}

% Umgebung "Satz":
\newenvironment{satz}[1][]
{% \begin{satz}
\medskip\noindent%
\refstepcounter{umgnr}% Zähler erhöhen
\textbf{Satz~\theumgnr \ifthenelse{\equal{#1}{}}{}{~(#1)}}
}
{% \end{satz}
\medskip
}

% Umgebung "Hilfssatz":
\newenvironment{hilfssatz}[1][]
{\medskip\noindent \refstepcounter{umgnr}%
\textbf{Hilfssatz~\theumgnr\ifthenelse{\equal{#1}{}}{}{~(#1)}}
}
{\medskip}

% Umgebung "Definition":
\newenvironment{definition}[1][]
{\medskip\noindent \refstepcounter{umgnr}%
\textbf{Definition~\theumgnr\ifthenelse{\equal{#1}{}}{}{~(#1)}}
}
{\medskip}

% Umgebung "Bemerkung":
\newenvironment{bemerkung}[1][]
{\medskip\noindent \refstepcounter{umgnr}%
\textbf{Bemerkung~\theumgnr\ifthenelse{\equal{#1}{}}{}{~(#1)}}
}
{\medskip}

% Umgebung "Beispiel":
\newenvironment{beispiel}[1][]
{\medskip\noindent \refstepcounter{umgnr}%
\textbf{Beispiel~\theumgnr\ifthenelse{\equal{#1}{}}{}{~(#1)}}
}
{
\hfill $\diamond$
\medskip
}

% Umgebung "Lemma":
\newenvironment{lemma}[1][]
{\medskip\noindent \refstepcounter{umgnr}%
\textbf{Lemma~\theumgnr\ifthenelse{\equal{#1}{}}{}{~(#1)}}
}
{\medskip}

% Umgebung "Korollar":
\newenvironment{korollar}[1][]
{\medskip\noindent \refstepcounter{umgnr}%
\textbf{Korollar~\theumgnr\ifthenelse{\equal{#1}{}}{}{~(#1)}}
}
{\medskip}

% Umgebung "Beweis":
\newenvironment{beweis}[1][]
{% \begin{beweis}
\textbf{Beweis\ifthenelse{\equal{#1}{}}{:}{~(#1):}}
\begin{itshape}
}
{% \end{beweis}
\end{itshape}
\hfill $\Box$
\medskip
}

%------------------------------------------------------------------------------
% Formatierung
%------------------------------------------------------------------------------
\usepackage{enumitem}
\usepackage{setspace}         % Paket für div. Abstände, z.B. ZA
%\onehalfspacing              % nur dann, wenn gefordert; ist sehr groß!!
\setlength{\parindent}{0pt}   % kein linker Einzug der ersten Absatzzeile
\setlength{\parskip}{1.4ex plus 0.35ex minus 0.3ex} % Absatzabstand, leicht variabel

% Tiefe, bis zu der Überschriften in das Inhaltsverzeichnis kommen
\setcounter{tocdepth}{3}      % ist Standard

%------------------------------------------------------------------------------
% Quellcode
%------------------------------------------------------------------------------
%https://en.wikibooks.org/wiki/LaTeX/Source_Code_Listings
% Beispiele für Quellcode
\usepackage{listings}
\lstdefinestyle{num_octave}{
  basicstyle=\scriptsize,        % the size of the fonts that are used for the code
  breakatwhitespace=false,         % sets if automatic breaks should only happen at whitespace
  breaklines=true,                 % sets automatic line breaking
  captionpos=b,                    % sets the caption-position to bottom
  commentstyle=\color{ForestGreen},    % comment style
  escapeinside={\%*}{*)},          % if you want to add LaTeX within your code
  extendedchars=true,              % lets you use non-ASCII characters; for 8-bits encodings only, does not work with UTF-8
  frame=single,	                   % adds a frame around the code
  %keepspaces=true,                 % keeps spaces in text, useful for keeping indentation of code (possibly needs columns=flexible)
  keywordstyle=\color{blue},       % keyword style
  language=Octave,                 % the language of the code
  numbers=left,                    % where to put the line-numbers; possible values are (none, left, right)
  numbersep=5pt,                   % how far the line-numbers are from the code
  numberstyle=\tiny, % the style that is used for the line-numbers
  numberbychapter=true
  rulecolor=\color{black},         % if not set, the frame-color may be changed on line-breaks within not-black text (e.g. comments (green here))
  stringstyle=\color{black},     % string literal style
  tabsize=4,	                   % sets default tabsize to 2 spaces
}
\renewcommand{\lstlistingname}{Programmcode}% Listing -> Programmcode
\renewcommand{\lstlistlistingname}{\lstlistingname s Verzeichnis}% List of Listings -> Programmcodes

  
%https://www.zaik.uni-koeln.de/AFS/teachings/ws0708/ORSeminar/latex/seminar.pdf
%https://en.wikibooks.org/wiki/LaTeX/Algorithms
\usepackage{algorithmic}
\usepackage{algorithm}

\usepackage{subfloat}
\usepackage{subcaption}
\usepackage{mathrsfs}

%verhindet, dass sich Fußnote auf zwei Seiten erstreckt
\interfootnotelinepenalty=10000 

%------------------------------------------------------------------------------
% Persönliche Daten
%------------------------------------------------------------------------------
\newcommand{\Heiko}{Heiko Karus}
\newcommand{\Philipp}{Philipp Gehring}
\newcommand{\Felix}{Felix Götz}

\newcommand{\Hemail}{Heiko.Karus@uni-ulm.de}
\newcommand{\Pemail}{Philipp.Gehring@uni-ulm.de}
\newcommand{\Femail}{Felix.Goetz@uni-ulm.de}

\newcommand{\Hmatnr}{1072378}
\newcommand{\Pmatnr}{1104946}
\newcommand{\Fmatnr}{1063352}

\newcommand{\titel}{Maximal range flight}
\newcommand{\jahr}{2021}
\newcommand{\gutachterA}{Prof. Dr. rer. nat. Dirk Lebiedz}
\newcommand{\gutachterB}{M.Sc. Jörn Dietrich}

\newcommand{\fakultaet}{Mathematik und Wirtschaftswissenschaften}
\newcommand{\institut}{Institut für numerische Mathematik}

%------------------------------------------------------------------------------
% Informationen, die LaTeX in die PDF-Datei schreibt
%------------------------------------------------------------------------------
\pdfinfo{
  /Author (Karus, Gehring, Goetz)
 % /Title (\titel)
  /Producer     (pdfeTex 3.14159-1.30.6-2.2)
  /Keywords ()
}

%\usepackage{hyperref}
\usepackage[colorlinks=true, urlcolor=blue, linkcolor=green]{hyperref}
\hypersetup{
pdftitle=\titel,
pdfauthor=\Heiko \Philipp \Felix,
pdfsubject={Projekt Numerik 4},
pdfproducer={pdfeTex 3.14159-1.30.6-2.2},
colorlinks=false,
pdfborder=0 0 0	% keine Box um die Links!
}

%------------------------------------------------------------------------------
% Weiter hinzugefügte Pakete
%------------------------------------------------------------------------------
\usepackage{xcolor, colortbl}

\counterwithout{footnote}{chapter}
\usepackage{multirow}
\usepackage{longtable}
%==============================================================================






%==============================================================================
%------------------------------------------------------------------------------
% Beginn Dokument
%------------------------------------------------------------------------------
% Trennungsregeln
\hyphenation{Sil-ben-trenn-ung}

\begin{document}

% Seitenlayout
\newgeometry{left=2.5cm, right=2.5cm, top=3cm, bottom=3cm}

\frontmatter

%------------------------------------------------------------------------------
% Titelseite
%------------------------------------------------------------------------------
\thispagestyle{empty}
\begin{addmargin*}[4mm]{-10mm}

\includegraphics[height=1.8cm]{images/00_Sonstiges/unilogo_bild}
\hfill
\includegraphics[height=1.8cm]{images/00_Sonstiges/unilogo_wort}\\[2em]

%\includegraphics[height=1.8cm]{images/00_Sonstiges/THU}
%\hfill
%\includegraphics[height=1.8cm]{images/00_Sonstiges/unilogo_bild}
%\includegraphics[height=1.8cm]{images/00_Sonstiges/unilogo_wort}\\[2em]

%\includegraphics[height=1.4cm]{images/unilogo_wort}
%\hfill
%\includegraphics[height=1.4cm]{images/THU_word}\\[1em]
%\includegraphics[height=1.4cm]{images/unilogo_bild}
%\hfill
%\includegraphics[height=1.4cm]{images/THU_logo}\\[1em]

{\footnotesize
%{\bfseries Universität Ulm} \textbar ~89069 Ulm \textbar ~Germany
\hspace*{130mm}\parbox[t]{35mm}{
\bfseries Fakultät für\\
\fakultaet\\
\mdseries \institut
}\\[2cm]

\parbox{140mm}{\bfseries \LARGE \titel}\\[2.5em]
{\footnotesize Projekt Numerik 4}\\[2em]

{\footnotesize \bfseries Vorgelegt von:}\\
{\footnotesize \Heiko \\ E-Mail: \Hemail \\ Matrikel-Nr.: \Hmatnr}\\ \\%[2em]
{\footnotesize \Philipp \\ E-Mail: \Pemail \\ Matrikel-Nr.: \Pmatnr}\\ \\%[2em]
{\footnotesize \Felix \\ E-Mail: \Femail \\ Matrikel-Nr.: \Fmatnr}\\ \\[2em]

{\footnotesize \bfseries Gutachter:}\\                     
{\footnotesize \gutachterA}\\ \\%[2em]
{\footnotesize \gutachterB}\\[2em]

{\footnotesize \jahr}
}
\end{addmargin*}

%------------------------------------------------------------------------------
% Impressum
%------------------------------------------------------------------------------
\clearpage
\thispagestyle{empty}
{ \small
  \flushleft
  Fassung \today \\\vfill
  \copyright~\jahr~\Heiko,~\Philipp,~\Felix\\[0.5em]
% Wenn Sie Ihre Arbeit unter einer freien Lizenz bereitstellen möchten, können Sie die nächste Zeile in Ihren Code aufnehmen. Bitte beachten Sie, dass Sie hierfür an allen Inhalten, inklusive enthaltener Abbildungen, die notwendigen Rechte benötigen! Beim Veröffentlichungsexemplar Ihrer Dissertation achten Sie bitte darauf, dass der Lizenztext nicht den Angaben in den Metadaten der genutzten Publikationsplattform widerspricht. Nähere Information zu den Creative Commons Lizenzen erhalten Sie hier: https://creativecommons.org/licenses/
%This work is licensed under the Creative Commons Attribution 4.0 International (CC BY 4.0) License. To view a copy of this license, visit \href{https://creativecommons.org/licenses/by/4.0/}{https://creativecommons.org/licenses/by/4.0/} or send a letter to Creative Commons, 543 Howard Street, 5th Floor, San Francisco, California, 94105, USA. \\
  Satz: PDF-\LaTeXe
}

% ab hier Zeilenabstand etwas größer 
\setstretch{1.2}

%------------------------------------------------------------------------------
% Vorwort, Kurzfassung und Abstract
%------------------------------------------------------------------------------
% Vorwort
%\clearpage
%\input{chapters/Vorwort}

%Kurzfassung
%\clearpage
%\input{chapters/Kurzfassung}

% Abstract
%\clearpage
%\input{chapters/Abstract}

%------------------------------------------------------------------------------
% Inhaltsverzeichnis
%------------------------------------------------------------------------------
\tableofcontents
\addcontentsline{toc}{chapter}{Inhaltsverzeichnis}
\setcounter{SeitenzahlSpeicher}{\value{page}}
%------------------------------------------------------------------------------
% Hauptteil
%------------------------------------------------------------------------------
\mainmatter
\chapter{Motivation und Ziel der Projektarbeit}
Die Berechnung von Flugtrajektorien zur Reichweitenmaximierung hat eine lange Historie, siehe \cite{Burrows1982, Murrieta2016, Schaback2017, Pierson1989}.  Dabei wurden verschiedene mathematische Techniken angewandt, die von Parametrisierungen von Trajektorien \cite{Burrows1982}, Energiebetrachtungen \cite{Calise1977}, über Mehrzieloptimierung mit verschiedenen Kostenfunktionen \cite{Gardi2016} bis hin zur Anwendung der Optimalsteuerungstheorie \cite{Javier2016}  reichen. 

Eine einfache Lösung für den Reichweiten maximierten Flug ist für den Horizontalflug bekannt, siehe beispielsweise \cite{Peckham1974}. Diese folgt aus der Maximierung des Verhältnisses $\sqrt{C_{L}/C_{D}}$ der Auftriebs- und Widerstandskoeffizienten. Daraus resultiert eine Geschwindigkeit, die um den Faktor $\sqrt{4/3}$ größer ist als die Geschwindigkeit zur Maximierung des Verhältnisses von Auftriebs- zu Widerstandskoeffizienten $C_{L}/C_{D}$ \cite{Schaback2017}. 

Diese Projektarbeit bietet eine Erweiterung auf den Steigflug und konzentriert sich dabei auf numerische Standardmethoden, die für das Lösen von Optimalsteuerungsproblemen verwendet werden. Modelliert wird dabei ein zweidimensionaler Flug eines Flugzeugs in der $x$-$h$-Ebene (Abbildung \ref{img:Flugzeug}), bei dem der Auftriebsbeiwert $C_L(t)$ und der Schub $T(t)$ gesteuert werden kann.

\begin{figure}[H]
    \begin{center}
        \includegraphics[width=\textwidth]{images/01_Modellaufbau/Flugzeug.pdf}
        \MyCaption{Freikörperdiagramm eines Flugzeuges in der $x$-$h$-Ebene}{mit angreifenden Kräften $L$ (Auftriebskraft), $D$ (Luftwiederstand), $W$ (Erdanziehungskraft) und $T$ (Schub) am Schwerpunkt $S$. Des Weiteren beschreibt $v$ die Geschwindigkeit und $\gamma$ den Anstellwinkel.}\label{img:Flugzeug}
    \end{center}
\end{figure}

Ziel ist es, das Flugzeug von einer gegebenen Anfangsposition so zu steuern, dass eine vorgegebene Reisehöhe $h_f = 10668\ m$ und ein Anstellwinkel $\gamma_f = 0\ ^{\circ}$ in einer Flugzeit von $1800 \ s$ erreicht  wird, wobei die zurückgelegte Strecke maximal wird. Dabei dürfen die Steuerbeschränkungen für den Schub und Auftriebsbeiwert nicht verletzt und ein maximaler Staudruck nicht überschritten werden. 

Beginnend mit der Herleitung der Gleichungen für den quasi-statischen Flug in \autoref{cha:optim}, folgt die Formulierung des Optimalsteuerungsproblems für direkte (\autoref{cha:direct}) und indirekte (\autoref{cha:indirect}) Lösungsverfahren. Anschließend folgt die numerische Umsetzung und Lösung des Optimalsteuerungsproblems (Kapitel \ref{kap:TUNU}), sowie die Diskussion der Ergebnisse (Kapitel \ref{kap:LSG}) .

Alle Modellrechnungen wurden für den Airbus A380-800 \cite{A380Tech} durchgeführt. \textit{MATLAB} wurde für alle numerischen Berechnungen verwendet. Hierzu wurde teils auf enthaltene Funktionen zurückgegriffen und teils eigene Funktionen zur Lösung von Optimalsteuerungsproblemen implementiert.





\chapter{Optimalsteuerungsproblem} \label{cha:optim}

Im folgenden wird ein Differentialgleichungssystem für den quasi-statischen Flug hergeleitet. Modelliert wird das Flugzeug A380-800 der Firma Airbus bei einem Steigflug. Es wird sowohl die Dynamik in vertikaler ($y$-Achse) als auch in horizontaler ($h$-Achse) Richtung berücksichtigt. Dabei seien
\begin{itemize}
    \item $x(t)$: $x$-Koordinate des Massenschwerpunktes $S$
    \item $h(t)$: $h$-Koordinate des Massenschwerpunktes $S$
    \item $v(t)$: Geschwindigkeit
    \item $\gamma(t)$: Anstellwinkel
    \item $T(t)$: Schub (Steuerung)
    \item $C_L(t)$: Auftriebsbeiwert (Steuerung)
\end{itemize}
Um die Kräfte welche auf das Flugzeug einwirken, berechnen zu können, werden folgende Hilfsgrößen benötigt:
\begin{itemize}
    \item Luftwiderstandsbeiwert: \[C_D(C_L(t)) := C_{D_0} + k \cdot C^2_L(t) \ \ \ \ \text{mit} \ \ \ \ k = \dfrac{1}{\pi \cdot e \cdot AR}\] wobei $C_{D0}$ der Nullluftwiderstandsbeiwert, $e$ die Oswaldfaktor und $AR$ die Streckung (engl. \textit{aspect ratio}) bezeichnet. Dabei sind $C_{D0}$ und $k$ abhängig von der Machzahl, jedoch wird dieser Effekt zur Vereinfachung ignoriert. 
    
    \item Luftdichte: \[\rho(h(t)) := \alpha \cdot e^{-\beta \cdot h(t)}\]  wobei für die Berechnung der höhenabhängigen Luftdichte ein einfaches Exponentialmodell verwendet wird.%https://wind-data.ch/tools/luftdichte.php

    \item Staudruck: \[q(v(t), h(t)) := \dfrac{\rho(h(t)) \cdot v^2(t)}{2} \]
\end{itemize}

Neben Gewichtskraft \(W\), Auftrieb \(L\) und Luftwiderstand \(D\) wird auch der Schub \(T\) als Kraft betrachtet. Daneben wird der Einfluss von Klappen, Spoiler und ausgefahrenen Fahrwerken vernachlässigt. Die nachfolgenden Gleichungen beziehen sich auf kurze Zeitintervalle, in denen die Geschwindigkeit $v$ und der Anstellwinkel $\gamma$ als konstant angesehen werden. Wie jedoch gezeigt wurde, führen sie zu nützlichen Gleichungen, die die langfristigen Änderungen von $v$ und $\gamma$ beschreiben. Die am Flugzeug angreifenden Kräfte lassen sich wie folgt berechnen:
\begin{itemize}
    \item Auftriebskraft: \[L(v(t), h(t), C_L(t)) := F \cdot C_L(t) \cdot q(v(t), h(t))\] wobei $F$ die wirksame Fläche, d.h. die von der Luft angeströmte Fläche, ist.
    %
    \item Luftwiderstand: \[D(v(t), h(t), C_L(t)) := F \cdot C_D(C_L(t)) \cdot q(v(t), h(t))\]
    \item Gewichtskraft: \[W = m \cdot g\] wobei $m$ die Masse des Flugzeugs und $g$ die Gravitationskonstante darstellt.
\end{itemize}

Mit dem 2. Newtons'schen Axiom $F = m \cdot a$ lässt sich die Differentialgleichung
\[F := m \cdot a \Rightarrow \dot{v}(t) = a(t) = \dfrac{F(t)}{m} = \dfrac{T(t) - D(v(t),h(t),C_L(t)) - W \sin(\gamma(t))}{m}\]
für die Geschwindigkeit $v$ aufstellen.
Mit der Gleichung für die Zentripetalkraft $F_{ZP} = \dfrac{m v^2}{r}$ lässt sich die Differentialgleichung
\[F_{ZP} := \dfrac{m v^2}{r} \Rightarrow \dot{\gamma}(t) = \dfrac{v(t)}{r} = \dfrac{F_{ZP}(t)}{m v(t)} = \dfrac{L(v(t),h(t),C_L(t)) - W \cos(\gamma(t))}{m v(t)}\]
für den Ansstellwinkel $\gamma$ aufstellen.
Die Differentialgleichungen für die $h(t)$ und $x(t)$ lassen mittels der Geschwindigkeit und des Anstellwinkels bestimmen.
\[\begin{split}
    \dot{x}(t) &= v(t) \cos(\gamma(t))\\\
    \dot{h}(t) &= v(t) \sin(\gamma(t))
\end{split} \]

Es ergibt sich somit das Optimalsteuerungsproblem (Problem \ref{prob:MaxRF}) mit den Funktionen $g : \R^{n_X} \to \R$, $f_0 : \R^{n_X} \times \R^{n_U} \to \R$, $f : \R^{n_X} \times \R^{n_U} \to \R^{n_X}$ und $U : [t_0,t_f] \to \R^m$ für $0 \leq (n_{\psi} = 2) \leq (n_X = 4)$ und $n_U = 2$.

\begin{problem}[Optimalsteuerungsproblem - Maximal-Range-Flight]\label{prob:MaxRF}
    Für das Optimalsteuerungsproblem ergibt sich mit dem Zustandsvektor
    \[X(t) = (h(t),\gamma(t),x(t),v(t))^T\]
    und der Steuerfunktion
    \[U(t) = (T(t),C_L(t))^T\]
    das Problem:
    \begin{align*}
        \min_{U} F(X,U) &:= g(X(t_f)) + \int_{t_0}^{t_f} f_0(X(t),U(t)) dt = -(x(t_f) - x_0) & & \\\
        \text{unter} \hspace{20mm} \dot{X}(t) &= f(X(t),U(t)) =     
         \begin{pmatrix}
         \dot{h}(t)  \\ 
         \dot{\gamma}(t)  \\ 
         \dot{x}(t)  \\ 
         \dot{v}(t)   \\ 
	 \end{pmatrix} 
        = 
        \begin{pmatrix}
            v(t) \sin(\gamma(t)) \\ 
            \dfrac{L(v(t),h(t),C_L(t)) - W \cos(\gamma(t))}{mv(t)} \\ 
            v(t) \cos(\gamma(t)) \\ 
            \dfrac{T(t) - D(v(t),h(t),C_L(t)) - W \sin(\gamma(t))}{m}
        \end{pmatrix} & & \\\
        (h,\gamma,x,v)(t_0) &= (h_0,\gamma_0,x_0,v_0) & & \\\
        (h,\gamma)(t_f) &= (h_f,\gamma_f) & & \\\
        q(v(t),h(t)) &\leq q_{\max}  \forall t \in [t_0,t_f]\\\
        U(t) &= (T(t),C_L(t))^T \in \mathcal{U} = \left[ 
        \begin{matrix}
            [T_{\min},T_{\max}] \\ 
            [C_{L, \min},C_{L, \max}]
        \end{matrix} 
        \right]  \forall t \in [t_0,t_f]
    \end{align*}
\end{problem}

Des Weiteren sei $\psi : \R^{n_X} \to \R^{n_{\psi}}$ eine $C^1$-Funktion
\[\psi(X(t_f)) = 
\begin{pmatrix}
    h(t_f) - h_f \\ 
    \gamma(t_f) - \gamma_f
\end{pmatrix} = 0_{n_{\psi}}\]
Das Optimalsteuerungsproblem (Problem \ref{prob:MaxRF}) stellt also ein autonomes Mayer-Problem der Form 
\begin{equation} \label{equ:mayer_problem}
    \begin{aligned}
        \min F(X,U) :&= g(X(t_f))  \\
        \text{unter}  \hspace{10mm} \dot{X}(t) &= f(X(t),U(t)) & & \forall t \in [t_0,t_f] \\
        %
        X(t_0) &= X_0 = (h_0,\gamma_0,x_0,v_0)^T & & \\
        \psi(X(t_f)) &= 0_{n_{\psi}} & & \\
        %
        q(X(t)) &\leq q_{\max} & & \forall t \in [t_0,t_f] \\
        U(t) &= (T(t),C_L(t))^T \in \mathcal{U}  & & \forall t \in [t_0,t_f] 
    \end{aligned}
\end{equation}
da, mit der zusätzlichen Beschränkung des Staudrucks $q(v(t),h(t))$ und der konkreten Funktion $f$
\begin{equation} \label{equ:state_space}
    f(X(t),U(t)) = \dot{X}(t) = \begin{pmatrix}
        v(t) \sin(\gamma(t)) \\ 
        \dfrac{F \alpha e^{-\beta h(t)} v(t) C_L(t)}{2m} - \dfrac{g \cos(\gamma(t))}{v(t)} \\ 
        v(t) \cos(\gamma(t)) \\ 
        \dfrac{T(t)}{m} - \dfrac{(C_{D_0} + k C_L^2(t)) F \alpha e^{-\beta h(t)} v^2(t)}{2m} - g \sin(\gamma(t))
    \end{pmatrix}
\end{equation}
Für das Modell werden die Parameter aus Tabelle \ref{tab:ProblemPara} in Anhang \ref{Anhang:ModellPara} verwendet.

\chapter{Technische Umsetzung}

Zur Umsetzung der Codes wurde das Programm Matlab verwendet.

\section{Matlab Funktionen}

\subsection{Numerische Optimierung von beschränkten Problemen}
Mit der Matlab Funktion \verb|fmincon| lassen sich beschränkte nichtlineare Optimierungen durchführen.


\subsection{Numerisches Lösen von gewöhnlichen Differentialgleichungen}
Zum lösen von gewöhnlichen steifen oder nichtsteifen Differentialgleichungen stehen einem die folgenden Matlab Funktionen zur Verfügung.

Vergleich der ODE Solver auf unser Beispiel







\section{Klassenstruktur des Problems}

UML


\chapter{Lösen des Optimalsteuerungsproblem mit direkten Lösungsverfahren}

Direkte Lösungsverfahren basieren auf einer Diskretisierung des Optimalsteuerungsproblems. Dadurch wird das Optimalsteuerungsproblem auf ein endlichdimensionales Optimierungsproblem transformiert, welches mit einem numerischen Optimierungsverfahren gelöst werden kann.

Betrachtet wird das Optimalsteuerungsroblem (Mayer-Problem mit Boxschranken) der folgenden Form. Finde für feste Zeitpunkte $t_0 < t_f$ einen Zustand $X \in W^{1, \infty} ([t_0,t_f], \R^{n_X})$ und eine Steuerung $U \in L^{\infty} ([t_0,t_f], \R^{n_U})$, sodass die Zielfunktion unter der Betrachtung der gegebenen Bedingungen minimal wird:
\begin{itemize}
\item \textbf{Zielfunktion:} $\varphi(X(t_0),X(t_f)) = -(x(t_f) - x_0)$
%
\item \textbf{Differentialgleichung:} $\dot{X}(t) = f(t,X(T),U(t)) \ \ \ \ \forall [t_0,t_f]$
%
\item \textbf{Randbedingungen:} $\psi(X(t_0),X(t_f)) = 0_{n_{\psi}}$
%
\item \textbf{Reine Zustandsbeschränkungen:} $s(t,X(t)) \leq 0_{n_s} \ \ \ \ \forall t \in [t_0,t_f]$
%
\item \textbf{Mengenbeschränkungen:} $U(t) \in \mathcal{U} = \lbrace U \in \R^{n_U} \mid U_{\min} \leq U \leq U_{\max} \rbrace \ \ \ \ \forall t \in [t_0, t_f]$
\end{itemize}
Die Mengenbeschränkungen werden auch als Boxbeschränkungen bezeichnet. 











\section{Algorithmus vollständige Diskretisierung}
Dieses gegebene Optimalsteuerungsproblem wird mit verschiedenen Techniken und Verfahren auf ein nichtlineares Optimierungsproblem umgeformt. Zunächst wird dabei ein Gitter erzeugt, welches das Problem diskretisiert. Dieses Gitter
\begin{equation}
\mathbb{G}_h := \lbrace t_0 < t_1 < ... < t_{N-1} = t_f \rbrace
\end{equation}
mit den Schrittweiten 
\begin{equation}
h_i = t_{i+1} - t_i \ \ \ \ \forall i = 0,...,N-2
\end{equation}
muss dabei nicht notwendig äquidistant unterteilt sein. Dieses Gitter stellt die Basis für die Parametrisierung der Steuerung und das Diskretisierungsverfahren für die Differentialgleichung:
\begin{itemize}
\item \textbf{Parametrisierung der Steuerung:} Dabei wird die Steuerfunktion $U$ durch eine von der Schrittweite abhängige Steuerfunktion $U_h$ ersetzt. Diese Steuerfunktion hängt dann nur noch von endlich vielen Parametern ab.
%
\item \textbf{Diskretisierungsverfahren für die Differentialgleichung:} Die Differentialgleichung des Problems wird mit Hilfe eines Diskretisierungsverfahren diskretisiert, um die Lösung im Optimierungsproblem verarbeiten zu können. Hierfür eignen sich zum Beispiel allgemeine oder spezielle Einschrittverfahren wie das explizite Eulerverfahren oder das implizite Runge-Kutta Verfahren.
\end{itemize}
Somit ergibt sich das diskretisierte Optimalsteuerungsproblem:

Finde die Gitterfunktionen \[X_h : \mathbb{G}_h \to \R^{n_X} \text{ mit } t_i \mapsto X_h(t_i) =: X_i\] und \[U_h : \mathbb{G}_h \to \R^{n_U} \text{ mit } t_i \mapsto U_h(t_i) =: U_i\] sodass die Zielfunktion \[\varphi(X_0,X_{N-1})\] minimal wird unter der Betrachtung der diskretisierten  Differentialgleichung (hier mit explizitem Eulerverfahren): \[X_{i+1} = X_{i} + h_i f(t_i,X_i,U_i) \ \ \ \ \forall i = 0,1,...,N-2\] den Randbedingungen: \[\psi(X_0,X_N) = 0_{n_{\psi}}\] den reinen Zustandsbeschränkungen: \[s(t_i,X_i) \leq 0_{n_s} \ \ \ \ \forall i = 0,1,...,N-1\] und den Mengenbeschränkungen: \[U_{\min} \leq U_i \leq U_{\max} \ \ \ \ \forall i = 0,1,...,N-1\]

Dieses endlichdimensionale, nichtlineare Optimierungsproblem lässt sich dann mit einem geeigneten Optimierungsverfahren lösen. Mögliche Optimierungsverfahren sind das SQP-Verfahren, das Innere-Punkte-Verfahren oder die Strategie der aktiven Menge. Hierfür muss das Optimierungsproblem in die Form:
\[\begin{split}
\min_{z \in S} F(z) &:= \varphi(X_0,X_{N-1}) \\\
\text{unter} \hspace{10mm} G(z) &\leq 0_{n_s \cdot N} \hspace{43mm} \text{(Ungleichungsnebenbedingungen)} \\\
H(z) &= 0_{n_X \cdot (N-1) + n_{\psi}} \hspace{32mm} \text{(Gleichungsnebenbedingungen)}\\\
z &\in S := \R^{n_x \cdot N} \times [U_{\min},U_{\max}]^{N} \hspace{10mm} \text{(Zulässigemenge)}
\end{split} \]
gebracht werden mit der Optimierungsvariablen 
\begin{equation}
z := (X_0,X_1,...,X_{N-1},U_0,U_1,...,U_{N-1})^T
\end{equation}
und den Funktionen 
\begin{equation}
G(z) := \begin{pmatrix}
c(t_0,X_0,U_0) \\ 
\vdots \\ 
c(t_{N-1},X_{N-1},U_{U-1})
\end{pmatrix} 
\end{equation}
\begin{equation}
H(z) := \begin{pmatrix}
X_0 + h_0 f(t_0,X_0,U_0) - X_1 \\ 
\vdots \\ 
X_{N-2} + h_{N-2} f(t_{N-2},X_{N-2},U_{U-2}) - X_{N-1} \\
\psi(X_0,X_N)
\end{pmatrix} 
\end{equation}

%\floatname{algorithm}{Klasse}
%\begin{algorithm}[H]
%\caption{MaximalRangeFlight}\label{algo:SISPF}
%\textbf{[$\lbrace s^i_t, w^i_t \rbrace^{N_s}_{i=1}$] = PFSIR [$\lbrace s^i_{t-1}, w^i_{t-1} \rbrace^{N_s}_{i=1}$]}
%\begin{algorithmic}
%\FOR {$i = 1,...,N_s$}
%\STATE $s^i_t \sim q(x_t \mid s^i_{i-1})$
%\STATE $w^i_t = p(z_t \mid s^i_t)$
%\ENDFOR
%\STATE $t = \sum_{i=1}^{N_s} w^i_t$
%\FOR {$i = 1,...,N_s$}
%\STATE $w^i_t = t^{-1} w^i_{t-1}$ (Normalisieren)
%\ENDFOR
%\STATE [$\lbrace s^{\ast j}_t, w^{j}_t, i^{j} \rbrace^{N_s}_{j=1}$] = RESAMPLE [$\lbrace s^i_t, w^{j}_t \rbrace^{N_s}_{j=1}$]
%\end{algorithmic}

%\floatname{algorithm}{Algorithmus}
%\begin{algorithm}[H]
%\caption{Vollständige Euler-Diskretisierung}\label{algo:SISPF}
%\textbf{[$\lbrace s^i_t, w^i_t \rbrace^{N_s}_{i=1}$] = PFSIR [$\lbrace s^i_{t-1}, w^i_{t-1} \rbrace^{N_s}_{i=1}$]}
%\begin{algorithmic}
%\FOR {$i = 1,...,N_s$}
%\STATE $s^i_t \sim q(x_t \mid s^i_{i-1})$
%\STATE $w^i_t = p(z_t \mid s^i_t)$
%\ENDFOR
%\STATE $t = \sum_{i=1}^{N_s} w^i_t$
%\FOR {$i = 1,...,N_s$}
%\STATE $w^i_t = t^{-1} w^i_{t-1}$ (Normalisieren)
%\ENDFOR
%\STATE [$\lbrace s^{\ast j}_t, w^{j}_t, i^{j} \rbrace^{N_s}_{j=1}$] = RESAMPLE [$\lbrace s^i_t, w^{j}_t \rbrace^{N_s}_{j=1}$]
%\end{algorithmic}















\section{Beschränkte nichtlineare Optimierungsverfahren}
SQP Verfahren \\
Innere Punkte Verfahren \\
Strategie der aktiven Menge













\section{Versuchsdurchführungen und Ergebnisse}
Gedanken wie der Verlauf aussehen könnte:\\
Am Anfang maximaler Schub und maximaler Auftriebsbeiwert, um so schnell wie möglich die gewünschte Reisehöhe zu erhalten. Auftriebsbeiwert kann dann reduziert werden, um die Geschwindigkeit zu erhöhen. Dies bewirkt eine höhere Geschwindigkeit, wodurch die zurückgelegte Strecke maximiert wird.



%Hallo Jörn,
%
%Nochmal vielen Dank für deine Hinweise und deine Mühen!!!
%Einige Punkte konnten wir schon umsetzen, teils sogar erfolgreich :)
%
%Folgende Punkte haben wir nun bei den direkten Verfahren umgesetzt bzw. getestet:
% - Performance und Fehler Verbesserung des Codes: Wir haben den Code nun so angepasst, sodass wir test files haben in denen wir alle Parameter, Startwerte etc. anpassen können. Welcher File welchen Versuch darstellt ist aktuell im Anhang der Doku festgehalten. Des Weiteren ist uns gestern noch ein Fehler in der nonlcon(obj,z) Funktion aufgefallen. Dabei haben wir unseren Vektor ceq  immer zu lang gewählt mit (n_x*N)+n_psi anstatt mit (n_x*(N-1))+n_psi so wie in der Lösung des Raketenwagen Beispiels. Diese Verbesserung wurde auch nochmal in der Doku festgehalten. Im Zuge dessen haben wir auch die darin enthaltene for Schleife vektorisiert.
% - Mit dieser Verbesserung/Korrektur laufen nun auch mal das Innere Punkte Verfahren. Das Verfahren der Strategie der aktiven Mengen jedoch noch nicht.
% - Du hast ja geschrieben wir sollen mal die impliziten und expliziten Verfahren mit vorgegebener Steuerung untersuchen. Leider war uns nicht ganz klar was du damit gemeint hast? Meinst du von unserem Startwert X_0 aus und einer vorgegebener Steuerung T(t) und C_L(t) die Lösung von h,gamma,x,v mit einem Impliziten und expliziten Verfahren berechnen?
% - Deine Hinweise bezüglich der Veränderung von Parametern wie der Endzeit wurden in den verschiedenen Versuchen ausprobiert. Leider kommen für uns nicht wirkliche aussagekräftige Ergebnisse raus. Oft sind die Ergebnisse auch deutlich entfernt von einem realen physikalischen Verhalten (z.B. Flughöhen von über 30.000m).
%
%​​​​​​​Bei dem Beispiel Raketenwagen mit den indirekten Lösungsverfahren konnten wir nur ein kleinen Erfolg erzielen (siehe Anhang Raketenwagen.pdf Kapitel 0.4).
%
%Bezüglich unserer aktuellen Probleme und geringen Erfahrung, wissen wir jetzt nicht genau wie wir weiter machen sollen. Es wäre deshalb sehr nett von dir, wenn du dir vielleicht die Zeit für ein Gespräch mit uns nehmen könntest?
%
%Mit freundlichen Grüßen
%
%Philipp, Heiko, Felix
 









\chapter{Formulierung des Optimalsteuerungsproblem für indirekte Lösungsverfahren} \label{cha:indirect}

Bei den indirekten Lösungsverfahren werden notwendige Bedingungen für eine optimale Lösung des Steuerungsproblems aufgestellt. Diese notwendigen Bedingungen für das Optimalsteuerungsproblem überführen das Problem in ein  Randwertproblem, welches mit verschiedenen Methoden, wie zum Beispiel dem Einfachschieß- oder Mehrfachschießverfahren \cite{Deuflhard2013,MultipleShoot1962} ausgewertet werden kann.

\section{Notwendige Optimalitätsbedinungen (Minimumpinzip von Pontryagin)}\label{kap:NOpt}
%Aus dem Modellaufbau ergeben sich die Definitionen 
%\begin{align*}
%    g(X(t_f)) &=  -(x(t_f) - x_0) & & (g : \R^n \to \R) \\
%    f_0(X(t),U(t)) &\equiv 0 & &(f_0 : \R^n \times \R^m \to \R^n)\\
%    f(X(t),U(t)) &= \dot{X}(t) = (\dot{h}(t),\dot{\gamma}(t),\dot{x}(t),\dot{v}(t))^T & &(f : \R^n \times \R^m \to \R^n) \\
%    U(t) &= (T(t),C_L(t))^T \in \mathcal{U} \left( = 
%    \left[
%        \begin{matrix}
%            [T_{\min},T_{\max}] \\ 
%            [C_{L, \min},C_{L, \max}]
%        \end{matrix} 
%    \right]\right) \subset \R^2 & &(U : [t_0,t_f] \to \R^m)
%\end{align*}
%Des Weiteren sei $\psi : \R^4 \to \R^2$ eine $C^1$-Funktion (mit $0 \leq (r = 2) \leq (n = 4)$) \[\psi(X(t_f)) = 
%\begin{pmatrix}
%    h(t_f) - h_f \\ 
%    \gamma(t_f) - \gamma_f
%\end{pmatrix} = 0\]
%
%So ergibt sich das autonome Mayer-Problem mit 
%\begin{equation} \label{equ:mayer_problem}
%    \begin{aligned}
%        \min F(X,U) &:= g(X(t_f)) =  -(x(t_f) - x_0) & & \\
%        \text{unter}  \hspace{10mm} \dot{X}(t) &= f(X(t),U(t)) = (\dot{h}(t),\dot{\gamma}(t),\dot{x}(t),\dot{v}(t))^T & & \forall t \in [t_0,t_f] \\
%        %
%        X(t_0) &= X_0 = (h_0,\gamma_0,x_0,v_0)^T & & \\
%        \psi(X(t_f)) &= 0 & & \\
%        %
%        q(X(t)) &\leq q_{\max} & & \forall t \in [t_0,t_f] \\
%        U(t) &= (T(t),C_L(t))^T \in \mathcal{U} \subset \R^2  & & \forall t \in [t_0,t_f] 
%    \end{aligned}
%\end{equation}
%
%Für die Funktion $f$ lässt sich konkret
%\begin{equation} \label{equ:state_space}
%    f(X(t),U(t)) = \dot{X}(t) = \begin{pmatrix}
%        v(t) \sin(\gamma(t)) \\ 
%        \dfrac{F \alpha e^{-\beta h(t)} v(t) C_L(t)}{2m} - \dfrac{g \cos(\gamma(t))}{v(t)} \\ 
%        v(t) \cos(\gamma(t)) \\ 
%        \dfrac{T(t)}{m} - \dfrac{(C_{D_0} + k C_L^2(t)) F \alpha e^{-\beta h(t)} v^2(t)}{2m} - g \sin(\gamma(t))
%    \end{pmatrix}
%\end{equation}

Für die Notwendigen Optimalitätsbedingungen muss zunächst die Hamilton-Funktion aufgestellt werden. Diese ergibt sich mit $\lambda_0 \in \R$ ($\lambda_0 \geq 0$) und $\lambda : [t_0,t_f] \to \R^{n_X}$ zu
\begin{equation} \label{equ:hamilt_func}
    \begin{split}
        H(X(t),U(t),\lambda(t)) &= \lambda_0 f_0(X(t),U(t)) + \lambda(t)^T f(X(t),U(t)) \\\
        &= \lambda(t)^T f(X(t),U(t)) \\\
        &= \lambda_1(t) \dot{h}(t) + \lambda_2(t) \dot{\gamma}(t) + \lambda_3(t) \dot{x}(t) + \lambda_4(t) \dot{v}(t) \\\
        &= \sin(\gamma(t)) v(t) \lambda_1 \\\
        &\hspace{7mm} + \dfrac{F \alpha e^{-\beta h(t)} C_L(t) v(t) \lambda_2(t)}{2m} - \dfrac{g \cos(\gamma(t)) \lambda_2(t)}{v(t)} \\\
        &\hspace{7mm} + \cos(\gamma(t)) v(t) \lambda_3(t) \\\
        &\hspace{7mm} + \dfrac{T(t) \lambda_4(t)}{m} - \dfrac{(C_{D_0} + k C_L^2(t)) F \alpha e^{-\beta h(t)} v^2(t) \lambda_4(t)}{2m} \\\
        &\hspace{7mm} - g \sin(\gamma(t)) \lambda_4(t)
    \end{split}
\end{equation}
Mit der Hamilton-Funktion (s. Gleichung \ref{equ:hamilt_func}) lassen sich nun die notwendigen Optimalitätsbedingungen des Minimumprinzips von Pontryagin \cite{OptiControl2018} untersuchen:
\begin{enumerate}
    \item \textbf{Minimumbedingung:} Es gilt an allen Stetigkeitsstellen $t \in [t_0,t_f]$ von $u^{\ast}(t)$ die Minimumbedingung \[H(X^{\ast}(t),U^{\ast}(t),\lambda(t)) = \min_{U(t) \in \mathcal{U}} H(X^{\ast}(t),U(t),\lambda(t))\] Leitet man nun nach der Steuerfunktion $U(t)$ ab, so ergibt sich für den unbeschränkten Fall der Steuerung die Minimumbedingung
    \[\dfrac{\partial}{\partial U} H(X^{\ast}(t),U^{\ast}(t),\lambda(t)) = \begin{pmatrix}
    \dfrac{\lambda_4(t)}{m} \\ 
    - \dfrac{k F \alpha e^{-\beta h(t)} v^2(t) \lambda_4(t) C_L(t)}{m} + \dfrac{F \alpha e^{-\beta h(t)} v(t) \lambda_2(t)}{2m}
    \end{pmatrix}^T \stackrel{!}{=} 0\]
    und 
    \[\dfrac{\partial^2}{\partial U^2} H(X^{\ast}(t),U^{\ast}(t),\lambda(t)) = \begin{pmatrix}
    0 & - \dfrac{k F \alpha e^{-\beta h(t)} v^2(t) \lambda_4(t)}{m} 
    \end{pmatrix} \stackrel{!}{\geq} 0\] wobei \[\sigma(x(t),\lambda(t)) := H_u(X^{\ast}(t),U^{\ast}(t),\lambda(t))\] Schaltfunktion genannt wird. Für die Betrachtung mit den Beschränkungen der Steuerfunktion ergibt sich für den Schub
\begin{equation}\label{func:SynSchub}
T(t) = \left\lbrace \begin{array}{ll}
T_{\min} & ,\text{falls } \lambda_4 > 0  \\ 
\text{beliebig} \in [T_{\min},T_{\max}] & ,\text{falls } \lambda_4 = 0  \\ 
T_{\max} & ,\text{falls } \lambda_4 < 0
\end{array} \right.
\end{equation}
Für die Bestimmung der Steuerfunktion des Auftriebsbeiwerts lässt sich die Hamilton-Funktion verkürzt mit den Termen $K_1(t)$ und $K_2(t)$ schreiben.
    \[\begin{split}
        \tilde{H}(X^{\ast}(t),C_L(t),\lambda(t)) &= \dfrac{F \alpha e^{-\beta h^{\ast}(t)} v^{\ast}(t) \lambda_2(t)}{2m} \cdot C_L(t) - \dfrac{k F \alpha e^{-\beta h^{\ast}(t)}  v^{\ast 2}(t) \lambda_4(t)}{2m} \cdot C_L^2(t) \\\
        &= K_1(t) C_L(t) - K_2(t) C_L^2(t)
    \end{split}\]
Soll nun $\tilde{H}(X^{\ast}(t),C_L(t),\lambda(t))$ minimal werden, so müssen die Folgenden Bedinungen überprüft werden:
    \begin{enumerate}
        \item[1.)] $\mathbf{K_1(t) < 0 \wedge K_2(t)} < 0$:
        \begin{enumerate}
            \item[1.1.)] Für die Ableitungen von $\tilde{H}$ ergeben sich
            \[\begin{split}
            \frac{d \tilde{H}}{d C_L} = - 2 K_2(t) C_L(t) + K_1(t) &\stackrel{!}{=} 0 \\\
            \frac{d^2 \tilde{H}}{d C_L^2} = - 2 K_2(t) &\geq 0 \Rightarrow \text{Minimum}
            \end{split}\]
            So folgt für die Steuerfunktion $C_L(t) = \dfrac{K_1(t)}{2 K_2(t)} > 0$.
            %
            \item[1.2.)] Falls $C_L(t) = \dfrac{K_1(t)}{2 K_2(t)} > C_{L, \max}$ so gilt  $C_L(t) = C_{L, \max}$.
        \end{enumerate}
        %
        \item[2.)] $\mathbf{K_1(t) = 0 \wedge K_2(t)} < 0$: Für die Ableitungen von $\tilde{H}$ ergeben sich
        \[\begin{split}
        \frac{d \tilde{H}}{d C_L} = - 2 K_2(t) C_L(t) &\stackrel{!}{=} 0 \\\
        \frac{d^2 \tilde{H}}{d C_L^2} = - 2 K_2(t) &\geq 0 \Rightarrow \text{Minimum}
        \end{split}\]
        So folgt für die Steuerfunktion $C_L(t) = 0 = C_{L, \min}$.
        %
        \item[3.)] $\mathbf{K_1(t) > 0 \wedge K_2(t)} < 0$: Für die Ableitungen von $\tilde{H}$ ergeben sich
        \[\begin{split}
        \frac{d \tilde{H}}{d C_L} = - 2 K_2(t) C_L(t) + K_1(t) &\stackrel{!}{=} 0 \\\
        \frac{d^2 \tilde{H}}{d C_L^2} = - 2 K_2(t) &\geq 0 \Rightarrow \text{Minimum}
        \end{split}\]
        So folgt für die Steuerfunktion $C_L(t) = \dfrac{K_1(t)}{2 K_2(t)} < 0$. Es gilt also $C_L(t) = C_{L, \min}$.
        %
        \item[4.)] $\mathbf{K_1(t) < 0 \wedge K_2(t)} = 0$: Für die Ableitungen von $\tilde{H}$ ergeben sich
        \[\begin{split}
        \frac{d \tilde{H}}{d C_L} = K_1(t) &\stackrel{!}{=} 0 \\\
        \frac{d^2 \tilde{H}}{d C_L^2} = 0 &\geq 0 \Rightarrow \text{Minimum}
        \end{split}\]
        So folgt für die Steuerfunktion $C_L(t) = C_{L, \max}$.
        %
        \item[5.)] $\mathbf{K_1(t) = 0 \wedge K_2(t)} = 0$: Daraus folgt für die Steuerfunktion $C_L(t) = \text{beliebig} \in [C_{L, \min},C_{L, \max}]$.
        %
        \item[6.)] $\mathbf{K_1(t) > 0 \wedge K_2(t)} = 0$: Für die Ableitungen von $\tilde{H}$ ergeben sich
        \[\begin{split}
        \frac{d \tilde{H}}{d C_L} = K_1(t) &\stackrel{!}{=} 0 \\\
        \frac{d^2 \tilde{H}}{d C_L^2} = 0 &\geq 0 \Rightarrow \text{Minimum}
        \end{split}\]
        So folgt für die Steuerfunktion $C_L(t) = C_{L, \min}$.
        %
        \item[7.)] $\mathbf{K_1(t) < 0 \wedge K_2(t)} > 0$: Für die Ableitungen von $\tilde{H}$ ergeben sich
        \[\begin{split}
        \frac{d \tilde{H}}{d C_L} = - 2 K_2(t) C_L(t) + K_1(t) &\stackrel{!}{=} 0 \\\
        \frac{d^2 \tilde{H}}{d C_L^2} = - 2 K_2(t) &\leq 0 \Rightarrow \text{Maximum}
        \end{split}\]
        So folgt für die Steuerfunktion $C_L(t) = \dfrac{K_1(t)}{2 K_2(t)} < 0$. Es gilt also $C_L(t) = C_{L, \max}$.
        %
        \item[8.)] $\mathbf{K_1(t) = 0 \wedge K_2(t)} > 0$: Für die Ableitungen von $\tilde{H}$ ergeben sich
        \[\begin{split}
        \frac{d \tilde{H}}{d C_L} = - 2 K_2(t) C_L(t) &\stackrel{!}{=} 0 \\\
        \frac{d^2 \tilde{H}}{d C_L^2} = - 2 K_2(t) &\leq 0 \Rightarrow \text{Maximum}
        \end{split}\]
        So folgt für die Steuerfunktion $C_L(t) = 0$. Es gilt also $C_L(t) = C_{L, \max}$.
        %
\item[9.)] $\mathbf{K_1(t) > 0 \wedge K_2(t)} > 0$: Für die Ableitungen von $\tilde{H}$ ergeben sich
\[\begin{split}
\frac{d \tilde{H}}{d C_L} = - 2 K_2(t) C_L(t) + K_1(t) &\stackrel{!}{=} 0 \\\
\frac{d^2 \tilde{H}}{d C_L^2} = - 2 K_2(t) &\leq 0 \Rightarrow \text{Maximum}
\end{split}\]
So folgt für die Steuerfunktion $C_L(t) = \dfrac{K_1(t)}{2 K_2(t)} > 0$. Für das Minimum bestimme man nun die Nullstellen von $\tilde{H}$:
\begin{enumerate}
\item[9.1.)] Für die Bestimmung der Nullstellen von $\tilde{H}$ gilt
\[\begin{split}
\tilde{H} \stackrel{!}{=} 0 &= - K_2(t) C_L^2(t) + K_1(t)C_L(t) \\\
&= (- K_2(t) C_L(t) + K_1(t)) C_L(t) \\\
0 &= K_1(t) - K_2(t)C_L(t) \\\
C_L(t) &= \dfrac{K_1(t)}{K_2(t)}
\end{split}\]
So gilt $\tilde{H} = 0$ bei $C_L(t) = 0$ und $C_L(t) = \dfrac{K_1(t)}{K_2(t)}$ Falls $C_L(t) = \dfrac{K_1(t)}{K_2(t)} < C_{L,\max}$, so gilt also $C_L(t) = C_{L, \max}$.
%
\item[9.2.)] Falls $C_L(t) = \dfrac{K_1(t)}{K_2(t)} > C_{L,\max}$, so gilt $C_L(t) = C_{L,\min}$.
\item[9.3.)] Falls $C_L(t) = \dfrac{K_1(t)}{K_2(t)} = C_{L,\max}$, so gilt $C_L(t) = C_{L,\max} \vee C_{L,\min}$.
\end{enumerate}
\end{enumerate}
Damit folgt für die Steuerfunktion des Auftriebsbeiwerts
\begin{equation}\label{func:SynAuftrieb}
C_L(t) = \left\lbrace 
\begin{array}{ll}
C_{L, \min} & ,\text{falls Bedingung } 2,3,6 \text{ oder } 9.2 \text{ gilt} \\ 
\text{beliebig} \in [C_{L, \min},C_{L, \max}] & ,\text{falls Bedingung } 5 \text{ gilt} \\ 
\dfrac{K_1(t)}{2 K_2(t)} & ,\text{falls Bedingung } 1.1 \text{ gilt} \\ 
C_{L, \max} & ,\text{falls Bedingung } 1.2,4,7,8 \text{ oder } 9.1 \text{ gilt} \\
C_L(t) = C_{L,\max} \vee C_{L,\min} & ,\text{falls Bedingung } 9.3 \text{ gilt} \\ 
\end{array} 
\right.
\end{equation}
%
\item \textbf{Adjungierte DGL:} Leitet man nach dem Zustandsvektor $X(t)$ ab, also $H_{X}(X^{\ast}(t),U^{\ast}(t),\lambda(t))$ so erhält man
\begin{equation}\label{func:AjgDGL}
\begin{split}
\dfrac{\partial}{\partial h} H &= - \dfrac{\alpha \beta F e^{-\beta h(t)} C_L(t) v(t) \lambda_2(t)}{2m} + \dfrac{(C_{D_0}+k C_L^2(t)) \alpha \beta F e^{-\beta h(t)} v^2(t) \lambda_4(t)}{2m} \\\
\dfrac{\partial}{\partial \gamma} H &= \cos(\gamma(t)) v(t) \lambda_1(t) + \dfrac{g \sin(\gamma(t)) \lambda_2(t)}{v(t)} - \sin(\gamma(t)) v(t) \lambda_3(t) - \cos(\gamma(t)) g \lambda_4(t) \\\
\dfrac{\partial}{\partial x} H &= 0 \\\
\dfrac{\partial}{\partial v} H &= \sin(\gamma(t)) \lambda_1(t) + \left( \dfrac{F \alpha e^{-\beta h(t)} C_L(t)}{2m} + \dfrac{g \cos(\gamma(t))}{v^2(t)} \right) \lambda_2(t) \\\
&\hspace{7mm} + \cos(\gamma(t)) \lambda_3(t) - \dfrac{(C_{D_0} + k C_L^2(t)) F \alpha e^{-\beta h(t)} v(t) \lambda_4(t)}{m}
\end{split}
\end{equation}
        wobei gilt 
        \[\dot{\lambda}(t)^T = - \dfrac{\partial}{\partial X} H = -H_{X} = \left( -\dfrac{\partial}{\partial h} H, -\dfrac{\partial}{\partial \gamma} H, -\dfrac{\partial}{\partial x} H, -\dfrac{\partial}{\partial v} H \right)\]
    %
    \item \textbf{Transversalitätsbedingung:} Im Endzeitpunkt $t_f$ gilt die Transversalitätsbedingung mit dem Vektor $\nu \in \R^{n_{\psi}}$ mit $(\lambda_0,\lambda(t),\nu) \neq 0$ für alle $t \in [t_0,t_f]$
        \[\begin{split}
            \lambda(t_f)^T &= \lambda_0 g_X(X^{\ast}(t_f)) + \nu^T \psi_X(X^{\ast}(t_f)) \\\
            &= \lambda_0 
            \begin{pmatrix}
            0 & 0 & -1 & 0
            \end{pmatrix}  
            + \nu^T 
            \begin{pmatrix}
            1 & 0 & 0 & 0 \\
            0 & 1 & 0 & 0 
            \end{pmatrix}  \\\
            &= \begin{pmatrix}
            \nu_1 & \nu_2 & -\lambda_0 & 0 
            \end{pmatrix}
        \end{split}\]
%(ist $\lambda_{0} =1$ da $x(t_f)$ frei ist und $\psi_X(X^{\ast}(t_f))$ hat vollen Zeilenrang???) 
    %
    \item \textbf{Konstanz:} Für autonome Systeme gilt 
\begin{equation}\label{func:HKonstanz}
H(X^{\ast}(t),U^{\ast}(t), \lambda(t)) = const \ \in [t_0,t_f]
\end{equation}
\end{enumerate}













%\section{Zweipunkt-Randwertproblem und Beschränkungsuntersuchungen}
%Um das Optimalsteuerungsproblem \ref{prob:MaxRF} mit indirekten Lösungsverfahren lösen zu können, muss dieses zunächst auf ein Zweipunkt-Randwertproblem umgestellt werden.
%
%\begin{problem}[Zweipunkt-Randwertproblem]\label{prob:ZweiRand}
%Seien $G : [t_0,t_f] \times \R^{n_Z} \to \R^{n_Z}$ und $R : \R^{n_Z} \times \R^{n_Z} \to \R^{n_Z}$ gegeben. Gesucht ist eine Lösung $Z$ des Randwertproblems
%\begin{align}
%\dot{Z}(t) &= G(t,Z(t),U(t)) \\
%R(Z(t_0),Z(t_f)) &= 0_{n_Z}
%\end{align}
%im Intervall $[t_0,t_f]$.
%\end{problem}
%
%Jedoch behandelt das Zweipunkt-Randwertproblem keine Bedingungen wie die Beschränkung des Staudrucks $q(v(t),h(t)) \leq q_{\max}$, wie in Problem \ref{prob:MaxRF} gefordert. Untersuchungen der Ergebnisse aus den Versuchen \ref{kap:Versuch11}, \ref{kap:Versuch31} und  \ref{kap:Versuch41} haben gezeigt, dass diese zu keinem Zeitpunkt den maximalen Wert $q_{\max}$ der Beschränkung erreicht haben (Abbildung \ref{img:test_1_1_staudruck}). Aus Vereinfachungsgründen, wird diese Beschränkung deshalb in diesem Kapitel nicht weiter berücksichtigt.
%
%\begin{figure}[H]
%\begin{center}
%\includegraphics[width=\textwidth]{../code/direct_sol/results/test_1_1_staudruck}
%\caption{Untersuchung, ob bei Versuch \ref{kap:Versuch11} der maximale Staudruck überschritten bzw. angenähert wird.}\label{img:test_1_1_staudruck}
%\end{center}
%\end{figure}

















\section{Aufstellen des Zweipunkt-Randwertproblems}\label{kap:ZPRand}
Um das Optimalsteuerungsproblem \ref{prob:MaxRF} mit indirekten Lösungsverfahren lösen zu können, muss dieses zunächst auf ein Zweipunkt-Randwertproblem umgestellt werden.

\begin{problem}[Zweipunkt-Randwertproblem]\label{prob:ZweiRand}
    Seien $G : [t_0,t_f] \times \R^{n_Z} \to \R^{n_Z}$ und $R : \R^{n_Z} \times \R^{n_Z} \to \R^{n_Z}$ gegeben. Gesucht ist eine Lösung $Z$ des Randwertproblems
    \begin{align}
        \dot{Z}(t) &= G(t,Z(t),U(t)) \\
        R(Z(t_0),Z(t_f)) &= 0_{n_Z}
    \end{align}
    im Intervall $[t_0,t_f]$.
\end{problem}

Mit den Optimalitätsbedingungen des Minimumprinzips von Pontryagin lässt sich das Steuerungsproblem in ein Randwertproblem überführen, welches aus den beiden Funktionen $G(t,Z(t),U(t))$ und $R(Z(t_0),Z(t_f)) = 0_{n_Z}$ für $n_Z = 8$ besteht:
\begin{itemize}
\item Für $G(t,Z(t),U(t))$ ergibt sich
\begin{equation} 
    \dot{Z}(t) = G(t,Z(t),U(t)) = 
    \begin{pmatrix}
        \dot{h}(t),\dot{\gamma}(t),\dot{x}(t),\dot{v}(t),\dot{\lambda}_1(t),\dot{\lambda}_2(t),\dot{\lambda}_3(t),\dot{\lambda}_4(t)
    \end{pmatrix}^T
\end{equation}
und für die Ableitung die Matrix
\begin{equation}\label{equ:jacobi}
    \dfrac{\partial G(t,Z(t),U(t))}{\partial Z} = 
    \begin{pmatrix}
        0 & J_G^{(1,2)} & 0 & J_G^{(1,4)} & 0 & 0 & 0 & 0 \\ 
        J_G^{(2,1)} & J_G^{(2,2)} & 0 & J_G^{(2,4)} & 0 & 0 & 0 & 0 \\ 
        0 & J_G^{(3,2)} & 0 & J_G^{(3,4)} & 0 & 0 & 0 & 0 \\ 
        J_G^{(4,1)} & J_G^{(4,2)} & 0 & J_G^{(4,4)} & 0 & 0 & 0 & 0 \\
        J_G^{(5,1)} & 0 & 0 & J_G^{(5,4)} & 0 & J_G^{(5,6)} & 0 & J_G^{(5,8)} \\
        0 & J_G^{(6,2)} & 0 & J_G^{(6,4)} & J_G^{(6,5)} & J_G^{(6,6)} & J_G^{(6,7)} & J_G^{(6,8)} \\
        0 & 0 & 0 & 0 & 0 & 0 & 0 & 0 \\
        J_G^{(8,1)} & J_G^{(8,2)} & 0 & J_G^{(8,4)} & J_G^{(8,5)} & J_G^{(8,6)} & J_G^{(8,7)} & J_G^{(8,8)}
    \end{pmatrix}
\end{equation}
mit den Einträgen \(J_G^{(i,j)}\), welche die parteiellen Ableitungen von \(G\) nach jeder Komponente von \(Z\) enthalten (vgl. \autoref{Anhang:Jacobi}).
%
\item Für $R(Z(t_0),Z(t_f)) = 0_{n_Z}$ müssen zunächst die Endbedingungen mit
\begin{align*}
X_i(t_f) &= c_i & & (i=1,...,n_{\psi}=2) \\
\lambda_i(t_f) &= \lambda_0 g_{X_i}(X^{\ast}(t_f)) & &(i=n_{\psi}+1,...,n_X=4)
\end{align*}
aus
\begin{align*}
c &= \begin{pmatrix} h_f & \gamma_f \end{pmatrix} \\
g_{X}(X^{\ast}(t_f)) &= \begin{pmatrix} 0 & 0 & -1 & 0 \end{pmatrix}
\end{align*}
gebildet werden. Es ergibt sich dann
\begin{equation}
R(Z(t_0),Z(t_f)) = 0_{n_Z} = \begin{pmatrix}
h(t_0) - h_0 \\ 
\gamma(t_0) - \gamma_0 \\
x(t_0) - x_0 \\ 
v(t_0) - v_0 \\ 
h(t_f) - h_f \\ 
\gamma(t_f) - \gamma_f \\
\lambda_3(t_f) + \lambda_0 \\ 
\lambda_4(t_f) - 0
\end{pmatrix}
\end{equation}
mit 
\begin{equation}\label{func:RZt0}
\dfrac{d R(Z(t_0),Z(t_f))}{d Z(t_0)} = \begin{pmatrix}
1 & 0 & 0 & 0 & 0 & 0 & 0 & 0 \\ 
0 & 1 & 0 & 0 & 0 & 0 & 0 & 0 \\ 
0 & 0 & 1 & 0 & 0 & 0 & 0 & 0 \\ 
0 & 0 & 0 & 1 & 0 & 0 & 0 & 0 \\
0 & 0 & 0 & 0 & 0 & 0 & 0 & 0 \\
0 & 0 & 0 & 0 & 0 & 0 & 0 & 0 \\
0 & 0 & 0 & 0 & 0 & 0 & 0 & 0 \\
0 & 0 & 0 & 0 & 0 & 0 & 0 & 0
\end{pmatrix}
\end{equation}
und
\begin{equation}\label{func:RZtf}
\dfrac{d R(Z(t_0),Z(t_f))}{d Z(t_f)} = \begin{pmatrix}
0 & 0 & 0 & 0 & 0 & 0 & 0 & 0 \\ 
0 & 0 & 0 & 0 & 0 & 0 & 0 & 0 \\ 
0 & 0 & 0 & 0 & 0 & 0 & 0 & 0 \\ 
0 & 0 & 0 & 0 & 0 & 0 & 0 & 0 \\
1 & 0 & 0 & 0 & 0 & 0 & 0 & 0 \\
0 & 1 & 0 & 0 & 0 & 0 & 0 & 0 \\
0 & 0 & 0 & 0 & 0 & 0 & 1 & 0 \\
0 & 0 & 0 & 0 & 0 & 0 & 0 & 1
\end{pmatrix}
\end{equation}
\end{itemize}





















%\section{Algorithmus Einfachschießverfahren}
%Für eine gegebene Startschätzung $\eta$ des Anfangswerts $y(a)$ besitze das Anfangswertproblem
%\[y'(t) = g(t, y(t)) \ \ \ \ y(a) = \eta\]
%die Lösung $y(t;\eta)$ auf $[a,b]$. Damit $y(t;\eta)$ auch die Randbedingung erfüllt, muss 
%\begin{equation}\label{func:SchiessF}
%F(\eta) := r(y(a;\eta), y(b;\eta)) = r(\eta, y(b;\eta)) = 0_{n_y}
%\end{equation}
%gelten. Gleichung \ref{func:SchiessF} ist also ein \textbf{nichtlineares Gleichungssystem} für die Funktion $F$. Anwendung des Newtonverfahrens führt auf das sogenannte Einfachschießverfahren:
%
%\begin{definition}[Algorithmus Einfachschießverfahren]\label{algo:EinfSchiess}
%Initialisierung: Wähle Startschätzung $\eta^{[0]} \in \R^{n_y}$ und setze $i = 0$:
%\begin{enumerate}
%\item Löse das Anfangswertproblem \[y'(t) = g(t, y(t)) \ \ \ \ y(a) = \eta^{[i]} \ \ \ \ (a \leq t \leq b)\] zur Berechnung von $F(\eta^{[i]})$ und berechne die Jacobimatrix \[F'(\eta^{[i]}) = r'_{y_a} (\eta^{[i]}, y(b;\eta^{[i]})) + r'_{y_b}(\eta^{[i]}, y(b;\eta^{[i]})) \cdot S(b)\] wobei $S$ Lösung der Sensitivitäts-Differentialgleichung \[S'(t) = g'_y(t, y(t;\eta^{[i]})) \cdot S(t) \ \ \ \ S(a) = I_{n_y \times n_y}\] ist.
%%
%\item Ist $F(\eta^{[i]}) = 0_{n_y}$ (oder ist ein anderes Abbruchkriterium) erfüllt, \textbf{STOP!}
%%
%\item Berechne die Newton-Richtung $d^{[i]}$ als Lösung des linearen Gleichungssystems \[F'(\eta^{[i]})d = -F(\eta^{[i]})\]
%%
%\item Setze $\eta^{[i+1]} = \eta^{[i]} + d^{[i]}$ und $i=i+1$ und gehe zu 1.).
%\end{enumerate}
%\end{definition}
%
%Die Ableitung $F'(\eta^{[i]})$ in Schritt 2.) des Einfachschießverfahrens \ref{algo:EinfSchiess} kann alternativ
%durch \textbf{finite Differenzen} approximiert werden:
%\[\dfrac{\partial}{\partial \eta_j} F(\eta) \approx \dfrac{F(\eta + h e_j) - F(\eta)}{h} \ \ \ \ (j=1,...,n_y)\]
%mit $e_j = j$-ter Einheitsvektor. Dieser Ansatz erfordert das Lösen von $n_y$ Anfangswertproblemen!


\chapter{Zusammenfassung und Ausblick}



%------------------------------------------------------------------------------
% Anhang
%------------------------------------------------------------------------------
\clearpage
\appendix
\pagenumbering{roman}
\stepcounter{SeitenzahlSpeicher}
\setcounter{page}{\theSeitenzahlSpeicher}
% hier Anhänge einbinden
\chapter{Direkte Lösungsverfahren - Versuche und Ergebnisse}\label{Anhang:DirektV}

\section{Versuch 0}\label{kap:Versuch0}
Bei diesem Versuch wird das gegebene Problem \ref{prob:MaxRF} mit den Parametern aus Tabelle \ref{tab:ProblemPara} ohne Veränderungen gelöst. Für die beiden Versuche werden die folgenden Einstellungen aus Tabelle \ref{tab:Versuch0} verwendet.
\begin{table}[H]
    \centering
    \caption{Einstellungen von Versuch 0.1 und 0.2.}\label{tab:Versuch0}
    \begin{tabularx}{.9\textwidth}{Zccc}
        \toprule
        \textbf{Einstellungen} & \textbf{Versuch 0.1} & \textbf{Versuch 0.2} \\
        \midrule
        Anzahl Diskretisierungen & $N = 100$ & $N = 100$ \\
        Lösungsverfahren der DGL & Explizites Euler Verfahren & Explizites Euler Verfahren \\
        Optimierungsverfahren & SQP-Verfahren & Innere-Punkte-Verfahren \\
        Startvektor & $z_0 = \begin{pmatrix}
        9000 \\ 
        5 \\ 
        800000 \\
        250 \\
        1259999 \\ 
        1.4
        \end{pmatrix} $ & $z_0 = \begin{pmatrix}
        9000 \\ 
        5 \\ 
        800000 \\
        250 \\
        1259999 \\ 
        1.4
        \end{pmatrix}$ \\
        \bottomrule
    \end{tabularx}
\end{table}
Für die Ergebnisse von Versuch 0.1 (Anhang \ref{kap:Versuch01}) und Versuch 0.2 (Anhang \ref{kap:Versuch02}) wird der folgende technische Aufwand (Tabelle \ref{tab:Versuch0_TA}) benötigt.
\begin{table}[H]
    \centering
    \caption{Technischer Aufwand von Versuch 0.1 und 0.2.}\label{tab:Versuch0_TA}
    \begin{tabularx}{.9\textwidth}{Zccc}
        \toprule
         & \textbf{Versuch 0.1} & \textbf{Versuch 0.2} \\
        \midrule
        Funktionswert der Zielfunktion & $-1768382.0703$ & $-1020675.0231$ \\
        Anzahl Iterationen & $5394$ & $14887$ \\
        Anzahl Funktionsauswertungen & $3251455$ & $8968016$ \\
        Exit Flag von \textit{MATLAB} & $-2$ & $-2$ \\
        Optimalität des Ergebnis & $15.2438$ & $886.3152$ \\
        Berechnungsdauer & $47.3956 \ min$ & $154.133 \ min$ \\
        \bottomrule
    \end{tabularx}
\end{table}



\subsection{Ergebnis von Versuch 0.1}\label{kap:Versuch01}
In Versuch 0.1 wird das Ergebnis in Abbildung \ref{img:test_0_1} erreicht. Dies stellt aufgrund des Abbruchkriteriums $EF = -2$ kein Optimum des Problems da (kein zulässiger Punkt gefunden). 
\begin{figure}[H]
\begin{center}
\includegraphics[width=\textwidth]{../code/direct_sol/results/test_0_1}
\MyCaption{Ergebnis von Versuch 0.1}{} \label{img:test_0_1}
\end{center}
\end{figure}
Der Verlauf des beschränkten Staudrucks ist in Abbildung \ref{img:test_0_1_staudruck} dargestellt.
\begin{figure}[H]
\begin{center}
\includegraphics[width=\textwidth]{../code/direct_sol/results/test_0_1_staudruck}
\MyCaption{Überprüfung Staudruck $q(v(t),h(t))$ von Versuch 0.1}{} \label{img:test_0_1_staudruck}
\end{center}
\end{figure}





\subsection{Ergebnis von Versuch 0.2}\label{kap:Versuch02}
In Versuch 0.1 wird das Ergebnis in Abbildung \ref{img:test_0_2} erreicht. Dies stellt aufgrund des Abbruchkriteriums $EF = -2$ kein Optimum des Problems da (kein zulässiger Punkt gefunden).
\begin{figure}[H]
\begin{center}
\includegraphics[width=\textwidth]{../code/direct_sol/results/test_0_2}
\MyCaption{Ergebnis von Versuch 0.2}{} \label{img:test_0_2}
\end{center}
\end{figure}
Der Verlauf des beschränkten Staudrucks ist in Abbildung \ref{img:test_0_2_staudruck} dargestellt.
\begin{figure}[H]
\begin{center}
\includegraphics[width=\textwidth]{../code/direct_sol/results/test_0_2_staudruck}
\MyCaption{Überprüfung Staudruck $q(v(t),h(t))$ von Versuch 0.2}{} \label{img:test_0_2_staudruck}
\end{center}
\end{figure}











\newpage
\section{Versuch 1}\label{kap:Versuch1}
Die Problemstellung \ref{prob:MaxRF} gibt den Parametern \ref{tab:ProblemPara} eine feste Endzeit $t_f = 1800 \ s$ beziehungsweise einen Zeitraum vor in welchem das Problem gelöst werden soll. In diesem Versuch wird diese nun verkürzt, auf $t_f = 300 \ s$ und $t_f = 350 \ s$. Für die beiden Versuche werden die folgenden Einstellungen aus Tabelle \ref{tab:Versuch1} verwendet.
\begin{table}[H]
    \centering
    \caption{Einstellungen von Versuch 1.1 und 1.2.}\label{tab:Versuch1}
    \begin{tabularx}{.9\textwidth}{Zccc}
        \toprule
        \textbf{Einstellungen} & \textbf{Versuch 1.1} & \textbf{Versuch 1.2} \\
        \midrule
        Reduzierte Endzeit & $t_f = 300 \ s$ & $t_f = 350 \ s$ \\
        Anzahl Diskretisierungen & $N = 100$ & $N = 100$ \\
        Lösungsverfahren der DGL & Explizites Euler Verfahren & Explizites Euler Verfahren \\
        Optimierungsverfahren & SQP-Verfahren & SQP-Verfahren \\
        Startvektor & $z_0 = \begin{pmatrix}
        9000 \\ 
        5 \\ 
        80000 \\
        250 \\
        1259999 \\ 
        1.4
        \end{pmatrix} $ & $z_0 = \begin{pmatrix}
        9000 \\ 
        5 \\ 
        800000 \\
        250 \\
        1259999 \\ 
        1.4
        \end{pmatrix}$ \\
        \bottomrule
    \end{tabularx}
\end{table}
Für die Ergebnisse von Versuch 1.1 (Anhang \ref{kap:Versuch11}) und Versuch 1.2 (Anhang \ref{kap:Versuch12}) wird der folgende technische Aufwand (Tabelle \ref{tab:Versuch1_TA}) benötigt.
\begin{table}[H]
    \centering
    \caption{Technischer Aufwand von Versuch 1.1 und 1.2.}\label{tab:Versuch1_TA}
    \begin{tabularx}{.9\textwidth}{Zccc}
        \toprule
         & \textbf{Versuch 1.1} & \textbf{Versuch 1.2} \\
        \midrule
        Funktionswert der Zielfunktion & $-55444.2939$ & $-57922.1277$ \\
        Anzahl Iterationen & $3023$ & $2989$ \\
        Anzahl Funktionsauswertungen & $1846035$ & $1802834$ \\
        Exit Flag von \textit{MATLAB} & $2$ & $2$ \\
        Optimalität des Ergebnis & $0.0012243$ & $0.00011134$ \\
        Berechnungsdauer & $70.7353 \ min$ & $21.3019 \ min$ \\
        \bottomrule
    \end{tabularx}
\end{table}




\subsection{Ergebnis von Versuch 1.1}\label{kap:Versuch11}
In Versuch 1.1 wird das Ergebnis in Abbildung \ref{img:test_1_1} erreicht. Dies stellt aufgrund des Abbruchkriteriums $EF = 2$ ein mögliches Optimum des Problems da, da die Änderung des Zustandsvektors $z$ kleiner ist die gegebene Schrittweitentoleranz von hier $SW_{tol} = 10^{-6}$ und die maximale Verletzung der Beschränkungen kleiner ist als Beschränkungstoleranz $BS_{tol} = 10^{-6}$.
\begin{figure}[H]
\begin{center}
\includegraphics[width=\textwidth]{../code/direct_sol/results/test_1_1}
\MyCaption{Ergebnis von Versuch 1.1}{} \label{img:test_1_1}
\end{center}
\end{figure}
Der Verlauf des beschränkten Staudrucks ist in Abbildung \ref{img:test_0_2_staudruck} dargestellt.
\begin{figure}[H]
\begin{center}
\includegraphics[width=\textwidth]{../code/direct_sol/results/test_1_1_staudruck}
\MyCaption{Überprüfung Staudruck $q(v(t),h(t))$ von Versuch 1.1}{} \label{img:test_1_1_staudruck}
\end{center}
\end{figure}




\subsection{Ergebnis von Versuch 1.2}\label{kap:Versuch12}
In Versuch 1.2 wird das Ergebnis in Abbildung \ref{img:test_1_2} erreicht. Dies stellt aufgrund des Abbruchkriteriums $EF = 2$ ein mögliches Optimum des Problems da, da die Änderung des Zustandsvektors $z$ kleiner ist die gegebene Schrittweitentoleranz von hier $SW_{tol} = 10^{-6}$ und die maximale Verletzung der Beschränkungen kleiner ist als Beschränkungstoleranz $BS_{tol} = 10^{-6}$. Die Unstetigkeiten im Verlauf des Anstellwinkels und der Geschwindigkeit deuten jedoch auf kein physikalisch plausibles Ergebnis hin.
\begin{figure}[H]
\begin{center}
\includegraphics[width=\textwidth]{../code/direct_sol/results/test_1_2}
\MyCaption{Ergebnis von Versuch 1.2}{} \label{img:test_1_2}
\end{center}
\end{figure}
Der Verlauf des beschränkten Staudrucks ist in Abbildung \ref{img:test_1_2_staudruck} dargestellt.
\begin{figure}[H]
\begin{center}
\includegraphics[width=\textwidth]{../code/direct_sol/results/test_1_2_staudruck}
\MyCaption{Überprüfung Staudruck $q(v(t),h(t))$ von Versuch 1.2}{} \label{img:test_1_2_staudruck}
\end{center}
\end{figure}
















\newpage
\section{Versuch 2}\label{kap:Versuch2}
Zusätzlich zur verkürzten Endzeit wie in Versuch \ref{kap:Versuch1} wird in Versuch 2 auch die Starthöhe des Flugzeuges angepasst. Für die beiden Versuche werden die folgenden Einstellungen aus Tabelle \ref{tab:Versuch2} verwendet.
\begin{table}[H]
    \centering
    \caption{Einstellungen von Versuch 2.1 und 2.2.}\label{tab:Versuch2}
    \begin{tabularx}{.9\textwidth}{Zccc}
        \toprule
        \textbf{Einstellungen} & \textbf{Versuch 2.1} & \textbf{Versuch 2.2} \\
        \midrule
        Reduzierte Endzeit & $t_f = 250 \ s$ & $t_f = 300 \ s$ \\
        Angepasste Starthöhe & $h_0 = 4000 \ m$ & $h_0 = 4000 \ m$ \\
        Anzahl Diskretisierungen & $N = 100$ & $N = 100$ \\
        Lösungsverfahren der DGL & Explizites Euler Verfahren & Explizites Euler Verfahren \\
        Optimierungsverfahren & SQP-Verfahren & SQP-Verfahren \\
        Startvektor & $z_0 = \begin{pmatrix}
        5000 \\ 
        5 \\ 
        8000 \\
        250 \\
        1259999 \\ 
        1.4
        \end{pmatrix} $ & $z_0 = \begin{pmatrix}
        9000 \\ 
        5 \\ 
        8000 \\
        250 \\
        1259999 \\ 
        1.4
        \end{pmatrix}$ \\
        \bottomrule
    \end{tabularx}
\end{table}
Für die Ergebnisse von Versuch 2.1 (Anhang \ref{kap:Versuch21}) und Versuch 2.2 (Anhang \ref{kap:Versuch22}) wird der folgende technische Aufwand (Tabelle \ref{tab:Versuch2_TA}) benötigt.
\begin{table}[H]
    \centering
    \caption{Technischer Aufwand von Versuch 2.1 und 2.2.}\label{tab:Versuch2_TA}
    \begin{tabularx}{.9\textwidth}{Zccc}
        \toprule
         & \textbf{Versuch 2.1} & \textbf{Versuch 2.2} \\
        \midrule
        Funktionswert der Zielfunktion & $-17067.0955$ & $-70301.9291$ \\
        Anzahl Iterationen & $1259$ & $4699$ \\
        Anzahl Funktionsauswertungen & $771293$ & $2826056$ \\
        Exit Flag von \textit{MATLAB} & $-2$ & $2$ \\
        Optimalität des Ergebnis & $116.9551$ & $0.0004094$ \\
        Berechnungsdauer & $398.7814 \ min$ & $20.6724 \ min$ \\
        \bottomrule
    \end{tabularx}
\end{table}




\subsection{Ergebnis von Versuch 2.1}\label{kap:Versuch21}
In Versuch 2.1 wird das Ergebnis in Abbildung \ref{img:test_2_1} erreicht. Dies stellt aufgrund des Abbruchkriteriums $EF = -2$ kein Optimum des Problems da (kein zulässiger Punkt gefunden).
\begin{figure}[H]
\begin{center}
\includegraphics[width=\textwidth]{../code/direct_sol/results/test_2_1}
\MyCaption{Ergebnis von Versuch 2.1}{} \label{img:test_2_1}
\end{center}
\end{figure}
Der Verlauf des beschränkten Staudrucks ist in Abbildung \ref{img:test_0_2_staudruck} dargestellt.
\begin{figure}[H]
\begin{center}
\includegraphics[width=\textwidth]{../code/direct_sol/results/test_2_1_staudruck}
\MyCaption{Überprüfung Staudruck $q(v(t),h(t))$ von Versuch 2.1}{} \label{img:test_2_1_staudruck}
\end{center}
\end{figure}




\subsection{Ergebnis von Versuch 2.2}\label{kap:Versuch22}
In Versuch 2.2 wird das Ergebnis in Abbildung \ref{img:test_1_2} erreicht. Dies stellt aufgrund des Abbruchkriteriums $EF = 2$ ein mögliches Optimum des Problems da, da die Änderung des Zustandsvektors $z$ kleiner ist die gegebene Schrittweitentoleranz von hier $SW_{tol} = 10^{-6}$ und die maximale Verletzung der Beschränkungen kleiner ist als Beschränkungstoleranz $BS_{tol} = 10^{-6}$.
\begin{figure}[H]
\begin{center}
\includegraphics[width=\textwidth]{../code/direct_sol/results/test_2_2}
\MyCaption{Ergebnis von Versuch 2.2}{} \label{img:test_2_2}
\end{center}
\end{figure}
Der Verlauf des beschränkten Staudrucks ist in Abbildung \ref{img:test_0_2_staudruck} dargestellt.
\begin{figure}[H]
\begin{center}
\includegraphics[width=\textwidth]{../code/direct_sol/results/test_2_2_staudruck}
\MyCaption{Überprüfung Staudruck $q(v(t),h(t))$ von Versuch 2.2}{} \label{img:test_2_2_staudruck}
\end{center}
\end{figure}














\newpage
\section{Versuch 3}\label{kap:Versuch3}
Zusätzlich zur verkürzten Endzeit wie in Versuch 1 (Anhang \ref{kap:Versuch1}) wird nun zusätzlich das Gewicht des Flugzeuges angepasst. Das maximale Startgewicht eines A380-800 beträgt max. $560000 \ kg$ \cite{A380Tech}. Gewählt wird eine Masse von $500000 \ kg$. Für die beiden Versuche werden die folgenden Einstellungen aus Tabelle \ref{tab:Versuch3} verwendet.
\begin{table}[H]
    \centering
    \caption{Einstellungen von Versuch 3.1 und 3.2.}\label{tab:Versuch3}
    \begin{tabularx}{.9\textwidth}{Zccc}
        \toprule
        \textbf{Einstellungen} & \textbf{Versuch 3.1} & \textbf{Versuch 3.2} \\
        \midrule
        Reduzierte Endzeit & $t_f = 550 \ s$ & $t_f = 600 \ s$ \\
        Angepasstes Startgewicht & $m = 500000 \ kg$ & $m = 500000 \ kg$ \\
        Anzahl Diskretisierungen & $N = 200$ & $N = 200$ \\
        Lösungsverfahren der DGL & Explizites Euler Verfahren & Explizites Euler Verfahren \\
        Optimierungsverfahren & SQP-Verfahren & SQP-Verfahren \\
        Startvektor & $z_0 = \begin{pmatrix}
        20 \\ 
        9 \\ 
        6000 \\
        90 \\
        1259999 \\ 
        1.47
        \end{pmatrix} $ & $z_0 = \begin{pmatrix}
        20 \\ 
        9 \\ 
        6000 \\
        90 \\
        1259999 \\ 
        1.47
        \end{pmatrix}$ \\
        \bottomrule
    \end{tabularx}
\end{table}
Für die Ergebnisse von Versuch 3.1 (Anhang \ref{kap:Versuch31}) und Versuch 3.2 (Anhang \ref{kap:Versuch32}) wird der folgende technische Aufwand (Tabelle \ref{tab:Versuch3_TA}) benötigt.
\begin{table}[H]
    \centering
    \caption{Technischer Aufwand von Versuch 3.1 und 3.2.}\label{tab:Versuch3_TA}
    \begin{tabularx}{.9\textwidth}{Zccc}
        \toprule
         & \textbf{Versuch 3.1} & \textbf{Versuch 3.2} \\
        \midrule
        Funktionswert der Zielfunktion & $-103327.3544$ & $-140216.741$ \\
        Anzahl Iterationen & $15763$ & $6864$ \\
        Anzahl Funktionsauswertungen & $18975672$ & $8260083$ \\
        Exit Flag von \textit{MATLAB} & $2$ & $2$ \\
        Optimalität des Ergebnis & $0.0011298$ & $0.00015495$ \\
        Berechnungsdauer & $394.887 \ min$ & $262.6699 \ min$ \\
        \bottomrule
    \end{tabularx}
\end{table}





\subsection{Ergebnis von Versuch 3.1}\label{kap:Versuch31}
In Versuch 3.1 wird das Ergebnis in Abbildung \ref{img:test_3_1} erreicht. Dies stellt aufgrund des Abbruchkriteriums $EF = 2$ ein mögliches Optimum des Problems da, da die Änderung des Zustandsvektors $z$ kleiner ist die gegebene Schrittweitentoleranz von hier $SW_{tol} = 10^{-12}$ und die maximale Verletzung der Beschränkungen kleiner ist als Beschränkungstoleranz $BS_{tol} = 10^{-6}$.
\begin{figure}[H]
\begin{center}
\includegraphics[width=\textwidth]{../code/direct_sol/results/test_3_1}
\MyCaption{Ergebnis von Versuch 3.1}{} \label{img:test_3_1}
\end{center}
\end{figure}
Der Verlauf des beschränkten Staudrucks ist in Abbildung \ref{img:test_3_1_staudruck} dargestellt.
\begin{figure}[H]
\begin{center}
\includegraphics[width=\textwidth]{../code/direct_sol/results/test_3_1_staudruck}
\MyCaption{Überprüfung Staudruck $q(v(t),h(t))$ von Versuch 3.1}{} \label{img:test_3_1_staudruck}
\end{center}
\end{figure}




\subsection{Ergebnis von Versuch 3.2}\label{kap:Versuch32}
In Versuch 3.2 wird das Ergebnis in Abbildung \ref{img:test_3_2} erreicht. Dies stellt aufgrund des Abbruchkriteriums $EF = 2$ ein mögliches Optimum des Problems da, da die Änderung des Zustandsvektors $z$ kleiner ist die gegebene Schrittweitentoleranz von hier $SW_{tol} = 10^{-12}$ und die maximale Verletzung der Beschränkungen kleiner ist als Beschränkungstoleranz $BS_{tol} = 10^{-6}$.
\begin{figure}[H]
\begin{center}
\includegraphics[width=\textwidth]{../code/direct_sol/results/test_3_2}
\MyCaption{Ergebnis von Versuch 3.2}{} \label{img:test_3_2}
\end{center}
\end{figure}
Der Verlauf des beschränkten Staudrucks ist in Abbildung \ref{img:test_3_2_staudruck} dargestellt.
\begin{figure}[H]
\begin{center}
\includegraphics[width=\textwidth]{../code/direct_sol/results/test_3_2_staudruck}
\MyCaption{Überprüfung Staudruck $q(v(t),h(t))$ von Versuch 3.2}{} \label{img:test_3_2_staudruck}
\end{center}
\end{figure}












\newpage
\section{Versuch 4}\label{kap:Versuch4}
Zusätzlich zur verkürzten Endzeit wie in Versuch 1 (Anhang \ref{kap:Versuch1}) wird nun zusätzlich das Gewicht des Flugzeuges angepasst, sowie Box-Beschränkungen für die Zustände gesetzt. Diese zusätzlichen Box-Beschränkungen spiegeln die technischen Eigenschaften des Flugzeuges A380-800 der Firma Airbus wieder \cite{A380Tech}:
\begin{itemize}
\item \textbf{Höchstgeschwindigkeit:} Die maximale Geschwindigkeit beträgt $960 \ \frac{km}{h}$ ($266 \ \frac{m}{s}$).
%
\item \textbf{Maximale Flughöhe:} Die maximale Flughöhe beträgt $13100 \ m$.
%
\item \textbf{Maximale Reichweite:} Die maximale Reichweite bei maximaler Auslastung beträgt $15200000 \ m$.
%
\item \textbf{Anstellwinkel:} Für die Boxschranken des Anstellwinkels wurde $\gamma \in [-80,80]$ in Grad gewählt. 
\end{itemize}
Damit ergibt sich das veränderte Optimalsteuerungsproblem:
\begin{align*}
\min_{U} F(X,U) &:= -(x(t_f) - x_0) \\
\text{unter} \hspace{20mm} \dot{X}(t) &= f(X(t),U(t)) = (\dot{h}(t),\dot{\gamma}(t),\dot{x}(t),\dot{v}(t))^T \\
(h,\gamma,x,v)(t_0) &= (h_0,\gamma_0,x_0,v_0) \\
(h,\gamma,x,v)(t_f) &= (h_f,\gamma_f) \hspace{36mm} \\
q(v(t),h(t)) &\leq q_{\max} & & \forall t \in [t_0,t_f] \\
h(t) &\in [0,13100] & & \forall t \in [t_0,t_f] \\
\gamma(t) &\in [-80,80] & & \forall t \in [t_0,t_f] \\
x(t) &\in [0,15200000] & & \forall t \in [t_0,t_f] \\
v(t) &\in [0,266] & & \forall t \in [t_0,t_f] \\
T(t) &\in [T_{\min},T_{\max}] & & \forall t \in [t_0,t_f] \\
C_L(t) &\in [C_{L, \min},C_{L, \max}] & & \forall t \in [t_0,t_f]
\end{align*}
Für die beiden Versuche werden die folgenden Einstellungen aus Tabelle \ref{tab:Versuch4} verwendet.
\begin{table}[H]
    \centering
    \caption{Einstellungen von Versuch 4.1 und 4.2}
    \begin{tabularx}{.9\textwidth}{Zccc}
        \toprule
        \textbf{Einstellungen} & \textbf{Versuch 4.1} & \textbf{Versuch 4.2} \\
        \midrule
        Reduzierte Endzeit & $t_f = 550 \ s$ & $t_f = 600 \ s$ \\
        Angepasstes Startgewicht & $m = 500000 \ kg$ & $m = 500000 \ kg$ \\
        Anzahl Diskretisierungen & $N = 100$ & $N = 100$ \\
        Lösungsverfahren der DGL & Explizites Euler Verfahren & Explizites Euler Verfahren \\
        Optimierungsverfahren & SQP-Verfahren & SQP-Verfahren \\
        Startvektor & $z_0 = \begin{pmatrix}
        20 \\ 
        9 \\ 
        6000 \\
        90 \\
        1259999 \\ 
        1.47
        \end{pmatrix} $ & $z_0 = \begin{pmatrix}
        20 \\ 
        9 \\ 
        6000 \\
        90 \\
        1259999 \\ 
        1.47
        \end{pmatrix}$ \\
        \bottomrule
    \end{tabularx}
\end{table}
Für die Ergebnisse von Versuch 4.1 (Anhang \ref{kap:Versuch41}) und Versuch 4.2 (Anhang \ref{kap:Versuch42}) wird der folgende technische Aufwand (Tabelle \ref{tab:Versuch4_TA}) benötigt.
\begin{table}[H]
    \centering
    \caption{Technischer Aufwand von Versuch 4.1 und 4.2.}\label{tab:Versuch4_TA}
    \begin{tabularx}{.9\textwidth}{Zccc}
        \toprule
         & \textbf{Versuch 4.1} & \textbf{Versuch 4.2} \\
        \midrule
        Funktionswert der Zielfunktion & $-104259.8227$ & $-132811.3929$ \\
        Anzahl Iterationen & $2105$ & $2813$ \\
        Anzahl Funktionsauswertungen & $1277653$ & $1692519$ \\
        Exit Flag von \textit{MATLAB} & $2$ & $2$ \\
        Optimalität des Ergebnis & $0.0023675$ & $0.0004094$ \\
        Berechnungsdauer & $19.1492 \ min$ & $0.00045547 \ min$ \\
        \bottomrule
    \end{tabularx}
\end{table}




\subsection{Ergebnis von Versuch 4.1}\label{kap:Versuch41}
In Versuch 4.1 wird das Ergebnis in Abbildung \ref{img:test_4_1} erreicht. Dies stellt aufgrund des Abbruchkriteriums $EF = 2$ ein mögliches Optimum des Problems da, da die Änderung des Zustandsvektors $z$ kleiner ist die gegebene Schrittweitentoleranz von hier $SW_{tol} = 10^{-6}$ und die maximale Verletzung der Beschränkungen kleiner ist als Beschränkungstoleranz $BS_{tol} = 10^{-6}$.
\begin{figure}[H]
\begin{center}
\includegraphics[width=\textwidth]{../code/direct_sol/results/test_4_1}
\MyCaption{Ergebnis von Versuch 4.1}{} \label{img:test_4_1}
\end{center}
\end{figure}
Der Verlauf des beschränkten Staudrucks ist in Abbildung \ref{img:test_4_1_staudruck} dargestellt.
\begin{figure}[H]
\begin{center}
\includegraphics[width=\textwidth]{../code/direct_sol/results/test_4_1_staudruck}
\MyCaption{Überprüfung Staudruck $q(v(t),h(t))$ von Versuch 4.1}{} \label{img:test_4_1_staudruck}
\end{center}
\end{figure}




\subsection{Ergebnis von Versuch 4.2}\label{kap:Versuch42}
In Versuch 4.2 wird das Ergebnis in Abbildung \ref{img:test_4_2} erreicht. Dies stellt aufgrund des Abbruchkriteriums $EF = 2$ ein mögliches Optimum des Problems da, da die Änderung des Zustandsvektors $z$ kleiner ist die gegebene Schrittweitentoleranz von hier $SW_{tol} = 10^{-6}$ und die maximale Verletzung der Beschränkungen kleiner ist als Beschränkungstoleranz $BS_{tol} = 10^{-6}$.
\begin{figure}[H]
\begin{center}
\includegraphics[width=\textwidth]{../code/direct_sol/results/test_4_2}
\MyCaption{Ergebnis von Versuch 4.2}{} \label{img:test_4_2}
\end{center}
\end{figure}
Der Verlauf des beschränkten Staudrucks ist in Abbildung \ref{img:test_4_2_staudruck} dargestellt.
\begin{figure}[H]
\begin{center}
\includegraphics[width=\textwidth]{../code/direct_sol/results/test_4_2_staudruck}
\MyCaption{Überprüfung Staudruck $q(v(t),h(t))$ von Versuch 4.2}{} \label{img:test_4_2_staudruck}
\end{center}
\end{figure}



%------------------------------------------------------------------------------
% Literatur-, Abbildungs-, Tabellen-, Formel- und Abkuerzungsverzeichnis
%------------------------------------------------------------------------------
\backmatter

% Literaturverzeichnis
\clearpage
\printbibliography

% Auflistung aller verwendeter Progammcodes
\clearpage
\lstlistoflistings
\addcontentsline{toc}{chapter}{Programmcodes Verzeichnis}


% Abbildungsverzeichnis
%\clearpage
%\listoffigures
%\addcontentsline{toc}{chapter}{\listfigurename}

% Tabellenverzeichnis
%\clearpage
%\listoftables
%\addcontentsline{toc}{chapter}{\listtablename}

% Formelverzeichnis
%\clearpage
%\chapter{Formelverzeichnis}


%%\addcontentsline{toc}{chapter}{Formelverzeichnis}

% Abkuerzungsverzeichnis
%\clearpage
%\chapter{Abkürzungsverzeichnis}


%%\addcontentsline{toc}{chapter}{Abkürzungsverzeichnis}

%------------------------------------------------------------------------------
% Eigenständigkeitserklärung
%------------------------------------------------------------------------------
\clearpage
\thispagestyle{empty}
\chapter{Eigenständigkeitserklärung}

\begin{tabbing}
\hspace{30mm}\=\hspace{60mm}\=\kill
Namen: \> \Heiko \> (Matrikelnummer: \Hmatnr) \\ 
  \>  \Philipp \> (Matrikelnummer: \Pmatnr) \\ 
  \>  \Felix \> (Matrikelnummer: \Fmatnr)
\end{tabbing} 

\minisec{Erklärung}

Wir erklären, dass wir die Arbeit selbständig verfasst und keine anderen als die angegebenen Quellen und Hilfsmittel verwendet haben.\vspace{2cm}

Ulm, den \dotfill

\hspace{10cm} {\footnotesize \Heiko}\\[2em]


Ulm, den \dotfill 

\hspace{10cm} {\footnotesize \Philipp} \\[2em]


Ulm, den \dotfill

\hspace{10cm} {\footnotesize \Felix}
\end{document}
%==============================================================================