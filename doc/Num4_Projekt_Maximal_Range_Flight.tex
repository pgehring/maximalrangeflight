%==============================================================================
%------------------------------------------------------------------------------
% Notizen, Bemerkungen, Quellen
%------------------------------------------------------------------------------
% Bemerkungen:
% 	- Kompilieren mit PDFLaTeX
%	- 

% Notizen:
%	- 

% Quellen:
%	-
%==============================================================================






%==============================================================================
%------------------------------------------------------------------------------
% Dokument Einstellungen
%------------------------------------------------------------------------------
\NeedsTeXFormat{LaTeX2e}
\documentclass[
a4paper,
11pt,
headsepline,           % Linie zw. Kopfzeile und Text
oneside,               % einseitig
numbers=noenddot,	   % keine Punkte nach den letzten Ziffern in Überschriften
bibliography=totoc,    % LV in IV
%DIV=15,                % Satzspiegel auf 15er Raster, schmalere Ränder   
%BCOR15mm              % Bindekorrektur
%,draft
]{scrbook}
\KOMAoptions{DIV=last} % Neuberechnung Satzspiegel nach Laden von Paket helvet
%==============================================================================






%==============================================================================
%------------------------------------------------------------------------------
% Seitenlayout
%------------------------------------------------------------------------------
\pagestyle{headings}
\usepackage{blindtext}

% für Texte in deutscher Sprache
\usepackage[ngerman]{babel}
\usepackage[utf8]{inputenc}
\usepackage[T1]{fontenc}

% Helvetica als Standard-Dokumentschrift
\usepackage[scaled]{helvet}
\renewcommand{\familydefault}{\sfdefault} 

\usepackage{geometry}
\usepackage{layout}

\newcounter{SeitenzahlSpeicher}

%------------------------------------------------------------------------------
% Grafiken
%------------------------------------------------------------------------------
\usepackage{graphicx}
\usepackage[font=footnotesize, labelfont=bf]{caption, subfig}
\usepackage{float} % um figure/Bild/Abbildung an bestimmte Stelle mit [H]
\usepackage{framed}

%------------------------------------------------------------------------------
% Literaturverzeichnis mit BibLKaTeX
%------------------------------------------------------------------------------
\usepackage[babel,german=quotes]{csquotes}
%\usepackage[backend=bibtex8]{biblatex}
\usepackage[backend=biber,bibencoding=utf8,style=numeric]{biblatex}
%\bibliography{bibliography}
\addbibresource{bibliography.bib}

%------------------------------------------------------------------------------
% Für Tabellen mit fester Gesamtbreite und variabler Spaltenbreite
%------------------------------------------------------------------------------
\usepackage{tabularx} 
\usepackage{booktabs}
\usepackage{longtable}

%------------------------------------------------------------------------------
% Besondere Schriftauszeichnungen
%------------------------------------------------------------------------------
\usepackage{url}              % \url{http://...} in Schreibmaschinenschrift
\usepackage{color}            % zum Setzen farbigen Textes
\usepackage[dvipsnames]{xcolor}

%------------------------------------------------------------------------------
% Pakete für Mathe-Umgebungen, -Symbole und Neudefinition von Kommandos
%------------------------------------------------------------------------------
\usepackage{amssymb, amsmath, amsthm} % Pakete für Mathe-Umgebungen und -Symbole

% Mengen Buchstaben:
\newcommand{\E}{\mathbb{E}}
\newcommand{\Q}{\mathbb{Q}}
\newcommand{\R}{\mathbb{R}}
\newcommand{\N}{\mathbb{N}}

%------------------------------------------------------------------------------
% Mathematische Umgebungen (Satz, Definition, Beweis ...)
%------------------------------------------------------------------------------
% Quellen:
%	Umgebungen: 	https://www.youtube.com/watch?v=_KFr29O7Jrc
%					https://de.overleaf.com/learn/latex/Environments
%	Zähler: https://www.youtube.com/watch?v=_NjyOVC54aQ

\usepackage{ifthen}

% Zähler für die Nummerierung der Umgebungen und dessen Referenzen
% 	Zu jedem Zähler umgnr gehört der Ausgabebefehl \theumgnr, der u.a. von 
% 	\ref aufgerufen wird (siehe Quelle).
\newcounter{umgnr}[chapter]
\renewcommand{\theumgnr}{\arabic{chapter}.\arabic{umgnr}}

% Umgebung "Satz":
\newenvironment{satz}[1][]
{% \begin{satz}
\medskip\noindent%
\refstepcounter{umgnr}% Zähler erhöhen
\textbf{Satz~\theumgnr \ifthenelse{\equal{#1}{}}{}{~(#1)}}
}
{% \end{satz}
\medskip
}

% Umgebung "Hilfssatz":
\newenvironment{hilfssatz}[1][]
{\medskip\noindent \refstepcounter{umgnr}%
\textbf{Hilfssatz~\theumgnr\ifthenelse{\equal{#1}{}}{}{~(#1)}}
}
{\medskip}

% Umgebung "Definition":
\newenvironment{definition}[1][]
{\medskip\noindent \refstepcounter{umgnr}%
\textbf{Definition~\theumgnr\ifthenelse{\equal{#1}{}}{}{~(#1)}}
}
{\medskip}

% Umgebung "Bemerkung":
\newenvironment{bemerkung}[1][]
{\medskip\noindent \refstepcounter{umgnr}%
\textbf{Bemerkung~\theumgnr\ifthenelse{\equal{#1}{}}{}{~(#1)}}
}
{\medskip}

% Umgebung "Beispiel":
\newenvironment{beispiel}[1][]
{\medskip\noindent \refstepcounter{umgnr}%
\textbf{Beispiel~\theumgnr\ifthenelse{\equal{#1}{}}{}{~(#1)}}
}
{
\hfill $\diamond$
\medskip
}

% Umgebung "Lemma":
\newenvironment{lemma}[1][]
{\medskip\noindent \refstepcounter{umgnr}%
\textbf{Lemma~\theumgnr\ifthenelse{\equal{#1}{}}{}{~(#1)}}
}
{\medskip}

% Umgebung "Korollar":
\newenvironment{korollar}[1][]
{\medskip\noindent \refstepcounter{umgnr}%
\textbf{Korollar~\theumgnr\ifthenelse{\equal{#1}{}}{}{~(#1)}}
}
{\medskip}

% Umgebung "Beweis":
\newenvironment{beweis}[1][]
{% \begin{beweis}
\textbf{Beweis\ifthenelse{\equal{#1}{}}{:}{~(#1):}}
\begin{itshape}
}
{% \end{beweis}
\end{itshape}
\hfill $\Box$
\medskip
}

%------------------------------------------------------------------------------
% Formatierung
%------------------------------------------------------------------------------
\usepackage{enumitem}
\usepackage{setspace}         % Paket für div. Abstände, z.B. ZA
%\onehalfspacing              % nur dann, wenn gefordert; ist sehr groß!!
\setlength{\parindent}{0pt}   % kein linker Einzug der ersten Absatzzeile
\setlength{\parskip}{1.4ex plus 0.35ex minus 0.3ex} % Absatzabstand, leicht variabel

% Tiefe, bis zu der Überschriften in das Inhaltsverzeichnis kommen
\setcounter{tocdepth}{3}      % ist Standard

%------------------------------------------------------------------------------
% Quellcode
%------------------------------------------------------------------------------
%https://en.wikibooks.org/wiki/LaTeX/Source_Code_Listings
% Beispiele für Quellcode
\usepackage{listings}
\lstdefinestyle{num_octave}{
  basicstyle=\scriptsize,        % the size of the fonts that are used for the code
  breakatwhitespace=false,         % sets if automatic breaks should only happen at whitespace
  breaklines=true,                 % sets automatic line breaking
  captionpos=b,                    % sets the caption-position to bottom
  commentstyle=\color{ForestGreen},    % comment style
  escapeinside={\%*}{*)},          % if you want to add LaTeX within your code
  extendedchars=true,              % lets you use non-ASCII characters; for 8-bits encodings only, does not work with UTF-8
  frame=single,	                   % adds a frame around the code
  %keepspaces=true,                 % keeps spaces in text, useful for keeping indentation of code (possibly needs columns=flexible)
  keywordstyle=\color{blue},       % keyword style
  language=Octave,                 % the language of the code
  numbers=left,                    % where to put the line-numbers; possible values are (none, left, right)
  numbersep=5pt,                   % how far the line-numbers are from the code
  numberstyle=\tiny, % the style that is used for the line-numbers
  numberbychapter=true
  rulecolor=\color{black},         % if not set, the frame-color may be changed on line-breaks within not-black text (e.g. comments (green here))
  stringstyle=\color{black},     % string literal style
  tabsize=4,	                   % sets default tabsize to 2 spaces
}
\renewcommand{\lstlistingname}{Programmcode}% Listing -> Programmcode
\renewcommand{\lstlistlistingname}{\lstlistingname s Verzeichnis}% List of Listings -> Programmcodes

  
%https://www.zaik.uni-koeln.de/AFS/teachings/ws0708/ORSeminar/latex/seminar.pdf
%https://en.wikibooks.org/wiki/LaTeX/Algorithms
\usepackage{algorithmic}
\usepackage{algorithm}

\usepackage{subfloat}
\usepackage{mathrsfs}

%verhindet, dass sich Fußnote auf zwei Seiten erstreckt
\interfootnotelinepenalty=10000 

%------------------------------------------------------------------------------
% Persönliche Daten
%------------------------------------------------------------------------------
\newcommand{\Heiko}{Heiko Karus}
\newcommand{\Philipp}{Philipp Gehring}
\newcommand{\Felix}{Felix Götz}

\newcommand{\Hemail}{Heiko.Karus@uni-ulm.de}
\newcommand{\Pemail}{Philipp.Gehring@uni-ulm.de}
\newcommand{\Femail}{Felix.Goetz@uni-ulm.de}

\newcommand{\Hmatnr}{1072378}
\newcommand{\Pmatnr}{1104946}
\newcommand{\Fmatnr}{1063352}

\newcommand{\titel}{Maximal range flight}
\newcommand{\jahr}{2021}
\newcommand{\gutachterA}{Prof. Dr. rer. nat. Dirk Lebiedz}
\newcommand{\gutachterB}{Jörn Dietrich, M.Sc.}

\newcommand{\fakultaet}{Mathematik und Wirtschaftswissenschaften}
\newcommand{\institut}{Institut für numerische Mathematik}

%------------------------------------------------------------------------------
% Informationen, die LaTeX in die PDF-Datei schreibt
%------------------------------------------------------------------------------
\pdfinfo{
  /Author (Karus, Gehring, Goetz)
 % /Title (\titel)
  /Producer     (pdfeTex 3.14159-1.30.6-2.2)
  /Keywords ()
}

%\usepackage{hyperref}
\usepackage[colorlinks=true, urlcolor=blue, linkcolor=green]{hyperref}
\hypersetup{
pdftitle=\titel,
pdfauthor=\Heiko \Philipp \Felix,
pdfsubject={Projekt Numerik 4},
pdfproducer={pdfeTex 3.14159-1.30.6-2.2},
colorlinks=false,
pdfborder=0 0 0	% keine Box um die Links!
}

%------------------------------------------------------------------------------
% Weiter hinzugefügte Pakete
%------------------------------------------------------------------------------
\usepackage{xcolor, colortbl}

\counterwithout{footnote}{chapter}
\usepackage{multirow}
%==============================================================================






%==============================================================================
%------------------------------------------------------------------------------
% Beginn Dokument
%------------------------------------------------------------------------------
% Trennungsregeln
\hyphenation{MATLAB}

\begin{document}

% Seitenlayout
\newgeometry{left=2.5cm, right=2.5cm, top=3cm, bottom=3cm}

\frontmatter

%------------------------------------------------------------------------------
% Titelseite
%------------------------------------------------------------------------------
\thispagestyle{empty}
\begin{addmargin*}[4mm]{-10mm}

\includegraphics[height=1.8cm]{images/00_Sonstiges/unilogo_bild}
\hfill
\includegraphics[height=1.8cm]{images/00_Sonstiges/unilogo_wort}\\[2em]

%\includegraphics[height=1.8cm]{images/00_Sonstiges/THU}
%\hfill
%\includegraphics[height=1.8cm]{images/00_Sonstiges/unilogo_bild}
%\includegraphics[height=1.8cm]{images/00_Sonstiges/unilogo_wort}\\[2em]

%\includegraphics[height=1.4cm]{images/unilogo_wort}
%\hfill
%\includegraphics[height=1.4cm]{images/THU_word}\\[1em]
%\includegraphics[height=1.4cm]{images/unilogo_bild}
%\hfill
%\includegraphics[height=1.4cm]{images/THU_logo}\\[1em]

{\footnotesize
%{\bfseries Universität Ulm} \textbar ~89069 Ulm \textbar ~Germany
\hspace*{130mm}\parbox[t]{35mm}{
\bfseries Fakultät für\\
\fakultaet\\
\mdseries \institut
}\\[2cm]

\parbox{140mm}{\bfseries \LARGE \titel}\\[2.5em]
{\footnotesize Projekt Numerik 4}\\[2em]

{\footnotesize \bfseries Vorgelegt von:}\\
{\footnotesize \Heiko \\ E-Mail: \Hemail \\ Matrikel-Nr.: \Hmatnr}\\ \\%[2em]
{\footnotesize \Philipp \\ E-Mail: \Pemail \\ Matrikel-Nr.: \Pmatnr}\\ \\%[2em]
{\footnotesize \Felix \\ E-Mail: \Femail \\ Matrikel-Nr.: \Fmatnr}\\ \\[2em]

{\footnotesize \bfseries Gutachter:}\\                     
{\footnotesize \gutachterA}\\ \\%[2em]
{\footnotesize \gutachterB}\\[2em]

{\footnotesize \jahr}
}
\end{addmargin*}

%------------------------------------------------------------------------------
% Impressum
%------------------------------------------------------------------------------
\clearpage
\thispagestyle{empty}
{ \small
  \flushleft
  Fassung \today \\\vfill
  \copyright~\jahr~\Heiko,~\Philipp,~\Felix\\[0.5em]
% Wenn Sie Ihre Arbeit unter einer freien Lizenz bereitstellen möchten, können Sie die nächste Zeile in Ihren Code aufnehmen. Bitte beachten Sie, dass Sie hierfür an allen Inhalten, inklusive enthaltener Abbildungen, die notwendigen Rechte benötigen! Beim Veröffentlichungsexemplar Ihrer Dissertation achten Sie bitte darauf, dass der Lizenztext nicht den Angaben in den Metadaten der genutzten Publikationsplattform widerspricht. Nähere Information zu den Creative Commons Lizenzen erhalten Sie hier: https://creativecommons.org/licenses/
%This work is licensed under the Creative Commons Attribution 4.0 International (CC BY 4.0) License. To view a copy of this license, visit \href{https://creativecommons.org/licenses/by/4.0/}{https://creativecommons.org/licenses/by/4.0/} or send a letter to Creative Commons, 543 Howard Street, 5th Floor, San Francisco, California, 94105, USA. \\
  Satz: PDF-\LaTeXe
}

% ab hier Zeilenabstand etwas größer 
\setstretch{1.2}

%------------------------------------------------------------------------------
% Vorwort, Kurzfassung und Abstract
%------------------------------------------------------------------------------
% Vorwort
%\clearpage
%\input{chapters/Vorwort}

%Kurzfassung
%\clearpage
%\input{chapters/Kurzfassung}

% Abstract
%\clearpage
%\input{chapters/Abstract}

%------------------------------------------------------------------------------
% Inhaltsverzeichnis
%------------------------------------------------------------------------------
\tableofcontents
\addcontentsline{toc}{chapter}{Inhaltsverzeichnis}
\setcounter{SeitenzahlSpeicher}{\value{page}}
%------------------------------------------------------------------------------
% Hauptteil
%------------------------------------------------------------------------------
\mainmatter
\chapter{Motivation und Ziel der Arbeit}
Die Berechnung von Flugtrajektorien zur Reichweitenmaximierung haben eine lange Historie, siehe  \cite{Burrows1982, Murrieta2016, Schaback2017, Pierson1989}.  Dabei wurden verschiedene mathematische Techniken angewandt, die von Energiebetrachtungen \cite{Calise1977}, Parametrisierungen von Trajektorien \cite{Burrows1982} über bestimmte Formen der Optimalsteuerungstheorie \cite{Javier2016} bis hin zur Mehrzieloptimierung mit verschiedenen Kostenfunktionen \cite{Gardi2016} reichen. 

Eine einfache Lösung für den entfernungsoptimalen Flug ist für den Fall des Horizontalflugs bekannt, siehe beispielsweise \cite{Peckham1974}. Diese folgt aus der Maximierung des Verhältnisses $\sqrt{C_{L}/C_{D}}$ der Auftriebs- und Widerstandskoeffizienten, was zu einer Geschwindigkeit führt, die um den Faktor $\sqrt{4/3}$ größer ist als die Geschwindigkeit zur Maximierung des Verhältnisses von Auftrieb zu Widerstand $C_{L}/C_{D}$ \cite{Schaback2017}.  

Diese Projektarbeit bietet eine Erweiterung auf den Steigflug und konzentriert sich dabei auf numerische Standardmethoden, die für das Lösen von Optimalsteuerungsproblemen verwendet werden. Modelliert wird ein zweidimensionaler Flug eines Flugzeugs in der $x$-$h$-Ebene (Abbildung \ref{img:Flugzeug}), bei dem man den Auftriebsbeiwert $C_L(t)$ und den Schub $T(t)$ steuern kann.

\begin{figure}[H]
    \begin{center}
        \includegraphics[width=\textwidth]{images/01_Modellaufbau/Flugzeug.pdf}
        \MyCaption{2D-Skizze eines Flugzeuges in der $x$-$h$-Ebene}{mit angreifenden Kräften $L$ (Auftriebskraft), $D$ (Luftwiederstand), $W$ (Erdanziehungskraft) und $T$ (Schub) am Schwerpunkt $S$. Des Weiteren beschreibt $v$ die Geschwindigkeit und $\gamma$ den Anstellwinkel.}\label{img:Flugzeug}
    \end{center}
\end{figure}

Das Ziel ist es, das Flugzeug von einer gegebenen Anfangsposition so zu steuern, dass eine vorgegebene Reisehöhe $h_f = 10668\ m$ und Anstellwinkel $\gamma_f = 0\ ^{\circ}$ in einer Flugzeit von $1800 \ s$ erreicht  wird, wobei die zurückgelegte Strecke maximal wird. Dabei dürfen die Steuerbeschränkungen für den Schub und Auftriebsbeiwert nicht verletzt und ein maximaler Staudruck nicht überschritten werden. 

Beginnend mit der Herleitung der Gleichungen für den quasi-statischen Flug in \autoref{cha:optim}, folgt die Formulierung des Optimalsteuerungsproblems für direkte (\autoref{cha:direct}) und indirekte (\autoref{cha:indirect}) Lösungsverfahren. Anschließend folgt die numerische Umsetzung und Lösung des Optimalsteuerungsproblems und sowie die Diskussion der Ergebnisse.

Alle Modellrechnungen wurden für den Airbus A380-800 \cite{A380Tech} durchgeführt. MATLAB wurde für alle numerischen Berechnungen, hauptsächlich für die Lösung von gewöhnlichen Differentialgleichungen und Optimierungsproblemen, verwendet.





\chapter{Optimalsteuerungsproblem} \label{cha:optim}

Modelliert werden soll das Flugzeug A380-800 der Firma Airbus. Dabei seien:
\begin{itemize}
    \item $x(t)$: $x$-Koordinate des Massenschwerpunktes $S$
    %
    \item $h(t)$: $h$-Koordinate des Massenschwerpunktes $S$
    %
    \item $v(t)$: Geschwindigkeit
    %
    \item $\gamma(t)$: Anstellwinkel
    %
    \item $T(t)$: Schub, Steuerung
    %
    \item $C_L(t)$: Auftriebsbeiwert, Steuerung
\end{itemize}
Um die Kräfte welche auf das Flugzeug einwirken berechnen zu können, werden folgende Hilfsgrößen benötigt:
\begin{itemize}
    \item Luftwiderstandsbeiwert: \[C_D(C_L(t)) := C_{D_0} + k \cdot C^2_L(t) \ \ \ \ \text{mit} \ \ \ \ k = \dfrac{1}{\pi \cdot e \cdot AR}\] wobei $C_{D0}$ der Nullluftwiderstandsbeiwert, $e$ die Oswaldfaktor und $AR$ die Streckung (engl. \textit{aspect ratio}) bezeichnet.
    %
    \item Luftdichte: \[\rho(h(t)) := \alpha \cdot e^{-\beta \cdot h(t)}\] %https://wind-data.ch/tools/luftdichte.php
    %
    \item Staudruck: \[q(v(t), h(t)) := \dfrac{\rho(h(t)) \cdot v^2(t)}{2} \]
\end{itemize}
Die Kräfte lassen sich dann berechnen mit:
\begin{itemize}
    \item Auftriebskraft: \[L(v(t), h(t), C_L(t)) := F \cdot C_L(t) \cdot q(v(t), h(t))\] wobei $F$ die wirksame Fläche, d.h. die von der Luft angeströmte Fläche, ist.
    %
    \item Luftwiderstand: \[D(v(t), h(t), C_L(t)) := F \cdot C_D(C_L(t)) \cdot q(v(t), h(t))\]
    \item Erdanziehungskraft: \[W = m \cdot g\]
\end{itemize}

Mit dem Newtonsschen Gesetz $F = m * a$ lässt sich die Differentialgleichungen für die Geschwindigkeit $v$ aufstellen:
\[\begin{split}
    F &:= m \cdot a \\\
    \Rightarrow \dot{v}(t) &= a(t) = \dfrac{F(t)}{m} = \dfrac{T(t) - D(v(t),h(t),C_L(t)) - W \sin(\gamma(t))}{m}
\end{split} \]
Mit der Gleichung für die Zentripetalkraft $F_{ZP} = \dfrac{m v^2}{r}$ lässt sich die Differentialgleichungen für den Ansstellwinkel $\gamma$ aufstellen:
\[\begin{split}
    F_{ZP} &:= \dfrac{m v^2}{r} \\\
    \Rightarrow \dot{\gamma}(t) &:= \dfrac{v(t)}{r} = \dfrac{F_{ZP}(t)}{m v(t)} = \dfrac{L(v(t),h(t),C_L(t)) - W \cos(\gamma(t))}{m v(t)}
\end{split} \]
Die Differentialgleichungen für die $h(t)$ und $x(t)$ lassen mittels Geschwindigkeit und Anstellwinkel bestimmen:
\[\begin{split}
    \dot{x}(t) &= v(t) \cos(\gamma(t))\\\
    \dot{h}(t) &= v(t) \sin(\gamma(t))
\end{split} \]
Es ergibt sich somit das Optimalsteuerungsproblem (Problem \ref{prob:MaxRF}).

\begin{problem}[Optimalsteuerungsproblem - Maximal-Range-Flight]\label{prob:MaxRF}
\begin{align*}
    \min_{T, C_L} F(h,\gamma,x,v,T,C_L) &:= -(x(t_f) - x_0) & & \\\
    \text{unter} \hspace{20mm} \dot{h}(t) &= v(t) \sin(\gamma(t)) \hspace{27mm} & & \\\
    \dot{\gamma}(t) &=  \dfrac{L(v(t),h(t),C_L(t)) - W \cos(\gamma(t))}{mv(t)} & & \\\
    \dot{x}(t) &= v(t) \cos(\gamma(t)) & & \\\
    \dot{v}(t) &= \dfrac{T(t) - D(v(t),h(t),C_L(t)) - W \sin(\gamma(t))}{m} & & \\\
    %
    (h,\gamma,x,v)(t_0) &= (h_0,\gamma_0,x_0,v_0) & & \\\
    (h,\gamma)(t_f) &= (h_f,\gamma_f) & & \\\
    %
    q(v(t),h(t)) &\leq q_{\max} \hspace{19.5mm} & & \forall t \in [t_0,t_f]\\\
    T(t) &\in [T_{\min},T_{\max}] & & \forall t \in [t_0,t_f] \\\
    C_L(t) &\in [C_{L, \min},C_{L, \max}] & & \forall t \in [t_0,t_f]
\end{align*}
\end{problem}

Für das Modell werden folgende Parameter aus Tabelle \ref{tab:ProblemPara} verwendet.
\begin{table}[H]
    \centering
    \caption{Problemparameter für das Flugzeug A380-800 der Firma Airbus.}\label{tab:ProblemPara}
    \begin{tabularx}{.9\textwidth}{lXrl}
        \toprule
        \textbf{Parameter}   & \textbf{Bedeutung} & \textbf{Wert} & \textbf{Einheit} \\ 
        \midrule
        $t_0$       & Anfangszeitpunkt & $0$ & $s$ \\ 
        $t_f$       & Endzeitpunkt & $1800$ & $s$ \\ 
        \hline
        $h_0$       & Anfangshöhe & $0$ & $m$ \\ 
        $\gamma_0$  & Anfangsanstellwinkel & $0.27$ & $^{\circ}$ \\
        $x_0$       & Anfangskoordinate & $0$ & $m$ \\ 
        $v_0$       & Anfangsgeschwindigkeit & $100$ & $\frac{m}{s}$ \\ 
        \hline
        $h_f$       & Endhöhe & $10668$ & $m$ \\ 
        $\gamma_f$  & Endanstellwinkel & $0$ & $^{\circ}$ \\
        \hline
        $\alpha$    &  & $1.247015$ & $1$\\ 
        $\beta$     &  & $0.000104$ & $1$\\
        $g$         & Erdbeschleunigung & $9.81$ & $\frac{m}{s^2}$ \\ 
        $C_{D_0}$   & Nullluftwiderstandsbeiwert & $0.032$ & $1$\\ 
        $AR$        & Streckung & $7.5$ & $1$\\ 
        $e$         & Oswaldfaktor & $0.8$ & $1$\\ 
        $F$         & wirksame Fläche & $845$ & $m^2$ \\ 
        $m$         & Masse & $276800$ & $kg$ \\ 
        $q_{\max}$  & maximaler Staudruck & $44154$ & $\frac{N}{m^2}$ \\
        $T_{\min}$  & minimale Schubkraft & $0$ & $N$ \\  
        $T_{\max}$  & maximale Schubkraft & $1260000$ & $N$ \\ 
        $C_{L, \min}$ & minimaler Auftriebsbeiwert & $0$ & $1$ \\ 
        $C_{L, \max}$ & maximaler Auftriebsbeiwert & $1.48$ & $1$ \\ 
        \bottomrule
    \end{tabularx} 
\end{table}



%\newpage
%\section{Überprüfung der optimalen Steuerung}
%Die Steuerung $T(t)$ (Schub) geht linear in die Hamilton-Funktion ein. Um die Hamilton-Funktion zu minimieren gilt für diese Bang-Bang Verhalten.
%
%Die Steuerung $C_L(t)$ geht nichtlinear in die Hamilton-Funktion ein.
%
%
%
%
%
%
%\section{Überprüfung der Hinreichenden Optimalitätsbedingungen}
%
%
%
%
%
%
%
%\section{Aufstellen des Randwertproblems}
%Muss also ein Mehrpunktrandwertproblem sein ???
%
%
%Mit den Optimalitätsbedingungen des Minimumprinzips von Pontryagin lässt sich das Steuerungsproblem in ein Randwertproblem überführen, welches aus den beiden Funktionen $r(t,Z(t))$ und $r_0(Z(t_0),Z(t_f)) = 0$ besteht. Für $r(t,Z(t))$ ergibt sich \[r(t,Z(t)) = \dot{Z}(t) = \begin{pmatrix}
%\dot{h}(t),\dot{\gamma}(t),\dot{x}(t),\dot{v}(t),\dot{\lambda}_1(t),\dot{\lambda}_2(t),\dot{\lambda}_3(t),\dot{\lambda}_4(t)
%\end{pmatrix}^T\] Für $r_0(Z(t_0),Z(t_f)) = 0$ müssen zunächst die Endbedingungen mit gebildet aus 
%\[\begin{split}
%X_i(t_f) &= c_i \hspace{25mm} (i=1,...,r) \\\
%\lambda_i(t_f) &= \lambda_0 g_{X_i}(X^{\ast}(t_f)) \hspace{5mm} (i=r+1,...,n)
%\end{split}\] gebildet werden. Es ergibt sich dann \[r_0(Z(t_0),Z(t_f)) = \begin{pmatrix}
%h(t_0) - h_0 \\ 
%\gamma(t_0) - \gamma_0 \\
%x(t_0) - x_0 \\ 
%v(t_0) - v_0 \\ 
%h(t_f) - h_f \\ 
%\gamma(t_f) - \gamma_f \\
%\lambda_3(t_f) + \lambda_0 \\ 
%\lambda_4(t_f) - 0
%\end{pmatrix}\]

\chapter{Technische Umsetzung}
Das in Gleichung \eqref{equ:mayer_problem} gezeigte Optimalsterungsproblem wird numerisch in MATLAB implementiert und gelöst. Im Folgenden Kapitel werden 
Methoden zur Lösung des Differentialgleichungssystems (vgl. Gleichung \eqref{equ:state_space}) verglichen wird die Implementierung der Lösung des 
Optimalsteuerungsproblems erläutert.

\section{Numerische Lösung von gewöhnlichen Differentialgleichungen}
Das in Kapitel \autoref{} gezeigte Modell des Flugzeugs wurde durch ein System von gewöhnlichen Differentialgleichungen abgebildet. Um dieses Modell
Lösen zu können, werden Methoden von MATLAB und eigens implementierte Methoden zur Integration verwendet.

Mit dem Vergleich der Methoden konnte die fehlerfreie Implementierung bestätigt werden und die für den Anwendungsfall effizienteste Methode 
ausgewählt werden.

% \begin{table}[htbp]
%     \caption{Untersuchte Einschrittalgorithmen}
%     \begin{tabularx}{\textwidth}{Xccc}
%         \toprule
%         Algorithmus & Rechenzeit & Reichenzeitverbesserung & max. Fehler \\
%         \midrule
        
%         \bottomrule
%     \end{tabularx}
% \end{table}

\begin{figure}[!htbp]
    \centering 
    \subfloat[\label{fig:methods_d_h}]{\includegraphics[width=0.45\textwidth]{../code/methods/results/methods_plot_d_h}}
    \qquad
    \subfloat[\label{fig:methods_d_gamma}]{\includegraphics[width=0.45\textwidth]{../code/methods/results/methods_plot_d_gamma}} \\

    \subfloat[\label{fig:methods_d_x}]{\includegraphics[width=0.45\textwidth]{../code/methods/results/methods_plot_d_x}}
    \qquad
    \subfloat[\label{fig:methods_d_v}]{\includegraphics[width=0.45\textwidth]{../code/methods/results/methods_plot_d_v}}
    \caption{Lösung des Differentialgleichungsmodells (vgl. \eqref{equ:state_space}). Die vier Zustandgrößen des Vektors wurden mit kontanten Steuerfunktionen \(T\) und \(C_L\) gelöst.} %
    % \subref{fig:methods_h} zeigt die gleichmäßig steigende Flughöhe mit kleiner Abweichung zwischend en Methoden. \subref{fig:methods_gamma} zeigt den %
    % Anstellwinkel des Flugzeuges mit deutlicher Abweichung zwischen den Algorithmen. \subref{fig:methods_x} zeigt die zurückgelegte Strecke des Flugzeuges.} %
    % \subref{fig:methods_v} zeigt die Geschwindigkeit des Flugzeuges.}
\end{figure}

\begin{figure}[!htbp]
    \centering 
    \subfloat[\label{fig:methods_i_h}]{\includegraphics[width=0.45\textwidth]{../code/methods/results/methods_plot_i_h}}
    \qquad
    \subfloat[\label{fig:methods_i_gamma}]{\includegraphics[width=0.45\textwidth]{../code/methods/results/methods_plot_i_gamma}} \\

    \subfloat[\label{fig:methods_i_x}]{\includegraphics[width=0.45\textwidth]{../code/methods/results/methods_plot_i_x}}
    \qquad
    \subfloat[\label{fig:methods_i_v}]{\includegraphics[width=0.45\textwidth]{../code/methods/results/methods_plot_i_v}} \\

    \subfloat[\label{fig:methods_i_h}]{\includegraphics[width=0.45\textwidth]{../code/methods/results/methods_plot_i_l1}}
    \qquad
    \subfloat[\label{fig:methods_i_gamma}]{\includegraphics[width=0.45\textwidth]{../code/methods/results/methods_plot_i_l2}} \\

    \subfloat[\label{fig:methods_i_h}]{\includegraphics[width=0.45\textwidth]{../code/methods/results/methods_plot_i_l3}}
    \qquad
    \subfloat[\label{fig:methods_i_gamma}]{\includegraphics[width=0.45\textwidth]{../code/methods/results/methods_plot_i_l4}} \\

    \caption{Lösung des Differentialgleichungsmodells. Die acht Zustandgrößen des Vektors wurden mit kontanten Steuerfunktionen \(T\) und \(C_L\) gelöst.} %
    % \subref{fig:methods_h} zeigt die gleichmäßig steigende Flughöhe mit kleiner Abweichung zwischend en Methoden. \subref{fig:methods_gamma} zeigt den %
    % Anstellwinkel des Flugzeuges mit deutlicher Abweichung zwischen den Algorithmen. \subref{fig:methods_x} zeigt die zurückgelegte Strecke des Flugzeuges.} %
    % \subref{fig:methods_v} zeigt die Geschwindigkeit des Flugzeuges.}
\end{figure}

Runtime Vergleich 

Ergebnisse Vergleich

Toleranzen

Unterschiedliche Steuerungen

Maximale differenz der Funktionswerte




\section{Klassenstruktur des Problems}

UML


\chapter{Lösen des Optimalsteuerungsproblem mit direkten Lösungsverfahren}



\section{Algorithmen}
%\floatname{algorithm}{Klasse}
%\begin{algorithm}[H]
%\caption{MaximalRangeFlight}\label{algo:SISPF}
%\textbf{[$\lbrace s^i_t, w^i_t \rbrace^{N_s}_{i=1}$] = PFSIR [$\lbrace s^i_{t-1}, w^i_{t-1} \rbrace^{N_s}_{i=1}$]}
%\begin{algorithmic}
%\FOR {$i = 1,...,N_s$}
%\STATE $s^i_t \sim q(x_t \mid s^i_{i-1})$
%\STATE $w^i_t = p(z_t \mid s^i_t)$
%\ENDFOR
%\STATE $t = \sum_{i=1}^{N_s} w^i_t$
%\FOR {$i = 1,...,N_s$}
%\STATE $w^i_t = t^{-1} w^i_{t-1}$ (Normalisieren)
%\ENDFOR
%\STATE [$\lbrace s^{\ast j}_t, w^{j}_t, i^{j} \rbrace^{N_s}_{j=1}$] = RESAMPLE [$\lbrace s^i_t, w^{j}_t \rbrace^{N_s}_{j=1}$]
%\end{algorithmic}

%\floatname{algorithm}{Algorithmus}
%\begin{algorithm}[H]
%\caption{Vollständige Euler-Diskretisierung}\label{algo:SISPF}
%\textbf{[$\lbrace s^i_t, w^i_t \rbrace^{N_s}_{i=1}$] = PFSIR [$\lbrace s^i_{t-1}, w^i_{t-1} \rbrace^{N_s}_{i=1}$]}
%\begin{algorithmic}
%\FOR {$i = 1,...,N_s$}
%\STATE $s^i_t \sim q(x_t \mid s^i_{i-1})$
%\STATE $w^i_t = p(z_t \mid s^i_t)$
%\ENDFOR
%\STATE $t = \sum_{i=1}^{N_s} w^i_t$
%\FOR {$i = 1,...,N_s$}
%\STATE $w^i_t = t^{-1} w^i_{t-1}$ (Normalisieren)
%\ENDFOR
%\STATE [$\lbrace s^{\ast j}_t, w^{j}_t, i^{j} \rbrace^{N_s}_{j=1}$] = RESAMPLE [$\lbrace s^i_t, w^{j}_t \rbrace^{N_s}_{j=1}$]
%\end{algorithmic}


Gedanken wie der Verlauf aussehen könnte:\\
Am Anfang maximaler Schub und maximaler Auftriebsbeiwert, um so schnell wie möglich die gewünschte Reisehöhe zu erhalten. Auftriebsbeiwert kann dann reduziert werden, um die Geschwindigkeit zu erhöhen. Dies bewirkt eine höhere Geschwindigkeit, wodurch die zurückgelegte Strecke maximiert wird.


Allgemein wird für die Matlab Funktion \verb|fmincon| die Startmatrix  \verb|prob.z_0| 
\[\begin{pmatrix}
[20,9,6000,90,1259999,1.47] \\ 
\vdots \\ 
\vdots
\end{pmatrix}\]
für die Optimierung verwendet. Es sind also alle Zustandsvektoren für jeden Diskretisierungspunkt identisch.















\newpage
\section{Versuch 1}\label{kap:Versuch1}
Die Problemstellung gibt eine Endzeit beziehungsweise einen Zeitraum vor in welchem das Problem gelöst werden soll. In diesem Versuch soll diese nun verkürzt werden, um Beobachtungen zu erhalten wie sich die optimale Lösung aufgrund der Veränderungen verhält.

\subsection{Ergebnis 1}
Verwendet wird:
\begin{itemize}
\item Reduzierte Endzeit: $t_f = 1400$
\end{itemize}
Umgesetzt in Matlab mit dem Programmcode \ref{code:v1.1}
\begin{lstlisting}[style=num_octave, caption={Matlab Programmcode für Versuch 1.}, label=code:v1.1]
N = 400;                     % Anzahl an Diskretisierungen
ode_methods = ode_methods(); % Methoden zum Loesen der ODE

%% Objekt der Problemklasse erhalten
prob = MaximalRangeFlight(20,9,6000,90,1259999,1.47,N,@ode_methods.explicit_euler);

%% Loesen mit fmincon
options = optimoptions('fmincon','Display','iter','Algorithm','sqp','MaxFunctionEvaluations',4000.0e+03,'MaxIterations',4.0e+05,'ConstraintTolerance',1e-8,'StepTolerance',1e-14);

options = optimoptions('fmincon','Display','iter','Algorithm','sqp','MaxFunctionEvaluations',1.000e+03,'MaxIterations',4.0e+05);
v1_solution = fmincon(@prob.F_sol,prob.z_0,[],[],[],[],prob.lb,prob.ub,@prob.nonlcon,options);
\end{lstlisting}

Es folgt das Ergebnis
 
%\begin{figure}[H]
%\begin{center}
%\includegraphics[width=.7\textwidth]{images/01_Modellaufbau/V3E1}
%\caption{Versuch 3 - Ergebnis 1}\label{img:V1E1}
%\end{center}
%\end{figure}








\newpage
\section{Versuch 2}
Zusätzlich zur verkürzten Endzeit wie in Versuch 1 (Kapitel \ref{kap:Versuch1}) wird nun zusätzlich die Starthöhe des Flugzeuges angepasst.

\subsection{Ergebnis 1}
Verwendet werden:
\begin{itemize}
\item Reduzierte Endzeit: $t_f = 1400$
\item Angepasste Starthöhe: $h_0 = $
\end{itemize}
Umgesetzt in Matlab mit dem Programmcode \ref{code:v2.1}
\begin{lstlisting}[style=num_octave, caption={Matlab Programmcode für Versuch 2.}, label=code:v2.1]
N = 400;                     % Anzahl an Diskretisierungen
ode_methods = ode_methods(); % Methoden zum Loesen der ODE

%% Objekt der Problemklasse erhalten
prob = MaximalRangeFlight(20,9,6000,90,1259999,1.47,N,@ode_methods.explicit_euler);

%% Loesen mit fmincon
options = optimoptions('fmincon','Display','iter','Algorithm','sqp','MaxFunctionEvaluations',4000.0e+03,'MaxIterations',4.0e+05,'ConstraintTolerance',1e-8,'StepTolerance',1e-14);

options = optimoptions('fmincon','Display','iter','Algorithm','sqp','MaxFunctionEvaluations',1.000e+03,'MaxIterations',4.0e+05);
v1_solution = fmincon(@prob.F_sol,prob.z_0,[],[],[],[],prob.lb,prob.ub,@prob.nonlcon,options);
\end{lstlisting}

Es folgt das Ergebnis
 
%\begin{figure}[H]
%\begin{center}
%\includegraphics[width=.7\textwidth]{images/01_Modellaufbau/V3E1}
%\caption{Versuch 3 - Ergebnis 1}\label{img:V1E1}
%\end{center}
%\end{figure}








\newpage
\section{Versuch 3}
Zusätzlich zur verkürzten Endzeit wie in Versuch 1 (Kapitel \ref{kap:Versuch1}) wird nun zusätzlich das Gewicht des Flugzeuges angepasst.

\subsection{Ergebnis 1}
Verwendet werden:
\begin{itemize}
\item Reduzierte Endzeit: $t_f = 1400$
\item Neues Startgewicht: $m = 500000$
\end{itemize}
Umgesetzt in Matlab mit dem Programmcode \ref{code:v3.1}
\begin{lstlisting}[style=num_octave, caption={Matlab Programmcode für Versuch 3.}, label=code:v3.1]
N = 400;                     % Anzahl an Diskretisierungen
ode_methods = ode_methods(); % Methoden zum Loesen der ODE

%% Objekt der Problemklasse erhalten
prob = MaximalRangeFlight(20,9,6000,90,1259999,1.47,N,@ode_methods.explicit_euler);

%% Loesen mit fmincon
options = optimoptions('fmincon','Display','iter','Algorithm','sqp','MaxFunctionEvaluations',4000.0e+03,'MaxIterations',4.0e+05,'ConstraintTolerance',1e-8,'StepTolerance',1e-14);

options = optimoptions('fmincon','Display','iter','Algorithm','sqp','MaxFunctionEvaluations',1.000e+03,'MaxIterations',4.0e+05);
v1_solution = fmincon(@prob.F_sol,prob.z_0,[],[],[],[],prob.lb,prob.ub,@prob.nonlcon,options);
\end{lstlisting}
 
%\begin{figure}[H]
%\begin{center}
%\includegraphics[width=.7\textwidth]{images/01_Modellaufbau/V3E1}
%\caption{Versuch 3 - Ergebnis 1}\label{img:V1E1}
%\end{center}
%\end{figure}










\newpage
\section{Versuch 4}
Zusätzlich zur verkürzten Endzeit wie in Versuch 1 (Kapitel \ref{kap:Versuch1}) wird nun zusätzlich das Gewicht des Flugzeuges angepasst, sowie Box-Beschränkungen für die Zustände gesetzt. Diese zusätzlichen Box-Beschränkungen spiegeln die realen Bedingungen des Flugzeuges A380-800 der Firma Airbus wieder. Es werden folgende Schranken gesetzt:
\begin{itemize}
\item \textbf{Startgeschwindigkeit:} Gewöhnliche Flugzeuge starten von einer Startbahn und benötigen zum Starten eine Mindestgeschwindigkeit relativ zur umgebenden Luft. Diese beträgt bei Verkehrsflugzeugen zwischen 250 (69,44 m/s) und 345 km/h (95,83 m/s). (Quelle: \url{https://de.wikipedia.org/wiki/Start_(Luftfahrt)})
%
\item \textbf{Steigflug:} Aus Lärmschutzgründen und aus Gründen des „Freimachens“ von Luftraum starten Verkehrsflugzeuge von großen Flughäfen oft mit steilem Neigungswinkel, etwa um 20° (abhängig von den Vorgaben des Herstellers, der Beladung, der Windverhältnisse und des Lotsen). Der Steigwinkel wird in der Regel verringert, sobald die Flughafenumgebung verlassen wurde. (Quelle: \url{https://de.wikipedia.org/wiki/Steigflug})
%
\item \textbf{Flugzeugmassen:} Die Maximale Startmasse eines A380-800 beträgt max. 569 t. Die Leermasse 275 t. (Quelle: \url{https://de.wikipedia.org/wiki/Airbus_A380})
%
\item \textbf{Höchstgeschwindigkeit:} 961 km/h (266,94 m/s). (Quelle: \url{https://de.wikipedia.org/wiki/Airbus_A380})
%
\item \textbf{Maximale Flughöhe:} 13100 m. (Quelle: \url{https://de.wikipedia.org/wiki/Airbus_A380})
%
\item \textbf{Maximale Reichweite:} 15200000 m (Quelle: \url{https://de.wikipedia.org/wiki/Airbus_A380})
%
\item \textbf{Anstellwinkel:} -90 - 90
\end{itemize}
Damit ergibt sich das veränderte Optimalsteuerungsproblem:
\[\begin{split}
\min_{T, C_L} F(h,\gamma,x,v,T,C_L) &:= -(x(t_f) - x_0) \\\
\text{unter} \hspace{20mm} \dot{h}(t) &= v(t) \sin(\gamma(t)) \hspace{27mm} \text{(Dynamik)} \\\
\dot{\gamma}(t) &=  \dfrac{1}{mv(t)} \left( L(v(t),h(t),C_L(t)) - mg \cos(\gamma(t)) \right) \\\
\dot{x}(t) &= v(t) \cos(\gamma(t))\\\
\dot{v}(t) &= \dfrac{1}{m} \left( T(t) - D(v(t),h(t),C_L(t)) - mg \sin(\gamma(t)) \right) \\\
%
(h,\gamma,x,v)(t_0) &= (h_0,\gamma_0,x_0,v_0) \hspace{26mm} \text{(Anfangsbedingungen)}\\\
(h,\gamma,x,v)(t_f) &= (h_f,\gamma_f) \hspace{36mm} \text{(Endbedingungen)}\\\
%
q(v(t),h(t)) &\leq q_{\max} \hspace{19.5mm} \forall t \in [t_0,t_f] \hspace{5mm} \text{(Zustandsbedingungen)}\\\
h(t) &\in [0,13100] \hspace{9.5mm} \forall t \in [t_0,t_f] \hspace{5mm} \text{(Boxbeschränkungen)}\\\
\gamma(t) &\in [-90,90] \hspace{9.5mm} \forall t \in [t_0,t_f]\\\
x(t) &\in [0,15200000] \hspace{9.5mm} \forall t \in [t_0,t_f]\\\
v(t) &\in [0,266] \hspace{9.5mm} \forall t \in [t_0,t_f]\\\
T(t) &\in [T_{\min},T_{\max}] \hspace{9.5mm} \forall t \in [t_0,t_f]\\\
C_L(t) &\in [C_{L, \min},C_{L, \max}] \hspace{3mm} \forall t \in [t_0,t_f]
\end{split} \]

\subsection{Ergebnis 1}
Verwendet werden:
\begin{itemize}
\item Reduzierte Endzeit: $t_f = 1400$
\item Neues Startgewicht: $m = 500000$
\end{itemize}
Umgesetzt in Matlab mit dem Programmcode \ref{code:v4.1}
\begin{lstlisting}[style=num_octave, caption={Matlab Programmcode für Versuch 3.}, label=code:v4.1]
N = 400;                     % Anzahl an Diskretisierungen
ode_methods = ode_methods(); % Methoden zum Loesen der ODE

%% Objekt der Problemklasse erhalten
prob = MaximalRangeFlight(20,9,6000,90,1259999,1.47,N,@ode_methods.explicit_euler);

%% Loesen mit fmincon
options = optimoptions('fmincon','Display','iter','Algorithm','sqp','MaxFunctionEvaluations',4000.0e+03,'MaxIterations',4.0e+05,'ConstraintTolerance',1e-8,'StepTolerance',1e-14);

options = optimoptions('fmincon','Display','iter','Algorithm','sqp','MaxFunctionEvaluations',1.000e+03,'MaxIterations',4.0e+05);
v1_solution = fmincon(@prob.F_sol,prob.z_0,[],[],[],[],prob.lb,prob.ub,@prob.nonlcon,options);
\end{lstlisting}
 
%\begin{figure}[H]
%\begin{center}
%\includegraphics[width=.7\textwidth]{images/01_Modellaufbau/V3E1}
%\caption{Versuch 3 - Ergebnis 1}\label{img:V1E1}
%\end{center}
%\end{figure}











\newpage
\section{Versuch 5}
Testet doch mal, ob ihr bei vorgegebener Steuerung einen deutlichen Unterschied zwischen expl. und impl. Verfahren feststellt. Falls nicht ist das schon mal ein deutliches Indiz dafür, dass ihr euch den Rechenaufwand eines impl. Verfahrens auch in der Optimierung sparen könnt. Ich denke letztere ist eher die Herausforderung bei eurem Projekt, da es wahrscheinlich viele lokale Minima gibt. Deswegen wäre es wahrscheinlich auch sinnvoll unterschiedliche Algorithmen (z.B. SQP) auszuprobieren. Dafür müsst ihr nur die entsprechende Option für fmincon setzen und müsst auch nicht unbedingt den Algorithmus im Detail verstehen. Sowas könnt ihr dann natürlich auch gerne in die Präsentation/Ausarbeitung aufnehmen.







\newpage
\section{Versuch 6}
Von Explizit auf implizites Verfahren

Ich denke auch nicht unbedingt, dass die Stabilität des ODE-Solvers oder die Steifheit der ODE das (Haupt-)Problem ist. Wahrscheinlich könnt ihr ein explizites Verfahren verwenden.











%\newpage
%\section{Überprüfung der optimalen Steuerung}
%Die Steuerung $T(t)$ (Schub) geht linear in die Hamilton-Funktion ein. Um die Hamilton-Funktion zu minimieren gilt für diese Bang-Bang Verhalten.
%
%Die Steuerung $C_L(t)$ geht nichtlinear in die Hamilton-Funktion ein.
%
%
%
%
%
%
%\section{Überprüfung der Hinreichenden Optimalitätsbedingungen}
%
%
%
%
%
%
%
%\section{Aufstellen des Randwertproblems}
%Muss also ein Mehrpunktrandwertproblem sein ???
%
%
%Mit den Optimalitätsbedingungen des Minimumprinzips von Pontryagin lässt sich das Steuerungsproblem in ein Randwertproblem überführen, welches aus den beiden Funktionen $r(t,Z(t))$ und $r_0(Z(t_0),Z(t_f)) = 0$ besteht. Für $r(t,Z(t))$ ergibt sich \[r(t,Z(t)) = \dot{Z}(t) = \begin{pmatrix}
%\dot{h}(t),\dot{\gamma}(t),\dot{x}(t),\dot{v}(t),\dot{\lambda}_1(t),\dot{\lambda}_2(t),\dot{\lambda}_3(t),\dot{\lambda}_4(t)
%\end{pmatrix}^T\] Für $r_0(Z(t_0),Z(t_f)) = 0$ müssen zunächst die Endbedingungen mit gebildet aus 
%\[\begin{split}
%X_i(t_f) &= c_i \hspace{25mm} (i=1,...,r) \\\
%\lambda_i(t_f) &= \lambda_0 g_{X_i}(X^{\ast}(t_f)) \hspace{5mm} (i=r+1,...,n)
%\end{split}\] gebildet werden. Es ergibt sich dann \[r_0(Z(t_0),Z(t_f)) = \begin{pmatrix}
%h(t_0) - h_0 \\ 
%\gamma(t_0) - \gamma_0 \\
%x(t_0) - x_0 \\ 
%v(t_0) - v_0 \\ 
%h(t_f) - h_f \\ 
%\gamma(t_f) - \gamma_f \\
%\lambda_3(t_f) + \lambda_0 \\ 
%\lambda_4(t_f) - 0
%\end{pmatrix}\]

\chapter{Lösen des Optimalsteuerungsproblem mit indirekten Lösungsverfahren}

\section{Aufstellen des Randwertproblems}
Mit den Optimalitätsbedingungen des Minimumprinzips von Pontryagin lässt sich das Steuerungsproblem in ein Randwertproblem überführen, welches aus den beiden Funktionen $g(t,Z(t),U(t))$ und $r_0(Z(t_0),Z(t_f)) = 0$ besteht. Für $g(t,Z(t),U(t))$ ergibt sich \[\dot{Z}(t) = r(t,Z(t)) = \begin{pmatrix}
\dot{h}(t),\dot{\gamma}(t),\dot{x}(t),\dot{v}(t),\dot{\lambda}_1(t),\dot{\lambda}_2(t),\dot{\lambda}_3(t),\dot{\lambda}_4(t)
\end{pmatrix}^T\]
und für die Ableitung
\[\dfrac{\partial g(t,Z(t),U(t))}{\partial Z} = J_g(t,Z(t),U(t)) = \begin{pmatrix}
0 & J_g^{(1,2)} & 0 & J_g^{(1,4)} & 0 & 0 & 0 & 0 \\ 
J_g^{(2,1)} & J_g^{(2,2)} & 0 & J_g^{(2,4)} & 0 & 0 & 0 & 0 \\ 
0 & J_g^{(3,2)} & 0 & J_g^{(3,4)} & 0 & 0 & 0 & 0 \\ 
J_g^{(4,1)} & J_g^{(4,2)} & 0 & J_g^{(4,4)} & 0 & 0 & 0 & 0 \\
J_g^{(5,1)} & 0 & 0 & J_g^{(5,4)} & 0 & J_g^{(5,6)} & 0 & J_g^{(5,8)} \\
0 & J_g^{(6,2)} & 0 & J_g^{(6,4)} & J_g^{(6,5)} & J_g^{(6,6)} & J_g^{(6,7)} & J_g^{(6,8)} \\
0 & 0 & 0 & 0 & 0 & 0 & 0 & 0 \\
J_g^{(8,1)} & J_g^{(8,2)} & 0 & J_g^{(8,4)} & J_g^{(8,5)} & J_g^{(8,6)} & J_g^{(8,7)} & J_g^{(8,8)}
\end{pmatrix}\]
mit:
\begin{align*}
J_g^{(1,2)} &= v(t) \cos(\gamma(t)) \\
J_g^{(1,4)} &= \sin(\gamma(t))
\end{align*}
\begin{align*}
J_g^{(2,1)} &= - \dfrac{F \alpha \beta e^{-\beta h(t)} v(t) C_L(t)}{2m} \\
J_g^{(2,2)} &= \dfrac{g \sin(\gamma(t))}{v(t)} \\
J_g^{(2,4)} &= \dfrac{F \alpha e^{-\beta h(t)} C_L(t)}{2m} + \dfrac{g \cos(\gamma(t))}{v^2(t)}
\end{align*}
\begin{align*}
J_g^{(3,2)} &= - v(t) \sin(\gamma(t)) \\
J_g^{(3,4)} &= \cos(\gamma(t))
\end{align*}
\begin{align*}
J_g^{(4,1)} &= \dfrac{(C_{D_0} + k C_L^2(t)) F \alpha \beta e^{-\beta h(t)} v^2(t)}{2m} \\
J_g^{(4,2)} &= - g \cos(\gamma(t)) \\
J_g^{(4,4)} &= -\dfrac{(C_{D_0} + k C_L^2(t)) F \alpha e^{-\beta h(t)} v(t)}{m} 
\end{align*}
\begin{align*}
J_g^{(5,1)} &= \dfrac{\alpha \beta^2 F e^{-\beta h(t)} C_L(t) v(t) \lambda_2(t)}{2m} - \dfrac{(C_{D_0}+k C_L^2(t)) \alpha \beta^2 F e^{-\beta h(t)} v^2(t) \lambda_4(t)}{2m} \\
J_g^{(5,4)} &= - \dfrac{\alpha \beta F e^{-\beta h(t)} C_L(t) \lambda_2(t)}{2m} + \dfrac{(C_{D_0}+k C_L^2(t)) \alpha \beta F e^{-\beta h(t)} v(t) \lambda_4(t)}{m} \\
J_g^{(5,6)} &= - \dfrac{\alpha \beta F e^{-\beta h(t)} C_L(t) v(t)}{2m}\\
J_g^{(5,8)} &= \dfrac{(C_{D_0}+k C_L^2(t)) \alpha \beta F e^{-\beta h(t)} v^2(t)}{2m}
\end{align*}
\begin{align*}
J_g^{(6,2)} &= -\sin(\gamma(t)) v(t) \lambda_1(t) + \dfrac{g \cos(\gamma(t)) \lambda_2(t)}{v(t)} - \cos(\gamma(t)) v(t) \lambda_3(t) + \sin(\gamma(t)) g \lambda_4(t) \\
J_g^{(6,4)} &= \cos(\gamma(t)) \lambda_1(t) - \dfrac{g \sin(\gamma(t)) \lambda_2(t)}{v^2(t)} - \sin(\gamma(t)) \lambda_3(t) \\
J_g^{(6,5)} &= \cos(\gamma(t)) v(t) \\
J_g^{(6,6)} &= \dfrac{g \sin(\gamma(t))}{v(t)} \\
J_g^{(6,7)} &= - \sin(\gamma(t)) v(t) \\
J_g^{(6,8)} &= - \cos(\gamma(t)) g
\end{align*}
\begin{align*}
J_g^{(8,1)} &= -\dfrac{F \alpha \beta e^{-\beta h(t)} C_L(t) \lambda_2(t)}{2m}  + \dfrac{(C_{D_0} + k C_L^2(t)) F \alpha \beta e^{-\beta h(t)} v(t) \lambda_4(t)}{m} \\
J_g^{(8,2)} &= \cos(\gamma(t)) \lambda_1(t) - \dfrac{g \sin(\gamma(t)) \lambda_2(t)}{v^2(t)} - \sin(\gamma(t)) \lambda_3(t) \\
J_g^{(8,4)} &= - \dfrac{2 g \cos(\gamma(t)) \lambda_2(t)}{v^3(t)} - \dfrac{(C_{D_0} + k C_L^2(t)) F \alpha e^{-\beta h(t)} \lambda_4(t)}{m} \\
J_g^{(8,5)} &= \sin(\gamma(t)) \\
J_g^{(8,6)} &= \dfrac{F \alpha e^{-\beta h(t)} C_L(t)}{2m} + \dfrac{g \cos(\gamma(t))}{v^2(t)} \\
J_g^{(8,7)} &= \cos(\gamma(t)) \\
J_g^{(8,8)} &= - \dfrac{(C_{D_0} + k C_L^2(t)) F \alpha e^{-\beta h(t)} v(t)}{m} 
\end{align*}

%\[\begin{split}
%J_g^{(1,2)} &= v(t) \cos(\gamma(t)) \\\
%J_g^{(1,4)} &= \sin(\gamma(t)) \\\
%J_g^{(2,1)} &= - \dfrac{F \alpha \beta e^{-\beta h(t)} v(t) C_L(t)}{2m} \\\
%J_g^{(2,2)} &= \dfrac{g \sin(\gamma(t))}{v(t)} \\\
%J_g^{(2,4)} &= \dfrac{F \alpha e^{-\beta h(t)} C_L(t)}{2m} + \dfrac{g \cos(\gamma(t))}{v^2(t)} \\\
%J_g^{(3,2)} &= - v(t) \sin(\gamma(t)) \\\
%J_g^{(3,4)} &= \cos(\gamma(t)) \\\
%J_g^{(4,1)} &= \dfrac{(C_{D_0} + k C_L^2(t)) F \alpha \beta e^{-\beta h(t)} v^2(t)}{2m} \\\
%J_g^{(4,2)} &= - g \cos(\gamma(t)) \\\
%J_g^{(4,4)} &= -\dfrac{(C_{D_0} + k C_L^2(t)) F \alpha e^{-\beta h(t)} v(t)}{m} \\\
%%
%J_g^{(5,1)} &= \dfrac{\alpha \beta^2 F e^{-\beta h(t)} C_L(t) v(t) \lambda_2(t)}{2m} - \dfrac{(C_{D_0}+k C_L^2(t)) \alpha \beta^2 F e^{-\beta h(t)} v^2(t) \lambda_4(t)}{2m} \\\
%J_g^{(5,4)} &= - \dfrac{\alpha \beta F e^{-\beta h(t)} C_L(t) \lambda_2(t)}{2m} + \dfrac{(C_{D_0}+k C_L^2(t)) \alpha \beta F e^{-\beta h(t)} v(t) \lambda_4(t)}{m} \\\
%J_g^{(5,6)} &= - \dfrac{\alpha \beta F e^{-\beta h(t)} C_L(t) v(t)}{2m}\\\
%J_g^{(5,8)} &= \dfrac{(C_{D_0}+k C_L^2(t)) \alpha \beta F e^{-\beta h(t)} v^2(t)}{2m} \\\
%%
%J_g^{(6,2)} &= -\sin(\gamma(t)) v(t) \lambda_1(t) + \dfrac{g \cos(\gamma(t)) \lambda_2(t)}{v(t)} - \cos(\gamma(t)) v(t) \lambda_3(t) + \sin(\gamma(t)) g(t) \lambda_4(t) \\\
%J_g^{(6,4)} &= \cos(\gamma(t)) \lambda_1(t) - \dfrac{g \sin(\gamma(t)) \lambda_2(t)}{v^2(t)} - \sin(\gamma(t)) \lambda_3(t) \\\
%J_g^{(6,5)} &= \cos(\gamma(t)) v(t) \\\
%J_g^{(6,6)} &= \dfrac{g \sin(\gamma(t))}{v(t)} \\\
%J_g^{(6,7)} &= - \sin(\gamma(t)) v(t) \\\
%J_g^{(6,8)} &= - \cos(\gamma(t)) g(t) \\\
%%
%J_g^{(8,1)} &= -\dfrac{F \alpha \beta e^{-\beta h(t)} C_L(t) \lambda_2(t)}{2m}  + \dfrac{(C_{D_0} + k C_L^2(t)) F \alpha \beta e^{-\beta h(t)} v(t) \lambda_4(t)}{m} \\\
%J_g^{(8,2)} &= \cos(\gamma(t)) \lambda_1(t) - \dfrac{g \sin(\gamma(t)) \lambda_2(t)}{v^2(t)} - \sin(\gamma(t)) \lambda_3(t) \\\
%J_g^{(8,4)} &= - \dfrac{2 g \cos(\gamma(t)) \lambda_2(t)}{v^3(t)} - \dfrac{(C_{D_0} + k C_L^2(t)) F \alpha e^{-\beta h(t)} \lambda_4(t)}{m} \\\
%J_g^{(8,5)} &= \sin(\gamma(t)) \\\
%J_g^{(8,6)} &= \dfrac{F \alpha e^{-\beta h(t)} C_L(t)}{2m} + \dfrac{g \cos(\gamma(t))}{v^2(t)} \\\
%J_g^{(8,7)} &= \cos(\gamma(t)) \\\
%J_g^{(8,8)} &= - \dfrac{(C_{D_0} + k C_L^2(t)) F \alpha e^{-\beta h(t)} v(t)}{m} 
%\end{split}\]



Für $r_0(Z(t_0),Z(t_f)) = 0$ müssen zunächst die Endbedingungen mit gebildet aus 
\[\begin{split}
X_i(t_f) &= c_i \hspace{25mm} (i=1,...,r) \\\
\lambda_i(t_f) &= \lambda_0 g_{X_i}(X^{\ast}(t_f)) \hspace{5mm} (i=r+1,...,n)
\end{split}\] gebildet werden. Es ergibt sich dann
\[r_0(Z(t_0),Z(t_f)) = \begin{pmatrix}
h(t_0) - h_0 \\ 
\gamma(t_0) - \gamma_0 \\
x(t_0) - x_0 \\ 
v(t_0) - v_0 \\ 
h(t_f) - h_f \\ 
\gamma(t_f) - \gamma_f \\
\lambda_3(t_f) + \lambda_0 \\ 
\lambda_4(t_f) - 0
\end{pmatrix}\]
mit 
\[\dfrac{d r(Z(t_0),Z(t_f))}{d Z(t_0)} = \begin{pmatrix}
1 & 0 & 0 & 0 & 0 & 0 & 0 & 0 \\ 
0 & 1 & 0 & 0 & 0 & 0 & 0 & 0 \\ 
0 & 0 & 1 & 0 & 0 & 0 & 0 & 0 \\ 
0 & 0 & 0 & 1 & 0 & 0 & 0 & 0 \\
0 & 0 & 0 & 0 & 0 & 0 & 0 & 0 \\
0 & 0 & 0 & 0 & 0 & 0 & 0 & 0 \\
0 & 0 & 0 & 0 & 0 & 0 & 0 & 0 \\
0 & 0 & 0 & 0 & 0 & 0 & 0 & 0
\end{pmatrix}\]
\[\dfrac{d r(Z(t_0),Z(t_f))}{d Z(t_f)} = \begin{pmatrix}
0 & 0 & 0 & 0 & 0 & 0 & 0 & 0 \\ 
0 & 0 & 0 & 0 & 0 & 0 & 0 & 0 \\ 
0 & 0 & 0 & 0 & 0 & 0 & 0 & 0 \\ 
0 & 0 & 0 & 0 & 0 & 0 & 0 & 0 \\
1 & 0 & 0 & 0 & 0 & 0 & 0 & 0 \\
0 & 1 & 0 & 0 & 0 & 0 & 0 & 0 \\
0 & 0 & 0 & 0 & 0 & 0 & 1 & 0 \\
0 & 0 & 0 & 0 & 0 & 0 & 0 & 1
\end{pmatrix}\]






















\section{Algorithmus Einfachschiessverfahren}
Für eine gegebene Startschätzung $\eta$ des Anfangswerts $y(a)$ besitze das Anfangswertproblem
\[y'(t) = g(t, y(t)) \ \ \ \ y(a) = \eta\]
die Lösung $y(t;\eta)$ auf $[a,b]$. Damit $y(t;\eta)$ auch die Randbedingung erfüllt, muss 
\begin{equation}\label{func:SchiessF}
F(\eta) := r(y(a;\eta), y(b;\eta)) = r(\eta, y(b;\eta)) = 0_{n_y}
\end{equation}
gelten. Gleichung \ref{func:SchiessF} ist also ein \textbf{nichtlineares Gleichungssystem} für die Funktion $F$. Anwendung des Newtonverfahrens führt auf das sogenannte Einfachschießverfahren:

\begin{definition}[Algorithmus Einfachschießverfahren]\label{algo:EinfSchiess}
Initialisierung: Wähle Startschätzung $\eta^{[0]} \in \R^{n_y}$ und setze $i = 0$:
\begin{enumerate}
\item Löse das Anfangswertproblem \[y'(t) = g(t, y(t)) \ \ \ \ y(a) = \eta^{[i]} \ \ \ \ (a \leq t \leq b)\] zur Berechnung von $F(\eta^{[i]})$ und berechne die Jacobimatrix \[F'(\eta^{[i]}) = r'_{y_a} (\eta^{[i]}, y(b;\eta^{[i]})) + r'_{y_b}(\eta^{[i]}, y(b;\eta^{[i]})) \cdot S(b)\] wobei $S$ Lösung der Sensitivitäts-Differentialgleichung \[S'(t) = g'_y(t, y(t;\eta^{[i]})) \cdot S(t) \ \ \ \ S(a) = I_{n_y \times n_y}\] ist.
%
\item Ist $F(\eta^{[i]}) = 0_{n_y}$ (oder ist ein anderes Abbruchkriterium) erfüllt, \textbf{STOP!}
%
\item Berechne die Newton-Richtung $d^{[i]}$ als Lösung des linearen Gleichungssystems \[F'(\eta^{[i]})d = -F(\eta^{[i]})\]
%
\item Setze $\eta^{[i+1]} = \eta^{[i]} + d^{[i]}$ und $i=i+1$ und gehe zu 1.).
\end{enumerate}
\end{definition}

Die Ableitung $F'(\eta^{[i]})$ in Schritt 2.) des Einfachschießverfahrens \ref{algo:EinfSchiess} kann alternativ
durch \textbf{finite Differenzen} approximiert werden:
\[\dfrac{\partial}{\partial \eta_j} F(\eta) \approx \dfrac{F(\eta + h e_j) - F(\eta)}{h} \ \ \ \ (j=1,...,n_y)\]
mit $e_j = j$-ter Einheitsvektor. Dieser Ansatz erfordert das Lösen von $n_y$ Anfangswertproblemen!


\chapter{Zusammenfassung und Ausblick}

Hinreichende Optimalitätsbedingungen



%------------------------------------------------------------------------------
% Anhang
%------------------------------------------------------------------------------
\clearpage
\appendix
\pagenumbering{roman}
\stepcounter{SeitenzahlSpeicher}
\setcounter{page}{\theSeitenzahlSpeicher}
% hier Anhänge einbinden
\chapter{Quelltexte}

In diesem Anhang sind einige wichtige Quelltexte aufgeführt.






%------------------------------------------------------------------------------
% Literatur-, Abbildungs-, Tabellen-, Formel- und Abkuerzungsverzeichnis
%------------------------------------------------------------------------------
\backmatter

% Literaturverzeichnis
\clearpage
\printbibliography

% Auflistung aller verwendeter Progammcodes
\clearpage
\lstlistoflistings
\addcontentsline{toc}{chapter}{Programmcodes Verzeichnis}


% Abbildungsverzeichnis
%\clearpage
%\listoffigures
%\addcontentsline{toc}{chapter}{\listfigurename}

% Tabellenverzeichnis
%\clearpage
%\listoftables
%\addcontentsline{toc}{chapter}{\listtablename}

% Formelverzeichnis
%\clearpage
%\input{chapters/Formelverzeichnis}
%%\addcontentsline{toc}{chapter}{Formelverzeichnis}

% Abkuerzungsverzeichnis
%\clearpage
%\chapter{Abkürzungsverzeichnis}

\begin{tabular}{ll}
SF & Sensorfusion\\
SHA & Servohydraulische Achsen\\
\end{tabular} 
%%\addcontentsline{toc}{chapter}{Abkürzungsverzeichnis}

%------------------------------------------------------------------------------
% Eigenständigkeitserklärung
%------------------------------------------------------------------------------
\clearpage
\thispagestyle{empty}
\chapter{Eigenständigkeitserklärung}

\begin{tabbing}
\hspace{30mm}\=\hspace{60mm}\=\kill
Namen: \> \Heiko \> (Matrikelnummer: \Hmatnr) \\ 
  \>  \Philipp \> (Matrikelnummer: \Pmatnr) \\ 
  \>  \Felix \> (Matrikelnummer: \Fmatnr)
\end{tabbing} 

\minisec{Erklärung}

Wir erklären, dass wir die Arbeit selbständig verfasst und keine anderen als die angegebenen Quellen und Hilfsmittel verwendet haben.\vspace{2cm}

Ulm, den \dotfill

\hspace{10cm} {\footnotesize \Heiko}\\[2em]


Ulm, den \dotfill 

\hspace{10cm} {\footnotesize \Philipp} \\[2em]


Ulm, den \dotfill

\hspace{10cm} {\footnotesize \Felix}
\end{document}
%==============================================================================