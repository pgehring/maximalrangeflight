%==============================================================================
%------------------------------------------------------------------------------
% Notizen, Bemerkungen, Quellen
%------------------------------------------------------------------------------
% Bemerkungen:
% 	- Kompilieren mit PDFLaTeX
%	- Römische Seitenzahlen im Anhang manuell anpassen !!!

% Notizen:
%	- 

% Quellen:
%	-
%==============================================================================






%==============================================================================
%------------------------------------------------------------------------------
% Dokument Einstellungen
%------------------------------------------------------------------------------
\NeedsTeXFormat{LaTeX2e}
\documentclass[
a4paper,
10pt, %12pt,
headsepline,           % Linie zw. Kopfzeile und Text
oneside,               % einseitig
numbers=noenddot,	   % keine Punkte nach den letzten Ziffern in Überschriften
bibliography=totoc,    % LV in IV
%DIV=15,                % Satzspiegel auf 15er Raster, schmalere Ränder   
%BCOR15mm              % Bindekorrektur
%,draft
]{scrbook}
\KOMAoptions{DIV=last} % Neuberechnung Satzspiegel nach Laden von Paket helvet
%==============================================================================






%==============================================================================
%------------------------------------------------------------------------------
% Seitenlayout
%------------------------------------------------------------------------------
\pagestyle{headings}
\usepackage{blindtext}

% für Texte in deutscher Sprache
\usepackage[ngerman]{babel}
\usepackage[utf8]{inputenc}
\usepackage[T1]{fontenc}

% Helvetica als Standard-Dokumentschrift
\usepackage[scaled]{helvet}
\renewcommand{\familydefault}{\sfdefault} 

\usepackage{geometry}
\usepackage{layout}

%------------------------------------------------------------------------------
% Grafiken
%------------------------------------------------------------------------------
\usepackage{graphicx}
\usepackage{float} % um figure/Bild/Abbildung an bestimmte Stelle mit [H]
\usepackage{framed}

%------------------------------------------------------------------------------
% Literaturverzeichnis mit BibLKaTeX
%------------------------------------------------------------------------------
\usepackage[babel,german=quotes]{csquotes}
%\usepackage[backend=bibtex8]{biblatex}
\usepackage[backend=biber,bibencoding=utf8,style=numeric]{biblatex}
%\bibliography{bibliography}
\addbibresource{bibliography.bib}

%------------------------------------------------------------------------------
% Für Tabellen mit fester Gesamtbreite und variabler Spaltenbreite
%------------------------------------------------------------------------------
\usepackage{tabularx} 

%------------------------------------------------------------------------------
% Besondere Schriftauszeichnungen
%------------------------------------------------------------------------------
\usepackage{url}              % \url{http://...} in Schreibmaschinenschrift
\usepackage{color}            % zum Setzen farbigen Textes

%------------------------------------------------------------------------------
% Pakete für Mathe-Umgebungen, -Symbole und Neudefinition von Kommandos
%------------------------------------------------------------------------------
\usepackage{amssymb, amsmath, amsthm} % Pakete für Mathe-Umgebungen und -Symbole

% Mengen Buchstaben:
\newcommand{\E}{\mathbb{E}}
\newcommand{\Q}{\mathbb{Q}}
\newcommand{\R}{\mathbb{R}}
\newcommand{\N}{\mathbb{N}}

%------------------------------------------------------------------------------
% Mathematische Umgebungen (Satz, Definition, Beweis ...)
%------------------------------------------------------------------------------
\newtheorem{myDef}{Definition}
\newtheorem{myTheorem}{Theorem}
\newtheorem{mySatz}{Satz}
\newtheorem{myBeweis}{Beweis}
\newtheorem*{myBemerkung}{Bemerkung}

%------------------------------------------------------------------------------
% Formatierung
%------------------------------------------------------------------------------
\usepackage{enumitem}
\usepackage{setspace}         % Paket für div. Abstände, z.B. ZA
%\onehalfspacing              % nur dann, wenn gefordert; ist sehr groß!!
\setlength{\parindent}{0pt}   % kein linker Einzug der ersten Absatzzeile
\setlength{\parskip}{1.4ex plus 0.35ex minus 0.3ex} % Absatzabstand, leicht variabel

% Tiefe, bis zu der Überschriften in das Inhaltsverzeichnis kommen
\setcounter{tocdepth}{3}      % ist Standard

%------------------------------------------------------------------------------
% Quellcode
%------------------------------------------------------------------------------
%https://en.wikibooks.org/wiki/LaTeX/Source_Code_Listings
% Beispiele für Quellcode
\usepackage{listings}
\lstset{language=Java,
  showstringspaces=false,
  frame=single,
  numbers=left,
  basicstyle=\ttfamily,
  numberstyle=\tiny}

\usepackage{subfloat}
\usepackage{subcaption}
\usepackage{mathrsfs}

%verhindet, dass sich Fußnote auf zwei Seiten erstreckt
\interfootnotelinepenalty=10000 

%------------------------------------------------------------------------------
% Persönliche Daten
%------------------------------------------------------------------------------
\newcommand{\Heiko}{Heiko Karus}
\newcommand{\Philipp}{Philipp Gehring}
\newcommand{\Felix}{Felix Götz}

\newcommand{\Hemail}{Heiko.Karus@uni-ulm.de}
\newcommand{\Pemail}{Philipp.Gehring@uni-ulm.de}
\newcommand{\Femail}{Felix.Goetz@uni-ulm.de}

\newcommand{\Hmatnr}{1072378}
\newcommand{\Pmatnr}{1104946}
\newcommand{\Fmatnr}{1063352}

\newcommand{\titel}{Maximal Range Flight}
\newcommand{\jahr}{2021}
\newcommand{\gutachterA}{Prof. Dr. rer. nat. Dirk Lebiedz}
\newcommand{\gutachterB}{M.Sc. Jörn Dietrich}

\newcommand{\fakultaet}{Mathematik und Wirtschaftswissenschaften}
\newcommand{\institut}{Institut für numerische Mathematik}

%------------------------------------------------------------------------------
% Informationen, die LaTeX in die PDF-Datei schreibt
%------------------------------------------------------------------------------
\pdfinfo{
  /Author (Karus, Gehring, Goetz)
 % /Title (\titel)
  /Producer     (pdfeTex 3.14159-1.30.6-2.2)
  /Keywords ()
}

%\usepackage{hyperref}
\usepackage[colorlinks=true, urlcolor=blue, linkcolor=green]{hyperref}
\hypersetup{
pdftitle=\titel,
pdfauthor=\Heiko \Philipp \Felix,
pdfsubject={Projekt Numerik 4},
pdfproducer={pdfeTex 3.14159-1.30.6-2.2},
colorlinks=false,
pdfborder=0 0 0	% keine Box um die Links!
}

%------------------------------------------------------------------------------
% Weiter hinzugefügte Pakete
%------------------------------------------------------------------------------
\usepackage{xcolor, colortbl}

\counterwithout{footnote}{chapter}
\usepackage{multirow}
\usepackage{longtable}
%==============================================================================






%==============================================================================
%------------------------------------------------------------------------------
% Beginn Dokument
%------------------------------------------------------------------------------
% Trennungsregeln
\hyphenation{Sil-ben-trenn-ung}

\begin{document}

% Seitenlayout
\newgeometry{left=2.5cm, right=2.5cm, top=3cm, bottom=3cm}

\frontmatter

%------------------------------------------------------------------------------
% Titelseite
%------------------------------------------------------------------------------
\thispagestyle{empty}
\begin{addmargin*}[4mm]{-10mm}

\includegraphics[height=1.8cm]{images/00_Sonstiges/unilogo_bild}
\hfill
\includegraphics[height=1.8cm]{images/00_Sonstiges/unilogo_wort}\\[2em]

%\includegraphics[height=1.8cm]{images/00_Sonstiges/THU}
%\hfill
%\includegraphics[height=1.8cm]{images/00_Sonstiges/unilogo_bild}
%\includegraphics[height=1.8cm]{images/00_Sonstiges/unilogo_wort}\\[2em]

%\includegraphics[height=1.4cm]{images/unilogo_wort}
%\hfill
%\includegraphics[height=1.4cm]{images/THU_word}\\[1em]
%\includegraphics[height=1.4cm]{images/unilogo_bild}
%\hfill
%\includegraphics[height=1.4cm]{images/THU_logo}\\[1em]

{\footnotesize
%{\bfseries Universität Ulm} \textbar ~89069 Ulm \textbar ~Germany
\hspace*{130mm}\parbox[t]{35mm}{
\bfseries Fakultät für\\
\fakultaet\\
\mdseries \institut
}\\[2cm]

\parbox{140mm}{\bfseries \LARGE \titel}\\[2.5em]
{\footnotesize Projekt Numerik 4}\\[2em]

{\footnotesize \bfseries Vorgelegt von:}\\
{\footnotesize \Heiko \\ E-Mail: \Hemail \\ Matrikel-Nr.: \Hmatnr}\\ \\%[2em]
{\footnotesize \Philipp \\ E-Mail: \Pemail \\ Matrikel-Nr.: \Pmatnr}\\ \\%[2em]
{\footnotesize \Felix \\ E-Mail: \Femail \\ Matrikel-Nr.: \Fmatnr}\\ \\[2em]

{\footnotesize \bfseries Gutachter:}\\                     
{\footnotesize \gutachterA}\\ \\%[2em]
{\footnotesize \gutachterB}\\[2em]

{\footnotesize \jahr}
}
\end{addmargin*}

%------------------------------------------------------------------------------
% Impressum
%------------------------------------------------------------------------------
\clearpage
\thispagestyle{empty}
{ \small
  \flushleft
  Fassung \today \\\vfill
  \copyright~\jahr~\Heiko,~\Philipp,~\Felix\\[0.5em]
% Wenn Sie Ihre Arbeit unter einer freien Lizenz bereitstellen möchten, können Sie die nächste Zeile in Ihren Code aufnehmen. Bitte beachten Sie, dass Sie hierfür an allen Inhalten, inklusive enthaltener Abbildungen, die notwendigen Rechte benötigen! Beim Veröffentlichungsexemplar Ihrer Dissertation achten Sie bitte darauf, dass der Lizenztext nicht den Angaben in den Metadaten der genutzten Publikationsplattform widerspricht. Nähere Information zu den Creative Commons Lizenzen erhalten Sie hier: https://creativecommons.org/licenses/
%This work is licensed under the Creative Commons Attribution 4.0 International (CC BY 4.0) License. To view a copy of this license, visit \href{https://creativecommons.org/licenses/by/4.0/}{https://creativecommons.org/licenses/by/4.0/} or send a letter to Creative Commons, 543 Howard Street, 5th Floor, San Francisco, California, 94105, USA. \\
  Satz: PDF-\LaTeXe
}

% ab hier Zeilenabstand etwas größer 
\setstretch{1.2}

%------------------------------------------------------------------------------
% Vorwort, Kurzfassung und Abstract
%------------------------------------------------------------------------------
% Vorwort
%\clearpage
%\input{chapters/Vorwort}

%Kurzfassung
%\clearpage
%\input{chapters/Kurzfassung}

% Abstract
%\clearpage
%\input{chapters/Abstract}

%------------------------------------------------------------------------------
% Inhaltsverzeichnis
%------------------------------------------------------------------------------
\tableofcontents
\addcontentsline{toc}{chapter}{Inhaltsverzeichnis}

%------------------------------------------------------------------------------
% Hauptteil
%------------------------------------------------------------------------------
\mainmatter
\chapter{Motivation und Ziel der Projektarbeit}
Die Berechnung von Flugtrajektorien zur Reichweitenmaximierung hat eine lange Historie, siehe \cite{Burrows1982, Murrieta2016, Schaback2017, Pierson1989}.  Dabei wurden verschiedene mathematische Techniken angewandt, die von Parametrisierungen von Trajektorien \cite{Burrows1982}, Energiebetrachtungen \cite{Calise1977}, über Mehrzieloptimierung mit verschiedenen Kostenfunktionen \cite{Gardi2016} bis hin zur Anwendung der Optimalsteuerungstheorie \cite{Javier2016}  reichen. 

Eine einfache Lösung für den Reichweiten maximierten Flug ist für den Horizontalflug bekannt, siehe beispielsweise \cite{Peckham1974}. Diese folgt aus der Maximierung des Verhältnisses $\sqrt{C_{L}/C_{D}}$ der Auftriebs- und Widerstandskoeffizienten. Daraus resultiert eine Geschwindigkeit, die um den Faktor $\sqrt{4/3}$ größer ist als die Geschwindigkeit zur Maximierung des Verhältnisses von Auftriebs- zu Widerstandskoeffizienten $C_{L}/C_{D}$ \cite{Schaback2017}. 

Diese Projektarbeit bietet eine Erweiterung auf den Steigflug und konzentriert sich dabei auf numerische Standardmethoden, die für das Lösen von Optimalsteuerungsproblemen verwendet werden. Modelliert wird dabei ein zweidimensionaler Flug eines Flugzeugs in der $x$-$h$-Ebene (Abbildung \ref{img:Flugzeug}), bei dem der Auftriebsbeiwert $C_L(t)$ und der Schub $T(t)$ gesteuert werden kann.

\begin{figure}[H]
    \begin{center}
        \includegraphics[width=\textwidth]{images/01_Modellaufbau/Flugzeug.pdf}
        \MyCaption{Freikörperdiagramm eines Flugzeuges in der $x$-$h$-Ebene}{mit angreifenden Kräften $L$ (Auftriebskraft), $D$ (Luftwiederstand), $W$ (Erdanziehungskraft) und $T$ (Schub) am Schwerpunkt $S$. Des Weiteren beschreibt $v$ die Geschwindigkeit und $\gamma$ den Anstellwinkel.}\label{img:Flugzeug}
    \end{center}
\end{figure}

Ziel ist es, das Flugzeug von einer gegebenen Anfangsposition so zu steuern, dass eine vorgegebene Reisehöhe $h_f = 10668\ m$ und ein Anstellwinkel $\gamma_f = 0\ ^{\circ}$ in einer Flugzeit von $1800 \ s$ erreicht  wird, wobei die zurückgelegte Strecke maximal wird. Dabei dürfen die Steuerbeschränkungen für den Schub und Auftriebsbeiwert nicht verletzt und ein maximaler Staudruck nicht überschritten werden. 

Beginnend mit der Herleitung der Gleichungen für den quasi-statischen Flug in \autoref{cha:optim}, folgt die Formulierung des Optimalsteuerungsproblems für direkte (\autoref{cha:direct}) und indirekte (\autoref{cha:indirect}) Lösungsverfahren. Anschließend folgt die numerische Umsetzung und Lösung des Optimalsteuerungsproblems (Kapitel \ref{kap:TUNU}), sowie die Diskussion der Ergebnisse (Kapitel \ref{kap:LSG}) .

Alle Modellrechnungen wurden für den Airbus A380-800 \cite{A380Tech} durchgeführt. \textit{MATLAB} wurde für alle numerischen Berechnungen verwendet. Hierzu wurde teils auf enthaltene Funktionen zurückgegriffen und teils eigene Funktionen zur Lösung von Optimalsteuerungsproblemen implementiert.





\chapter{Optimalsteuerungsproblem} \label{cha:optim}

Im folgenden wird ein Differentialgleichungssystem für den quasi-statischen Flug hergeleitet. Modelliert wird das Flugzeug A380-800 der Firma Airbus bei einem Steigflug. Es wird sowohl die Dynamik in vertikaler ($y$-Achse) als auch in horizontaler ($h$-Achse) Richtung berücksichtigt. Dabei seien
\begin{itemize}
    \item $x(t)$: $x$-Koordinate des Massenschwerpunktes $S$
    \item $h(t)$: $h$-Koordinate des Massenschwerpunktes $S$
    \item $v(t)$: Geschwindigkeit
    \item $\gamma(t)$: Anstellwinkel
    \item $T(t)$: Schub (Steuerung)
    \item $C_L(t)$: Auftriebsbeiwert (Steuerung)
\end{itemize}
Um die Kräfte welche auf das Flugzeug einwirken, berechnen zu können, werden folgende Hilfsgrößen benötigt:
\begin{itemize}
    \item Luftwiderstandsbeiwert: \[C_D(C_L(t)) := C_{D_0} + k \cdot C^2_L(t) \ \ \ \ \text{mit} \ \ \ \ k = \dfrac{1}{\pi \cdot e \cdot AR}\] wobei $C_{D0}$ der Nullluftwiderstandsbeiwert, $e$ die Oswaldfaktor und $AR$ die Streckung (engl. \textit{aspect ratio}) bezeichnet. Dabei sind $C_{D0}$ und $k$ abhängig von der Machzahl, jedoch wird dieser Effekt zur Vereinfachung ignoriert. 
    
    \item Luftdichte: \[\rho(h(t)) := \alpha \cdot e^{-\beta \cdot h(t)}\]  wobei für die Berechnung der höhenabhängigen Luftdichte ein einfaches Exponentialmodell verwendet wird.%https://wind-data.ch/tools/luftdichte.php

    \item Staudruck: \[q(v(t), h(t)) := \dfrac{\rho(h(t)) \cdot v^2(t)}{2} \]
\end{itemize}

Neben Gewichtskraft \(W\), Auftrieb \(L\) und Luftwiderstand \(D\) wird auch der Schub \(T\) als Kraft betrachtet. Daneben wird der Einfluss von Klappen, Spoiler und ausgefahrenen Fahrwerken vernachlässigt. Die nachfolgenden Gleichungen beziehen sich auf kurze Zeitintervalle, in denen die Geschwindigkeit $v$ und der Anstellwinkel $\gamma$ als konstant angesehen werden. Wie jedoch gezeigt wurde, führen sie zu nützlichen Gleichungen, die die langfristigen Änderungen von $v$ und $\gamma$ beschreiben. Die am Flugzeug angreifenden Kräfte lassen sich wie folgt berechnen:
\begin{itemize}
    \item Auftriebskraft: \[L(v(t), h(t), C_L(t)) := F \cdot C_L(t) \cdot q(v(t), h(t))\] wobei $F$ die wirksame Fläche, d.h. die von der Luft angeströmte Fläche, ist.
    %
    \item Luftwiderstand: \[D(v(t), h(t), C_L(t)) := F \cdot C_D(C_L(t)) \cdot q(v(t), h(t))\]
    \item Gewichtskraft: \[W = m \cdot g\] wobei $m$ die Masse des Flugzeugs und $g$ die Gravitationskonstante darstellt.
\end{itemize}

Mit dem 2. Newtons'schen Axiom $F = m \cdot a$ lässt sich die Differentialgleichung
\[F := m \cdot a \Rightarrow \dot{v}(t) = a(t) = \dfrac{F(t)}{m} = \dfrac{T(t) - D(v(t),h(t),C_L(t)) - W \sin(\gamma(t))}{m}\]
für die Geschwindigkeit $v$ aufstellen.
Mit der Gleichung für die Zentripetalkraft $F_{ZP} = \dfrac{m v^2}{r}$ lässt sich die Differentialgleichung
\[F_{ZP} := \dfrac{m v^2}{r} \Rightarrow \dot{\gamma}(t) = \dfrac{v(t)}{r} = \dfrac{F_{ZP}(t)}{m v(t)} = \dfrac{L(v(t),h(t),C_L(t)) - W \cos(\gamma(t))}{m v(t)}\]
für den Ansstellwinkel $\gamma$ aufstellen.
Die Differentialgleichungen für die $h(t)$ und $x(t)$ lassen mittels der Geschwindigkeit und des Anstellwinkels bestimmen.
\[\begin{split}
    \dot{x}(t) &= v(t) \cos(\gamma(t))\\\
    \dot{h}(t) &= v(t) \sin(\gamma(t))
\end{split} \]

Es ergibt sich somit das Optimalsteuerungsproblem (Problem \ref{prob:MaxRF}) mit den Funktionen $g : \R^{n_X} \to \R$, $f_0 : \R^{n_X} \times \R^{n_U} \to \R$, $f : \R^{n_X} \times \R^{n_U} \to \R^{n_X}$ und $U : [t_0,t_f] \to \R^m$ für $0 \leq (n_{\psi} = 2) \leq (n_X = 4)$ und $n_U = 2$.

\begin{problem}[Optimalsteuerungsproblem - Maximal-Range-Flight]\label{prob:MaxRF}
    Für das Optimalsteuerungsproblem ergibt sich mit dem Zustandsvektor
    \[X(t) = (h(t),\gamma(t),x(t),v(t))^T\]
    und der Steuerfunktion
    \[U(t) = (T(t),C_L(t))^T\]
    das Problem:
    \begin{align*}
        \min_{U} F(X,U) &:= g(X(t_f)) + \int_{t_0}^{t_f} f_0(X(t),U(t)) dt = -(x(t_f) - x_0) & & \\\
        \text{unter} \hspace{20mm} \dot{X}(t) &= f(X(t),U(t)) =     
         \begin{pmatrix}
         \dot{h}(t)  \\ 
         \dot{\gamma}(t)  \\ 
         \dot{x}(t)  \\ 
         \dot{v}(t)   \\ 
	 \end{pmatrix} 
        = 
        \begin{pmatrix}
            v(t) \sin(\gamma(t)) \\ 
            \dfrac{L(v(t),h(t),C_L(t)) - W \cos(\gamma(t))}{mv(t)} \\ 
            v(t) \cos(\gamma(t)) \\ 
            \dfrac{T(t) - D(v(t),h(t),C_L(t)) - W \sin(\gamma(t))}{m}
        \end{pmatrix} & & \\\
        (h,\gamma,x,v)(t_0) &= (h_0,\gamma_0,x_0,v_0) & & \\\
        (h,\gamma)(t_f) &= (h_f,\gamma_f) & & \\\
        q(v(t),h(t)) &\leq q_{\max}  \forall t \in [t_0,t_f]\\\
        U(t) &= (T(t),C_L(t))^T \in \mathcal{U} = \left[ 
        \begin{matrix}
            [T_{\min},T_{\max}] \\ 
            [C_{L, \min},C_{L, \max}]
        \end{matrix} 
        \right]  \forall t \in [t_0,t_f]
    \end{align*}
\end{problem}

Des Weiteren sei $\psi : \R^{n_X} \to \R^{n_{\psi}}$ eine $C^1$-Funktion
\[\psi(X(t_f)) = 
\begin{pmatrix}
    h(t_f) - h_f \\ 
    \gamma(t_f) - \gamma_f
\end{pmatrix} = 0_{n_{\psi}}\]
Das Optimalsteuerungsproblem (Problem \ref{prob:MaxRF}) stellt also ein autonomes Mayer-Problem der Form 
\begin{equation} \label{equ:mayer_problem}
    \begin{aligned}
        \min F(X,U) :&= g(X(t_f))  \\
        \text{unter}  \hspace{10mm} \dot{X}(t) &= f(X(t),U(t)) & & \forall t \in [t_0,t_f] \\
        %
        X(t_0) &= X_0 = (h_0,\gamma_0,x_0,v_0)^T & & \\
        \psi(X(t_f)) &= 0_{n_{\psi}} & & \\
        %
        q(X(t)) &\leq q_{\max} & & \forall t \in [t_0,t_f] \\
        U(t) &= (T(t),C_L(t))^T \in \mathcal{U}  & & \forall t \in [t_0,t_f] 
    \end{aligned}
\end{equation}
da, mit der zusätzlichen Beschränkung des Staudrucks $q(v(t),h(t))$ und der konkreten Funktion $f$
\begin{equation} \label{equ:state_space}
    f(X(t),U(t)) = \dot{X}(t) = \begin{pmatrix}
        v(t) \sin(\gamma(t)) \\ 
        \dfrac{F \alpha e^{-\beta h(t)} v(t) C_L(t)}{2m} - \dfrac{g \cos(\gamma(t))}{v(t)} \\ 
        v(t) \cos(\gamma(t)) \\ 
        \dfrac{T(t)}{m} - \dfrac{(C_{D_0} + k C_L^2(t)) F \alpha e^{-\beta h(t)} v^2(t)}{2m} - g \sin(\gamma(t))
    \end{pmatrix}
\end{equation}
Für das Modell werden die Parameter aus Tabelle \ref{tab:ProblemPara} in Anhang \ref{Anhang:ModellPara} verwendet.

\chapter{Technische Umsetzung}

Zur Umsetzung der Codes wurde das Programm Matlab verwendet.

\section{Matlab Funktionen}

\subsection{Numerische Optimierung von beschränkten Problemen}
Mit der Matlab Funktion \verb|fmincon| lassen sich beschränkte nichtlineare Optimierungen durchführen.


\subsection{Numerisches Lösen von gewöhnlichen Differentialgleichungen}
Zum lösen von gewöhnlichen steifen oder nichtsteifen Differentialgleichungen stehen einem die folgenden Matlab Funktionen zur Verfügung.

Vergleich der ODE Solver auf unser Beispiel







\section{Klassenstruktur des Problems}

UML


\chapter{Zusammenfassung und Ausblick}



%------------------------------------------------------------------------------
% Anhang
%------------------------------------------------------------------------------
\backmatter
\pagenumbering{roman}
\setcounter{page}{5} % ACHTUNG MANUELL ANPASSEN !!!!!!!!!!!!!!!!!!!!!!!!!!!!!!!

\appendix
% hier Anhänge einbinden
%\input{Anhang.tex}
%\chapter{Quelltexte}

In diesem Anhang sind einige wichtige Quelltexte aufgeführt.

\begin{lstlisting}
public class Hello {
    public static void main(String[] args) {
        System.out.println("Hello World");
    }
}
\end{lstlisting}


\backmatter

%------------------------------------------------------------------------------
% Literatur-, Abbildungs-, Tabellen-, Formel- und Abkuerzungsverzeichnis
%------------------------------------------------------------------------------
% Literaturverzeichnis
\clearpage
\printbibliography

% Abbildungsverzeichnis
%\clearpage
%\listoffigures
%\addcontentsline{toc}{chapter}{\listfigurename}

% Tabellenverzeichnis
%\clearpage
%\listoftables
%\addcontentsline{toc}{chapter}{\listtablename}

% Formelverzeichnis
%\clearpage
%\chapter{Formelverzeichnis}


%%\addcontentsline{toc}{chapter}{Formelverzeichnis}

% Abkuerzungsverzeichnis
%\clearpage
%\chapter{Abkürzungsverzeichnis}


%%\addcontentsline{toc}{chapter}{Abkürzungsverzeichnis}

%------------------------------------------------------------------------------
% Eigenständigkeitserklärung
%------------------------------------------------------------------------------
\clearpage
\thispagestyle{empty}
\chapter{Eigenständigkeitserklärung}

\begin{tabbing}
\hspace{30mm}\=\hspace{60mm}\=\kill
Namen: \> \Heiko \> (Matrikelnummer: \Hmatnr) \\ 
  \>  \Philipp \> (Matrikelnummer: \Pmatnr) \\ 
  \>  \Felix \> (Matrikelnummer: \Fmatnr)
\end{tabbing} 

\minisec{Erklärung}

Wir erklären, dass wir die Arbeit selbständig verfasst und keine anderen als die angegebenen Quellen und Hilfsmittel verwendet haben.\vspace{2cm}

Ulm, den \dotfill

\hspace{10cm} {\footnotesize \Heiko}\\[2em]


Ulm, den \dotfill 

\hspace{10cm} {\footnotesize \Philipp} \\[2em]


Ulm, den \dotfill

\hspace{10cm} {\footnotesize \Felix}
\end{document}
%==============================================================================