\chapter{Lösen des Optimalsteuerungsproblem mit indirekten Lösungsverfahren}

Nächster Schritt also:
\begin{itemize}
\item Problem auf ein Mehrfachschiesverfahren anwenden?
\end{itemize}


















%\newpage
%\section{Überprüfung der optimalen Steuerung}
%Die Steuerung $T(t)$ (Schub) geht linear in die Hamilton-Funktion ein. Um die Hamilton-Funktion zu minimieren gilt für diese Bang-Bang Verhalten.
%
%Die Steuerung $C_L(t)$ geht nichtlinear in die Hamilton-Funktion ein.
%
%
%
%
%
%
%\section{Überprüfung der Hinreichenden Optimalitätsbedingungen}
%
%
%
%
%
%
%
%\section{Aufstellen des Randwertproblems}
%Muss also ein Mehrpunktrandwertproblem sein ???
%
%
%Mit den Optimalitätsbedingungen des Minimumprinzips von Pontryagin lässt sich das Steuerungsproblem in ein Randwertproblem überführen, welches aus den beiden Funktionen $r(t,Z(t))$ und $r_0(Z(t_0),Z(t_f)) = 0$ besteht. Für $r(t,Z(t))$ ergibt sich \[r(t,Z(t)) = \dot{Z}(t) = \begin{pmatrix}
%\dot{h}(t),\dot{\gamma}(t),\dot{x}(t),\dot{v}(t),\dot{\lambda}_1(t),\dot{\lambda}_2(t),\dot{\lambda}_3(t),\dot{\lambda}_4(t)
%\end{pmatrix}^T\] Für $r_0(Z(t_0),Z(t_f)) = 0$ müssen zunächst die Endbedingungen mit gebildet aus 
%\[\begin{split}
%X_i(t_f) &= c_i \hspace{25mm} (i=1,...,r) \\\
%\lambda_i(t_f) &= \lambda_0 g_{X_i}(X^{\ast}(t_f)) \hspace{5mm} (i=r+1,...,n)
%\end{split}\] gebildet werden. Es ergibt sich dann \[r_0(Z(t_0),Z(t_f)) = \begin{pmatrix}
%h(t_0) - h_0 \\ 
%\gamma(t_0) - \gamma_0 \\
%x(t_0) - x_0 \\ 
%v(t_0) - v_0 \\ 
%h(t_f) - h_f \\ 
%\gamma(t_f) - \gamma_f \\
%\lambda_3(t_f) + \lambda_0 \\ 
%\lambda_4(t_f) - 0
%\end{pmatrix}\]
