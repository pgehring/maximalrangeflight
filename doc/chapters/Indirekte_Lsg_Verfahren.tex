\chapter{Formulierung des Optimalsteuerungsproblem für indirekte Lösungsverfahren} \label{cha:indirect}

Bei den indirekten Lösungsverfahren werden notwendige Bedingungen für eine optimale Lösung des Steuerungsproblems aufgestellt. Diese notwendigen Bedingungen für das Optimalsteuerungsproblem überführen das Problem in ein  Randwertproblem, welches mit verschiedenen Methoden, wie zum Beispiel dem Einfachschieß- oder Mehrfachschießverfahren \cite{Deuflhard2013,MultipleShoot1962} ausgewertet werden kann.

\section{Notwendige Optimalitätsbedinungen (Minimumpinzip von Pontryagin)}\label{kap:NOpt}
%Aus dem Modellaufbau ergeben sich die Definitionen 
%\begin{align*}
%    g(X(t_f)) &=  -(x(t_f) - x_0) & & (g : \R^n \to \R) \\
%    f_0(X(t),U(t)) &\equiv 0 & &(f_0 : \R^n \times \R^m \to \R^n)\\
%    f(X(t),U(t)) &= \dot{X}(t) = (\dot{h}(t),\dot{\gamma}(t),\dot{x}(t),\dot{v}(t))^T & &(f : \R^n \times \R^m \to \R^n) \\
%    U(t) &= (T(t),C_L(t))^T \in \mathcal{U} \left( = 
%    \left[
%        \begin{matrix}
%            [T_{\min},T_{\max}] \\ 
%            [C_{L, \min},C_{L, \max}]
%        \end{matrix} 
%    \right]\right) \subset \R^2 & &(U : [t_0,t_f] \to \R^m)
%\end{align*}
%Des Weiteren sei $\psi : \R^4 \to \R^2$ eine $C^1$-Funktion (mit $0 \leq (r = 2) \leq (n = 4)$) \[\psi(X(t_f)) = 
%\begin{pmatrix}
%    h(t_f) - h_f \\ 
%    \gamma(t_f) - \gamma_f
%\end{pmatrix} = 0\]
%
%So ergibt sich das autonome Mayer-Problem mit 
%\begin{equation} \label{equ:mayer_problem}
%    \begin{aligned}
%        \min F(X,U) &:= g(X(t_f)) =  -(x(t_f) - x_0) & & \\
%        \text{unter}  \hspace{10mm} \dot{X}(t) &= f(X(t),U(t)) = (\dot{h}(t),\dot{\gamma}(t),\dot{x}(t),\dot{v}(t))^T & & \forall t \in [t_0,t_f] \\
%        %
%        X(t_0) &= X_0 = (h_0,\gamma_0,x_0,v_0)^T & & \\
%        \psi(X(t_f)) &= 0 & & \\
%        %
%        q(X(t)) &\leq q_{\max} & & \forall t \in [t_0,t_f] \\
%        U(t) &= (T(t),C_L(t))^T \in \mathcal{U} \subset \R^2  & & \forall t \in [t_0,t_f] 
%    \end{aligned}
%\end{equation}
%
%Für die Funktion $f$ lässt sich konkret
%\begin{equation} \label{equ:state_space}
%    f(X(t),U(t)) = \dot{X}(t) = \begin{pmatrix}
%        v(t) \sin(\gamma(t)) \\ 
%        \dfrac{F \alpha e^{-\beta h(t)} v(t) C_L(t)}{2m} - \dfrac{g \cos(\gamma(t))}{v(t)} \\ 
%        v(t) \cos(\gamma(t)) \\ 
%        \dfrac{T(t)}{m} - \dfrac{(C_{D_0} + k C_L^2(t)) F \alpha e^{-\beta h(t)} v^2(t)}{2m} - g \sin(\gamma(t))
%    \end{pmatrix}
%\end{equation}

Für die notwendigen Optimalitätsbedingungen muss zunächst die Hamilton-Funktion aufgestellt werden. Diese ergibt sich mit $\lambda_0 \in \R$ ($\lambda_0 \geq 0$) und $\lambda : [t_0,t_f] \to \R^{n_X}$ zu
\begin{equation} \label{equ:hamilt_func}
    \begin{split}
        H(X(t),U(t),\lambda(t)) &= \lambda_0 f_0(X(t),U(t)) + \lambda(t)^T f(X(t),U(t)) \\\
        &= \lambda(t)^T f(X(t),U(t)) \\\
        &= \lambda_1(t) \dot{h}(t) + \lambda_2(t) \dot{\gamma}(t) + \lambda_3(t) \dot{x}(t) + \lambda_4(t) \dot{v}(t) \\\
        &= \sin(\gamma(t)) v(t) \lambda_1 \\\
        &\hspace{7mm} + \dfrac{F \alpha e^{-\beta h(t)} C_L(t) v(t) \lambda_2(t)}{2m} - \dfrac{g \cos(\gamma(t)) \lambda_2(t)}{v(t)} \\\
        &\hspace{7mm} + \cos(\gamma(t)) v(t) \lambda_3(t) \\\
        &\hspace{7mm} + \dfrac{T(t) \lambda_4(t)}{m} - \dfrac{(C_{D_0} + k C_L^2(t)) F \alpha e^{-\beta h(t)} v^2(t) \lambda_4(t)}{2m} \\\
        &\hspace{7mm} - g \sin(\gamma(t)) \lambda_4(t)
    \end{split}
\end{equation}
Mit der Hamilton-Funktion (Gleichung \eqref{equ:hamilt_func}) lassen sich nun die notwendigen Optimalitätsbedingungen des Minimumprinzips von Pontryagin \cite{OptiControl2018} untersuchen:
\begin{enumerate}
    \item \textbf{Minimumbedingung:} Es gilt an allen Stetigkeitsstellen $t \in [t_0,t_f]$ von $u^{\ast}(t)$ die Minimumbedingung \[H(X^{\ast}(t),U^{\ast}(t),\lambda(t)) = \min_{U(t) \in \mathcal{U}} H(X^{\ast}(t),U(t),\lambda(t))\] Leitet man nun nach der Steuerfunktion $U(t)$ ab, so ergibt sich für den unbeschränkten Fall der Steuerung die Minimumbedingung
    \[\dfrac{\partial}{\partial U} H(X^{\ast}(t),U^{\ast}(t),\lambda(t)) = \begin{pmatrix}
    \dfrac{\lambda_4(t)}{m} \\ 
    - \dfrac{k F \alpha e^{-\beta h(t)} v^2(t) \lambda_4(t) C_L(t)}{m} + \dfrac{F \alpha e^{-\beta h(t)} v(t) \lambda_2(t)}{2m}
    \end{pmatrix}^T \stackrel{!}{=} 0\]
    und 
    \[\dfrac{\partial^2}{\partial U^2} H(X^{\ast}(t),U^{\ast}(t),\lambda(t)) = \begin{pmatrix}
    0 & - \dfrac{k F \alpha e^{-\beta h(t)} v^2(t) \lambda_4(t)}{m} 
    \end{pmatrix} \stackrel{!}{\geq} 0\] wobei \[\sigma(x(t),\lambda(t)) := H_u(X^{\ast}(t),U^{\ast}(t),\lambda(t))\] Schaltfunktion genannt wird. Für die Betrachtung mit den Beschränkungen der Steuerfunktion ergibt sich für den Schub
\begin{equation}\label{func:SynSchub}
T(t) = \left\lbrace \begin{array}{ll}
T_{\min} & ,\text{falls } \lambda_4 > 0  \\ 
\text{beliebig} \in [T_{\min},T_{\max}] & ,\text{falls } \lambda_4 = 0  \\ 
T_{\max} & ,\text{falls } \lambda_4 < 0
\end{array} \right.
\end{equation}
Für die Bestimmung der Steuerfunktion des Auftriebsbeiwerts lässt sich die Hamilton-Funktion verkürzt mit den Termen $K_1(t)$ und $K_2(t)$ schreiben.
    \[\begin{split}
        \tilde{H}(X^{\ast}(t),C_L(t),\lambda(t)) &= \dfrac{F \alpha e^{-\beta h^{\ast}(t)} v^{\ast}(t) \lambda_2(t)}{2m} \cdot C_L(t) - \dfrac{k F \alpha e^{-\beta h^{\ast}(t)}  v^{\ast 2}(t) \lambda_4(t)}{2m} \cdot C_L^2(t) \\\
        &= K_1(t) C_L(t) - K_2(t) C_L^2(t)
    \end{split}\]
Soll nun $\tilde{H}(X^{\ast}(t),C_L(t),\lambda(t))$ minimal werden, so müssen die folgenden Bedingungen überprüft werden:
    \begin{enumerate}
        \item[1.)] $\mathbf{K_1(t) < 0 \wedge K_2(t)} < 0$:
        \begin{enumerate}
            \item[1.1.)] Für die Ableitungen von $\tilde{H}$ ergeben sich
            \[\begin{split}
            \frac{d \tilde{H}}{d C_L} = - 2 K_2(t) C_L(t) + K_1(t) &\stackrel{!}{=} 0 \\\
            \frac{d^2 \tilde{H}}{d C_L^2} = - 2 K_2(t) &\geq 0 \Rightarrow \text{Minimum}
            \end{split}\]
            So folgt für die Steuerfunktion $C_L(t) = \dfrac{K_1(t)}{2 K_2(t)} > 0$.
            %
            \item[1.2.)] Falls $C_L(t) = \dfrac{K_1(t)}{2 K_2(t)} > C_{L, \max}$ so gilt  $C_L(t) = C_{L, \max}$.
        \end{enumerate}
        %
        \item[2.)] $\mathbf{K_1(t) = 0 \wedge K_2(t)} < 0$: Für die Ableitungen von $\tilde{H}$ ergeben sich
        \[\begin{split}
        \frac{d \tilde{H}}{d C_L} = - 2 K_2(t) C_L(t) &\stackrel{!}{=} 0 \\\
        \frac{d^2 \tilde{H}}{d C_L^2} = - 2 K_2(t) &\geq 0 \Rightarrow \text{Minimum}
        \end{split}\]
        So folgt für die Steuerfunktion $C_L(t) = 0 = C_{L, \min}$.
        %
        \item[3.)] $\mathbf{K_1(t) > 0 \wedge K_2(t)} < 0$: Für die Ableitungen von $\tilde{H}$ ergeben sich
        \[\begin{split}
        \frac{d \tilde{H}}{d C_L} = - 2 K_2(t) C_L(t) + K_1(t) &\stackrel{!}{=} 0 \\\
        \frac{d^2 \tilde{H}}{d C_L^2} = - 2 K_2(t) &\geq 0 \Rightarrow \text{Minimum}
        \end{split}\]
        So folgt für die Steuerfunktion $C_L(t) = \dfrac{K_1(t)}{2 K_2(t)} < 0$. Es gilt also $C_L(t) = C_{L, \min}$.
        %
        \item[4.)] $\mathbf{K_1(t) < 0 \wedge K_2(t)} = 0$: Für die Ableitungen von $\tilde{H}$ ergeben sich
        \[\begin{split}
        \frac{d \tilde{H}}{d C_L} = K_1(t) &\stackrel{!}{=} 0 \\\
        \frac{d^2 \tilde{H}}{d C_L^2} = 0 &\geq 0 \Rightarrow \text{Minimum}
        \end{split}\]
        So folgt für die Steuerfunktion $C_L(t) = C_{L, \max}$.
        %
        \item[5.)] $\mathbf{K_1(t) = 0 \wedge K_2(t)} = 0$: Daraus folgt für die Steuerfunktion $C_L(t) = \text{beliebig} \in [C_{L, \min},C_{L, \max}]$.
        %
        \item[6.)] $\mathbf{K_1(t) > 0 \wedge K_2(t)} = 0$: Für die Ableitungen von $\tilde{H}$ ergeben sich
        \[\begin{split}
        \frac{d \tilde{H}}{d C_L} = K_1(t) &\stackrel{!}{=} 0 \\\
        \frac{d^2 \tilde{H}}{d C_L^2} = 0 &\geq 0 \Rightarrow \text{Minimum}
        \end{split}\]
        So folgt für die Steuerfunktion $C_L(t) = C_{L, \min}$.
        %
        \item[7.)] $\mathbf{K_1(t) < 0 \wedge K_2(t)} > 0$: Für die Ableitungen von $\tilde{H}$ ergeben sich
        \[\begin{split}
        \frac{d \tilde{H}}{d C_L} = - 2 K_2(t) C_L(t) + K_1(t) &\stackrel{!}{=} 0 \\\
        \frac{d^2 \tilde{H}}{d C_L^2} = - 2 K_2(t) &\leq 0 \Rightarrow \text{Maximum}
        \end{split}\]
        So folgt für die Steuerfunktion $C_L(t) = \dfrac{K_1(t)}{2 K_2(t)} < 0$. Es gilt also $C_L(t) = C_{L, \max}$.
        %
        \item[8.)] $\mathbf{K_1(t) = 0 \wedge K_2(t)} > 0$: Für die Ableitungen von $\tilde{H}$ ergeben sich
        \[\begin{split}
        \frac{d \tilde{H}}{d C_L} = - 2 K_2(t) C_L(t) &\stackrel{!}{=} 0 \\\
        \frac{d^2 \tilde{H}}{d C_L^2} = - 2 K_2(t) &\leq 0 \Rightarrow \text{Maximum}
        \end{split}\]
        So folgt für die Steuerfunktion $C_L(t) = 0$. Es gilt also $C_L(t) = C_{L, \max}$.
        %
\item[9.)] $\mathbf{K_1(t) > 0 \wedge K_2(t)} > 0$: Für die Ableitungen von $\tilde{H}$ ergeben sich
\[\begin{split}
\frac{d \tilde{H}}{d C_L} = - 2 K_2(t) C_L(t) + K_1(t) &\stackrel{!}{=} 0 \\\
\frac{d^2 \tilde{H}}{d C_L^2} = - 2 K_2(t) &\leq 0 \Rightarrow \text{Maximum}
\end{split}\]
So folgt für die Steuerfunktion $C_L(t) = \dfrac{K_1(t)}{2 K_2(t)} > 0$. Für das Minimum bestimme man nun die Nullstellen von $\tilde{H}$:
\begin{enumerate}
\item[9.1.)] Für die Bestimmung der Nullstellen von $\tilde{H}$ gilt
\[\begin{split}
\tilde{H} \stackrel{!}{=} 0 &= - K_2(t) C_L^2(t) + K_1(t)C_L(t) \\\
&= (- K_2(t) C_L(t) + K_1(t)) C_L(t) \\\
0 &= K_1(t) - K_2(t)C_L(t) \\\
C_L(t) &= \dfrac{K_1(t)}{K_2(t)}
\end{split}\]
So gilt $\tilde{H} = 0$ bei $C_L(t) = 0$ und $C_L(t) = \dfrac{K_1(t)}{K_2(t)}$ Falls $C_L(t) = \dfrac{K_1(t)}{K_2(t)} < C_{L,\max}$, so gilt also $C_L(t) = C_{L, \max}$.
%
\item[9.2.)] Falls $C_L(t) = \dfrac{K_1(t)}{K_2(t)} > C_{L,\max}$, so gilt $C_L(t) = C_{L,\min}$.
\item[9.3.)] Falls $C_L(t) = \dfrac{K_1(t)}{K_2(t)} = C_{L,\max}$, so gilt $C_L(t) = C_{L,\max} \vee C_{L,\min}$.
\end{enumerate}
\end{enumerate}
Damit folgt für die Steuerfunktion des Auftriebsbeiwerts
\begin{equation}\label{func:SynAuftrieb}
C_L(t) = \left\lbrace 
\begin{array}{ll}
C_{L, \min} & ,\text{falls Bedingung } 2,3,6 \text{ oder } 9.2 \text{ gilt} \\ 
\text{beliebig} \in [C_{L, \min},C_{L, \max}] & ,\text{falls Bedingung } 5 \text{ gilt} \\ 
\dfrac{K_1(t)}{2 K_2(t)} & ,\text{falls Bedingung } 1.1 \text{ gilt} \\ 
C_{L, \max} & ,\text{falls Bedingung } 1.2,4,7,8 \text{ oder } 9.1 \text{ gilt} \\
C_L(t) = C_{L,\max} \vee C_{L,\min} & ,\text{falls Bedingung } 9.3 \text{ gilt} \\ 
\end{array} 
\right.
\end{equation}
%
\item \textbf{Adjungierte DGL:} Leitet man nach dem Zustandsvektor $X(t)$ ab, also $H_{X}(X^{\ast}(t),U^{\ast}(t),\lambda(t))$ so erhält man
\begin{equation}\label{func:AjgDGL}
\begin{split}
\dfrac{\partial}{\partial h} H &= - \dfrac{\alpha \beta F e^{-\beta h(t)} C_L(t) v(t) \lambda_2(t)}{2m} + \dfrac{(C_{D_0}+k C_L^2(t)) \alpha \beta F e^{-\beta h(t)} v^2(t) \lambda_4(t)}{2m} \\\
\dfrac{\partial}{\partial \gamma} H &= \cos(\gamma(t)) v(t) \lambda_1(t) + \dfrac{g \sin(\gamma(t)) \lambda_2(t)}{v(t)} - \sin(\gamma(t)) v(t) \lambda_3(t) - \cos(\gamma(t)) g \lambda_4(t) \\\
\dfrac{\partial}{\partial x} H &= 0 \\\
\dfrac{\partial}{\partial v} H &= \sin(\gamma(t)) \lambda_1(t) + \left( \dfrac{F \alpha e^{-\beta h(t)} C_L(t)}{2m} + \dfrac{g \cos(\gamma(t))}{v^2(t)} \right) \lambda_2(t) \\\
&\hspace{7mm} + \cos(\gamma(t)) \lambda_3(t) - \dfrac{(C_{D_0} + k C_L^2(t)) F \alpha e^{-\beta h(t)} v(t) \lambda_4(t)}{m}
\end{split}
\end{equation}
        wobei gilt 
        \[\dot{\lambda}(t)^T = - \dfrac{\partial}{\partial X} H = -H_{X} = \left( -\dfrac{\partial}{\partial h} H, -\dfrac{\partial}{\partial \gamma} H, -\dfrac{\partial}{\partial x} H, -\dfrac{\partial}{\partial v} H \right)\]
    %
    \item \textbf{Transversalitätsbedingung:} Im Endzeitpunkt $t_f$ gilt die Transversalitätsbedingung mit dem Vektor $\nu \in \R^{n_{\psi}}$ mit $(\lambda_0,\lambda(t),\nu) \neq 0$ für alle $t \in [t_0,t_f]$
        \[\begin{split}
            \lambda(t_f)^T &= \lambda_0 g_X(X^{\ast}(t_f)) + \nu^T \psi_X(X^{\ast}(t_f)) \\\
            &= \lambda_0 
            \begin{pmatrix}
            0 & 0 & -1 & 0
            \end{pmatrix}  
            + \nu^T 
            \begin{pmatrix}
            1 & 0 & 0 & 0 \\
            0 & 1 & 0 & 0 
            \end{pmatrix}  \\\
            &= \begin{pmatrix}
            \nu_1 & \nu_2 & -\lambda_0 & 0 
            \end{pmatrix}
        \end{split}\]
%(ist $\lambda_{0} =1$ da $x(t_f)$ frei ist und $\psi_X(X^{\ast}(t_f))$ hat vollen Zeilenrang???) 
    %
    \item \textbf{Konstanz:} Für autonome Systeme gilt 
\begin{equation}\label{func:HKonstanz}
H(X^{\ast}(t),U^{\ast}(t), \lambda(t)) = const \ \in [t_0,t_f]
\end{equation}
\end{enumerate}













%\section{Zweipunkt-Randwertproblem und Beschränkungsuntersuchungen}
%Um das Optimalsteuerungsproblem \ref{prob:MaxRF} mit indirekten Lösungsverfahren lösen zu können, muss dieses zunächst auf ein Zweipunkt-Randwertproblem umgestellt werden.
%
%\begin{problem}[Zweipunkt-Randwertproblem]\label{prob:ZweiRand}
%Seien $G : [t_0,t_f] \times \R^{n_Z} \to \R^{n_Z}$ und $R : \R^{n_Z} \times \R^{n_Z} \to \R^{n_Z}$ gegeben. Gesucht ist eine Lösung $Z$ des Randwertproblems
%\begin{align}
%\dot{Z}(t) &= G(t,Z(t),U(t)) \\
%R(Z(t_0),Z(t_f)) &= 0_{n_Z}
%\end{align}
%im Intervall $[t_0,t_f]$.
%\end{problem}
%
%Jedoch behandelt das Zweipunkt-Randwertproblem keine Bedingungen wie die Beschränkung des Staudrucks $q(v(t),h(t)) \leq q_{\max}$, wie in Problem \ref{prob:MaxRF} gefordert. Untersuchungen der Ergebnisse aus den Versuchen \ref{kap:Versuch11}, \ref{kap:Versuch31} und  \ref{kap:Versuch41} haben gezeigt, dass diese zu keinem Zeitpunkt den maximalen Wert $q_{\max}$ der Beschränkung erreicht haben (Abbildung \ref{img:test_1_1_staudruck}). Aus Vereinfachungsgründen, wird diese Beschränkung deshalb in diesem Kapitel nicht weiter berücksichtigt.
%
%\begin{figure}[H]
%\begin{center}
%\includegraphics[width=\textwidth]{../code/direct_sol/results/test_1_1_staudruck}
%\caption{Untersuchung, ob bei Versuch \ref{kap:Versuch11} der maximale Staudruck überschritten bzw. angenähert wird.}\label{img:test_1_1_staudruck}
%\end{center}
%\end{figure}

















\section{Aufstellen des Zweipunkt-Randwertproblems}\label{kap:ZPRand}
Um das Optimalsteuerungsproblem \ref{prob:MaxRF} mit indirekten Lösungsverfahren lösen zu können, muss dieses zunächst auf ein Zweipunkt-Randwertproblem \cite{Gerdts2011} umgestellt werden.

\begin{problem}[Zweipunkt-Randwertproblem]\label{prob:ZweiRand}
    Seien $G : [t_0,t_f] \times \R^{n_Z} \to \R^{n_Z}$ und $R : \R^{n_Z} \times \R^{n_Z} \to \R^{n_Z}$ gegeben. Gesucht ist eine Lösung $Z$ des Randwertproblems
    \begin{align}
        \dot{Z}(t) &= G(t,Z(t),U(t)) \\
        R(Z(t_0),Z(t_f)) &= 0_{n_Z}
    \end{align}
    im Intervall $[t_0,t_f]$.
\end{problem}

Mit den Optimalitätsbedingungen des Minimumprinzips von Pontryagin lässt sich das Steuerungsproblem in ein Randwertproblem überführen, welches aus den beiden Funktionen $G(t,Z(t),U(t))$ und $R(Z(t_0),Z(t_f)) = 0_{n_Z}$ für $n_Z = 8$ besteht:
\begin{itemize}
\item Für $G(t,Z(t),U(t))$ ergibt sich
\begin{equation} 
    \dot{Z}(t) = G(t,Z(t),U(t)) = 
    \begin{pmatrix}
        \dot{h}(t),\dot{\gamma}(t),\dot{x}(t),\dot{v}(t),\dot{\lambda}_1(t),\dot{\lambda}_2(t),\dot{\lambda}_3(t),\dot{\lambda}_4(t)
    \end{pmatrix}^T
\end{equation}
und für die Ableitung die Matrix
\begin{equation}\label{equ:jacobi}
    \dfrac{\partial G(t,Z(t),U(t))}{\partial Z} = 
    \begin{pmatrix}
        0 & J_G^{(1,2)} & 0 & J_G^{(1,4)} & 0 & 0 & 0 & 0 \\ 
        J_G^{(2,1)} & J_G^{(2,2)} & 0 & J_G^{(2,4)} & 0 & 0 & 0 & 0 \\ 
        0 & J_G^{(3,2)} & 0 & J_G^{(3,4)} & 0 & 0 & 0 & 0 \\ 
        J_G^{(4,1)} & J_G^{(4,2)} & 0 & J_G^{(4,4)} & 0 & 0 & 0 & 0 \\
        J_G^{(5,1)} & 0 & 0 & J_G^{(5,4)} & 0 & J_G^{(5,6)} & 0 & J_G^{(5,8)} \\
        0 & J_G^{(6,2)} & 0 & J_G^{(6,4)} & J_G^{(6,5)} & J_G^{(6,6)} & J_G^{(6,7)} & J_G^{(6,8)} \\
        0 & 0 & 0 & 0 & 0 & 0 & 0 & 0 \\
        J_G^{(8,1)} & J_G^{(8,2)} & 0 & J_G^{(8,4)} & J_G^{(8,5)} & J_G^{(8,6)} & J_G^{(8,7)} & J_G^{(8,8)}
    \end{pmatrix}
\end{equation}
mit den Einträgen \(J_G^{(i,j)}\), welche die partiellen Ableitungen von \(G\) nach jeder Komponente von \(Z\) enthalten (vgl. \autoref{Anhang:Jacobi}).
%
\item Für $R(Z(t_0),Z(t_f)) = 0_{n_Z}$ müssen zunächst die Endbedingungen mit
\begin{align*}
X_i(t_f) &= c_i & & (i=1,...,n_{\psi}=2) \\
\lambda_i(t_f) &= \lambda_0 g_{X_i}(X^{\ast}(t_f)) & &(i=n_{\psi}+1,...,n_X=4)
\end{align*}
aus
\begin{align*}
c &= \begin{pmatrix} h_f & \gamma_f \end{pmatrix} \\
g_{X}(X^{\ast}(t_f)) &= \begin{pmatrix} 0 & 0 & -1 & 0 \end{pmatrix}
\end{align*}
gebildet werden. Es ergibt sich dann
\begin{equation}
R(Z(t_0),Z(t_f)) = 0_{n_Z} = \begin{pmatrix}
h(t_0) - h_0 \\ 
\gamma(t_0) - \gamma_0 \\
x(t_0) - x_0 \\ 
v(t_0) - v_0 \\ 
h(t_f) - h_f \\ 
\gamma(t_f) - \gamma_f \\
\lambda_3(t_f) + \lambda_0 \\ 
\lambda_4(t_f) - 0
\end{pmatrix}
\end{equation}
mit 
\begin{equation}\label{func:RZt0}
\dfrac{d R(Z(t_0),Z(t_f))}{d Z(t_0)} = \begin{pmatrix}
1 & 0 & 0 & 0 & 0 & 0 & 0 & 0 \\ 
0 & 1 & 0 & 0 & 0 & 0 & 0 & 0 \\ 
0 & 0 & 1 & 0 & 0 & 0 & 0 & 0 \\ 
0 & 0 & 0 & 1 & 0 & 0 & 0 & 0 \\
0 & 0 & 0 & 0 & 0 & 0 & 0 & 0 \\
0 & 0 & 0 & 0 & 0 & 0 & 0 & 0 \\
0 & 0 & 0 & 0 & 0 & 0 & 0 & 0 \\
0 & 0 & 0 & 0 & 0 & 0 & 0 & 0
\end{pmatrix}
\end{equation}
und
\begin{equation}\label{func:RZtf}
\dfrac{d R(Z(t_0),Z(t_f))}{d Z(t_f)} = \begin{pmatrix}
0 & 0 & 0 & 0 & 0 & 0 & 0 & 0 \\ 
0 & 0 & 0 & 0 & 0 & 0 & 0 & 0 \\ 
0 & 0 & 0 & 0 & 0 & 0 & 0 & 0 \\ 
0 & 0 & 0 & 0 & 0 & 0 & 0 & 0 \\
1 & 0 & 0 & 0 & 0 & 0 & 0 & 0 \\
0 & 1 & 0 & 0 & 0 & 0 & 0 & 0 \\
0 & 0 & 0 & 0 & 0 & 0 & 1 & 0 \\
0 & 0 & 0 & 0 & 0 & 0 & 0 & 1
\end{pmatrix}
\end{equation}
\end{itemize}





















%\section{Algorithmus Einfachschießverfahren}
%Für eine gegebene Startschätzung $\eta$ des Anfangswerts $y(a)$ besitze das Anfangswertproblem
%\[y'(t) = g(t, y(t)) \ \ \ \ y(a) = \eta\]
%die Lösung $y(t;\eta)$ auf $[a,b]$. Damit $y(t;\eta)$ auch die Randbedingung erfüllt, muss 
%\begin{equation}\label{func:SchiessF}
%F(\eta) := r(y(a;\eta), y(b;\eta)) = r(\eta, y(b;\eta)) = 0_{n_y}
%\end{equation}
%gelten. Gleichung \ref{func:SchiessF} ist also ein \textbf{nichtlineares Gleichungssystem} für die Funktion $F$. Anwendung des Newtonverfahrens führt auf das sogenannte Einfachschießverfahren:
%
%\begin{definition}[Algorithmus Einfachschießverfahren]\label{algo:EinfSchiess}
%Initialisierung: Wähle Startschätzung $\eta^{[0]} \in \R^{n_y}$ und setze $i = 0$:
%\begin{enumerate}
%\item Löse das Anfangswertproblem \[y'(t) = g(t, y(t)) \ \ \ \ y(a) = \eta^{[i]} \ \ \ \ (a \leq t \leq b)\] zur Berechnung von $F(\eta^{[i]})$ und berechne die Jacobimatrix \[F'(\eta^{[i]}) = r'_{y_a} (\eta^{[i]}, y(b;\eta^{[i]})) + r'_{y_b}(\eta^{[i]}, y(b;\eta^{[i]})) \cdot S(b)\] wobei $S$ Lösung der Sensitivitäts-Differentialgleichung \[S'(t) = g'_y(t, y(t;\eta^{[i]})) \cdot S(t) \ \ \ \ S(a) = I_{n_y \times n_y}\] ist.
%%
%\item Ist $F(\eta^{[i]}) = 0_{n_y}$ (oder ist ein anderes Abbruchkriterium) erfüllt, \textbf{STOP!}
%%
%\item Berechne die Newton-Richtung $d^{[i]}$ als Lösung des linearen Gleichungssystems \[F'(\eta^{[i]})d = -F(\eta^{[i]})\]
%%
%\item Setze $\eta^{[i+1]} = \eta^{[i]} + d^{[i]}$ und $i=i+1$ und gehe zu 1.).
%\end{enumerate}
%\end{definition}
%
%Die Ableitung $F'(\eta^{[i]})$ in Schritt 2.) des Einfachschießverfahrens \ref{algo:EinfSchiess} kann alternativ
%durch \textbf{finite Differenzen} approximiert werden:
%\[\dfrac{\partial}{\partial \eta_j} F(\eta) \approx \dfrac{F(\eta + h e_j) - F(\eta)}{h} \ \ \ \ (j=1,...,n_y)\]
%mit $e_j = j$-ter Einheitsvektor. Dieser Ansatz erfordert das Lösen von $n_y$ Anfangswertproblemen!

