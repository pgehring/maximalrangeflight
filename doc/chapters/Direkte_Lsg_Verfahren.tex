\chapter{Lösen des Optimalsteuerungsproblem mit direkten Lösungsverfahren}

Direkte Lösungsverfahren basieren auf einer Diskretisierung des Optimalsteuerungsproblems. Dadurch wird das Optimalsteuerungsproblem auf ein endlichdimensionales Optimierungsproblem transformiert, welches mit einem numerischen Optimierungsverfahren gelöst werden kann.

Das gegebene Optimalsteuerungsroblem (Mayer-Problem mit Boxschranken) lässt sich also in ein Optimierungsproblem der folgenden Form umstellen. Finde für feste Zeitpunkte $t_0 < t_f$ einen Zustand $x \in W^{1, \infty} ([t_0,t_f], \R^{n_x})$ und eine Steuerung $u \in L^{\infty} ([t_0,t_f], \R^{n_u})$, sodass:
\begin{itemize}
\item \textbf{Zielfunktion:} Die Zielfunktion \[\varphi\]
\end{itemize}












\section{Algorithmus vollständige Diskretisierung}
Vollständige Diskretisierung mit explizit aber auch impliziten Verfahren

%\floatname{algorithm}{Klasse}
%\begin{algorithm}[H]
%\caption{MaximalRangeFlight}\label{algo:SISPF}
%\textbf{[$\lbrace s^i_t, w^i_t \rbrace^{N_s}_{i=1}$] = PFSIR [$\lbrace s^i_{t-1}, w^i_{t-1} \rbrace^{N_s}_{i=1}$]}
%\begin{algorithmic}
%\FOR {$i = 1,...,N_s$}
%\STATE $s^i_t \sim q(x_t \mid s^i_{i-1})$
%\STATE $w^i_t = p(z_t \mid s^i_t)$
%\ENDFOR
%\STATE $t = \sum_{i=1}^{N_s} w^i_t$
%\FOR {$i = 1,...,N_s$}
%\STATE $w^i_t = t^{-1} w^i_{t-1}$ (Normalisieren)
%\ENDFOR
%\STATE [$\lbrace s^{\ast j}_t, w^{j}_t, i^{j} \rbrace^{N_s}_{j=1}$] = RESAMPLE [$\lbrace s^i_t, w^{j}_t \rbrace^{N_s}_{j=1}$]
%\end{algorithmic}

%\floatname{algorithm}{Algorithmus}
%\begin{algorithm}[H]
%\caption{Vollständige Euler-Diskretisierung}\label{algo:SISPF}
%\textbf{[$\lbrace s^i_t, w^i_t \rbrace^{N_s}_{i=1}$] = PFSIR [$\lbrace s^i_{t-1}, w^i_{t-1} \rbrace^{N_s}_{i=1}$]}
%\begin{algorithmic}
%\FOR {$i = 1,...,N_s$}
%\STATE $s^i_t \sim q(x_t \mid s^i_{i-1})$
%\STATE $w^i_t = p(z_t \mid s^i_t)$
%\ENDFOR
%\STATE $t = \sum_{i=1}^{N_s} w^i_t$
%\FOR {$i = 1,...,N_s$}
%\STATE $w^i_t = t^{-1} w^i_{t-1}$ (Normalisieren)
%\ENDFOR
%\STATE [$\lbrace s^{\ast j}_t, w^{j}_t, i^{j} \rbrace^{N_s}_{j=1}$] = RESAMPLE [$\lbrace s^i_t, w^{j}_t \rbrace^{N_s}_{j=1}$]
%\end{algorithmic}





Allgemein wird für die Matlab Funktion \verb|fmincon| die Startmatrix  \verb|prob.z_0| 
\[\begin{pmatrix}
[20,9,6000,90,1259999,1.47] \\ 
\vdots \\ 
\vdots
\end{pmatrix}\]
für die Optimierung verwendet. Es sind also alle Zustandsvektoren für jeden Diskretisierungspunkt identisch.









\section{Beschränkte nichtlineare Optimierungsverfahren}








\section{Versuchsdurchführungen und Ergebnisse}
Gedanken wie der Verlauf aussehen könnte:\\
Am Anfang maximaler Schub und maximaler Auftriebsbeiwert, um so schnell wie möglich die gewünschte Reisehöhe zu erhalten. Auftriebsbeiwert kann dann reduziert werden, um die Geschwindigkeit zu erhöhen. Dies bewirkt eine höhere Geschwindigkeit, wodurch die zurückgelegte Strecke maximiert wird.












