\chapter{Formulierung des Optimalsteuerungsproblem für direkte Lösungsverfahren} \label{cha:direct}

Direkte Lösungsverfahren basieren auf einer Diskretisierung des Optimalsteuerungsproblems. Dadurch wird das Optimalsteuerungsproblem (Problem \ref{prob:MaxRF}) auf ein endlichdimensionales Optimierungsproblem transformiert, welches mit einem numerischen Optimierungsverfahren gelöst werden kann.












\section{Diskretisierung des Optimalsteuerungsproblems}
Unter der Diskretisierung wird die Aufgabe verstanden, für feste Zeitpunkte $t_0 < t_f$ einen Zustand $X \in W^{1, \infty} ([t_0,t_f], \R^{n_X})$ und eine Steuerung $U \in L^{\infty} ([t_0,t_f], \R^{n_U})$ zu finden, sodass die Zielfunktion unter der Betrachtung der gegebenen Bedingungen minimal wird.

Für die Diskretisierung wird daher zunächst ein Gitter erzeugt, welches das Problem in explizite $N$ Diskretisierungspunkte unterteilt. Dieses Gitter 
\[\mathbb{G}_h := \lbrace t_0 < t_1 < ... < t_{N-1} = t_f \rbrace\]
mit den Schrittweiten 
\[h_i = t_{i+1} - t_i \ \ \ \ \forall i = 0,...,N-2\]
muss dabei nicht notwendig äquidistant unterteilt sein. Es stellt die Basis für die Parametrisierung der Steuerung und das Diskretisierungsverfahren für die Differentialgleichung:
\begin{itemize}
\item \textbf{Parametrisierung der Steuerung:} Dabei wird die Steuerfunktion $U$ durch eine von der Schrittweite abhängige Steuerfunktion $U_h$ ersetzt. Diese Steuerfunktion hängt dann nur noch von endlich vielen Parametern ab.
%
\item \textbf{Diskretisierungsverfahren für die Differentialgleichung:} Die Differentialgleichung des Problems wird mit Hilfe eines Diskretisierungsverfahren diskretisiert, um die Lösung im Optimierungsproblem verarbeiten zu können. Hierfür eignen sich zum Beispiel allgemeine oder spezielle Einschrittverfahren wie das explizite Eulerverfahren oder das implizite Runge-Kutta Verfahren.
%
\item \textbf{Ranbedingungen und Beschränkungen:} Die Anfangs- und Endbedingungen aus Problem \ref{prob:MaxRF} werden zusammen in einer Randbedingungsfunktion $\psi : \R^{n_X} \times \R^{n_X} \to \R^{n_{\psi}}$ mit $n_{\psi} = 8$
\[\psi(X_0,X_N) = 0_{n_{\psi}} = \begin{pmatrix}
h_0(t_0) - h_0 \\ 
\gamma_0(t_0) - \gamma_0 \\
x_0(t_0) - x_0 \\ 
v_0(t_0) - v_0 \\ 
h_N(t_f) - h_f \\ 
\gamma_N(t_f) - \gamma_f \\
0 - 0 \\ 
0 - 0
\end{pmatrix}\] formuliert. Die Beschränkung des Staudrucks $q(v(t),h(t))$ wird in der reinen Zustandsbeschränkungsfunktion $s : \mathbb{G}_h \times \R^{n_X} \to \R^{n_s}$ mit $n_s = 1$
\[s(t_i,X_i) = q(v_i(t_i),h_i(t_i)) - q_{max} \leq 0_{n_s} \ \ \ \ \forall i = 0,1,...,N-1\] formuliert.
\end{itemize}
Somit geht das Optimalsteuerungsproblem (Problem \ref{prob:MaxRF}) in das diskretisierte Optimalsteuerungsproblem (Problem \ref{prob:VolldisOpt}) über.

\begin{problem}[Vollständig diskretisiertes Optimalsteuerungsproblem]\label{prob:VolldisOpt}
    Finde die Gitterfunktionen \[X_h : \mathbb{G}_h \to \R^{n_X} \text{ mit } t_i \mapsto X_h(t_i) =: X_i\] und \[U_h : \mathbb{G}_h \to \R^{n_U} \text{ mit } t_i \mapsto U_h(t_i) =: U_i\] sodass die Zielfunktion \[\varphi(X_0,X_{N-1})\] minimal wird unter der Betrachtung der diskretisierten  Differentialgleichung (hier mit explizitem Eulerverfahren): \[X_{i+1} = X_{i} + h_i f(t_i,X_i,U_i) \ \ \ \ \forall i = 0,1,...,N-2\] den Randbedingungen: \[\psi(X_0,X_N) = 0_{n_{\psi}}\] den reinen Zustandsbeschränkungen: \[s(t_i,X_i) \leq 0_{n_s} \ \ \ \ \forall i = 0,1,...,N-1\] und den Mengenbeschränkungen: \[U_{\min} \leq U_i \leq U_{\max} \ \ \ \ \forall i = 0,1,...,N-1\]
\end{problem}











%\newpage
%
%Betrachtet wird das Optimalsteuerungsroblem (Mayer-Problem mit Boxschranken) der folgenden Form. Finde für feste Zeitpunkte $t_0 < t_f$ einen Zustand $X \in W^{1, \infty} ([t_0,t_f], \R^{n_X})$ und eine Steuerung $U \in L^{\infty} ([t_0,t_f], \R^{n_U})$, sodass die Zielfunktion unter der Betrachtung der gegebenen Bedingungen minimal wird:
%\begin{itemize}
%    \item \textbf{Zielfunktion:} $\varphi(X(t_0),X(t_f)) = -(x(t_f) - x_0)$
%    \item \textbf{Differentialgleichung:} $\dot{X}(t) = f(t,X(T),U(t)) \ \ \ \ \forall [t_0,t_f]$
%    \item \textbf{Randbedingungen:} $\psi(X(t_0),X(t_f)) = 0_{n_{\psi}}$
%    \item \textbf{Reine Zustandsbeschränkungen:} $s(t,X(t)) \leq 0_{n_s} \ \ \ \ \forall t \in [t_0,t_f]$
%    \item \textbf{Mengenbeschränkungen:} $U(t) \in \mathcal{U} = \lbrace U \in \R^{n_U} \mid U_{\min} \leq U \leq U_{\max} \rbrace \ \ \ \ \forall t \in [t_0, t_f]$
%\end{itemize}
%Die Mengenbeschränkungen werden auch als Boxbeschränkungen bezeichnet. 
%
%
%Dieses gegebene Optimalsteuerungsproblem wird mit verschiedenen Techniken und Verfahren auf ein nichtlineares Optimierungsproblem umgeformt. Zunächst wird dabei ein Gitter erzeugt, welches das Problem diskretisiert. Dieses Gitter 
%\begin{equation}
%    \mathbb{G}_h := \lbrace t_0 < t_1 < ... < t_{N-1} = t_f \rbrace
%\end{equation}
%mit $N$, der Anzahl an Diskretisierungspunkten und mit den Schrittweiten 
%\begin{equation}
%    h_i = t_{i+1} - t_i \ \ \ \ \forall i = 0,...,N-2
%\end{equation}
%muss dabei nicht notwendig äquidistant unterteilt sein. Es stellt die Basis für die Parametrisierung der Steuerung und das Diskretisierungsverfahren für die Differentialgleichung:
%\begin{itemize}
%    \item \textbf{Parametrisierung der Steuerung:} Dabei wird die Steuerfunktion $U$ durch eine von der Schrittweite abhängige Steuerfunktion $U_h$ ersetzt. Diese Steuerfunktion hängt dann nur noch von endlich vielen Parametern ab.
%
%    \item \textbf{Diskretisierungsverfahren für die Differentialgleichung:} Die Differentialgleichung des Problems wird mit Hilfe eines Diskretisierungsverfahren diskretisiert, um die Lösung im Optimierungsproblem verarbeiten zu können. Hierfür eignen sich zum Beispiel allgemeine oder spezielle Einschrittverfahren wie das explizite Eulerverfahren oder das implizite Runge-Kutta Verfahren.
%\end{itemize}
%Somit ergibt sich das diskretisierte Optimalsteuerungsproblem (Problem \ref{prob:VolldisOpt}).
%
%\begin{problem}[Vollständig diskretisiertes Optimalsteuerungsproblem]\label{prob:VolldisOpt}
%    Finde die Gitterfunktionen \[X_h : \mathbb{G}_h \to \R^{n_X} \text{ mit } t_i \mapsto X_h(t_i) =: X_i\] und \[U_h : \mathbb{G}_h \to \R^{n_U} \text{ mit } t_i \mapsto U_h(t_i) =: U_i\] sodass die Zielfunktion \[\varphi(X_0,X_{N-1})\] minimal wird unter der Betrachtung der diskretisierten  Differentialgleichung (hier mit explizitem Eulerverfahren): \[X_{i+1} = X_{i} + h_i f(t_i,X_i,U_i) \ \ \ \ \forall i = 0,1,...,N-2\] den Randbedingungen: \[\psi(X_0,X_N) = 0_{n_{\psi}}\] den reinen Zustandsbeschränkungen: \[s(t_i,X_i) \leq 0_{n_s} \ \ \ \ \forall i = 0,1,...,N-1\] und den Mengenbeschränkungen: \[U_{\min} \leq U_i \leq U_{\max} \ \ \ \ \forall i = 0,1,...,N-1\]
%\end{problem}












\section{Aufstellen des Optimierungsproblems}
Dieses diskretisierte Optimalsteuerungsproblem lässt sich dann mit einem geeigneten numerischen Optimierungsverfahren lösen. Mögliche Optimierungsverfahren sind das SQP-Verfahren oder das Innere-Punkte-Verfahren. Hierfür muss Problem \ref{prob:VolldisOpt} in die Form des endlichdimensionale, nichtlineare Optimierungsproblem \ref{prob:EndNichtOpt} gebracht werden.

\begin{problem}[Endlichdimensionales, nichtlineares Optimierungsproblem]\label{prob:EndNichtOpt}
    Endlichdimensionales, nichtlineares Optimierungsproblem der Form
    \begin{align*}
        \min_{z \in S} F(z) &:= \varphi(X_0,X_{N-1}) & & \\
        \text{unter} \hspace{10mm} G(z) &\leq 0_{n_s \cdot N} & &  \text{(Ungleichungsnebenbedingungen)} \\
        H(z) &= 0_{n_X \cdot (N-1) + n_{\psi}} & & \text{(Gleichungsnebenbedingungen)}\\
        z &\in S := \R^{n_X \cdot N} \times [U_{\min},U_{\max}]^{N} & &  \text{(Zulässigemenge)}
    \end{align*}
    mit der Optimierungsvariable
    \begin{equation}
        z := (X_0,X_1,...,X_{N-1},U_0,U_1,...,U_{N-1})^T
    \end{equation}
    und den Nebenbedingungsfunktionen 
    \begin{equation}
        G(z) := 
        \begin{pmatrix}
            s(t_0,X_0,U_0) \\ 
            \vdots \\ 
            s(t_{N-1},X_{N-1},U_{U-1})
        \end{pmatrix} 
    \end{equation}
    und
    \begin{equation}
        H(z) := 
        \begin{pmatrix}
            X_0 + h_0 f(t_0,X_0,U_0) - X_1 \\ 
            \vdots \\ 
            X_{N-2} + h_{N-2} f(t_{N-2},X_{N-2},U_{U-2}) - X_{N-1} \\
            \psi(X_0,X_N)
        \end{pmatrix} 
    \end{equation}
\end{problem}



 








