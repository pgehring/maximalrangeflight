\chapter{Motivation und Ziel der Arbeit}
Modelliert werden soll ein 2-dimensionaler Flug eines Flugzeugs in der x-h-Ebene (Abbildung \ref{img:Flugzeug}), bei dem man den Auftriebsbeiwert $C_L(t)$ und den Schub $T(t)$ steuern kann. Diese Steuerungen sind jedoch in ihrer Möglichkeit beschränkt. Ebenfalls darf während des Fluges ein maximaler Staudruck nicht überschritten werden.

\begin{figure}[H]
    \begin{center}
        \includegraphics[width=\textwidth]{images/01_Modellaufbau/Flugzeug.pdf}
        \caption{2D-Skizze eines Flugzeuges in der $x$-$h$-Ebene mit angreifenden Kräften $L$ (Auftriebskraft), $D$ (Luftwiederstand), $W$ (Erdanziehungskraft) und $T$ (Schub) am Schwerpunkt $S$. Des Weiteren beschreibt $v$ die Geschwindigkeit und $\gamma$ den Anstellwinkel.}\label{img:Flugzeug}
    \end{center}
\end{figure}

Das Ziel ist es, das Flugzeug von einer gegebenen Anfangsposition so zu steuern, dass eine vorgegebene Reisehöhe $h_f = 10668\ m$ und Anstellwinkel $\gamma_f = 0\ ^{\circ}$ während der Dauer von $1800 \ s$ erreicht erreicht wird, wobei die zurückgelegte Strecke maximal wird.
