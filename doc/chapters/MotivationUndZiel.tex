\chapter{Motivation und Ziel der Arbeit}
Modelliert werden soll ein 2-dimensionaler Flug eines Flugzeugs in der x-h-Ebene (Abbildung \ref{img:Flugzeug}), bei dem man den Auftriebsbeiwert und den Schub steuern kann, wobei ein maximaler Staudruck nicht überschritten werden darf.

\begin{figure}[H]
\begin{center}
\includegraphics[width=.7\textwidth]{images/01_Modellaufbau/Flugzeug}
\caption{2D-Skizze eines Flugzeuges in der $x$-$h$-Ebene mit angreifenden Kräften $L$ (Auftriebskraft), $D$ (Luftwiederstand), $W$ (Erdanziehungskraft) und $T$ (Schub) am Schwerpunkt $S$. Des Weiteren beschreibt $v$ die Geschwindigkeit und $\gamma$ den Anstellwinkel.}\label{img:Flugzeug}
\end{center}
\end{figure}

Das Ziel ist den Flug eines Flugzeuges von einer gegebenen Anfangsposition so zu steuern, dass eine vorgegebene Reisehöhe $h_f = 10668 \ [m]$ erreicht wird, der Anstellwinkel dort $\gamma_f = 0 \ [Grad]$ und die Reichweite maximal ist.
