\chapter{Diskussion der numerischen Versuche und Ergebnisse}\label{kap:LSG}

\section{Direkte Lösungsverfahren - Ergebnisse und Probleme}
Das Optimalsteuerungsproblem \ref{prob:MaxRF} mit den Parametern aus Tabelle \ref{tab:ProblemPara} ist in die Form von Problem \ref{prob:EndNichtOpt} umgestellt und wird technisch in Anhang \ref{Anhang:DirektV} mit \textit{MATLAB} umgesetzt und gelöst.

Zunächst werden beide Optimierungsverfahren, welche in \ref{kap:OptVerfahren} vorgestellt werden, auf das Problem in Versuch 0 (Anhang \ref{kap:Versuch0}) angewendet. Während das SQP-Verfahren in Versuch 0.1 (Anhang \ref{kap:Versuch01}) zu einem Ergebnis kommt, welches plausibel erscheint, liefert das Innere-Punkte-Verfahren in Versuch 0.1 (Anhang \ref{kap:Versuch02}) ein stark oszilierendes, nicht plausibles Ergebnis.

Daher wurde im weiteren Vorgehen eine numerische Untersuchung der beiden Optimierungsverfahren durchgeführt. Für die Untersuchung wird die Anzahl an benötigten Iterationen, der erreichte Funktionswert und die Zulässigkeit des Ergebnisses herangezogen. In Abbildung \ref{img:Vergleich_SQP_IP} zeigt sich, dass das SQP-Verfahren für den selben Startwert, sowohl schneller konvergiert, als auch einen kleineren Funktionswert nach Abbruch des Verfahrens besitzt. Gründe für dieses Verhalten könnten darin liegen, dass sich beide Steuerungen und teilweise auch die Zustandsvariablen am Rand der zulässigen Menge bewegen und die Barrierefunktion beim IP-Verfahren die Iterierten von den Grenzen der Ungleichheitsbedingungen fernhält \cite{Matlab2016}. In den nachfolgenden numerischen Versuchen wird deshalb stets das SQP-Verfahren verwendet.

\begin{figure}[htbp]
    \begin{center}
        \includegraphics{images/03_TechnischeUmsetzung/Vergleich_SQP_IP.pdf}
        \caption{Vergleich von IP- und SQP-Verfahren.}\label{img:Vergleich_SQP_IP}
    \end{center}
\end{figure}

Zwar scheint das Ergebnis in Versuch 0.1 noch plausibel zu sein, aber auch hier wird wie in Versuch 0.2 die Abbruchbedingung $EF = -2$ erreicht (kein zulässiger Punkt). Auch physikalischer Betrachtung sind diese Ergebnisse für das Flugzeug A380-800 nicht plausibel. Die Strategie der Steuerung liegt in einer Art Parabelflug, da die Trajektorie der Höhe einer Parabel gleicht (Abbildung \ref{img:test_0_1}). Dabei wird das Flugzeug möglichst schnell auf eine maximale Höhe gesteuert, um von diesem Punkt aus dann auf die Endhöhe $h_f$ mit abnehmender Geschwindigkeit zu segeln. Die physikalischen Grenzen $h_{\max} = 13.100 \ m$ und $v_{\max} = 266 \ \frac{m}{s}$ aus \cite{A380Tech} von den berechneten Höhen und Geschwindigkeiten werden dabei deutlich überschritten.

Der Hauptgrund für dieses Problem liegt in der gegebenen festes Endzeit $t_f$ und den fehlenden physikalischen Grenzen des Flugzeuges A380-800. Vor allem die fest gegebene Endzeit $t_f$ wirft zunächst die Frage auf, ob für diese Zeitdauer das Steuerungsproblem überhaupt lösbar ist. Um diese Frage zu beantworten wurde das Optimalsteuerungsproblem \ref{prob:MaxRF} und deren Zielfunktion wie in Anhang \ref{kap:OptTf} transformiert auf das Optimalsteuerungsproblem \ref{prob:MaxRFEndzeit}. Bei diesem liegt das Ziel nun darin, die optimale = minimale Endzeit $t_f$ unter Einhaltung der Start- und Endbedingungen zu bestimmen. Aus den Versuchen 1 (Anhang \ref{kap:Versuch0_OptTf}) und 2 (Anhang \ref{kap:Versuch1_OptTf}) ergeben sich die optimalen Endzeiten $t_f = 279.2113 \ s$ und $t_f = 515.8404 \ s$. Da diese kleiner sind als als die gegebene Endzeit $t_f = 1800 \ s$, ist das Steuerungsproblem demnach lösbar.

In weiteren Versuchen mit dem Steuerungsproblem \ref{prob:MaxRF}, Anhang \ref{kap:Versuch1}, \ref{kap:Versuch2}, \ref{kap:Versuch3} und \ref{kap:Versuch4} variieren verschiedene Parameter und die gegebene Endzeit $t_f$, beziehungsweise werden neue Parameter und Beschränkungen eingeführt. Ziel dieser Versuche ist es, somit aussagekräftige Ergebnisse für das Problem zu erhalten, welche auch die physikalischen Kriterien erfüllen:
\begin{itemize}
\item \textbf{Versuch 1:} In Versuch 1 (Anhang \ref{kap:Versuch1}) wird die gegebenen Endzeit $t_f$ von $1800 \ s$ einmal auf $t_f = 300 \ s$ (Versuch 1.1, Anhang \ref{kap:Versuch11}) und einmal auf $t_f = 350 \ s$ (Versuch 1.2, Anhang \ref{kap:Versuch12}) reduziert. 
%
\item \textbf{Versuch 2:} In Versuch 2 (Anhang \ref{kap:Versuch2}) wird neben der gegebenen Endzeit $t_f$ auch die Starthöhe $h_0 = 4000 \ m$ angepasst. Die gegebene Endzeit wird von $1800 \ s$ einmal auf $t_f = 300 \ s$ (Versuch 2.1, Anhang \ref{kap:Versuch21}) und einmal auf $t_f = 350 \ s$ (Versuch 2.2, Anhang \ref{kap:Versuch22}) reduziert.
%
\item \textbf{Versuch 3:} In Versuch 3 (Anhang \ref{kap:Versuch3}) wird das Startgewicht $m$ und die Endzeit $t_f$ angepasst. Damit wird das Startgewicht einem real vorkommenden Startgewicht $m = 500000 \ kg$ für das Flugzeug A380-800 angepasst. Die gegebenen Endzeit $t_f$ wird von $1800 \ s$ einmal auf $t_f = 550 \ s$ (Versuch 3.1, Anhang \ref{kap:Versuch31}) und einmal auf $t_f = 600 \ s$ (Versuch 3.2, Anhang \ref{kap:Versuch32}) reduziert.
%
\item \textbf{Versuch 4:} In Versuch 4 (Anhang \ref{kap:Versuch4}) werden zum angepassten Startgewicht $m = 500000 \ kg$ und der angepassten Endzeit $t_f$ auch noch zusätzliche physikalische Boxschranken gesetzt, die das Problem und das Flugzeug betreffen. Die gegebenen Endzeit $t_f$ wird von $1800 \ s$ einmal auf $t_f = 550 \ s$ (Versuch 4.1, Anhang \ref{kap:Versuch41}) und einmal auf $t_f = 600 \ s$ (Versuch 4.2, Anhang \ref{kap:Versuch42}) reduziert.
\end{itemize}
Grundsätzlich wurde bei diesen Versuchen kein Ergebnis mit dem Abbruchkriterium $EF = 1$ erreicht, also ein mögliches lokales oder globales Optimum, sondern maximal nur Lösungen mit dem Abbruchkriterium $EF = 2$. Daher gilt es zunächst kritisch den Ergebnissen gegenüber zu sein. Es zeichnet sich jedoch bei der Mehrheit der Ergebnisse eine gemeinsame Steuerstrategie heraus, welche auch mit den Ergebnissen aus den Versuchen 0 und 1 (Anhang \ref{kap:Versuch0_OptTf} und \ref{kap:Versuch1_OptTf}) korrelieren.

Für die Beschreibung dieser gemeinsamen Steuerstrategie wird das Ergebnis von Versuch 1.1 (Abbildung \ref{img:test_1_1}) betrachtet. Darin ist zu beobachten, dass die Steuerung Schub sich nur der Schranken der zulässigen Menge bedient, also entweder $T_{\max}$ oder $T_{\min}$ einstellt. Wie aus der Funktion \ref{equ:state_space} ersichtlich wird, beeinflusst der Schub lediglich die Änderung der Geschwindigkeit und damit auch das Verhalten der Geschwindigkeit. Bei der Steuerung Anstellwinkel zeichnet sich ein anderes Verhalten ab. Hier werden nicht nur die Schranken der zullässigen Menge, sondern auch reelle Werte dazwischen eingestellt. Ebenfalls aus \ref{equ:state_space} ersichtlich, beeinflusst der Auftriebsbeiwert nicht nur die Änderung der Geschwindigkeit, sondern auch die Änderung des Anstellwinkels. Die gemeinsame Strategie aus Schub und Auftriebsbeiwert lautet zu Beginn daher, das Flugzeug möglichst schnell mit $T_{\max}$ auf eine hohe Geschwindigkeit beschleunigen und diese dann zu halten. Hierdurch zeichnet sich eine hohe gleichbleibende Änderung der zu optimierenden Reichweite aus. Um das Endziel $h_f = 10668 \ m$ und mit dem Antstellwinkel $\gamma_f = 0 \ ^\circ$ zu erreichen, wird kurz vor erreichen der Endzeit $t_f$ der Schub vollständig auf $T_{\min}$ gesetzt und ein maximaler Auftriebsbeiwert $C_{L, \max}$ eingestellt. Diese Einstellungen rufen schnelle Veränderungen des Anstellwinkels und der Geschwindigkeit hervor. Da die Änderung der Flughöhe genau von diesen beiden Zustandsvariablen abhängig ist, wird durch deren Reduzierung ein Nicken des Flugzeuges erreicht, welches am Verlauf der Flughöhe ersichtlich ist. Abschließend wird so die Endposition des horizontal Fluges auf einer Höhe von $h_f = 10668 \ m$ erreicht, wobei die Steuerung so stark ein greift, dass die Geschwindigkeit nahe bei $v_f = 0 \ \frac{m}{s}$ liegt. Dieses Verhalten scheint aber aufgrund des gegebenen Modells und Problems plausibel zu sein.

Bei dem Ergebnis aus Versuch 3.2 (Anhang \ref{kap:Versuch32}) zeigt sich der Einfluss der gesetzten Endzeit $t_f$. Während sich die Steuerung zunächst wie bei der Mehrheit verhält, wird vor allem die zweite Hälfte des Zeitraumes anders gestaltet. Es steht der Steuerung ein so großes Zeitfenster zu Verfügung, um das gewünschte Endziel zu erreichen, sodass nur eine Anpassung des Auftriebsbeiwertes nötig ist. Das hat zur Folge, dass der Schub konstant auf $T_{\max}$ gehalten wird, um die Geschwindigkeit monoton zu erhöhen. Über die Anpassung des Auftriebsbeiwertes wird ab der zweiten Zeithälfte dann der Anstellwinkel reduziert und damit auch die Höhe, wodurch das Endziel erreicht wird.

Das Ergebnis aus Versuch 4.2 (Anhang \ref{kap:Versuch42}) verhält sich ähnlich zum Ergebnis aus Versuch 3.2. Jedoch ist hier zu beobachten, dass mit der möglichen Steuerung die physikalische Grenze der maximal Geschwindigkeit erreicht wird und deshalb die Steuerung reduziert werden muss. Diese wird dahingehend angepasst, dass die Geschwindigkeit auf maximalem Niveau $v_{\max}$ bis zum Endpunkt gehalten wird, um die Reichweite zu maximieren.

Die Ergebnisse aus den Versuchen 1.2 (Anhang \ref{kap:Versuch12}) und 2.1 (Anhang \ref{kap:Versuch21}) sind nicht plausibel, da bei beiden negative Geschwindigkeiten erreicht werden.

Allgemein wurde bei allen Versuchen die Anzahl $N = 100$ an Diskretisierungen verwendet. Bei Versuch 3 wurde jedoch die Anzahl auf $N = 200$ erhöht, da $N=100$ zu unzulässigen Ergebnissen führte. Dies hat zwar eine höhere Berechnungsdauer zur Folge, verbesserte allerdings die Ergebnisse enorm.

Ein weiteres Problem war die die Wahl eines Startvektors $z_0$. Schon geringe Abweichungen von den aktuell gewählten Startvektoren führt zu einem Abbruch des Optimierungsalgorithmus oder zu einer unzulässigen falschen Lösung. Ausgewählt wurden diese über die heuristische Methode \glqq Versuch und Irrtum\grqq{} (engl. \textit{trial and error}). Es wurde jedoch darauf geachtet, das die Werte der Zustandsvariablen in der physikalischen zulässigen Menge liegen. Die Wahl des Startvektors beeinflusste demnach nicht nur das Ergebnis, sondern auch die Berechnungsdauer und die Anzahl an benötigten Iterationen und Funktionsauswertungen (siehe Tabelle \ref{tab:Versuch0_TA}, \ref{tab:Versuch1_TA}, \ref{tab:Versuch2_TA}, \ref{tab:Versuch3_TA}, \ref{tab:Versuch4_TA} und \ref{tab:Versuch_TA}).








\section{Indirekte Lösungsverfahren - Ergebnisse und Probleme}\label{kap:LSGIndirekt}
Für die indirekten Lösungsverfahren werden die notwendigen Optimalitätsbedinungen für das gegebene Optimalsteuerungsproblem \ref{prob:MaxRF} in Kapitel \ref{kap:NOpt} aufgestellt. Aus diesen Bedinungen gehen die expliziten Synthese-Steuerungen für den Schub \ref{func:SynSchub} und den Auftriebsbeiwert \ref{func:SynAuftrieb}, die Adjungierte DGL \ref{func:AjgDGL}, sowie die Konstanz der Hamilton Funktion \ref{func:HKonstanz} hervor. Diese Funktionen und Bedingungen werden anschließend dafür verwendet, um das Optimalsteuerungsproblem auf ein Zweipunkt-Randwertproblem zu transformieren (Kapitel \ref{kap:ZPRand}).

Jedoch behandelt das Zweipunkt-Randwertproblem (Problem \ref{prob:ZweiRand}) keine Bedingungen, wie die geforderte Beschränkung des Staudrucks $q(v(t),h(t)) \leq q_{\max}$. Untersuchungen der Ergebnisse aus den Versuchen mit den direkten Lösungsverfahren (Anhang \ref{Anhang:DirektV}, Abbildung \ref{img:test_1_1_staudruck}, \ref{img:test_2_2_staudruck}, \ref{img:test_3_1_staudruck}, \ref{img:test_3_2_staudruck},\ref{img:test_4_1_staudruck} und \ref{img:test_1_2_staudruck} und Anhang \label{kap:OptTf}, Abbildung \ref{img:test_0_staudruck_OptTf} und \ref{img:test_1_staudruck_OptTf}) haben gezeigt, dass diese zu keinem Zeitpunkt den maximalen Wert $q_{\max}$ der Beschränkung erreicht haben. Aus Vereinfachungsgründen, wird diese Beschränkung deshalb im weiteren Vorgehen nicht berücksichtigt.

Nach der technischen Umsetzung des Einfach- und Mehrfachschießverfahren und Implementierung des Zweipunkt-Randwertproblems aus Kapitel \ref{kap:ZPRand} in \textit{MATLAB} ergab sich jedoch ein entscheidendes Prolem für die Durchführbarkeit der indirekten Lösungsverfahren. Die Jakobi-Matrix, des in den Schießverfahren enthaltenen Newtonverfahrens ist singulär, besitzt also keinen vollen Rang und ist demnach auch nicht invertierbar. Es lässt sich demnach kein eindeutiger Newtonschritt berechnen, welcher den Startwert in Richtung der Lösung des Nullstellenproblems anpasst. Diese Singularität rührt aus den Singularitäten der Matrizen \ref{equ:jacobi}, \ref{func:RZt0} und \ref{func:RZtf}, aus welchen die Jakobimatrix entsteht. Die indirekten Lösungsverfahren sind für das gegebene Optimalsteuerproblem \ref{prob:MaxRF} demnach nicht anwendbar.