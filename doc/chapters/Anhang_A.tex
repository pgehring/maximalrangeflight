\chapter{Direkte Lösungsverfahren - Versuche und Ergebnisse}\label{Anhang:DirektV}

\section{Versuch 0}\label{kap:Versuch0}
Bei diesem Versuch wird das gegebene Problem \ref{prob:MaxRF} mit den Parametern aus Tabelle \ref{tab:ProblemPara} ohne Veränderungen gelöst. Für die beiden Versuche werden die Einstellungen aus Tabelle \ref{tab:Versuch0} verwendet.
\begin{table}[H]
    \centering
    \caption{Einstellungen von Versuch 0.1 und 0.2.}\label{tab:Versuch0}
    \begin{tabularx}{.9\textwidth}{Zccc}
        \toprule
        \textbf{Einstellungen} & \textbf{Versuch 0.1} & \textbf{Versuch 0.2} \\
        \midrule
        Anzahl Diskretisierungen & $N = 100$ & $N = 100$ \\
        Lösungsverfahren der DGL & Explizites Euler Verfahren & Explizites Euler Verfahren \\
        Optimierungsverfahren & SQP-Verfahren & Innere-Punkte-Verfahren \\
        Startvektor & $z_0 = \begin{pmatrix}
        9000 \\ 
        5 \\ 
        800000 \\
        250 \\
        1259999 \\ 
        1.4
        \end{pmatrix} $ & $z_0 = \begin{pmatrix}
        9000 \\ 
        5 \\ 
        800000 \\
        250 \\
        1259999 \\ 
        1.4
        \end{pmatrix}$ \\
        \bottomrule
    \end{tabularx}
\end{table}
Für die Ergebnisse von Versuch 0.1 (Anhang \ref{kap:Versuch01}) und Versuch 0.2 (Anhang \ref{kap:Versuch02}) wird der folgende technische Aufwand (Tabelle \ref{tab:Versuch0_TA}) benötigt.
\begin{table}[H]
    \centering
    \caption{Technischer Aufwand von Versuch 0.1 und 0.2.}\label{tab:Versuch0_TA}
    \begin{tabularx}{.9\textwidth}{Zccc}
        \toprule
         & \textbf{Versuch 0.1} & \textbf{Versuch 0.2} \\
        \midrule
        Funktionswert der Zielfunktion & $-1768382.0703$ & $-1020675.0231$ \\
        Anzahl Iterationen & $5394$ & $14887$ \\
        Anzahl Funktionsauswertungen & $3251455$ & $8968016$ \\
        Exit Flag von \textit{MATLAB} & $-2$ & $-2$ \\
        Optimalität des Ergebnis & $15.2438$ & $886.3152$ \\
        Berechnungsdauer & $47.3956 \ min$ & $154.133 \ min$ \\
        \bottomrule
    \end{tabularx}
\end{table}



\subsection{Ergebnis von Versuch 0.1}\label{kap:Versuch01}
In Versuch 0.1 wird das Ergebnis in Abbildung \ref{img:test_0_1} erreicht. Dies stellt aufgrund des Abbruchkriteriums $EF = -2$ kein Optimum des Problems dar (kein zulässiger Punkt gefunden). 
\begin{figure}[H]
\begin{center}
\includegraphics[width=\textwidth]{../code/direct_sol/results/test_0_1}
\MyCaption{Ergebnis von Versuch 0.1}{} \label{img:test_0_1}
\end{center}
\end{figure}
Der Verlauf des beschränkten Staudrucks ist in Abbildung \ref{img:test_0_1_staudruck} dargestellt.
\begin{figure}[H]
\begin{center}
\includegraphics[width=\textwidth]{../code/direct_sol/results/test_0_1_staudruck}
\MyCaption{Überprüfung Staudruck $q(v(t),h(t))$ von Versuch 0.1}{} \label{img:test_0_1_staudruck}
\end{center}
\end{figure}





\subsection{Ergebnis von Versuch 0.2}\label{kap:Versuch02}
In Versuch 0.1 wird das Ergebnis in Abbildung \ref{img:test_0_2} erreicht. Dies stellt aufgrund des Abbruchkriteriums $EF = -2$ kein Optimum des Problems dar (kein zulässiger Punkt gefunden).
\begin{figure}[H]
\begin{center}
\includegraphics[width=\textwidth]{../code/direct_sol/results/test_0_2}
\MyCaption{Ergebnis von Versuch 0.2}{} \label{img:test_0_2}
\end{center}
\end{figure}
Der Verlauf des beschränkten Staudrucks ist in Abbildung \ref{img:test_0_2_staudruck} dargestellt.
\begin{figure}[H]
\begin{center}
\includegraphics[width=\textwidth]{../code/direct_sol/results/test_0_2_staudruck}
\MyCaption{Überprüfung Staudruck $q(v(t),h(t))$ von Versuch 0.2}{} \label{img:test_0_2_staudruck}
\end{center}
\end{figure}











\newpage
\section{Versuch 1}\label{kap:Versuch1}
Die Problemstellung \ref{prob:MaxRF} gibt den Parametern \ref{tab:ProblemPara} eine feste Endzeit $t_f = 1800 \ s$ beziehungsweise einen Zeitraum vor, in welchem das Problem gelöst werden soll. In diesem Versuch wird diese nun verkürzt auf $t_f = 300 \ s$ und $t_f = 350 \ s$. Für die beiden Versuche werden die Einstellungen aus Tabelle \ref{tab:Versuch1} verwendet.
\begin{table}[H]
    \centering
    \caption{Einstellungen von Versuch 1.1 und 1.2.}\label{tab:Versuch1}
    \begin{tabularx}{.9\textwidth}{Zccc}
        \toprule
        \textbf{Einstellungen} & \textbf{Versuch 1.1} & \textbf{Versuch 1.2} \\
        \midrule
        Reduzierte Endzeit & $t_f = 300 \ s$ & $t_f = 350 \ s$ \\
        Anzahl Diskretisierungen & $N = 100$ & $N = 100$ \\
        Lösungsverfahren der DGL & Explizites Euler Verfahren & Explizites Euler Verfahren \\
        Optimierungsverfahren & SQP-Verfahren & SQP-Verfahren \\
        Startvektor & $z_0 = \begin{pmatrix}
        9000 \\ 
        5 \\ 
        80000 \\
        250 \\
        1259999 \\ 
        1.4
        \end{pmatrix} $ & $z_0 = \begin{pmatrix}
        9000 \\ 
        5 \\ 
        800000 \\
        250 \\
        1259999 \\ 
        1.4
        \end{pmatrix}$ \\
        \bottomrule
    \end{tabularx}
\end{table}
Für die Ergebnisse von Versuch 1.1 (Anhang \ref{kap:Versuch11}) und Versuch 1.2 (Anhang \ref{kap:Versuch12}) wird der folgende technische Aufwand (Tabelle \ref{tab:Versuch1_TA}) benötigt.
\begin{table}[H]
    \centering
    \caption{Technischer Aufwand von Versuch 1.1 und 1.2.}\label{tab:Versuch1_TA}
    \begin{tabularx}{.9\textwidth}{Zccc}
        \toprule
         & \textbf{Versuch 1.1} & \textbf{Versuch 1.2} \\
        \midrule
        Funktionswert der Zielfunktion & $-55444.2939$ & $-57922.1277$ \\
        Anzahl Iterationen & $3023$ & $2989$ \\
        Anzahl Funktionsauswertungen & $1846035$ & $1802834$ \\
        Exit Flag von \textit{MATLAB} & $2$ & $2$ \\
        Optimalität des Ergebnis & $0.0012243$ & $0.00011134$ \\
        Berechnungsdauer & $70.7353 \ min$ & $21.3019 \ min$ \\
        \bottomrule
    \end{tabularx}
\end{table}




\subsection{Ergebnis von Versuch 1.1}\label{kap:Versuch11}
In Versuch 1.1 wird das Ergebnis in Abbildung \ref{img:test_1_1} erreicht. Dies stellt aufgrund des Abbruchkriteriums $EF = 2$ ein mögliches Optimum dar, da die Änderung des Zustandsvektors $z$ kleiner ist die Schrittweitentoleranz $SW_{tol} = 10^{-6}$ und die maximale Verletzung der Beschränkungen kleiner ist als Beschränkungstoleranz $BS_{tol} = 10^{-6}$.
\begin{figure}[H]
\begin{center}
\includegraphics[width=\textwidth]{../code/direct_sol/results/test_1_1}
\MyCaption{Ergebnis von Versuch 1.1}{} \label{img:test_1_1}
\end{center}
\end{figure}
Der Verlauf des beschränkten Staudrucks ist in Abbildung \ref{img:test_0_2_staudruck} dargestellt.
\begin{figure}[H]
\begin{center}
\includegraphics[width=\textwidth]{../code/direct_sol/results/test_1_1_staudruck}
\MyCaption{Überprüfung Staudruck $q(v(t),h(t))$ von Versuch 1.1}{} \label{img:test_1_1_staudruck}
\end{center}
\end{figure}




\subsection{Ergebnis von Versuch 1.2}\label{kap:Versuch12}
In Versuch 1.2 wird das Ergebnis in Abbildung \ref{img:test_1_2} erreicht. Dies stellt aufgrund des Abbruchkriteriums $EF = 2$ ein mögliches Optimum dar, da die Änderung des Zustandsvektors $z$ kleiner ist die Schrittweitentoleranz $SW_{tol} = 10^{-6}$ und die maximale Verletzung der Beschränkungen kleiner ist als Beschränkungstoleranz $BS_{tol} = 10^{-6}$. Die Unstetigkeiten im Verlauf des Anstellwinkels und der Geschwindigkeit deuten jedoch auf kein physikalisch plausibles Ergebnis hin.
\begin{figure}[H]
\begin{center}
\includegraphics[width=\textwidth]{../code/direct_sol/results/test_1_2}
\MyCaption{Ergebnis von Versuch 1.2}{} \label{img:test_1_2}
\end{center}
\end{figure}
Der Verlauf des beschränkten Staudrucks ist in Abbildung \ref{img:test_1_2_staudruck} dargestellt.
\begin{figure}[H]
\begin{center}
\includegraphics[width=\textwidth]{../code/direct_sol/results/test_1_2_staudruck}
\MyCaption{Überprüfung Staudruck $q(v(t),h(t))$ von Versuch 1.2}{} \label{img:test_1_2_staudruck}
\end{center}
\end{figure}
















\newpage
\section{Versuch 2}\label{kap:Versuch2}
Zusätzlich zur verkürzten Endzeit wie in Versuch \ref{kap:Versuch1} wird in Versuch 2 auch die Starthöhe des Flugzeuges angepasst. Für die beiden Versuche werden die Einstellungen aus Tabelle \ref{tab:Versuch2} verwendet.
\begin{table}[H]
    \centering
    \caption{Einstellungen von Versuch 2.1 und 2.2.}\label{tab:Versuch2}
    \begin{tabularx}{.9\textwidth}{Zccc}
        \toprule
        \textbf{Einstellungen} & \textbf{Versuch 2.1} & \textbf{Versuch 2.2} \\
        \midrule
        Reduzierte Endzeit & $t_f = 250 \ s$ & $t_f = 300 \ s$ \\
        Angepasste Starthöhe & $h_0 = 4000 \ m$ & $h_0 = 4000 \ m$ \\
        Anzahl Diskretisierungen & $N = 100$ & $N = 100$ \\
        Lösungsverfahren der DGL & Explizites Euler Verfahren & Explizites Euler Verfahren \\
        Optimierungsverfahren & SQP-Verfahren & SQP-Verfahren \\
        Startvektor & $z_0 = \begin{pmatrix}
        5000 \\ 
        5 \\ 
        8000 \\
        250 \\
        1259999 \\ 
        1.4
        \end{pmatrix} $ & $z_0 = \begin{pmatrix}
        9000 \\ 
        5 \\ 
        8000 \\
        250 \\
        1259999 \\ 
        1.4
        \end{pmatrix}$ \\
        \bottomrule
    \end{tabularx}
\end{table}
Für die Ergebnisse von Versuch 2.1 (Anhang \ref{kap:Versuch21}) und Versuch 2.2 (Anhang \ref{kap:Versuch22}) wird der folgende technische Aufwand (Tabelle \ref{tab:Versuch2_TA}) benötigt.
\begin{table}[H]
    \centering
    \caption{Technischer Aufwand von Versuch 2.1 und 2.2.}\label{tab:Versuch2_TA}
    \begin{tabularx}{.9\textwidth}{Zccc}
        \toprule
         & \textbf{Versuch 2.1} & \textbf{Versuch 2.2} \\
        \midrule
        Funktionswert der Zielfunktion & $-17067.0955$ & $-70301.9291$ \\
        Anzahl Iterationen & $1259$ & $4699$ \\
        Anzahl Funktionsauswertungen & $771293$ & $2826056$ \\
        Exit Flag von \textit{MATLAB} & $-2$ & $2$ \\
        Optimalität des Ergebnis & $116.9551$ & $0.0004094$ \\
        Berechnungsdauer & $398.7814 \ min$ & $20.6724 \ min$ \\
        \bottomrule
    \end{tabularx}
\end{table}




\subsection{Ergebnis von Versuch 2.1}\label{kap:Versuch21}
In Versuch 2.1 wird das Ergebnis in Abbildung \ref{img:test_2_1} erreicht. Dies stellt aufgrund des Abbruchkriteriums $EF = -2$ kein Optimum des Problems dar (kein zulässiger Punkt gefunden).
\begin{figure}[H]
\begin{center}
\includegraphics[width=\textwidth]{../code/direct_sol/results/test_2_1}
\MyCaption{Ergebnis von Versuch 2.1}{} \label{img:test_2_1}
\end{center}
\end{figure}
Der Verlauf des beschränkten Staudrucks ist in Abbildung \ref{img:test_0_2_staudruck} dargestellt.
\begin{figure}[H]
\begin{center}
\includegraphics[width=\textwidth]{../code/direct_sol/results/test_2_1_staudruck}
\MyCaption{Überprüfung Staudruck $q(v(t),h(t))$ von Versuch 2.1}{} \label{img:test_2_1_staudruck}
\end{center}
\end{figure}




\subsection{Ergebnis von Versuch 2.2}\label{kap:Versuch22}
In Versuch 2.2 wird das Ergebnis in Abbildung \ref{img:test_1_2} erreicht. Dies stellt aufgrund des Abbruchkriteriums $EF = 2$ ein mögliches Optimum dar, da die Änderung des Zustandsvektors $z$ kleiner ist die Schrittweitentoleranz $SW_{tol} = 10^{-6}$ und die maximale Verletzung der Beschränkungen kleiner ist als Beschränkungstoleranz $BS_{tol} = 10^{-6}$.
\begin{figure}[H]
\begin{center}
\includegraphics[width=\textwidth]{../code/direct_sol/results/test_2_2}
\MyCaption{Ergebnis von Versuch 2.2}{} \label{img:test_2_2}
\end{center}
\end{figure}
Der Verlauf des beschränkten Staudrucks ist in Abbildung \ref{img:test_0_2_staudruck} dargestellt.
\begin{figure}[H]
\begin{center}
\includegraphics[width=\textwidth]{../code/direct_sol/results/test_2_2_staudruck}
\MyCaption{Überprüfung Staudruck $q(v(t),h(t))$ von Versuch 2.2}{} \label{img:test_2_2_staudruck}
\end{center}
\end{figure}














\newpage
\section{Versuch 3}\label{kap:Versuch3}
Zusätzlich zur verkürzten Endzeit wie in Versuch 1 (Anhang \ref{kap:Versuch1}) wird nun zusätzlich das Gewicht des Flugzeuges angepasst. Das maximale Startgewicht eines A380-800 beträgt max. $560000 \ kg$ \cite{A380Tech}. Gewählt wird eine Masse von $500000 \ kg$. Für die beiden Versuche werden die Einstellungen aus Tabelle \ref{tab:Versuch3} verwendet.
\begin{table}[H]
    \centering
    \caption{Einstellungen von Versuch 3.1 und 3.2.}\label{tab:Versuch3}
    \begin{tabularx}{.9\textwidth}{Zccc}
        \toprule
        \textbf{Einstellungen} & \textbf{Versuch 3.1} & \textbf{Versuch 3.2} \\
        \midrule
        Reduzierte Endzeit & $t_f = 550 \ s$ & $t_f = 600 \ s$ \\
        Angepasstes Startgewicht & $m = 500000 \ kg$ & $m = 500000 \ kg$ \\
        Anzahl Diskretisierungen & $N = 200$ & $N = 200$ \\
        Lösungsverfahren der DGL & Explizites Euler Verfahren & Explizites Euler Verfahren \\
        Optimierungsverfahren & SQP-Verfahren & SQP-Verfahren \\
        Startvektor & $z_0 = \begin{pmatrix}
        20 \\ 
        9 \\ 
        6000 \\
        90 \\
        1259999 \\ 
        1.47
        \end{pmatrix} $ & $z_0 = \begin{pmatrix}
        20 \\ 
        9 \\ 
        6000 \\
        90 \\
        1259999 \\ 
        1.47
        \end{pmatrix}$ \\
        \bottomrule
    \end{tabularx}
\end{table}
Für die Ergebnisse von Versuch 3.1 (Anhang \ref{kap:Versuch31}) und Versuch 3.2 (Anhang \ref{kap:Versuch32}) wird der folgende technische Aufwand (Tabelle \ref{tab:Versuch3_TA}) benötigt.
\begin{table}[H]
    \centering
    \caption{Technischer Aufwand von Versuch 3.1 und 3.2.}\label{tab:Versuch3_TA}
    \begin{tabularx}{.9\textwidth}{Zccc}
        \toprule
         & \textbf{Versuch 3.1} & \textbf{Versuch 3.2} \\
        \midrule
        Funktionswert der Zielfunktion & $-103327.3544$ & $-140216.741$ \\
        Anzahl Iterationen & $15763$ & $6864$ \\
        Anzahl Funktionsauswertungen & $18975672$ & $8260083$ \\
        Exit Flag von \textit{MATLAB} & $2$ & $2$ \\
        Optimalität des Ergebnis & $0.0011298$ & $0.00015495$ \\
        Berechnungsdauer & $394.887 \ min$ & $262.6699 \ min$ \\
        \bottomrule
    \end{tabularx}
\end{table}





\subsection{Ergebnis von Versuch 3.1}\label{kap:Versuch31}
In Versuch 3.1 wird das Ergebnis in Abbildung \ref{img:test_3_1} erreicht. Dies stellt aufgrund des Abbruchkriteriums $EF = 2$ ein mögliches Optimum dar, da die Änderung des Zustandsvektors $z$ kleiner ist die Schrittweitentoleranz $SW_{tol} = 10^{-12}$ und die maximale Verletzung der Beschränkungen kleiner ist als Beschränkungstoleranz $BS_{tol} = 10^{-6}$.
\begin{figure}[H]
\begin{center}
\includegraphics[width=\textwidth]{../code/direct_sol/results/test_3_1}
\MyCaption{Ergebnis von Versuch 3.1}{} \label{img:test_3_1}
\end{center}
\end{figure}
Der Verlauf des beschränkten Staudrucks ist in Abbildung \ref{img:test_3_1_staudruck} dargestellt.
\begin{figure}[H]
\begin{center}
\includegraphics[width=\textwidth]{../code/direct_sol/results/test_3_1_staudruck}
\MyCaption{Überprüfung Staudruck $q(v(t),h(t))$ von Versuch 3.1}{} \label{img:test_3_1_staudruck}
\end{center}
\end{figure}




\subsection{Ergebnis von Versuch 3.2}\label{kap:Versuch32}
In Versuch 3.2 wird das Ergebnis in Abbildung \ref{img:test_3_2} erreicht. Dies stellt aufgrund des Abbruchkriteriums $EF = 2$ ein mögliches Optimum dar, da die Änderung des Zustandsvektors $z$ kleiner ist die Schrittweitentoleranz $SW_{tol} = 10^{-12}$ und die maximale Verletzung der Beschränkungen kleiner ist als Beschränkungstoleranz $BS_{tol} = 10^{-6}$.
\begin{figure}[H]
\begin{center}
\includegraphics[width=\textwidth]{../code/direct_sol/results/test_3_2}
\MyCaption{Ergebnis von Versuch 3.2}{} \label{img:test_3_2}
\end{center}
\end{figure}
Der Verlauf des beschränkten Staudrucks ist in Abbildung \ref{img:test_3_2_staudruck} dargestellt.
\begin{figure}[H]
\begin{center}
\includegraphics[width=\textwidth]{../code/direct_sol/results/test_3_2_staudruck}
\MyCaption{Überprüfung Staudruck $q(v(t),h(t))$ von Versuch 3.2}{} \label{img:test_3_2_staudruck}
\end{center}
\end{figure}












\newpage
\section{Versuch 4}\label{kap:Versuch4}
Zusätzlich zur verkürzten Endzeit wie in Versuch 1 (Anhang \ref{kap:Versuch1}) wird nun zusätzlich das Gewicht des Flugzeuges angepasst, sowie Box-Beschränkungen für die Zustände gesetzt. Diese zusätzlichen Box-Beschränkungen spiegeln die technischen Eigenschaften des Flugzeuges A380-800 der Firma Airbus wieder \cite{A380Tech}:
\begin{itemize}
\item \textbf{Höchstgeschwindigkeit:} Die maximale Geschwindigkeit beträgt $960 \ \frac{km}{h}$ ($266 \ \frac{m}{s}$).
%
\item \textbf{Maximale Flughöhe:} Die maximale Flughöhe beträgt $13100 \ m$.
%
\item \textbf{Maximale Reichweite:} Die maximale Reichweite bei maximaler Auslastung beträgt $15200000 \ m$.
%
\item \textbf{Anstellwinkel:} Für die Boxschranken des Anstellwinkels wurde $\gamma \in [-80^{\circ} , 80^{\circ}]$ gewählt. 
\end{itemize}
Damit ergibt sich das veränderte Optimalsteuerungsproblem:
\begin{align*}
\min_{U} F(X,U) &:= -(x(t_f) - x_0) \\
\text{unter} \hspace{20mm} \dot{X}(t) &= f(X(t),U(t)) = (\dot{h}(t),\dot{\gamma}(t),\dot{x}(t),\dot{v}(t))^T \\
(h,\gamma,x,v)(t_0) &= (h_0,\gamma_0,x_0,v_0) \\
(h,\gamma,x,v)(t_f) &= (h_f,\gamma_f) \hspace{36mm} \\
q(v(t),h(t)) &\leq q_{\max} & & \forall t \in [t_0,t_f] \\
h(t) &\in [0,13100] & & \forall t \in [t_0,t_f] \\
\gamma(t) &\in [-80,80] & & \forall t \in [t_0,t_f] \\
x(t) &\in [0,15200000] & & \forall t \in [t_0,t_f] \\
v(t) &\in [0,266] & & \forall t \in [t_0,t_f] \\
T(t) &\in [T_{\min},T_{\max}] & & \forall t \in [t_0,t_f] \\
C_L(t) &\in [C_{L, \min},C_{L, \max}] & & \forall t \in [t_0,t_f]
\end{align*}
Für die beiden Versuche werden die Einstellungen aus Tabelle \ref{tab:Versuch4} verwendet.
\begin{table}[H]
    \centering
    \caption{Einstellungen von Versuch 4.1 und 4.2}
    \begin{tabularx}{.9\textwidth}{Zccc}
        \toprule
        \textbf{Einstellungen} & \textbf{Versuch 4.1} & \textbf{Versuch 4.2} \\
        \midrule
        Reduzierte Endzeit & $t_f = 550 \ s$ & $t_f = 600 \ s$ \\
        Angepasstes Startgewicht & $m = 500000 \ kg$ & $m = 500000 \ kg$ \\
        Anzahl Diskretisierungen & $N = 100$ & $N = 100$ \\
        Lösungsverfahren der DGL & Explizites Euler Verfahren & Explizites Euler Verfahren \\
        Optimierungsverfahren & SQP-Verfahren & SQP-Verfahren \\
        Startvektor & $z_0 = \begin{pmatrix}
        20 \\ 
        9 \\ 
        6000 \\
        90 \\
        1259999 \\ 
        1.47
        \end{pmatrix} $ & $z_0 = \begin{pmatrix}
        20 \\ 
        9 \\ 
        6000 \\
        90 \\
        1259999 \\ 
        1.47
        \end{pmatrix}$ \\
        \bottomrule
    \end{tabularx}
\end{table}
Für die Ergebnisse von Versuch 4.1 (Anhang \ref{kap:Versuch41}) und Versuch 4.2 (Anhang \ref{kap:Versuch42}) wird der folgende technische Aufwand (Tabelle \ref{tab:Versuch4_TA}) benötigt.
\begin{table}[H]
    \centering
    \caption{Technischer Aufwand von Versuch 4.1 und 4.2.}\label{tab:Versuch4_TA}
    \begin{tabularx}{.9\textwidth}{Zccc}
        \toprule
         & \textbf{Versuch 4.1} & \textbf{Versuch 4.2} \\
        \midrule
        Funktionswert der Zielfunktion & $-104259.8227$ & $-132811.3929$ \\
        Anzahl Iterationen & $2105$ & $2813$ \\
        Anzahl Funktionsauswertungen & $1277653$ & $1692519$ \\
        Exit Flag von \textit{MATLAB} & $2$ & $2$ \\
        Optimalität des Ergebnis & $0.0023675$ & $0.0004094$ \\
        Berechnungsdauer & $19.1492 \ min$ & $0.00045547 \ min$ \\
        \bottomrule
    \end{tabularx}
\end{table}




\subsection{Ergebnis von Versuch 4.1}\label{kap:Versuch41}
In Versuch 4.1 wird das Ergebnis in Abbildung \ref{img:test_4_1} erreicht. Dies stellt aufgrund des Abbruchkriteriums $EF = 2$ ein mögliches Optimum dar, da die Änderung des Zustandsvektors $z$ kleiner ist die Schrittweitentoleranz $SW_{tol} = 10^{-6}$ und die maximale Verletzung der Beschränkungen kleiner ist als Beschränkungstoleranz $BS_{tol} = 10^{-6}$.
\begin{figure}[H]
\begin{center}
\includegraphics[width=\textwidth]{../code/direct_sol/results/test_4_1}
\MyCaption{Ergebnis von Versuch 4.1}{} \label{img:test_4_1}
\end{center}
\end{figure}
Der Verlauf des beschränkten Staudrucks ist in Abbildung \ref{img:test_4_1_staudruck} dargestellt.
\begin{figure}[H]
\begin{center}
\includegraphics[width=\textwidth]{../code/direct_sol/results/test_4_1_staudruck}
\MyCaption{Überprüfung Staudruck $q(v(t),h(t))$ von Versuch 4.1}{} \label{img:test_4_1_staudruck}
\end{center}
\end{figure}




\subsection{Ergebnis von Versuch 4.2}\label{kap:Versuch42}
In Versuch 4.2 wird das Ergebnis in Abbildung \ref{img:test_4_2} erreicht. Dies stellt aufgrund des Abbruchkriteriums $EF = 2$ ein mögliches Optimum dar, da die Änderung des Zustandsvektors $z$ kleiner ist die Schrittweitentoleranz $SW_{tol} = 10^{-6}$ und die maximale Verletzung der Beschränkungen kleiner ist als Beschränkungstoleranz $BS_{tol} = 10^{-6}$.
\begin{figure}[H]
\begin{center}
\includegraphics[width=\textwidth]{../code/direct_sol/results/test_4_2}
\MyCaption{Ergebnis von Versuch 4.2}{} \label{img:test_4_2}
\end{center}
\end{figure}
Der Verlauf des beschränkten Staudrucks ist in Abbildung \ref{img:test_4_2_staudruck} dargestellt.
\begin{figure}[H]
\begin{center}
\includegraphics[width=\textwidth]{../code/direct_sol/results/test_4_2_staudruck}
\MyCaption{Überprüfung Staudruck $q(v(t),h(t))$ von Versuch 4.2}{} \label{img:test_4_2_staudruck}
\end{center}
\end{figure}