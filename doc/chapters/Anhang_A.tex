\chapter{Programmcode direkte Lösungsverfahren und Versuchsprotokolle}\label{Anhang:DirektV}

\section{Aufbau Programmcode}
Für das Lösen des Optimalsteuerungsproblem mit direkten Lösungsverfahren wurde folgende Programmcodestruktur aufgebaut:
\begin{itemize}
\item \texttt{config}: In diesem Ordner sind die Versuchsskripte abgespeichert. Beispielhaft ist in Programmcode \ref{code:direct_test_4_1} das Versuchsskript von Versuch Null Ergebnis eins (Kapitel \ref{kap:Versuch41}) dargestellt.
%
\item \texttt{results}: In diesem Ordner werden nach erfolgreicher Berechnung des entsprechenden Versuchs die Lösungsdaten und Lösungsgraphik, als zum Beispiel \texttt{test\_0\_1.txt} und \texttt{test\_0\_1.png}, abgespeichert.
\item \texttt{main.m}: Mit diesem Matlab Skript wird das entsprechende Versuchsskript geladen und mit dem direkten Lösungsverfahren berechnet.
%
\item \texttt{maximalrangeflight.m}: In dieser Klasse (Programmcode \ref{code:direct_maximalrangeflight}) sind alle Parameter und Funktionen des Problems beziehungsweise Funktionen für das Lösen mit der Matlab Funktion \texttt{fmincon} definiert. 
%
\item \texttt{ode\_methods}: Diese Klasse (Programmcode \ref{code:direct_odemethods}) stellt explizite und implizite Algorithmen zum lösen von gewöhnlichen Differentialgleichungen zur Verfügung. Diese werden in der Problemklasse \texttt{maximalrangeflight.m} benötigt.
\end{itemize}

Allgemein wird für die Matlab Funktion \verb|fmincon| die Startmatrix  \verb|prob.z_0| 
\[\begin{pmatrix}
[20,9,6000,90,1259999,1.47] \\ 
\vdots \\ 
\vdots
\end{pmatrix}\]
für die Optimierung verwendet. Es sind also alle Zustandsvektoren für jeden Diskretisierungspunkt identisch.

Nachfolgend sind die Programmcodes für das direkte Lösungsverfahren aufgelistet:

\begin{lstlisting}[style=num_octave, caption={Matlab Programmcode von Versuchsskript \texttt{test\_4\_1.m}.}, label=code:direct_test_4_1]
%% Speicher Parameter
results_name = 'test_4_1';
%% Testparameter
N = 100;               % Anzahl an Diskretisierungen
z_0 = [     20,...     % h_start in [m]
             9,...     % gamma_start in [Grad]  
          6000,...     % x_start in [m]
            90,...     % v_start in [m/s]
       1259999,...     % T_start in [N]
          1.47];       % C_L_start in []
X_0 = [   0;           % h_0 in [m]
       0.27;           % gamma_0 in [Grad]
          0;           % x_0 in [m]
        100];          % v_0 in [m/s]
X_T = [10668;          % h_t in [m]
           0];         % gamma_t  in [Grad]
params = [       0,... % t_0:   Anfangszeitpunkt in [s]
               550,... % t_f:   Endzeitpunkt in [s]
          1.247015,... % alpha: Parameter zur Berechung der Luftdichte in []
          0.000104,... % beta: 
              9.81,... % g:     Erdbeschleunigung in [N/s^2]
             0.032,... % C_D_0: Nullluftwiderstandsbeiwert in []
               0.8,... % e:     Oswaldfaktor in []
               845,... % F:     Wirksame Flaeche in [m^2]
               7.5,... % AR:    Fluegelstreckung in []
            500000,... % m:     Leergewicht des A380 in [kg]
             44154];   % q_max: Maximaler Staudruck in [N/m^2]  
t = linspace(params(1),params(2),N);
%% Boxbeschraenkungen
lb = [     0.0,...
         -80.0,...
           0.0,...
           0.0,...
           0.0,...     % T_min:   Minimale Schubkraft in [N]
           0.0];       % C_L_min: Minimaler Auftriebsbeiwert in []
ub = [   13100,...
          80.0,...
      15200000,...
           266,...
       1260000,...     % T_max:   Maximale Schubkraft in [N]
          1.48];       % C_L_max: Maximaler Auftriebsbeiwert in []
%% Loesungsmethode der ODE und Objekt der Problemklasse erhalten
ode_methods = ode_methods();
ode_method = @ode_methods.explicit_euler;
prob = MaximalRangeFlight(N,t,z_0,X_0,X_T,params,lb,ub,ode_method);
%% Optionen fuer fmincon von Matlab
options = optimoptions('fmincon','Display','iter','Algorithm','sqp','MaxFunctionEvaluations',2000.0e+03,'MaxIterations',4.0e+05,'UseParallel',true);
\end{lstlisting}







\begin{lstlisting}[style=num_octave, caption={Matlab Programmcode von \texttt{main.m}.}, label=code:direct_main]
clear  variables; close  all; clc;
%% Memory paths
addpath('../utils');
addpath('./config');
addpath('./results');
%% Loading the corresponding configuration file
test_4_1;
%% Solving the control problem with fmincon
prob_sol = fmincon(@prob.F_sol,prob.z_0,[],[],[],[],prob.lb,prob.ub,@prob.nonlcon,options);
%% Plot the solution
plotter = Plotter();
titles = ["Flughoehe","Anstellwinkel","Zurueckgelegte Streckte",...
          "Geschwindigkeit","\textbf{Steuerung 1: Schub}",...
          "\textbf{Steuerung 2: Auftriebsbeiwert}"];
labels = ["$h_{sol}$ in $[m]$","$\gamma_{sol}$ in $[^{\circ}]$",...
          "$x_{sol}$ in $[m]$","$v_{sol}$ in $[\frac{m}{s}]$",...
          "$T_{sol}$ in $[N]$","$C_{L_{sol}}$ in $[1]$"];
frame_prop = [0.5,0.5,0.5,0.5,2,2];
line_style = ["b-","b-","b-","b-","r-","r-"];
order = [3,1,5,2,4,6];
fig = plotter.plot_fmincon(prob.t,prob_sol,results_name,titles,labels,order,frame_prop,line_style);
%% Save the graphics and the data
saveas(fig,strcat('./results/',results_name,'.png'));
writematrix(prob_sol,strcat('./results/',results_name,'.txt'));
\end{lstlisting}







\begin{lstlisting}[style=num_octave, caption={Matlab Programmcode von \texttt{maximalrangeflight.m}.}, label=code:direct_maximalrangeflight]
classdef MaximalRangeFlight
    properties
        %% Parameters for the optimal control problem
        t_0                 % Starting time in [s]
        t_f                 % End time in [s]
        X_0 = [   0;        % h_0 in [m]
               0.27;        % gamma_0 in [rad]
                  0;        % x_0 in [m]
                100];       % v_0 in [m/s]
        X_T = [10668;       % h_t in [m]
                   0];      % gamma_t in [Grad]
        alpha = 1.247015;   % Parameters for calculating the air density in []
        beta = 0.000104;    % Parameters for calculating the air density in []
        g = 9.81;           % Gravitational acceleration in [N/s^2]
        C_D_0 = 0.032;      % Zero air resistance coefficient in []             
        e = 0.8;            % Oswaldfactor in []
        F = 845;            % Effective area in [m^2]
        AR = 7.5;           % Aspect ratio in []
        k;                  % Factor for calculation of the drag coefficient in []
        m = 276800;         % Empty weight of the A380 in [kg]
        q_max = 44154;      % Maximum dynamic pressure in [N/m^2]
        T_min = 0;          % Minimum thrust in [N]
        T_max = 1260000;    % Maximum thrust in [N]
        C_L_min = 0.0;      % Minimum lift coefficient in []
        C_L_max = 1.48;     % Maximum lift coefficient in []
        %% Problem sizes
        n_x = 4;            % Size of the state vector
        n_u = 2;            % Control vector size
        n_s = 1;            % Size of the pure state constraint vector
        n_psi = 8;          % Size of the boundary condition vector
        %% Variables for the calculation
        N;                  % Number of discretisations
        t;                  % Variable for the time vector
        ode_method;         % Implicit or explicit iterative methods to approximate the solution of a ODE
        z_0;                % Array for the start values for fmincon
        lb;                 % lower bounds for fmincon
        ub;                 % upper bounds for fmincon
    end
    methods
        %% Constructor: Set initial conditions of the direct solver
        function obj = MaximalRangeFlight(N,t,z_0,X_0,X_T,params,lb,ub,ode_method)
            % Grid size
            obj.N = N;
            obj.t = t;
            % Method for solving the ODE
            obj.ode_method = ode_method;
            % Start vector fmincon
            if obj.N == size(z_0,1)
                obj.z_0 = z_0;
            else
                obj.z_0 = z_0 .* ones(obj.N,obj.n_x+obj.n_u);
            end
            % Box conditions
            obj.lb = lb .* ones(obj.N,obj.n_x+obj.n_u);
            obj.ub = ub .* ones(obj.N,obj.n_x+obj.n_u);
            % Parameters
            obj.t_0 = params(1);
            obj.t_f = params(2);
            obj.X_0 = X_0;
            obj.X_T = X_T;
            obj.alpha = params(3);
            obj.beta = params(4);
            obj.g = params(5);
            obj.C_D_0 = params(6);
            obj.e = params(7);
            obj.F = params(8);
            obj.AR = params(9);
            obj.k = 1/(pi*obj.e*obj.AR);
            obj.m = params(10);
            obj.q_max = params(11);
        end
        %% Differential equation
        function X = f(obj,t,z)
            X = [z(4)*sind(z(2)),...
                 (1/(2*obj.m*z(4))) * (obj.F*z(6)*(z(4))^2*obj.alpha*exp(-obj.beta*z(1)) - 2*obj.m*obj.g*cosd(z(2))),...
                 z(4)*cosd(z(2)),...
                 (1/(2*obj.m)) * (2*z(5) + (-obj.C_D_0 - obj.k*(z(6))^2)*obj.F*(z(4))^2*obj.alpha*exp(-obj.beta*z(1)) - 2*obj.m*obj.g*sind(z(2)))];
        end
        %% Objective function
        function T = F_sol(obj,z)
            T = -(z(end,3)-obj.X_0(3));
        end
        %% Nonlinear constraint functions
        function [c,ceq] = nonlcon(obj,z)
            % Equality constraints
            c = 0.5 * obj.alpha * exp(-obj.beta*z(:,1)) .* (z(:,4)).^2 - obj.q_max; % (n_c+n_s)*(N+1)=(0+1)*((100-1)+1)=100
            % Inequality constraints
            x = obj.ode_method(@obj.f,obj.t,z,obj.N,obj.n_x);
            ceq = zeros(1,obj.n_x*(obj.N-1)+obj.n_psi); % (n_x*N)+n_psi=(4*(100-1))+8=404
            ceq(1:obj.n_x*(obj.N-1)) = (z(1:end-1,1:obj.n_x) + x - z(2:end,1:obj.n_x))';
            ceq((obj.n_x*(obj.N-1)+1):(obj.n_x*(obj.N-1)+obj.n_psi)) = [  z(1,1)-obj.X_0(1),...
                                                                          z(1,2)-obj.X_0(2),...
                                                                          z(1,3)-obj.X_0(3),...
                                                                          z(1,4)-obj.X_0(4),...
                                                                        z(end,1)-obj.X_T(1),...
                                                                        z(end,2)-obj.X_T(2),...
                                                                                          0,...
                                                                                          0];
        end
    end
end
\end{lstlisting}













\newpage
\section{Versuch 0}\label{kap:Versuch0}
Bei diesem Versuch wird das gegebene Problem ohne Veränderungen mit dem direkten Lösungsverfahren gelöst.

\subsection{Ergebnis 1}\label{kap:Versuch01}
Verwendet wird:
\begin{itemize}
\item Anzahl an Diskretisierungen: $N = 100$ 
\item Startvektor für \texttt{fmincon}: $z_0 = (9000,5,800000,250,1259999,1.4)$
\item Explizites Euler Verfahren zum Lösen der DGL
\end{itemize}
Es folgt das Ergebnis
\begin{figure}[H]
\begin{center}
\includegraphics[width=\textwidth]{../code/direct_sol/results/test_0_1}
\caption{Versuch 1 - Ergebnis 1}\label{img:test_0_1}
\end{center}
\end{figure}

\begin{figure}[H]
\begin{center}
\includegraphics[width=\textwidth]{../code/direct_sol/results/test_0_1_staudruck}
\caption{Untersuchung, ob sich der Staudruck bei Versuch \ref{kap:Versuch01} dem maximalen Staudruck annähert.}\label{img:test_0_1_staudruck}
\end{center}
\end{figure}




\newpage
\subsection{Ergebnis 2}\label{kap:Versuch02}
Verwendet wird:
\begin{itemize}
\item Anzahl an Diskretisierungen: $N = 100$ 
\item Startvektor für \texttt{fmincon}: $z_0 = (9000,5,800000,250,1259999,1.4)$
\item Explizites Euler Verfahren zum Lösen der DGL
%
\item Innere-Punkte-Verfahren
\end{itemize}
Es folgt das Ergebnis
\begin{figure}[H]
\begin{center}
\includegraphics[width=\textwidth]{../code/direct_sol/results/test_0_2}
\caption{Versuch 1 - Ergebnis 2}\label{img:test_0_2}
\end{center}
\end{figure}

\begin{figure}[H]
\begin{center}
\includegraphics[width=\textwidth]{../code/direct_sol/results/test_0_2_staudruck}
\caption{Untersuchung, ob sich der Staudruck bei Versuch \ref{kap:Versuch02} dem maximalen Staudruck annähert.}\label{img:test_0_2_staudruck}
\end{center}
\end{figure}











\newpage
\section{Versuch 1}\label{kap:Versuch1}
Die Problemstellung gibt eine Endzeit beziehungsweise einen Zeitraum vor in welchem das Problem gelöst werden soll. In diesem Versuch soll diese nun verkürzt werden, um Beobachtungen zu erhalten wie sich die optimale Lösung aufgrund der Veränderungen verhält. %$z_0 = (20,9,6000,90,1259999,1.47)$

\subsection{Ergebnis 1}\label{kap:Versuch11}
Verwendet wird:
\begin{itemize}
\item Anzahl an Diskretisierungen: $N = 100$ 
\item Reduzierte Endzeit: $t_f = 300 \ [s]$
\item Startvektor für \texttt{fmincon}: $z_0 = (9000,5,800000,250,1259999,1.4)$
\item Explizites Euler Verfahren zum Lösen der DGL
%
\item SQP-Verfahren
\end{itemize}
Es folgt das Ergebnis
\begin{figure}[H]
\begin{center}
\includegraphics[width=\textwidth]{../code/direct_sol/results/test_1_1}
\caption{Versuch 1 - Ergebnis 1}\label{img:test_1_1}
\end{center}
\end{figure}

\begin{figure}[H]
\begin{center}
\includegraphics[width=\textwidth]{../code/direct_sol/results/test_1_1_staudruck}
\caption{Untersuchung, ob sich der Staudruck bei Versuch \ref{kap:Versuch11} dem maximalen Staudruck annähert.}\label{img:test_1_1_staudruck}
\end{center}
\end{figure}



\newpage
\subsection{Ergebnis 2}\label{kap:Versuch12}
Verwendet wird:
\begin{itemize}
\item Anzahl an Diskretisierungen: $N = 100$ 
\item Reduzierte Endzeit: $t_f = 350 \ [s]$
\item Startvektor für \texttt{fmincon}: $z_0 = (9000,5,800000,250,1259999,1.4)$
\item Explizites Euler Verfahren zum Lösen der DGL
%
\item SQP-Verfahren
\end{itemize}
Es folgt das Ergebnis
\begin{figure}[H]
\begin{center}
\includegraphics[width=\textwidth]{../code/direct_sol/results/test_1_2}
\caption{Versuch 1 - Ergebnis 2}\label{img:test_1_2}
\end{center}
\end{figure}

\begin{figure}[H]
\begin{center}
\includegraphics[width=\textwidth]{../code/direct_sol/results/test_1_2_staudruck}
\caption{Untersuchung, ob sich der Staudruck bei Versuch \ref{kap:Versuch12} dem maximalen Staudruck annähert.}\label{img:test_1_2_staudruck}
\end{center}
\end{figure}



\newpage
\subsection{Ergebnis 3}\label{kap:Versuch13}
Verwendet wird:
\begin{itemize}
\item Anzahl an Diskretisierungen: $N = 100$ 
\item Reduzierte Endzeit: $t_f = 300 \ [s]$
\item Startvektor für \texttt{fmincon}: $z_0 = (9000,5,800000,250,1259999,1.4)$
\item Explizites Euler Verfahren zum Lösen der DGL
%
\item Innere-Punkte-Verfahren
\end{itemize}
Es folgt das Ergebnis
\begin{figure}[H]
\begin{center}
\includegraphics[width=\textwidth]{../code/direct_sol/results/test_1_3}
\caption{Versuch 1 - Ergebnis 3}\label{img:test_1_3}
\end{center}
\end{figure}

\begin{figure}[H]
\begin{center}
\includegraphics[width=\textwidth]{../code/direct_sol/results/test_1_3_staudruck}
\caption{Untersuchung, ob sich der Staudruck bei Versuch \ref{kap:Versuch13} dem maximalen Staudruck annähert.}\label{img:test_1_3_staudruck}
\end{center}
\end{figure}















\newpage
\section{Versuch 2}\label{kap:Versuch2}
Zusätzlich zur verkürzten Endzeit wie in Versuch 1 (Kapitel \ref{kap:Versuch1}) wird nun zusätzlich die Starthöhe des Flugzeuges angepasst.

\subsection{Ergebnis 1}\label{kap:Versuch21}
Verwendet werden:
\begin{itemize}
\item Anzahl an Diskretisierungen: $N = 100$ 
\item Reduzierte Endzeit: $t_f = 250 \ [s]$
\item Angepasste Starthöhe: $h_0 = 4000 \ [m]$
\item Startvektor für \texttt{fmincon}: $z_0 = (5000,5,8000,250,1259999,1.4)$
\item Explizites Euler Verfahren zum Lösen der DGL
%
\item SQP-Verfahren
\end{itemize}
Es folgt das Ergebnis
\begin{figure}[H]
\begin{center}
\includegraphics[width=\textwidth]{../code/direct_sol/results/test_2_1}
\caption{Versuch 2 - Ergebnis 1}\label{img:test_2_1}
\end{center}
\end{figure}

\begin{figure}[H]
\begin{center}
\includegraphics[width=\textwidth]{../code/direct_sol/results/test_2_1_staudruck}
\caption{Untersuchung, ob sich der Staudruck bei Versuch \ref{kap:Versuch21} dem maximalen Staudruck annähert.}\label{img:test_2_1_staudruck}
\end{center}
\end{figure}



\newpage
\subsection{Ergebnis 2}\label{kap:Versuch22}
Verwendet werden:
\begin{itemize}
\item Anzahl an Diskretisierungen: $N = 100$ 
\item Reduzierte Endzeit: $t_f = 300 \ [s]$
\item Angepasste Starthöhe: $h_0 = 4000 \ [m]$
\item Startvektor für \texttt{fmincon}: $z_0 = (5000,5,8000,250,1259999,1.4)$
\item Explizites Euler Verfahren zum Lösen der DGL
%
\item SQP-Verfahren
\end{itemize}
Es folgt das Ergebnis
\begin{figure}[H]
\begin{center}
\includegraphics[width=\textwidth]{../code/direct_sol/results/test_2_2}
\caption{Versuch 2 - Ergebnis 3}\label{img:test_2_2}
\end{center}
\end{figure}

\begin{figure}[H]
\begin{center}
\includegraphics[width=\textwidth]{../code/direct_sol/results/test_2_2_staudruck}
\caption{Untersuchung, ob sich der Staudruck bei Versuch \ref{kap:Versuch22} dem maximalen Staudruck annähert.}\label{img:test_2_2_staudruck}
\end{center}
\end{figure}



\newpage
\subsection{Ergebnis 3}\label{kap:Versuch23}
Verwendet werden:
\begin{itemize}
\item Anzahl an Diskretisierungen: $N = 100$ 
\item Reduzierte Endzeit: $t_f = 250 \ [s]$
\item Angepasste Starthöhe: $h_0 = 4000 \ [m]$
\item Startvektor für \texttt{fmincon}: $z_0 = (5000,5,8000,250,1259999,1.4)$
\item Explizites Euler Verfahren zum Lösen der DGL.
%
\item Innere-Punkte-Verfahren
\end{itemize}
Es folgt das Ergebnis
\begin{figure}[H]
\begin{center}
\includegraphics[width=\textwidth]{../code/direct_sol/results/test_2_3}
\caption{Versuch 2 - Ergebnis 3}\label{img:test_23}
\end{center}
\end{figure}

\begin{figure}[H]
\begin{center}
\includegraphics[width=\textwidth]{../code/direct_sol/results/test_2_3_staudruck}
\caption{Untersuchung, ob sich der Staudruck bei Versuch \ref{kap:Versuch23} dem maximalen Staudruck annähert.}\label{img:test_2_3_staudruck}
\end{center}
\end{figure}













\newpage
\section{Versuch 3}\label{kap:Versuch3}
Zusätzlich zur verkürzten Endzeit wie in Versuch 1 (Kapitel \ref{kap:Versuch1}) wird nun zusätzlich das Gewicht des Flugzeuges angepasst. Das maximale Startgewicht eines A380-800 beträgt max. $560.000 \ [kg]$ \cite{A380Tech}. Gewählt wird eine Masse von $500.000 \ [kg]$.

\subsection{Ergebnis 1}\label{kap:Versuch31}
Verwendet werden:
\begin{itemize}
\item Anzahl an Diskretisierungen: $N = 100$ 
\item Reduzierte Endzeit: $t_f = 550 \ [s]$
\item Neues Startgewicht: $m = 500000 \ [kg]$
\item Startvektor für \texttt{fmincon}: $z_0 = (20,9,6000,90,1259999,1.47)$
\item Explizites Euler Verfahren zum Lösen der DGL
%
\item SQP-Verfahren
\end{itemize}
Es folgt das Ergebnis
\begin{figure}[H]
\begin{center}
\includegraphics[width=\textwidth]{../code/direct_sol/results/test_3_1}
\caption{Versuch 3 - Ergebnis 1}\label{img:test_3_1}
\end{center}
\end{figure}

\begin{figure}[H]
\begin{center}
\includegraphics[width=\textwidth]{../code/direct_sol/results/test_3_1_staudruck}
\caption{Untersuchung, ob sich der Staudruck bei Versuch \ref{kap:Versuch31} dem maximalen Staudruck annähert.}\label{img:test_3_1_staudruck}
\end{center}
\end{figure}



\newpage
\subsection{Ergebnis 2}\label{kap:Versuch32}
Verwendet werden:
\begin{itemize}
\item Anzahl an Diskretisierungen: $N = 100$ 
\item Reduzierte Endzeit: $t_f = 600 \ [s]$
\item Neues Startgewicht: $m = 500000 \ [kg]$
\item Startvektor für \texttt{fmincon}: $z_0 = (20,9,6000,90,1259999,1.47)$
\item Explizites Euler Verfahren zum Lösen der DGL
%
\item SQP-Verfahren
\end{itemize}
Es folgt das Ergebnis
\begin{figure}[H]
\begin{center}
\includegraphics[width=\textwidth]{../code/direct_sol/results/test_3_2}
\caption{Versuch 3 - Ergebnis 2}\label{img:test_3_2}
\end{center}
\end{figure}

\begin{figure}[H]
\begin{center}
\includegraphics[width=\textwidth]{../code/direct_sol/results/test_3_2_staudruck}
\caption{Untersuchung, ob sich der Staudruck bei Versuch \ref{kap:Versuch32} dem maximalen Staudruck annähert.}\label{img:test_3_2_staudruck}
\end{center}
\end{figure}



\newpage
\subsection{Ergebnis 3}\label{kap:Versuch33}
Verwendet werden:
\begin{itemize}
\item Anzahl an Diskretisierungen: $N = 100$ 
\item Reduzierte Endzeit: $t_f = 550 \ [s]$
\item Neues Startgewicht: $m = 500000 \ [kg]$
\item Startvektor für \texttt{fmincon}: $z_0 = (20,9,6000,90,1259999,1.47)$
\item Explizites Euler Verfahren zum Lösen der DGL
%
\item Innere-Punkte-Verfahren
\end{itemize}
Es folgt das Ergebnis
\begin{figure}[H]
\begin{center}
\includegraphics[width=\textwidth]{../code/direct_sol/results/test_3_3}
\caption{Versuch 3 - Ergebnis 3}\label{img:test_3_3}
\end{center}
\end{figure}

\begin{figure}[H]
\begin{center}
\includegraphics[width=\textwidth]{../code/direct_sol/results/test_3_3_staudruck}
\caption{Untersuchung, ob sich der Staudruck bei Versuch \ref{kap:Versuch33} dem maximalen Staudruck annähert.}\label{img:test_3_3_staudruck}
\end{center}
\end{figure}











\newpage
\section{Versuch 4}\label{kap:Versuch4}
Zusätzlich zur verkürzten Endzeit wie in Versuch 1 (Kapitel \ref{kap:Versuch1}) wird nun zusätzlich das Gewicht des Flugzeuges angepasst, sowie Box-Beschränkungen für die Zustände gesetzt. Diese zusätzlichen Box-Beschränkungen spiegeln die technischen Eigenschaften des Flugzeuges A380-800 der Firma Airbus wieder \cite{A380Tech}:
\begin{itemize}
\item \textbf{Höchstgeschwindigkeit:} Die maximale Geschwindigkeit beträgt $960 \ [km/h]$ ($266 \ [m/s]$).
%
\item \textbf{Maximale Flughöhe:} Die maximale Flughöhe beträgt $13.100 \ [m]$.
%
\item \textbf{Maximale Reichweite:} Die maximale Reichweite bei maximaler Auslastung beträgt $15.200.000 \ [m]$.
%
\item \textbf{Anstellwinkel:} Für die Boxschranken des Anstellwinkels wurde $\gamma \in [-80,80] \ [^{\circ}]$ gewählt. 
\end{itemize}
Damit ergibt sich das veränderte Optimalsteuerungsproblem:
\begin{align*}
\min_{T, C_L} F(h,\gamma,x,v,T,C_L) &:= -(x(t_f) - x_0) \\
\text{unter} \hspace{20mm} \dot{h}(t) &= v(t) \sin(\gamma(t)) \\
\dot{\gamma}(t) &=  \dfrac{1}{mv(t)} \left( L(v(t),h(t),C_L(t)) - mg \cos(\gamma(t)) \right) \\
\dot{x}(t) &= v(t) \cos(\gamma(t))\\
\dot{v}(t) &= \dfrac{1}{m} \left( T(t) - D(v(t),h(t),C_L(t)) - mg \sin(\gamma(t)) \right) \\
(h,\gamma,x,v)(t_0) &= (h_0,\gamma_0,x_0,v_0) \\
(h,\gamma,x,v)(t_f) &= (h_f,\gamma_f) \hspace{36mm} \\
q(v(t),h(t)) &\leq q_{\max} & & \forall t \in [t_0,t_f] \\
h(t) &\in [0,13100] & & \forall t \in [t_0,t_f] \\
\gamma(t) &\in [-80,80] & & \forall t \in [t_0,t_f] \\
x(t) &\in [0,15200000] & & \forall t \in [t_0,t_f] \\
v(t) &\in [0,266] & & \forall t \in [t_0,t_f] \\
T(t) &\in [T_{\min},T_{\max}] & & \forall t \in [t_0,t_f] \\
C_L(t) &\in [C_{L, \min},C_{L, \max}] & & \forall t \in [t_0,t_f]
\end{align*}

\newpage
\subsection{Ergebnis 1}\label{kap:Versuch41}
Verwendet werden:
\begin{itemize}
\item Anzahl an Diskretisierungen: $N = 100$ 
\item Reduzierte Endzeit: $t_f = 550 \ [s]$
\item Neues Startgewicht: $m = 500000 \ [kg]$
\item Startvektor für \texttt{fmincon}: $z_0 = (20,9,6000,90,1259999,1.47)$
\item Explizites Euler Verfahren zum Lösen der DGL
%
\item SQP-Verfahren
\end{itemize}
Es folgt das Ergebnis
\begin{figure}[H]
\begin{center}
\includegraphics[width=\textwidth]{../code/direct_sol/results/test_4_1}
\caption{Versuch 4 - Ergebnis 1}\label{img:test_4_1}
\end{center}
\end{figure}

\begin{figure}[H]
\begin{center}
\includegraphics[width=\textwidth]{../code/direct_sol/results/test_4_1_staudruck}
\caption{Untersuchung, ob sich der Staudruck bei Versuch \ref{kap:Versuch41} dem maximalen Staudruck annähert.}\label{img:test_4_1_staudruck}
\end{center}
\end{figure}



\newpage
\subsection{Ergebnis 2}\label{kap:Versuch42}
Verwendet werden:
\begin{itemize}
\item Anzahl an Diskretisierungen: $N = 100$ 
\item Reduzierte Endzeit: $t_f = 600 \ [s]$
\item Neues Startgewicht: $m = 500000 \ [kg]$
\item Startvektor für \texttt{fmincon}: $z_0 = (20,9,6000,90,1259999,1.47)$
\item Explizites Euler Verfahren zum Lösen der DGL
%
\item SQP-Verfahren
\end{itemize}
Es folgt das Ergebnis
\begin{figure}[H]
\begin{center}
\includegraphics[width=\textwidth]{../code/direct_sol/results/test_4_2}
\caption{Versuch 4 - Ergebnis 2}\label{img:test_4_2}
\end{center}
\end{figure}

\begin{figure}[H]
\begin{center}
\includegraphics[width=\textwidth]{../code/direct_sol/results/test_4_2_staudruck}
\caption{Untersuchung, ob sich der Staudruck bei Versuch \ref{kap:Versuch42} dem maximalen Staudruck annähert.}\label{img:test_4_2_staudruck}
\end{center}
\end{figure}



\newpage
\subsection{Ergebnis 3}\label{kap:Versuch43}
Verwendet werden:
\begin{itemize}
\item Anzahl an Diskretisierungen: $N = 100$ 
\item Reduzierte Endzeit: $t_f = 550 \ [s]$
\item Neues Startgewicht: $m = 500000 \ [kg]$
\item Startvektor für \texttt{fmincon}: $z_0 = (20,9,6000,90,1259999,1.47)$
\item Explizites Euler Verfahren zum Lösen der DGL
%
\item Innere-Punkte-Verfahren
\end{itemize}
Es folgt das Ergebnis
\begin{figure}[H]
\begin{center}
\includegraphics[width=\textwidth]{../code/direct_sol/results/test_4_3}
\caption{Versuch 4 - Ergebnis 3}\label{img:test_4_3}
\end{center}
\end{figure}

\begin{figure}[H]
\begin{center}
\includegraphics[width=\textwidth]{../code/direct_sol/results/test_4_3_staudruck}
\caption{Untersuchung, ob sich der Staudruck bei Versuch \ref{kap:Versuch43} dem maximalen Staudruck annähert.}\label{img:test_4_3_staudruck}
\end{center}
\end{figure}





%\newpage
%\section{Versuch 5}
%Testet doch mal, ob ihr bei vorgegebener Steuerung einen deutlichen Unterschied zwischen expl. und impl. Verfahren feststellt. Falls nicht ist das schon mal ein deutliches Indiz dafür, dass ihr euch den Rechenaufwand eines impl. Verfahrens auch in der Optimierung sparen könnt. Ich denke letztere ist eher die Herausforderung bei eurem Projekt, da es wahrscheinlich viele lokale Minima gibt. Deswegen wäre es wahrscheinlich auch sinnvoll unterschiedliche Algorithmen (z.B. SQP) auszuprobieren. Dafür müsst ihr nur die entsprechende Option für fmincon setzen und müsst auch nicht unbedingt den Algorithmus im Detail verstehen. Sowas könnt ihr dann natürlich auch gerne in die Präsentation/Ausarbeitung aufnehmen.
%
%\textbf{Steht noch aus !!!}





%\newpage
%\section{Versuch 6}
%Von Explizit auf implizites Verfahren
%
%Ich denke auch nicht unbedingt, dass die Stabilität des ODE-Solvers oder die Steifheit der ODE das (Haupt-)Problem ist. Wahrscheinlich könnt ihr ein explizites Verfahren verwenden.
%
%\textbf{Steht noch aus !!!}







%\newpage
%\section{Versuch 7}
%Gitterverfeinerung mit Runge-Kutta 3 und 4 Verfahren.
%
%\textbf{Steht noch aus !!!}










%\begin{itemize}
%\item \textbf{Startgeschwindigkeit:} Gewöhnliche Flugzeuge starten von einer Startbahn und benötigen zum Starten eine Mindestgeschwindigkeit relativ zur umgebenden Luft. Diese beträgt bei Verkehrsflugzeugen zwischen 250 (69,44 m/s) und 345 km/h (95,83 m/s). (Quelle: \url{https://de.wikipedia.org/wiki/Start_(Luftfahrt)})
%%
%\item \textbf{Steigflug:} Aus Lärmschutzgründen und aus Gründen des „Freimachens“ von Luftraum starten Verkehrsflugzeuge von großen Flughäfen oft mit steilem Neigungswinkel, etwa um 20° (abhängig von den Vorgaben des Herstellers, der Beladung, der Windverhältnisse und des Lotsen). Der Steigwinkel wird in der Regel verringert, sobald die Flughafenumgebung verlassen wurde. (Quelle: \url{https://de.wikipedia.org/wiki/Steigflug})
%%
%\item \textbf{Flugzeugmassen:} Die Maximale Startmasse eines A380-800 beträgt max. 569 t. Die Leermasse 275 t. (Quelle: \url{https://de.wikipedia.org/wiki/Airbus_A380})
%%
%\item \textbf{Höchstgeschwindigkeit:} 961 km/h (266,94 m/s). (Quelle: \url{https://de.wikipedia.org/wiki/Airbus_A380})
%%
%\item \textbf{Maximale Flughöhe:} 13100 m. (Quelle: \url{https://de.wikipedia.org/wiki/Airbus_A380})
%%
%\item \textbf{Maximale Reichweite:} 15200000 m (Quelle: \url{https://de.wikipedia.org/wiki/Airbus_A380})
%%
%\item \textbf{Anstellwinkel:} $\gamma \in [-80,80]$
%\end{itemize}