\chapter{Bestimmung der optimalen Endzeit des Modells}\label{kap:OptTf}

\section{Transformation des gegebenen Optimalsteuerungsproblems}
Bei Problem \ref{prob:MaxRF} ist das Ziel die zurückgelegte Strecke $x(t)$ zu maximieren. Dieses Problem lässt sich aber auch in ein Problem transformieren, bei welchem die Endzeit $t_f$, unter Einhaltung der Start- und Endbedingungen, minimiert werden soll. Hierfür müssen die Funktionen $g$ und $f_0$ angepasst werden, um anschließend das Problem mit freier Endzeit auf ein Problem mit fester Endzeit zu transformieren \cite{Gerdts2011}.

Für die Funktionen $g : \R^{n_X} \to \R$ und $f_0 : \R^{n_X} \times \R^{n_U} \to \R$ ergibt sich somit
\begin{itemize}
\item $g(X(t_f)) \equiv 0$
%
\item $f_0(X(t),U(t)) = 1$ 
\end{itemize}
Anschließend wird eine neue feste Endzeit $\tilde{t}_f = 1$, sowie eine neue Zeitvariable $s \in [t_0,\tilde{t}_f]$ mit der Transformation $t = s \cdot t_f$ definiert. Für die neue Zeitvariable gilt nun
\[\begin{split}
\tilde{X}(s) &= x(s \cdot t_f) \\\
\tilde{U}(s) &= u(s \cdot t_f)
\end{split}\] 
Nun wird nach der Zeitvariablen $s$ differenziert:
\[\dfrac{d \tilde{X}(s)}{ds} = \dfrac{d X(s \cdot t_f)}{ds} = t_f \dfrac{d x(s \cdot t_f)}{dt} = t_f f(s \cdot t_f,\tilde{X}(s),\tilde{U}(s)) \]
Damit lässt sich das Zielfunktional umschreiben. Aus
\[\begin{split}
F(X,U) &= g(x(t_f)) + \int^{t_f}_{t_0} f_0(t,X(t),U(t)) dt \\\
\Rightarrow F(\tilde{X},\tilde{U}) &= g(\tilde{X}(\tilde{t}_f)) + t_f \int^{\tilde{t}_f = 1}_{t_0 = 0} f_0(s \cdot t_f,\tilde{X}(s),\tilde{U}(s)) ds 
\end{split}\]
Zusätzlich benötigt man eine neue Zustandsvariable $\tilde{X}_5(s) \equiv t_f$, für die gilt: \[\dfrac{d \tilde{X}_5(s)}{ds} = 0 \ \ \ \ \tilde{X}_5(t_0) =\ \text{frei} \ \ \ \ \tilde{X}_5(\tilde{t}_f) =\ \text{frei}\], sodass die Endzeit $t_f$ minimiert wird. Somit ergibt sich das neue autonome Lagrange-Problem (Problem \ref{prob:MaxRFEndzeit}) mit den Funktionen $f_0 : \R^{n_{\tilde{X}}} \times \R^{n_{\tilde{U}}} \to \R$, $f : \R^{n_{\tilde{X}}} \times \R^{n_{\tilde{U}}} \to \R^{n_{\tilde{X}}}$ und $\tilde{U} : [t_0,\tilde{t}_f] \to \R^m$ für $0 \leq (n_{\psi} = 2) \leq (n_{\tilde{X}} = 5)$ und $n_{\tilde{U}} = 2$.

\begin{problem}[Angepasstes Optimalsteuerungsproblem - Maximal-Range-Flight]\label{prob:MaxRFEndzeit}
Für das Optimalsteuerungsproblem ergibt sich mit dem Zustandsvektor
\[\tilde{X}(s) = (h(s),\gamma(s),x(s),v(s),t_f)^T\]
und der Steuerfunktion
\[\tilde{U}(s) = (T(s),C_L(s))^T\]
das Problem:
\begin{align*}
\min_{U} F(\tilde{X},\tilde{U}) &:= t_f \int^{\tilde{t}_f = 1}_{t_0 = 0} f_0(s \cdot t_f,\tilde{X}(s),\tilde{U}(s)) ds = t_f & & \\\
\text{unter} \hspace{10mm} \dot{\tilde{X}}(t) &= t_f f(s \cdot t_f,\tilde{X}(s),\tilde{U}(s))\\\
&= \begin{pmatrix}
t_f v(t) \sin(\gamma(t)) \\ 
t_f \left( \dfrac{F \alpha e^{-\beta h(t)} v(t) C_L(t)}{2m} - \dfrac{g \cos(\gamma(t))}{v(t)} \right) \\ 
t_f v(t) \cos(\gamma(t)) \\ 
t_f \left( \dfrac{T(t)}{m} - \dfrac{(C_{D_0} + k C_L^2(t)) F \alpha e^{-\beta h(t)} v^2(t)}{2m} - g \sin(\gamma(t)) \right) \\
0
\end{pmatrix} & & \\\
(h,\gamma,x,v)(t_0) &= (h_0,\gamma_0,x_0,v_0) & & \\\
(h,\gamma)(\tilde{t}_f) &= (h_f,\gamma_f) & & \\\
q(v(s),h(s)) &\leq q_{\max} & & \forall s \in [t_0,\tilde{t}_f]\\\
U(s) &= (T(s),C_L(s))^T \in \mathcal{U} & & \forall s \in [t_0,\tilde{t}_f]
\end{align*}
Des Weiteren sei $\psi : \R^{n_{\tilde{X}}} \to \R^{n_{\psi}}$ eine $C^1$-Funktion
\[\psi(X(\tilde{t}_f)) = 
\begin{pmatrix}
    h(\tilde{t}_f) - h_f \\ 
    \gamma(\tilde{t}_f) - \gamma_f
\end{pmatrix} = 0_{n_{\psi}}\]
\end{problem}











\section{Versuche}
Bei diesem Versuch wird das neu aufgestellte Problem \ref{prob:MaxRFEndzeit} mit den Parametern aus Tabelle \ref{tab:ProblemPara} gelöst, wobei bei Versuch \ref{kap:Versuch1_OptTf} das Startgewicht angepasst wird. Es ergeben sich somit für die beiden Versuche die folgenden Einstellungen aus Tabelle \ref{tab:Versuche_OptTf}.
\begin{table}[H]
    \centering
    \caption{Einstellungen von Versuch 0 und 1.}\label{tab:Versuche_OptTf}
    \begin{tabularx}{.9\textwidth}{Zccc}
        \toprule
        \textbf{Einstellungen} & \textbf{Versuch 0} & \textbf{Versuch 1} \\
        \midrule
        Angepasstes Startgewicht &  & $m = 500000 \ kg$ \\
        Anzahl Diskretisierungen & $N = 100$ & $N = 100$ \\
        Lösungsverfahren der DGL & Explizites Euler Verfahren & Explizites Euler Verfahren \\
        Optimierungsverfahren & SQP-Verfahren & SQP-Verfahren \\
        Startvektor & $z_0 = \begin{pmatrix}
        9000 \\ 
        5 \\ 
        800000 \\
        250 \\
        1259999 \\ 
        1,4
        \end{pmatrix} $ & $z_0 = \begin{pmatrix}
        9000 \\ 
        5 \\ 
        800000 \\
        250 \\
        1259999 \\ 
        1,4
        \end{pmatrix}$ \\
        \bottomrule
    \end{tabularx}
\end{table}
Für die Ergebnisse von Versuch 0 (Anhang \ref{kap:Versuch0_OptTf}) und Versuch 1 (Anhang \ref{kap:Versuch1_OptTf}) wird der folgende technische Aufwand (Tabelle \ref{tab:Versuch_TA}) benötigt.
\begin{table}[H]
    \centering
    \caption{Technischer Aufwand von Versuch 0 und 1.}\label{tab:Versuch_TA}
    \begin{tabularx}{.9\textwidth}{Zccc}
        \toprule
         & \textbf{Versuch 0} & \textbf{Versuch 1} \\
        \midrule
        Funktionswert der Zielfunktion & $279,2113$ & $515,8404$ \\
        Anzahl Iterationen & $7153$ & $9551$ \\
        Anzahl Funktionsauswertungen & $5034191$ & $6715691$ \\
        Exit Flag von \textit{MATLAB} & $1$ & $1$ \\
        Optimalität des Ergebnis & $1,2877 \cdot 10^{-07}$ & $1,1043 \cdot 10^{-07}$ \\
        Berechnungsdauer & $47,5949 \ min$ & $57,8622 \ min$ \\
        \bottomrule
    \end{tabularx}
\end{table}




\subsection{Ergebnis Versuch 0}\label{kap:Versuch0_OptTf}
In Versuch 0 wird das Ergebnis in Abbildung \ref{img:test_0_OptTf} erreicht. Dies stellt aufgrund des Abbruchkriteriums $EF = 1$ ein Optimum des Problems da, da das Optimalitätsmaß kleiner als die Optimalitätstoleranz $OT_{tol} = 10^{-6}$ und die maximale Verletzung der Beschränkungen kleiner ist als Beschränkungstoleranz $BS_{tol} = 10^{-6}$ ist.
\begin{figure}[H]
\begin{center}
\includegraphics[width=\textwidth]{../code/Freie_Endzeit_sol/results/test_0}
\caption{Ergebnis von Versuch 0.}\label{img:test_0_OptTf}
\end{center}
\end{figure}
Der Verlauf des Beschränkten Staudrucks ist in Abbildung \ref{img:test_0_staudruck_OptTf} dargestellt.
\begin{figure}[H]
\begin{center}
\includegraphics[width=\textwidth]{../code/Freie_Endzeit_sol/results/test_0_staudruck}
\caption{Überprüfung Staudruck $q(v(t),h(t))$ von Versuch 0.} \label{img:test_0_staudruck_OptTf}
\end{center}
\end{figure}





\subsection{Ergebnis Versuch 1}\label{kap:Versuch1_OptTf}
In Versuch 1 wird das Ergebnis in Abbildung \ref{img:test_1_OptTf} erreicht. Dies stellt aufgrund des Abbruchkriteriums $EF = 1$ ein Optimum des Problems da, da das Optimalitätsmaß kleiner als die Optimalitätstoleranz $OT_{tol} = 10^{-6}$ und die maximale Verletzung der Beschränkungen kleiner ist als Beschränkungstoleranz $BS_{tol} = 10^{-6}$ ist.
\begin{figure}[H]
\begin{center}
\includegraphics[width=\textwidth]{../code/Freie_Endzeit_sol/results/test_1}
\caption{Ergebnis von Versuch 1.}\label{img:test_1_OptTf}
\end{center}
\end{figure}
Der Verlauf des Beschränkten Staudrucks ist in Abbildung \ref{img:test_1_staudruck_OptTf} dargestellt.
\begin{figure}[H]
\begin{center}
\includegraphics[width=\textwidth]{../code/Freie_Endzeit_sol/results/test_1_staudruck}
\caption{Überprüfung Staudruck $q(v(t),h(t))$ von Versuch 1.}\label{img:test_1_staudruck_OptTf}
\end{center}
\end{figure}