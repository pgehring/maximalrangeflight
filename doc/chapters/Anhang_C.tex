\chapter{Modellparameter}\label{Anhang:ModellPara}

\begin{table}[H]
    \centering
    \caption{Modellparameter für das Flugzeug A380-800 der Firma Airbus.}\label{tab:ProblemPara}
    \begin{tabularx}{.9\textwidth}{lXrl}
        \toprule
        \textbf{Parameter}   & \textbf{Bedeutung} & \textbf{Wert} & \textbf{Einheit} \\ 
        \midrule
        $t_0$       & Anfangszeitpunkt & $0$ & $s$ \\ 
        $t_f$       & Endzeitpunkt & $1800$ & $s$ \\ 
        \hline
        $h_0$       & Anfangshöhe & $0$ & $m$ \\ 
        $\gamma_0$  & Anfangsanstellwinkel & $0.27$ & $^{\circ}$ \\
        $x_0$       & Anfangskoordinate & $0$ & $m$ \\ 
        $v_0$       & Anfangsgeschwindigkeit & $100$ & $\frac{m}{s}$ \\ 
        \hline
        $h_f$       & Endhöhe & $10668$ & $m$ \\ 
        $\gamma_f$  & Endanstellwinkel & $0$ & $^{\circ}$ \\
        \hline
        $\alpha$    & Konstante für die Luftdichte & $1.247015$ & $\frac{kg}{m^3}$\\ 
        $\beta$     & Konstante für die Luftdichte & $0.000104$ & $\frac{1}{m}$\\
        $g$         & Erdbeschleunigung & $9.81$ & $\frac{m}{s^2}$ \\ 
        $C_{D_0}$   & Nullluftwiderstandsbeiwert & $0.032$ & $1$\\ 
        $AR$        & Streckung & $7.5$ & $1$\\ 
        $e$         & Oswaldfaktor & $0.8$ & $1$\\ 
        $F$         & wirksame Fläche & $845$ & $m^2$ \\ 
        $m$         & Masse & $276800$ & $kg$ \\ 
        $q_{\max}$  & maximaler Staudruck & $44154$ & $\frac{N}{m^2}$ \\
        $T_{\min}$  & minimale Schubkraft & $0$ & $N$ \\  
        $T_{\max}$  & maximale Schubkraft & $1260000$ & $N$ \\ 
        $C_{L, \min}$ & minimaler Auftriebsbeiwert & $0$ & $1$ \\ 
        $C_{L, \max}$ & maximaler Auftriebsbeiwert & $1.48$ & $1$ \\ 
        \bottomrule
    \end{tabularx} 
\end{table}

\chapter{Zweipunkt-Randwertproblem des Optimalssteuerungsproblems} \label{Anhang:Jacobi}
Das Randwertproblem aus Problem \ref{prob:ZweiRand} benötigt die Ableitung der Funktion \(G(t, Z(t), U(t))\) bezüglich aller Einträge von \(Z\), also
\begin{equation} 
    \dot{Z}(t) = G(t,Z(t),U(t)) = 
    \begin{pmatrix}
        \dot{h}(t),\dot{\gamma}(t),\dot{x}(t),\dot{v}(t),\dot{\lambda}_1(t),\dot{\lambda}_2(t),\dot{\lambda}_3(t),\dot{\lambda}_4(t)
    \end{pmatrix}^T
\end{equation}

Die bereits in \autoref{cha:indirect} eingeführte Jacobimatrix der Matrix \(G\) ergibt sich zu
\begin{equation} \label{equ:jacobi}
    \dfrac{\partial G(t,Z(t),U(t))}{\partial Z} = 
    \begin{pmatrix}
        0 & J_G^{(1,2)} & 0 & J_G^{(1,4)} & 0 & 0 & 0 & 0 \\ 
        J_G^{(2,1)} & J_G^{(2,2)} & 0 & J_G^{(2,4)} & 0 & 0 & 0 & 0 \\ 
        0 & J_G^{(3,2)} & 0 & J_G^{(3,4)} & 0 & 0 & 0 & 0 \\ 
        J_G^{(4,1)} & J_G^{(4,2)} & 0 & J_G^{(4,4)} & 0 & 0 & 0 & 0 \\
        J_G^{(5,1)} & 0 & 0 & J_G^{(5,4)} & 0 & J_G^{(5,6)} & 0 & J_G^{(5,8)} \\
        0 & J_G^{(6,2)} & 0 & J_G^{(6,4)} & J_G^{(6,5)} & J_G^{(6,6)} & J_G^{(6,7)} & J_G^{(6,8)} \\
        0 & 0 & 0 & 0 & 0 & 0 & 0 & 0 \\
        J_G^{(8,1)} & J_G^{(8,2)} & 0 & J_G^{(8,4)} & J_G^{(8,5)} & J_G^{(8,6)} & J_G^{(8,7)} & J_G^{(8,8)}
    \end{pmatrix}
\end{equation}

Die Einträge in Gleichung \eqref{equ:jacobi} werden für die numerische Lösung des Randwertproblem benötigt und können wie folgt aufgeschrieben werden.
\begin{align}
    J_G^{(1,2)} &= v(t) \cos(\gamma(t)) \\
    J_G^{(1,4)} &= \sin(\gamma(t)) \\
    J_G^{(2,1)} &= - \dfrac{F \alpha \beta e^{-\beta h(t)} v(t) C_L(t)}{2m} \\
    J_G^{(2,2)} &= \dfrac{g \sin(\gamma(t))}{v(t)} \\
    J_G^{(2,4)} &= \dfrac{F \alpha e^{-\beta h(t)} C_L(t)}{2m} + \dfrac{g \cos(\gamma(t))}{v^2(t)} \\
    J_G^{(3,2)} &= - v(t) \sin(\gamma(t)) \\
    J_G^{(3,4)} &= \cos(\gamma(t)) \\
    J_G^{(4,1)} &= \dfrac{(C_{D_0} + k C_L^2(t)) F \alpha \beta e^{-\beta h(t)} v^2(t)}{2m} \\
    J_G^{(4,2)} &= - g \cos(\gamma(t)) \\
    J_G^{(4,4)} &= -\dfrac{(C_{D_0} + k C_L^2(t)) F \alpha e^{-\beta h(t)} v(t)}{m}  \\
    J_G^{(5,1)} &= \dfrac{\alpha \beta^2 F e^{-\beta h(t)} C_L(t) v(t) \lambda_2(t)}{2m} - \dfrac{(C_{D_0}+k C_L^2(t)) \alpha \beta^2 F e^{-\beta h(t)} v^2(t) \lambda_4(t)}{2m} \\
    J_G^{(5,4)} &= - \dfrac{\alpha \beta F e^{-\beta h(t)} C_L(t) \lambda_2(t)}{2m} + \dfrac{(C_{D_0}+k C_L^2(t)) \alpha \beta F e^{-\beta h(t)} v(t) \lambda_4(t)}{m} \\
    J_G^{(5,6)} &= - \dfrac{\alpha \beta F e^{-\beta h(t)} C_L(t) v(t)}{2m}\\
    J_G^{(5,8)} &= \dfrac{(C_{D_0}+k C_L^2(t)) \alpha \beta F e^{-\beta h(t)} v^2(t)}{2m} \\
    J_G^{(6,2)} &= -\sin(\gamma(t)) v(t) \lambda_1(t) + \dfrac{g \cos(\gamma(t)) \lambda_2(t)}{v(t)} - \cos(\gamma(t)) v(t) \lambda_3(t) + \sin(\gamma(t)) g \lambda_4(t) \\
    J_G^{(6,4)} &= \cos(\gamma(t)) \lambda_1(t) - \dfrac{g \sin(\gamma(t)) \lambda_2(t)}{v^2(t)} - \sin(\gamma(t)) \lambda_3(t) \\
    J_G^{(6,5)} &= \cos(\gamma(t)) v(t) \\
    J_G^{(6,6)} &= \dfrac{g \sin(\gamma(t))}{v(t)} \\
    J_G^{(6,7)} &= - \sin(\gamma(t)) v(t) \\
    J_G^{(6,8)} &= - \cos(\gamma(t)) g \\
    J_G^{(8,1)} &= -\dfrac{F \alpha \beta e^{-\beta h(t)} C_L(t) \lambda_2(t)}{2m}  + \dfrac{(C_{D_0} + k C_L^2(t)) F \alpha \beta e^{-\beta h(t)} v(t) \lambda_4(t)}{m} \\
    J_G^{(8,2)} &= \cos(\gamma(t)) \lambda_1(t) - \dfrac{g \sin(\gamma(t)) \lambda_2(t)}{v^2(t)} - \sin(\gamma(t)) \lambda_3(t) \\
    J_G^{(8,4)} &= - \dfrac{2 g \cos(\gamma(t)) \lambda_2(t)}{v^3(t)} - \dfrac{(C_{D_0} + k C_L^2(t)) F \alpha e^{-\beta h(t)} \lambda_4(t)}{m} \\
    J_G^{(8,5)} &= \sin(\gamma(t)) \\
    J_G^{(8,6)} &= \dfrac{F \alpha e^{-\beta h(t)} C_L(t)}{2m} + \dfrac{g \cos(\gamma(t))}{v^2(t)} \\
    J_G^{(8,7)} &= \cos(\gamma(t)) \\
    J_G^{(8,8)} &= - \dfrac{(C_{D_0} + k C_L^2(t)) F \alpha e^{-\beta h(t)} v(t)}{m} 
\end{align}