\chapter{Optimalsteuerungsproblem}

\section{Problemformulierung \glqq Maximal range flight\grqq{}}
Modelliert werden soll das Flugzeug A380-800 der Firma Airbus. Dabei seien:
\begin{itemize}
    \item $x(t)$: $x$-Koordinate des Massenschwerpunktes $S$
    %
    \item $h(t)$: $h$-Koordinate des Massenschwerpunktes $S$
    %
    \item $v(t)$: Geschwindigkeit
    %
    \item $\gamma(t)$: Anstellwinkel
    %
    \item $T(t)$: Schub, Steuerung
    %
    \item $C_L(t)$: Auftriebsbeiwert, Steuerung
\end{itemize}
Um die Kräfte welche auf das Flugzeug einwirken berechnen zu können, werden folgende Hilfsgrößen benötigt:
\begin{itemize}
    \item Luftwiderstandsbeiwert: \[C_D(C_L(t)) := C_{D_0} + k \cdot C^2_L(t) \ \ \ \ \text{mit} \ \ \ \ k = \dfrac{1}{\pi \cdot e \cdot AR}\] wobei $C_{D0}$ der Nullluftwiderstandsbeiwert, $e$ die Oswaldfaktor und $AR$ die Streckung (engl. \textit{aspect ratio}) bezeichnet.
    %
    \item Luftdichte: \[\rho(h(t)) := \alpha \cdot e^{-\beta \cdot h(t)}\] %https://wind-data.ch/tools/luftdichte.php
    %
    \item Staudruck: \[q(v(t), h(t)) := \dfrac{\rho(h(t)) \cdot v^2(t)}{2} \]
\end{itemize}
Die Kräfte lassen sich dann berechnen mit:
\begin{itemize}
    \item Auftriebskraft: \[L(v(t), h(t), C_L(t)) := F \cdot C_L(t) \cdot q(v(t), h(t))\] wobei $F$ die wirksame Fläche, d.h. die von der Luft angeströmte Fläche, ist.
    %
    \item Luftwiderstand: \[D(v(t), h(t), C_L(t)) := F \cdot C_D(C_L(t)) \cdot q(v(t), h(t))\]
    \item Erdanziehungskraft: \[W = m \cdot g\]
\end{itemize}

Mit dem Newtonsschen Gesetz $F = m * a$ lässt sich die Differentialgleichungen für die Geschwindigkeit $v$ aufstellen:
\[\begin{split}
    F &:= m \cdot a \\\
    \Rightarrow \dot{v}(t) &= a(t) = \dfrac{F(t)}{m} = \dfrac{T(t) - D(v(t),h(t),C_L(t)) - W \sin(\gamma(t))}{m}
\end{split} \]
Mit der Gleichung für die Zentripetalkraft $F_{ZP} = \dfrac{m v^2}{r}$ lässt sich die Differentialgleichungen für den Ansstellwinkel $\gamma$ aufstellen:
\[\begin{split}
    F_{ZP} &:= \dfrac{m v^2}{r} \\\
    \Rightarrow \dot{\gamma}(t) &:= \dfrac{v(t)}{r} = \dfrac{F_{ZP}(t)}{m v(t)} = \dfrac{L(v(t),h(t),C_L(t)) - W \cos(\gamma(t))}{m v(t)}
\end{split} \]
Die Differentialgleichungen für die $h(t)$ und $x(t)$ lassen mittels Geschwindigkeit und Anstellwinkel bestimmen:
\[\begin{split}
    \dot{x}(t) &= v(t) \cos(\gamma(t))\\\
    \dot{h}(t) &= v(t) \sin(\gamma(t))
\end{split} \]
Es ergibt sich somit das Optimalsteuerungsproblem:
\begin{align*}
    \min_{T, C_L} F(h,\gamma,x,v,T,C_L) &:= -(x(t_f) - x_0) & & \\\
    \text{unter} \hspace{20mm} \dot{h}(t) &= v(t) \sin(\gamma(t)) \hspace{27mm} & & \\\
    \dot{\gamma}(t) &=  \dfrac{L(v(t),h(t),C_L(t)) - W \cos(\gamma(t))}{mv(t)} & & \\\
    \dot{x}(t) &= v(t) \cos(\gamma(t)) & & \\\
    \dot{v}(t) &= \dfrac{T(t) - D(v(t),h(t),C_L(t)) - W \sin(\gamma(t))}{m} & & \\\
    %
    (h,\gamma,x,v)(t_0) &= (h_0,\gamma_0,x_0,v_0) & & \\\
    (h,\gamma)(t_f) &= (h_f,\gamma_f) & & \\\
    %
    q(v(t),h(t)) &\leq q_{\max} \hspace{19.5mm} & & \forall t \in [t_0,t_f]\\\
    T(t) &\in [T_{\min},T_{\max}] & & \forall t \in [t_0,t_f] \\\
    C_L(t) &\in [C_{L, \min},C_{L, \max}] & & \forall t \in [t_0,t_f]
\end{align*}
Für das Modell werden folgende Parameter aus Tabelle \ref{tab:ProblemPara} verwendet.
\begin{table}[H]
    \begin{center}
        \caption{Problem Parameter für das Flugzeug A380-800 der Firma Airbus.}\label{tab:ProblemPara}
        \begin{tabular}{|l|l|r|l|}
            \hline
            Parameter & Bedeutung & Wert & Einheit \\ 
            \hline 
            \hline
            $t_0$ & Anfangszeitpunkt & $0$ & $[s]$ \\ 
            $t_f$ & Endzeitpunkt & $1800$ & $[s]$ \\ 
            \hline
            $h_0$ & Anfangshöhe & $0$ & $[m]$ \\ 
            $\gamma_0$ & Anfangsanstellwinkel & $0.27$ & $[Grad]$ \\
            $x_0$ & Anfangskoordinate & $0$ & $[m]$ \\ 
            $v_0$ & Anfangsgeschwindigkeit & $100$ & $[m/s]$ \\ 
            \hline
            $h_f$ & Endhöhe & $10668$ & $[m]$ \\ 
            $\gamma_f$ & Endanstellwinkel & $0$ & $[Grad]$ \\
            \hline
            $\alpha$ &  & $1.247015$ & $[]$ \\ 
            $\beta$ &  & $0.000104$ & $[]$ \\
            $g$ & Erdbeschleunigung & $9.81$ & $[m/s^2]$ \\ 
            $C_{D_0}$ & Nullluftwiderstandsbeiwert & $0.032$ & $[]$ \\ 
            $AR$ & Streckung & $7.5$ & $[]$ \\ 
            $e$ & Oswaldfaktor & $0.8$ & $[]$ \\ 
            $F$ & wirksame Fläche & $845$ & $[m^2]$ \\ 
            $m$ & Masse & $276800$ & $[kg]$ \\ 
            $q_{\max}$ & maximaler Staudruck & $44154$ & $[N/m^2]$ \\
            $T_{\min}$ & minimale Schubkraft & $0$ & $[N]$ \\  
            $T_{\max}$ & maximale Schubkraft & $1260000$ & $[N]$ \\ 
            $C_{L, \min}$ & minimaler Auftriebsbeiwert & $0$ & $[]$ \\ 
            $C_{L, \max}$ & maximaler Auftriebsbeiwert & $1.48$ & $[]$ \\ 
            \hline
        \end{tabular} 
    \end{center}
\end{table}


\section{Anwendung Minimumpinzip von Pontryagin}
Aus dem Modellaufbau ergeben sich die Definitionen 
\begin{align*}
    g(X(t_f)) &=  -(x(t_f) - x_0) & & (g : \R^n \to \R) \\
    f_0(X(t),U(t)) &\equiv 0 & &(f_0 : \R^n \times \R^m \to \R^n)\\
    f(X(t),U(t)) &= \dot{X}(t) = (\dot{h}(t),\dot{\gamma}(t),\dot{x}(t),\dot{v}(t))^T & &(f : \R^n \times \R^m \to \R^n) \\
    U(t) &= (T(t),C_L(t))^T \in \mathcal{U} \left( = 
    \left[
        \begin{matrix}
            [T_{\min},T_{\max}] \\ 
            [C_{L, \min},C_{L, \max}]
        \end{matrix} 
    \right]\right) \subset \R^2 & &(U : [t_0,t_f] \to \R^m)
\end{align*}
Des Weiteren sei $\psi : \R^4 \to \R^2$ eine $C^1$-Funktion (mit $0 \leq (r = 2) \leq (n = 4)$) \[\psi(X(t_f)) = 
\begin{pmatrix}
    h(t_f) - h_f \\ 
    \gamma(t_f) - \gamma_f
\end{pmatrix} = 0\]

So ergibt sich das autonome Mayer-Problem mit 
\begin{equation} \label{equ:mayer_problem}
    \begin{aligned}
        \min F(X,U) &:= g(X(t_f)) =  -(x(t_f) - x_0) & & \\
        \text{unter}  \hspace{10mm} \dot{X}(t) &= f(X(t),U(t)) = (\dot{h}(t),\dot{\gamma}(t),\dot{x}(t),\dot{v}(t))^T & & \forall t \in [t_0,t_f] \\
        %
        X(t_0) &= X_0 = (h_0,\gamma_0,x_0,v_0)^T & & \\
        \psi(X(t_f)) &= 0 & & \\
        %
        q(X(t)) &\leq q_{\max} & & \forall t \in [t_0,t_f] \\
        U(t) &= (T(t),C_L(t))^T \in \mathcal{U} \subset \R^2  & & \forall t \in [t_0,t_f] 
    \end{aligned}
\end{equation}

Für die Funktion $f$ lässt sich konkret
\begin{equation} \label{equ:state_space}
    f(X(t),U(t)) = \dot{X}(t) = \begin{pmatrix}
        v(t) \sin(\gamma(t)) \\ 
        \dfrac{F \alpha e^{-\beta h(t)} v(t) C_L(t)}{2m} - \dfrac{g \cos(\gamma(t))}{v(t)} \\ 
        v(t) \cos(\gamma(t)) \\ 
        \dfrac{T(t)}{m} - \dfrac{(C_{D_0} + k C_L^2(t)) F \alpha e^{-\beta h(t)} v^2(t)}{2m} - g \sin(\gamma(t))
    \end{pmatrix}
\end{equation}

Für die Hamilton-Funktion ergibt sich mit $\lambda_0 \in \R$ ($\lambda_0 \geq 0$) und $\lambda : [t_0,t_f] \to \R^n$
\begin{equation} \label{equ:hamilt_func}
    \begin{split}
        H(X(t),U(t),\lambda(t)) &= \lambda_0 f_0(X(t),U(t)) + \lambda(t)^T f(X(t),U(t)) \\\
        &= \lambda(t)^T f(X(t),U(t)) \\\
        &= \lambda_1(t) \dot{h}(t) + \lambda_2(t) \dot{\gamma}(t) + \lambda_3(t) \dot{x}(t) + \lambda_4(t) \dot{v}(t) \\\
        &= \sin(\gamma(t)) v(t) \lambda_1 \\\
        &\hspace{7mm} + \dfrac{F \alpha e^{-\beta h(t)} C_L(t) v(t) \lambda_2(t)}{2m} - \dfrac{g \cos(\gamma(t)) \lambda_2(t)}{v(t)} \\\
        &\hspace{7mm} + \cos(\gamma(t)) v(t) \lambda_3(t) \\\
        &\hspace{7mm} + \dfrac{T(t) \lambda_4(t)}{m} - \dfrac{(C_{D_0} + k C_L^2(t)) F \alpha e^{-\beta h(t)} v^2(t) \lambda_4(t)}{2m} - g \sin(\gamma(t)) \lambda_4(t)
    \end{split}
\end{equation}

Damit lassen sich nun die Optimalitätsbedingungen des Minimumprinzips von Pontryagin aufstellen:
\begin{enumerate}
    \item \textbf{Minimumbedingung:} Es gilt an allen Stetigkeitsstellen $t \in [t_0,t_f]$ von $u^{\ast}(t)$ die Minimumbedingung \[H(X^{\ast}(t),U^{\ast}(t),\lambda(t)) = \min_{U(t) \in \mathcal{U}} H(X^{\ast}(t),U(t),\lambda(t))\] Leitet man nun nach der Steuerfunktion $U(t)$ ab, so ergibt sich für den unbeschränkten Fall der Steuerung die Minimumbedingung
    \[\dfrac{\partial}{\partial U} H(X^{\ast}(t),U^{\ast}(t),\lambda(t)) = \begin{pmatrix}
    \dfrac{\lambda_4(t)}{m} \\ 
    - \dfrac{k F \alpha e^{-\beta h(t)} v^2(t) \lambda_4(t) C_L(t)}{m} + \dfrac{F \alpha e^{-\beta h(t)} v(t) \lambda_2(t)}{2m}
    \end{pmatrix}^T \stackrel{!}{=} 0\]
    und 
    \[\dfrac{\partial^2}{\partial U^2} H(X^{\ast}(t),U^{\ast}(t),\lambda(t)) = \begin{pmatrix}
    0 & - \dfrac{k F \alpha e^{-\beta h(t)} v^2(t) \lambda_4(t)}{m} 
    \end{pmatrix} \stackrel{!}{\geq} 0\] wobei \[\sigma(x(t),\lambda(t)) := H_u(X^{\ast}(t),U^{\ast}(t),\lambda(t))\] Schaltfunktion genannt wird. Für die Betrachtung mit den Beschränkungen der Steuerfunktion ergibt sich für den Schub:
    \[T(t) = \left\lbrace \begin{array}{ll}
    T_{\min} & ,\text{falls } \lambda_4 > 0  \\ 
    \text{beliebig} \in [T_{\min},T_{\max}] & ,\text{falls } \lambda_4 = 0  \\ 
    T_{\max} & ,\text{falls } \lambda_4 < 0
    \end{array} \right.\]
    Für die Bestimmung der Steuerfunktion des Auftriebbeiwerts lässt sich die Hamilton-Funktion verkürzt schreiben mit den Thermen $K_1(t)$ und $K_2(t)$
    \[
    \begin{split}
        \tilde{H}(X^{\ast}(t),C_L(t),\lambda(t)) &= \dfrac{F \alpha e^{-\beta h^{\ast}(t)} v^{\ast}(t) \lambda_2(t)}{2m} \cdot C_L(t) - \dfrac{k F \alpha e^{-\beta h^{\ast}(t)}  v^{\ast 2}(t) \lambda_4(t)}{2m} \cdot C_L^2(t) \\\
        &= K_1(t) C_L(t) - K_2(t) C_L^2(t)
    \end{split}
    \]
    Soll nun $\tilde{H}(X^{\ast}(t),C_L(t),\lambda(t))$ minimal werden, so müssen die Folgenden Bedinungen überprüft werden:
    \begin{enumerate}
        \item[1.)] $\mathbf{K_1(t) < 0 \wedge K_2(t)} < 0$:
        \begin{enumerate}
            \item[1.1.)] Für die Ableitungen von $\tilde{H}$ ergeben sich
            \[\begin{split}
            - 2 K_2(t) C_L(t) + K_1(t) &\stackrel{!}{=} 0 \\\
            - 2 K_2(t) &\geq 0 \Rightarrow \text{Minimum}
            \end{split}\]
            So folgt für die Steuerfunktion $C_L(t) = \dfrac{K_1(t)}{2 K_2(t)} > 0$.
            %
            \item[1.2.)] Falls $C_L(t) = \dfrac{K_1(t)}{2 K_2(t)} > C_{L, \max}$ so gilt  $C_L(t) = C_{L, \max}$.
        \end{enumerate}
        %
        \item[2.)] $\mathbf{K_1(t) = 0 \wedge K_2(t)} < 0$: Für die Ableitungen von $\tilde{H}$ ergeben sich
        \[\begin{split}
        - 2 K_2(t) C_L(t) &\stackrel{!}{=} 0 \\\
        - 2 K_2(t) &\geq 0 \Rightarrow \text{Minimum}
        \end{split}\]
        So folgt für die Steuerfunktion $C_L(t) = 0 = C_{L, \min}$.
        %
        \item[3.)] $\mathbf{K_1(t) > 0 \wedge K_2(t)} < 0$: Für die Ableitungen von $\tilde{H}$ ergeben sich
        \[\begin{split}
        - 2 K_2(t) C_L(t) + K_1(t) &\stackrel{!}{=} 0 \\\
        - 2 K_2(t) &\geq 0 \Rightarrow \text{Minimum}
        \end{split}\]
        So folgt für die Steuerfunktion $C_L(t) = \dfrac{K_1(t)}{2 K_2(t)} < 0$. Es gilt also $C_L(t) = C_{L, \min}$.
        %
        \item[4.)] $\mathbf{K_1(t) < 0 \wedge K_2(t)} = 0$: Für die Ableitungen von $\tilde{H}$ ergeben sich
        \[\begin{split}
        K_1(t) &\stackrel{!}{=} 0 \\\
        0 &\geq 0 \Rightarrow \text{Minimum}
        \end{split}\]
        So folgt für die Steuerfunktion $C_L(t) = C_{L, \max}$.
        %
        \item[5.)] $\mathbf{K_1(t) = 0 \wedge K_2(t)} = 0$: Daraus folgt für die Steuerfunktion $C_L(t) = \text{beliebig} \in [C_{L, \min},C_{L, \max}]$.
        %
        \item[6.)] $\mathbf{K_1(t) > 0 \wedge K_2(t)} = 0$: Für die Ableitungen von $\tilde{H}$ ergeben sich
        \[\begin{split}
        K_1(t) &\stackrel{!}{=} 0 \\\
        0 &\geq 0 \Rightarrow \text{Minimum}
        \end{split}\]
        So folgt für die Steuerfunktion $C_L(t) = C_{L, \min}$.
        %
        \item[7.)] $\mathbf{K_1(t) < 0 \wedge K_2(t)} > 0$: Für die Ableitungen von $\tilde{H}$ ergeben sich
        \[\begin{split}
        - 2 K_2(t) C_L(t) + K_1(t) &\stackrel{!}{=} 0 \\\
        - 2 K_2(t) &\leq 0 \Rightarrow \text{Maximum}
        \end{split}\]
        So folgt für die Steuerfunktion $C_L(t) = \dfrac{K_1(t)}{2 K_2(t)} < 0$. Es gilt also $C_L(t) = C_{L, \max}$.
        %
        \item[8.)] $\mathbf{K_1(t) = 0 \wedge K_2(t)} > 0$: Für die Ableitungen von $\tilde{H}$ ergeben sich
        \[\begin{split}
        - 2 K_2(t) C_L(t) &\stackrel{!}{=} 0 \\\
        - 2 K_2(t) &\leq 0 \Rightarrow \text{Maximum}
        \end{split}\]
        So folgt für die Steuerfunktion $C_L(t) = 0$. Es gilt also $C_L(t) = C_{L, \max}$.
        %
        \item[9.)] $\mathbf{K_1(t) > 0 \wedge K_2(t)} > 0$: Für die Ableitungen von $\tilde{H}$ ergeben sich
        \[\begin{split}
        - 2 K_2(t) C_L(t) + K_1(t) &\stackrel{!}{=} 0 \\\
        - 2 K_2(t) &\leq 0 \Rightarrow \text{Maximum}
        \end{split}\]
        So folgt für die Steuerfunktion $C_L(t) = \dfrac{K_1(t)}{2 K_2(t)} > 0$. Es gilt also $C_L(t) = C_{L, \min}$.
    \end{enumerate}
    Damit folgt für die Steuerfunktion des Auftriebsbeiwerts
    \[C_L(t) = \left\lbrace 
    \begin{array}{ll}
        C_{L, \min} & ,\text{falls Bedingung } 2,3,6 \text{ oder } 9 \text{ gilt} \\ 
        \text{beliebig} \in [C_{L, \min},C_{L, \max}] & ,\text{falls Bedingung } 5 \text{ gilt} \\ 
        \dfrac{K_1(t)}{2 K_2(t)} & ,\text{falls Bedingung } 1.1 \text{ gilt} \\ 
        C_{L, \max} & ,\text{falls Bedingung } 1.2,4,7 \text{ oder } 8 \text{ gilt}  
    \end{array} 
    \right.\]
    %
    \item \textbf{Adjungierte DGL:} Leitet man nach dem Zustandsvektor $X(t)$ ab, also $H_{X}(X^{\ast}(t),U^{\ast}(t),\lambda(t))$ so erhält man 
        \[\begin{split}
            \dfrac{\partial}{\partial h} H &= - \dfrac{\alpha \beta F e^{-\beta h(t)} C_L(t) v(t) \lambda_2(t)}{2m} + \dfrac{(C_{D_0}+k C_L^2(t)) \alpha \beta F e^{-\beta h(t)} v^2(t) \lambda_4(t)}{2m} \\\
            \dfrac{\partial}{\partial \gamma} H &= \cos(\gamma(t)) v(t) \lambda_1(t) + \dfrac{g \sin(\gamma(t)) \lambda_2(t)}{v(t)} - \sin(\gamma(t)) v(t) \lambda_3(t) - \cos(\gamma(t)) g \lambda_4(t) \\\
            \dfrac{\partial}{\partial x} H &= 0 \\\
            \dfrac{\partial}{\partial v} H &= \sin(\gamma(t)) \lambda_1(t) + \left( \dfrac{F \alpha e^{-\beta h(t)} C_L(t)}{2m} + \dfrac{g \cos(\gamma(t))}{v^2(t)} \right) \lambda_2(t) \\\
            &\hspace{7mm} + \cos(\gamma(t)) \lambda_3(t) - \dfrac{(C_{D_0} + k C_L^2(t)) F \alpha e^{-\beta h(t)} v(t) \lambda_4(t)}{m} \\\ 
        \end{split}\]
        wobei gilt 
        \[\dot{\lambda}(t)^T = - \dfrac{\partial}{\partial X} H = -H_{X} = \left( -\dfrac{\partial}{\partial h} H, -\dfrac{\partial}{\partial \gamma} H, -\dfrac{\partial}{\partial x} H, -\dfrac{\partial}{\partial v} H \right)\]
    %
    \item \textbf{Transversalitätsbedingung:} Im Endzeitpunkt $t_f$ gilt die Transversalitätsbedingung mit dem Vektor $\nu \in \R^r$ mit $(\lambda_0,\lambda(t),\nu) \neq 0$ für alle $t \in [t_0,t_f]$
        \[\begin{split}
            \lambda(t_f)^T &= \lambda_0 g_X(X^{\ast}(t_f)) + \nu^T \psi_X(X^{\ast}(t_f)) \\\
            &= \lambda_0 
            \begin{pmatrix}
            0 & 0 & -1 & 0
            \end{pmatrix}  
            + \nu^T 
            \begin{pmatrix}
            1 & 0 & 0 & 0 \\
            0 & 1 & 0 & 0 
            \end{pmatrix}  \\\
            &= \begin{pmatrix}
            \nu_1 & \nu_2 & -\lambda_0 & 0 
            \end{pmatrix}
        \end{split}\]
    (ist $\lambda_{0} =1$ da $x(t_f)$ frei ist und $\psi_X(X^{\ast}(t_f))$ hat vollen Zeilenrang???) 
    %
    \item \textbf{Konstanz:} Für autonome Systeme gilt \[H(X^{\ast}(t),U^{\ast}(t), \lambda(t)) = const \ \in [t_0,t_f]\]
\end{enumerate}

































%\newpage
%\section{Überprüfung der optimalen Steuerung}
%Die Steuerung $T(t)$ (Schub) geht linear in die Hamilton-Funktion ein. Um die Hamilton-Funktion zu minimieren gilt für diese Bang-Bang Verhalten.
%
%Die Steuerung $C_L(t)$ geht nichtlinear in die Hamilton-Funktion ein.
%
%
%
%
%
%
%\section{Überprüfung der Hinreichenden Optimalitätsbedingungen}
%
%
%
%
%
%
%
%\section{Aufstellen des Randwertproblems}
%Muss also ein Mehrpunktrandwertproblem sein ???
%
%
%Mit den Optimalitätsbedingungen des Minimumprinzips von Pontryagin lässt sich das Steuerungsproblem in ein Randwertproblem überführen, welches aus den beiden Funktionen $r(t,Z(t))$ und $r_0(Z(t_0),Z(t_f)) = 0$ besteht. Für $r(t,Z(t))$ ergibt sich \[r(t,Z(t)) = \dot{Z}(t) = \begin{pmatrix}
%\dot{h}(t),\dot{\gamma}(t),\dot{x}(t),\dot{v}(t),\dot{\lambda}_1(t),\dot{\lambda}_2(t),\dot{\lambda}_3(t),\dot{\lambda}_4(t)
%\end{pmatrix}^T\] Für $r_0(Z(t_0),Z(t_f)) = 0$ müssen zunächst die Endbedingungen mit gebildet aus 
%\[\begin{split}
%X_i(t_f) &= c_i \hspace{25mm} (i=1,...,r) \\\
%\lambda_i(t_f) &= \lambda_0 g_{X_i}(X^{\ast}(t_f)) \hspace{5mm} (i=r+1,...,n)
%\end{split}\] gebildet werden. Es ergibt sich dann \[r_0(Z(t_0),Z(t_f)) = \begin{pmatrix}
%h(t_0) - h_0 \\ 
%\gamma(t_0) - \gamma_0 \\
%x(t_0) - x_0 \\ 
%v(t_0) - v_0 \\ 
%h(t_f) - h_f \\ 
%\gamma(t_f) - \gamma_f \\
%\lambda_3(t_f) + \lambda_0 \\ 
%\lambda_4(t_f) - 0
%\end{pmatrix}\]
