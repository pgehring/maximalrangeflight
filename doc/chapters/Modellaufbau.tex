\chapter{Modellaufbau}
Modelliert werden soll das Flugzeug A380-800 der Firma Airbus. Dabei seien:
\begin{itemize}
\item $x(t)$: $x$-Koordinate des Massenschwerpunktes $S$
%
\item $h(t)$: $h$-Koordinate des Massenschwerpunktes $S$
%
\item $v(t)$: Geschwindigkeit
%
\item $\gamma(t)$: Anstellwinkel
%
\item $T(t)$: Schub, Steuerung
%
\item $C_L(t)$: Auftriebsbeiwert, Steuerung
\end{itemize}
Um die Kräfte welche auf das Flugzeug einwirken berechnen zu können, werden folgende Hilfsgrößen benötigt:
\begin{itemize}
\item Luftwiderstandsbeiwert: \[C_D(C_L(t)) := C_{D_0} + k C^2_L(t) \ \ \ \ \text{mit} \ \ \ \ k = \dfrac{1}{\pi e AR}\] wobei $C_{D0}$ der Nullluftwiderstandsbeiwert, $e$ die Oswaldfaktor und $AR$ die Streckung (engl. \textit{aspect ratio}) bezeichnet.
%
\item Staudruck: \[q(v(t), h(t)) := \dfrac{1}{2} \cdot \rho(h(t)) \cdot v^2(t)\]
%
\item Luftdichte: \[\rho(h(t)) := \alpha e^{-\beta h(t)}\]
\end{itemize}
Die Kräfte lassen sich dann berechnen mit:
\begin{itemize}
\item Auftriebskraft: \[L(v(t), h(t), C_L(t)) := F \cdot C_L(t) \cdot q(v(t), h(t))\] wobei $F$ die wirksame Fläche, d.h. die von der Luft angeströmte Fläche, ist.
%
\item Luftwiderstand: \[D(v(t), h(t), C_L(t)) := F \cdot C_D(C_L(t)) \cdot q(v(t), h(t))\]
\item Erdanziehungskraft: \[W = mg\]
\end{itemize}

Für das Modell werden folgende Parameter verwendet:
\begin{center}
\begin{tabular}{|l|l|r|l|}
\hline
Parameter & Bedeutung & Wert & Einheit \\ 
\hline 
$\alpha$ &  & $1.247015$ & $[]$ \\ 
$\beta$ &  & $0.000104$ & $[]$ \\ 
$g$ & Erdbeschleunigung & $9.81$ & $[m/s^2]$ \\ 
$t_f$ & Endzeitpunkt & $1800$ & $[s]$ \\ 
$C_{D_0}$ & Nullluftwiderstandsbeiwert & $0.032$ & $[]$ \\ 
$AR$ & Streckung & $7.5$ & $[]$ \\ 
$e$ & Oswaldfaktor & $0.8$ & $[]$ \\ 
$F$ & wirksame Fläche & $845$ & $[m^2]$ \\ 
$m$ & Masse & $276800$ & $[kg]$ \\ 
$q_{\max}$ & maximaler Staudruck & $44154$ & $[N/m^2]$ \\ 
$T_{\max}$ & maximale Schubkraft & $1260000$ & $[N]$ \\ 
$C_{L, \min}$ & minimaler Auftriebsbeiwert & $0$ & $[]$ \\ 
$C_{L, \max}$ & maximaler Auftriebsbeiwert & $1.48$ & $[]$ \\ 
\hline
\end{tabular} 
\end{center}

Es ergibt sich das Optimalsteuerungsproblem:
\[\begin{split}
\min_{T, C_L} F(x,v,h,\gamma,T,C_L) &:= -(x(t_f) - x_0) \\\
\text{unter}  \hspace{20mm} \dot{x}(t) &= v(t) \cos(\gamma(t)) \ \qquad \qquad \qquad  \ \ \text{(Dynamik)}\\\
\dot{h}(t) &= v(t) \sin(\gamma(t)) \\\
\dot{v}(t) &= \dfrac{1}{m} \left( T(t) - D(v(t),h(t),C_L(t)) - mg \sin(\gamma(t)) \right) \\\
\dot{\gamma}(t) &=  \dfrac{1}{mv(t)} \left( L(v(t),h(t),C_L(t)) - mg \cos(\gamma(t)) \right) \\\
%
(x,h,v,\gamma)(0) &= (x_0,h_0,v_0,\gamma_0) \qquad \qquad \qquad \ \ \ \ \text{(Anfangsbedingungen)}\\\
(h,\gamma)(t_f) &= (h_f,0) \qquad \qquad \qquad \qquad \qquad  \  \text{(Endbedingungen)}\\\
%
q(v(t),h(t)) &\leq q_{\max} \ \ \ \ \forall t \in [0,t_f] \qquad \qquad  \ \ \ \ \ \ \text{(Steuer- und Zustandsbedingungen)}\\\
T(t) &\in [0,T_{\max}] \ \ \ \ \forall t \in [0,t_f] \\\
C_L(t) &\in [C_{L, \max},C_{L, \max}] \ \ \ \ \forall t \in [0,t_f] \\\
\end{split} \]

Die Start- und Endbedingungen sind gegeben mit:
\begin{equation}
\begin{pmatrix}
x_0 = 0 \\ 
v_0 = 0 \\ 
h_0 = 0 \\ 
\gamma_0 = 0.27 \ [Grad]
\end{pmatrix} \ \ \ \ \text{und} \ \ \ \ \begin{pmatrix}
h_f = 10668 \ [m] \\ 
\gamma_f = 0 \ [Grad]
\end{pmatrix}  
\end{equation}









\section{Anwendung Minimumpinzip von Pontryagin}
Sei nun $\dot{X}(t) = (\dot{x}(t), \dot{h}(t),\dot{v}(t),\dot{\gamma}(t))^T$, so ergibt sich das autonome Mayer-Problem mit \[\min_{T,C_L} F(X,T,C_L) := g(X(t_f)) =  -(x(t_f) - x_0)\] Für die Hamilton-Funktion gilt daher $f_0(X(t),T(t),C_L(t)) \equiv 0$ und $f(X(t),T(t),C_L(t)) = \dot{X}(t)$ mit
\[\dot{X}(t) = \begin{pmatrix}
v(t) \cos(\gamma(t)) \\ 
v(t) \sin(\gamma(t)) \\ 
\dfrac{1}{2m} \left(2T(t) + (-C_{D_0}-k C_L^2(t)) F  v^2(t) \alpha e^{-\beta h(t)} - 2 m g \sin(\gamma(t)) \right) \\ 
\dfrac{1}{2mv(t)} \left(F C_L(t)v^2(t) \alpha e^{-\beta h(t)} - 2 m g \cos(\gamma(t)) \right)
\end{pmatrix} \]
Für die Hamilton-Funktion ergibt sich somit 
\[\begin{split}
H(X(t),T(t),C_L(t)) &= \lambda_0 f_0(t,X(t),T(t),C_L(t)) + \lambda^T f(t,X(t),T(t),C_L(t)) \\\
&= \lambda^T f(t,X(t),T(t),C_L(t)) \\\
&= \lambda_1 \dot{x}(t) + \lambda_2 \dot{h}(t) + \lambda_3 \dot{v}(t) + \lambda_4 \dot{\gamma}(t) \\\
&= \lambda_1 v(t) \cos(\gamma(t)) \\\
&\hspace{7mm} + \lambda_2 v(t) \sin(\gamma(t)) \\\
&\hspace{7mm} + \dfrac{\lambda_3}{2m} \left(2T(t) + (-C_{D_0}-k C_L^2(t)) F  v^2(t) \alpha e^{-\beta h(t)} - 2 m g \sin(\gamma(t)) \right) \\\
&\hspace{7mm} + \dfrac{\lambda_4}{2mv(t)} \left(F C_L(t)v^2(t) \alpha e^{-\beta h(t)} - 2 m g \cos(\gamma(t)) \right) 
\end{split}\]
Leitet man nun nach den Steuerfunktionen ab, so ergibt sich für die Minimumbedingung
\[H_{T}(X^{\ast}(t),T^{\ast}(t),C_L(t),\lambda(t)) = \dfrac{\lambda_3}{m} \stackrel{!}{=} 0\]
und 
\[H_{C_L}(X^{\ast}(t),T(t),C_L^{\ast}(t),\lambda(t)) = - \dfrac{\lambda_3 k F v^2(t) \alpha e^{-\beta h(t)}}{m} C_L(t)  + \dfrac{\lambda_4 F v(t) \alpha e^{-\beta h(t)}}{2m} \stackrel{!}{=} 0 \]
Leitet man nach dem Zustandsvektor $X(t)$ ab, also $H_{X}(X^{\ast}(t),T^{\ast}(t),C_L^{\ast}(t),\lambda(t))$ so erhält man 
\[H_{X} = \begin{pmatrix}
0 \\ 
\dfrac{\lambda_3}{2m} C_{DL}(t) v^2(t) F \beta \rho(h(t))   - \dfrac{\lambda_4}{2m} C_L(t) v(t) F \beta \rho(h(t))  \\ 
\lambda_1 \cos(\gamma(t)) + \lambda_2 \sin(\gamma(t)) - \dfrac{\lambda_3}{m} C_{DL}(t)  v(t) F \rho(h(t))   - \dfrac{\lambda_4}{2m} \left(C_L(t) v(t) F \rho(h(t))  + \dfrac{2mg}{v(t)^2}\cos(\gamma(t))\right) \\ 
-\lambda_1 v(t) \sin(\gamma(t)) + \lambda_2 v(t) \cos(\gamma(t)) - \lambda_3 g \cos(\gamma(t)) + \lambda_4 \dfrac{g}{v(t)} \sin(\gamma(t)) 
\end{pmatrix} \]
mit \(C_{DL}(t) = C_{D_0}+k C_L^2(t)\) und \(\rho(h(t)) = \alpha e^{-\beta h(t)}\)
wobei gilt $\dot{\lambda}(t)^T = -H_{X}$. 

Im Endzeitpunkt $t_f$ gilt die Transversalitätsbedingung 
\[\begin{split}
\lambda(t_f)^T &= \lambda_0 g_X(X^{\ast}(t_f)) + \nu^T \psi_X(X^{\ast}(t_f)) \\\
&= \lambda_0 \begin{pmatrix}
-1 & 0 & 0 & 0
\end{pmatrix}  + \nu^T \begin{pmatrix}
0 & 0 & 0 & 0 \\
0 & 1 & 0 & 0 \\ 
0 & 0 & 0 & 0 \\
0 & 0 & 0 & 1 
\end{pmatrix}  \\\
&= \begin{pmatrix}
-\lambda_{0,x} & \nu_2 & 0 & \nu_4 
\end{pmatrix}
\end{split}\]
(ist $\lambda_{0} =1$ da $x(t_f)$ frei ist ???) Außerdem gilt für autonome Systeme \[H(X^{\ast}(t),T^{\ast}(t),C_L^{\ast}(t), \lambda(t)) = const \ \in [0, T]\]





\section{Aufstellen des Randwertproblems}
Es ergibt sich somit das Randwertproblem, welches aus den beiden Funktionen $r(t,Z(t))$ und $r_0(Z(t_0),Z(t_f)) = 0$ besteht:
\[\dot{Z}(t) = r(t,Z(t)) = \begin{pmatrix}
\dot{x}(t) \\
\dot{h}(t) \\
\dot{v}(t) \\
\dot{\gamma}(t) \\
\dot{\lambda}_1(t) \\ 
\dot{\lambda}_2(t) \\ 
\dot{\lambda}_3(t) \\ 
\dot{\lambda}_4(t)
\end{pmatrix}  \ \ \ \ r_0(Z(t_0),Z(t_f)) = \begin{pmatrix}
x(t_0) - x_0 \\ 
v(t_0) - v_0 \\ 
h(t_0) - h_0 \\ 
\gamma(t_0) - \gamma_0 \\
\lambda_1(t_f) + \lambda_{0,x} \\ 
h(t_f) - h_f \\ 
\lambda_3(t_f) - 0 \\ 
\gamma(t_f) - \gamma_f
\end{pmatrix}\]
% Für die einzelnen Einträge des Vektors $\dot{Z}(t)$ ergibt sich demnach:
% \[\dot{x}(t) = v(t) \cos(\gamma(t)) \]

% \[\dot{h}(t) = v(t) \sin(\gamma(t)) \]

% \[\dot{v}(t) = \dfrac{1}{2m} \left(2T(t) + (-C_{D_0}-k C_L^2(t)) F  v^2(t) \alpha e^{-\beta h(t)} - 2 m g \sin(\gamma(t)) \right) \]

% \[\dot{\gamma}(t) = \dfrac{1}{2mv(t)} \left(F C_L(t)v^2(t) \alpha e^{-\beta h(t)} - 2 m g \cos(\gamma(t)) \right)\]

% \[\dot{\lambda}_1(t) = 0\]

% \[\dot{\lambda}_2(t) = - \left( \dfrac{(C_{D_0}+k C_L^2(t)) v^2(t) \lambda_3 - C_L(t) v(t) \lambda_4}{2m} \alpha F \beta e^{-\beta h(t)} \right)\]

% \[\dot{\lambda}_3(t) = - \left( \lambda_1 \cos(\gamma(t)) + \lambda_2 \sin(\gamma(t)) + \dfrac{(-C_{D_0}-k C_L^2(t)) 2 v(t) \lambda_3 + C_L(t) \lambda_4}{2m}   F \alpha e^{-\beta h(t)} + \dfrac{g \cos(\gamma(t)) \lambda_4}{v^2(t)} \right)\]

<<<<<<< HEAD
\[\dot{\lambda}_4(t) = - \left( \dfrac{- \lambda_1 v^2(t) + \lambda_4 g}{v(t)} \sin(\gamma(t)) + (\lambda_2 v(t) - \lambda_3 g) \cos(\gamma(t)) \right)\]
=======
% \[\dot{\lambda}_4(t) = - \left( \dfrac{- \lambda_1 v(t) - \lambda_4 g}{v(t)} \sin(\gamma(t)) + (\lambda_2 v(t) + \lambda_3 g) \cos(\gamma(t)) \right)\]
>>>>>>> fixed bvp
