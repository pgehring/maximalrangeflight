\chapter{Optimalsteuerungsproblem} \label{cha:optim}

Modelliert werden soll das Flugzeug A380-800 der Firma Airbus. Dabei seien:
\begin{itemize}
    \item $x(t)$: $x$-Koordinate des Massenschwerpunktes $S$
    %
    \item $h(t)$: $h$-Koordinate des Massenschwerpunktes $S$
    %
    \item $v(t)$: Geschwindigkeit
    %
    \item $\gamma(t)$: Anstellwinkel
    %
    \item $T(t)$: Schub, Steuerung
    %
    \item $C_L(t)$: Auftriebsbeiwert, Steuerung
\end{itemize}
Um die Kräfte welche auf das Flugzeug einwirken berechnen zu können, werden folgende Hilfsgrößen benötigt:
\begin{itemize}
    \item Luftwiderstandsbeiwert: \[C_D(C_L(t)) := C_{D_0} + k \cdot C^2_L(t) \ \ \ \ \text{mit} \ \ \ \ k = \dfrac{1}{\pi \cdot e \cdot AR}\] wobei $C_{D0}$ der Nullluftwiderstandsbeiwert, $e$ die Oswaldfaktor und $AR$ die Streckung (engl. \textit{aspect ratio}) bezeichnet.
    %
    \item Luftdichte: \[\rho(h(t)) := \alpha \cdot e^{-\beta \cdot h(t)}\] %https://wind-data.ch/tools/luftdichte.php
    %
    \item Staudruck: \[q(v(t), h(t)) := \dfrac{\rho(h(t)) \cdot v^2(t)}{2} \]
\end{itemize}
Die Kräfte lassen sich dann berechnen mit:
\begin{itemize}
    \item Auftriebskraft: \[L(v(t), h(t), C_L(t)) := F \cdot C_L(t) \cdot q(v(t), h(t))\] wobei $F$ die wirksame Fläche, d.h. die von der Luft angeströmte Fläche, ist.
    %
    \item Luftwiderstand: \[D(v(t), h(t), C_L(t)) := F \cdot C_D(C_L(t)) \cdot q(v(t), h(t))\]
    \item Erdanziehungskraft: \[W = m \cdot g\]
\end{itemize}

Mit dem Newtonsschen Gesetz $F = m * a$ lässt sich die Differentialgleichungen für die Geschwindigkeit $v$ aufstellen:
\[\begin{split}
    F &:= m \cdot a \\\
    \Rightarrow \dot{v}(t) &= a(t) = \dfrac{F(t)}{m} = \dfrac{T(t) - D(v(t),h(t),C_L(t)) - W \sin(\gamma(t))}{m}
\end{split} \]
Mit der Gleichung für die Zentripetalkraft $F_{ZP} = \dfrac{m v^2}{r}$ lässt sich die Differentialgleichungen für den Ansstellwinkel $\gamma$ aufstellen:
\[\begin{split}
    F_{ZP} &:= \dfrac{m v^2}{r} \\\
    \Rightarrow \dot{\gamma}(t) &:= \dfrac{v(t)}{r} = \dfrac{F_{ZP}(t)}{m v(t)} = \dfrac{L(v(t),h(t),C_L(t)) - W \cos(\gamma(t))}{m v(t)}
\end{split} \]
Die Differentialgleichungen für die $h(t)$ und $x(t)$ lassen mittels Geschwindigkeit und Anstellwinkel bestimmen:
\[\begin{split}
    \dot{x}(t) &= v(t) \cos(\gamma(t))\\\
    \dot{h}(t) &= v(t) \sin(\gamma(t))
\end{split} \]
Es ergibt sich somit das Optimalsteuerungsproblem (Problem \ref{prob:MaxRF}).

\begin{problem}[Optimalsteuerungsproblem - Maximal-Range-Flight]\label{prob:MaxRF}
\begin{align*}
    \min_{T, C_L} F(h,\gamma,x,v,T,C_L) &:= -(x(t_f) - x_0) & & \\\
    \text{unter} \hspace{20mm} \dot{h}(t) &= v(t) \sin(\gamma(t)) \hspace{27mm} & & \\\
    \dot{\gamma}(t) &=  \dfrac{L(v(t),h(t),C_L(t)) - W \cos(\gamma(t))}{mv(t)} & & \\\
    \dot{x}(t) &= v(t) \cos(\gamma(t)) & & \\\
    \dot{v}(t) &= \dfrac{T(t) - D(v(t),h(t),C_L(t)) - W \sin(\gamma(t))}{m} & & \\\
    %
    (h,\gamma,x,v)(t_0) &= (h_0,\gamma_0,x_0,v_0) & & \\\
    (h,\gamma)(t_f) &= (h_f,\gamma_f) & & \\\
    %
    q(v(t),h(t)) &\leq q_{\max} \hspace{19.5mm} & & \forall t \in [t_0,t_f]\\\
    T(t) &\in [T_{\min},T_{\max}] & & \forall t \in [t_0,t_f] \\\
    C_L(t) &\in [C_{L, \min},C_{L, \max}] & & \forall t \in [t_0,t_f]
\end{align*}
\end{problem}

Für das Modell werden folgende Parameter aus Tabelle \ref{tab:ProblemPara} verwendet.
\begin{table}[H]
    \centering
    \caption{Problemparameter für das Flugzeug A380-800 der Firma Airbus.}\label{tab:ProblemPara}
    \begin{tabularx}{.9\textwidth}{lXrl}
        \toprule
        \textbf{Parameter}   & \textbf{Bedeutung} & \textbf{Wert} & \textbf{Einheit} \\ 
        \midrule
        $t_0$       & Anfangszeitpunkt & $0$ & $s$ \\ 
        $t_f$       & Endzeitpunkt & $1800$ & $s$ \\ 
        \hline
        $h_0$       & Anfangshöhe & $0$ & $m$ \\ 
        $\gamma_0$  & Anfangsanstellwinkel & $0.27$ & $^{\circ}$ \\
        $x_0$       & Anfangskoordinate & $0$ & $m$ \\ 
        $v_0$       & Anfangsgeschwindigkeit & $100$ & $\frac{m}{s}$ \\ 
        \hline
        $h_f$       & Endhöhe & $10668$ & $m$ \\ 
        $\gamma_f$  & Endanstellwinkel & $0$ & $^{\circ}$ \\
        \hline
        $\alpha$    &  & $1.247015$ & $1$\\ 
        $\beta$     &  & $0.000104$ & $1$\\
        $g$         & Erdbeschleunigung & $9.81$ & $\frac{m}{s^2}$ \\ 
        $C_{D_0}$   & Nullluftwiderstandsbeiwert & $0.032$ & $1$\\ 
        $AR$        & Streckung & $7.5$ & $1$\\ 
        $e$         & Oswaldfaktor & $0.8$ & $1$\\ 
        $F$         & wirksame Fläche & $845$ & $m^2$ \\ 
        $m$         & Masse & $276800$ & $kg$ \\ 
        $q_{\max}$  & maximaler Staudruck & $44154$ & $\frac{N}{m^2}$ \\
        $T_{\min}$  & minimale Schubkraft & $0$ & $N$ \\  
        $T_{\max}$  & maximale Schubkraft & $1260000$ & $N$ \\ 
        $C_{L, \min}$ & minimaler Auftriebsbeiwert & $0$ & $1$ \\ 
        $C_{L, \max}$ & maximaler Auftriebsbeiwert & $1.48$ & $1$ \\ 
        \bottomrule
    \end{tabularx} 
\end{table}



%\newpage
%\section{Überprüfung der optimalen Steuerung}
%Die Steuerung $T(t)$ (Schub) geht linear in die Hamilton-Funktion ein. Um die Hamilton-Funktion zu minimieren gilt für diese Bang-Bang Verhalten.
%
%Die Steuerung $C_L(t)$ geht nichtlinear in die Hamilton-Funktion ein.
%
%
%
%
%
%
%\section{Überprüfung der Hinreichenden Optimalitätsbedingungen}
%
%
%
%
%
%
%
%\section{Aufstellen des Randwertproblems}
%Muss also ein Mehrpunktrandwertproblem sein ???
%
%
%Mit den Optimalitätsbedingungen des Minimumprinzips von Pontryagin lässt sich das Steuerungsproblem in ein Randwertproblem überführen, welches aus den beiden Funktionen $r(t,Z(t))$ und $r_0(Z(t_0),Z(t_f)) = 0$ besteht. Für $r(t,Z(t))$ ergibt sich \[r(t,Z(t)) = \dot{Z}(t) = \begin{pmatrix}
%\dot{h}(t),\dot{\gamma}(t),\dot{x}(t),\dot{v}(t),\dot{\lambda}_1(t),\dot{\lambda}_2(t),\dot{\lambda}_3(t),\dot{\lambda}_4(t)
%\end{pmatrix}^T\] Für $r_0(Z(t_0),Z(t_f)) = 0$ müssen zunächst die Endbedingungen mit gebildet aus 
%\[\begin{split}
%X_i(t_f) &= c_i \hspace{25mm} (i=1,...,r) \\\
%\lambda_i(t_f) &= \lambda_0 g_{X_i}(X^{\ast}(t_f)) \hspace{5mm} (i=r+1,...,n)
%\end{split}\] gebildet werden. Es ergibt sich dann \[r_0(Z(t_0),Z(t_f)) = \begin{pmatrix}
%h(t_0) - h_0 \\ 
%\gamma(t_0) - \gamma_0 \\
%x(t_0) - x_0 \\ 
%v(t_0) - v_0 \\ 
%h(t_f) - h_f \\ 
%\gamma(t_f) - \gamma_f \\
%\lambda_3(t_f) + \lambda_0 \\ 
%\lambda_4(t_f) - 0
%\end{pmatrix}\]
