\chapter{Optimalsteuerungsproblem} \label{cha:optim}

Modelliert werden soll das Flugzeug A380-800 der Firma Airbus. Dabei seien:
\begin{itemize}
    \item $x(t)$: $x$-Koordinate des Massenschwerpunktes $S$
    %
    \item $h(t)$: $h$-Koordinate des Massenschwerpunktes $S$
    %
    \item $v(t)$: Geschwindigkeit
    %
    \item $\gamma(t)$: Anstellwinkel
    %
    \item $T(t)$: Schub, Steuerung
    %
    \item $C_L(t)$: Auftriebsbeiwert, Steuerung
\end{itemize}
Um die Kräfte welche auf das Flugzeug einwirken berechnen zu können, werden folgende Hilfsgrößen benötigt:
\begin{itemize}
    \item Luftwiderstandsbeiwert: \[C_D(C_L(t)) := C_{D_0} + k \cdot C^2_L(t) \ \ \ \ \text{mit} \ \ \ \ k = \dfrac{1}{\pi \cdot e \cdot AR}\] wobei $C_{D0}$ der Nullluftwiderstandsbeiwert, $e$ die Oswaldfaktor und $AR$ die Streckung (engl. \textit{aspect ratio}) bezeichnet.
    %
    \item Luftdichte: \[\rho(h(t)) := \alpha \cdot e^{-\beta \cdot h(t)}\] %https://wind-data.ch/tools/luftdichte.php
    %
    \item Staudruck: \[q(v(t), h(t)) := \dfrac{\rho(h(t)) \cdot v^2(t)}{2} \]
\end{itemize}
Die Kräfte lassen sich dann berechnen mit:
\begin{itemize}
    \item Auftriebskraft: \[L(v(t), h(t), C_L(t)) := F \cdot C_L(t) \cdot q(v(t), h(t))\] wobei $F$ die wirksame Fläche, d.h. die von der Luft angeströmte Fläche, ist.
    %
    \item Luftwiderstand: \[D(v(t), h(t), C_L(t)) := F \cdot C_D(C_L(t)) \cdot q(v(t), h(t))\]
    \item Erdanziehungskraft: \[W = m \cdot g\]
\end{itemize}

Mit dem 2. Newtons'schen Axiom $F = m \cdot a$ lässt sich die Differentialgleichungen für die Geschwindigkeit $v$ aufstellen:
\[F := m \cdot a \Rightarrow \dot{v}(t) = a(t) = \dfrac{F(t)}{m} = \dfrac{T(t) - D(v(t),h(t),C_L(t)) - W \sin(\gamma(t))}{m}\]
Mit der Gleichung für die Zentripetalkraft $F_{ZP} = \dfrac{m v^2}{r}$ lässt sich die Differentialgleichungen für den Ansstellwinkel $\gamma$ aufstellen:
\[F_{ZP} := \dfrac{m v^2}{r} \Rightarrow \dot{\gamma}(t) = \dfrac{v(t)}{r} = \dfrac{F_{ZP}(t)}{m v(t)} = \dfrac{L(v(t),h(t),C_L(t)) - W \cos(\gamma(t))}{m v(t)}\]
Die Differentialgleichungen für die $h(t)$ und $x(t)$ lassen mittels Geschwindigkeit und Anstellwinkel bestimmen:
\[\begin{split}
    \dot{x}(t) &= v(t) \cos(\gamma(t))\\\
    \dot{h}(t) &= v(t) \sin(\gamma(t))
\end{split} \]
%Es ergibt sich somit das Optimalsteuerungsproblem (Problem \ref{prob:MaxRF}) mit dem Zustandsvektor
%\begin{equation}
%X(t) = (h(t),\gamma(t),x(t),v(t))^T
%\end{equation}
%und dem Steuervektor 
%\begin{equation}
%U(t) = (T(t),C_L(t))^T
%\end{equation}
%Für das Modell werden folgende Parameter aus Tabelle \ref{tab:ProblemPara} aus Anhang \ref{Anhang:ModellPara} verwendet.

%\begin{problem}[Optimalsteuerungsproblem - Maximal-Range-Flight]\label{prob:MaxRF}
%\begin{align*}
%    \min_{T, C_L} F(h,\gamma,x,v,T,C_L) &:= -(x(t_f) - x_0) & & \\\
%    \text{unter} \hspace{20mm} \dot{h}(t) &= v(t) \sin(\gamma(t)) \hspace{27mm} & & \\\
%    \dot{\gamma}(t) &=  \dfrac{L(v(t),h(t),C_L(t)) - W \cos(\gamma(t))}{mv(t)} & & \\\
%    \dot{x}(t) &= v(t) \cos(\gamma(t)) & & \\\
%    \dot{v}(t) &= \dfrac{T(t) - D(v(t),h(t),C_L(t)) - W \sin(\gamma(t))}{m} & & \\\
%    %
%    (h,\gamma,x,v)(t_0) &= (h_0,\gamma_0,x_0,v_0) & & \\\
%    (h,\gamma)(t_f) &= (h_f,\gamma_f) & & \\\
%    %
%    q(v(t),h(t)) &\leq q_{\max} \hspace{19.5mm} & & \forall t \in [t_0,t_f]\\\
%    T(t) &\in [T_{\min},T_{\max}] & & \forall t \in [t_0,t_f] \\\
%    C_L(t) &\in [C_{L, \min},C_{L, \max}] & & \forall t \in [t_0,t_f]
%\end{align*}
%\end{problem}

Es ergibt sich somit das Optimalsteuerungsproblem (Problem \ref{prob:MaxRF}) mit den Funktionen $g : \R^{n_X} \to \R$, $f_0 : \R^{n_X} \times \R^{n_U} \to \R$, $f : \R^{n_X} \times \R^{n_U} \to \R^{n_X}$ und $U : [t_0,t_f] \to \R^m$ für $0 \leq (n_{\psi} = 2) \leq (n_X = 4)$ und $n_U = 2$.

\begin{problem}[Optimalsteuerungsproblem - Maximal-Range-Flight]\label{prob:MaxRF}
Für das Optimalsteuerungsproblem ergibt sich mit dem Zustandsvektor
\[X(t) = (h(t),\gamma(t),x(t),v(t))^T\]
und der Steuerfunktion
\[U(t) = (T(t),C_L(t))^T\]
das Problem:
\begin{align*}
\min_{U} F(X,U) &:= g(X(t_f)) + \int_{t_0}^{t_f} f_0(X(t),U(t)) dt = -(x(t_f) - x_0) & & \\\
\text{unter} \hspace{10mm} \dot{X}(t) &= f(X(t),U(t)) = (\dot{h}(t),\dot{\gamma}(t),\dot{x}(t),\dot{v}(t))^T  & & \\\
&= \begin{pmatrix}
v(t) \sin(\gamma(t)) \\ 
\dfrac{L(v(t),h(t),C_L(t)) - W \cos(\gamma(t))}{mv(t)} \\ 
v(t) \cos(\gamma(t)) \\ 
\dfrac{T(t) - D(v(t),h(t),C_L(t)) - W \sin(\gamma(t))}{m}
\end{pmatrix} & & \\\
(h,\gamma,x,v)(t_0) &= (h_0,\gamma_0,x_0,v_0) & & \\\
(h,\gamma)(t_f) &= (h_f,\gamma_f) & & \\\
q(v(t),h(t)) &\leq q_{\max} & & \forall t \in [t_0,t_f]\\\
U(t) &= (T(t),C_L(t))^T \in \mathcal{U} = \left[ \begin{matrix}
[T_{\min},T_{\max}] \\ 
[C_{L, \min},C_{L, \max}]
\end{matrix} 
\right] & & \forall t \in [t_0,t_f]
\end{align*}
\end{problem}

Des Weiteren sei $\psi : \R^{n_X} \to \R^{n_{\psi}}$ eine $C^1$-Funktion
\[\psi(X(t_f)) = 
\begin{pmatrix}
    h(t_f) - h_f \\ 
    \gamma(t_f) - \gamma_f
\end{pmatrix} = 0_{n_{\psi}}\]
Das Optimalsteuerungsproblem (Problem \ref{prob:MaxRF}) stellt also ein autonomes Mayer-Problem der Form 
\begin{equation} \label{equ:mayer_problem}
    \begin{aligned}
        \min F(X,U) &:= g(X(t_f)) & & \\
        \text{unter}  \hspace{10mm} \dot{X}(t) &= f(X(t),U(t)) & & \forall t \in [t_0,t_f] \\
        %
        X(t_0) &= X_0 = (h_0,\gamma_0,x_0,v_0)^T & & \\
        \psi(X(t_f)) &= 0_{n_{\psi}} & & \\
        %
        q(X(t)) &\leq q_{\max} & & \forall t \in [t_0,t_f] \\
        U(t) &= (T(t),C_L(t))^T \in \mathcal{U}  & & \forall t \in [t_0,t_f] 
    \end{aligned}
\end{equation}
da, mit der zusätzlichen Beschränkung des Staudrucks $q(v(t),h(t))$ und der konkreten Funktion $f$
\begin{equation} \label{equ:state_space}
    f(X(t),U(t)) = \dot{X}(t) = \begin{pmatrix}
        v(t) \sin(\gamma(t)) \\ 
        \dfrac{F \alpha e^{-\beta h(t)} v(t) C_L(t)}{2m} - \dfrac{g \cos(\gamma(t))}{v(t)} \\ 
        v(t) \cos(\gamma(t)) \\ 
        \dfrac{T(t)}{m} - \dfrac{(C_{D_0} + k C_L^2(t)) F \alpha e^{-\beta h(t)} v^2(t)}{2m} - g \sin(\gamma(t))
    \end{pmatrix}
\end{equation}
Für das Modell werden die Parameter aus Tabelle \ref{tab:ProblemPara} aus Anhang \ref{Anhang:ModellPara} verwendet.






%\begin{theo}[Optimalsteuerungsproblem - Maximal-Range-Flight]{thm:theorem1}
%There exist two irrational numbers $x$, $y$ such that $x^y$ is rational.
%\begin{align*}
%    \min_{T, C_L} F(h,\gamma,x,v,T,C_L) &:= -(x(t_f) - x_0) & & \\\
%    \text{unter} \hspace{20mm} \dot{h}(t) &= v(t) \sin(\gamma(t)) \hspace{27mm} & & \\\
%    \dot{\gamma}(t) &=  \dfrac{L(v(t),h(t),C_L(t)) - W \cos(\gamma(t))}{mv(t)} & & \\\
%    \dot{x}(t) &= v(t) \cos(\gamma(t)) & & \\\
%    \dot{v}(t) &= \dfrac{T(t) - D(v(t),h(t),C_L(t)) - W \sin(\gamma(t))}{m} & & \\\
%    %
%    (h,\gamma,x,v)(t_0) &= (h_0,\gamma_0,x_0,v_0) & & \\\
%    (h,\gamma)(t_f) &= (h_f,\gamma_f) & & \\\
%    %
%    q(v(t),h(t)) &\leq q_{\max} \hspace{19.5mm} & & \forall t \in [t_0,t_f]\\\
%    T(t) &\in [T_{\min},T_{\max}] & & \forall t \in [t_0,t_f] \\\
%    C_L(t) &\in [C_{L, \min},C_{L, \max}] & & \forall t \in [t_0,t_f]
%\end{align*}
%\end{theo}