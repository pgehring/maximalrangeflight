\chapter{Modellaufbau}
Modelliert werden soll das Flugzeug A380-800 der Firma Airbus. Dabei seien:
\begin{itemize}
\item $x(t)$: $x$-Koordinate des Massenschwerpunktes $S$
%
\item $h(t)$: $h$-Koordinate des Massenschwerpunktes $S$
%
\item $v(t)$: Geschwindigkeit
%
\item $\gamma(t)$: Anstellwinkel
%
\item $T(t)$: Schub, Steuerung
%
\item $C_L(t)$: Auftriebsbeiwert, Steuerung
\end{itemize}
Um die Kräfte welche auf das Flugzeug einwirken berechnen zu können, werden folgende Hilfsgrößen benötigt:
\begin{itemize}
\item Luftwiderstandsbeiwert: \[C_D(C_L(t)) := C_{D_0} + k C^2_L(t) \ \ \ \ \text{mit} \ \ \ \ k = \dfrac{1}{\pi e AR}\] wobei $C_{D0}$ der Nullluftwiderstandsbeiwert, $e$ die Oswaldfaktor und $AR$ die Streckung (engl. \textit{aspect ratio}) bezeichnet.
%
\item Staudruck: \[q(v(t), h(t)) := \dfrac{1}{2} \cdot \rho(h(t)) \cdot v^2(t)\]
%
\item Luftdichte: \[\rho(h(t)) := \alpha e^{-\beta h(t)}\]
\end{itemize}
Die Kräfte lassen sich dann berechnen mit:
\begin{itemize}
\item Auftriebskraft: \[L(v(t), h(t), C_L(t)) := F \cdot C_L(t) \cdot q(v(t), h(t))\] wobei $F$ die wirksame Fläche, d.h. die von der Luft angeströmte Fläche, ist.
%
\item Luftwiderstand: \[D(v(t), h(t), C_L(t)) := F \cdot C_D(C_L(t)) \cdot q(v(t), h(t))\]
\item Erdanziehungskraft: \[W = mg\]
\end{itemize}

Für das Modell werden folgende Parameter verwendet:
\begin{center}
\begin{tabular}{|l|l|r|l|}
\hline
Parameter & Bedeutung & Wert & Einheit \\ 
\hline 
\hline
$t_0$ & Anfangszeitpunkt & $0$ & $[s]$ \\ 
$t_f$ & Endzeitpunkt & $1800$ & $[s]$ \\ 
\hline
$x_0$ & Anfangskoordinate & $0$ & $[m]$ \\ 
$v_0$ & Anfangsgeschwindigkeit & $100$ & $[m/s]$ \\ 
$h_0$ & Anfangshöhe & $0$ & $[m]$ \\ 
$\gamma_0$ & Anfangsanstellwinkel & $0.27$ & $[Grad]$ \\
\hline
$h_f$ & Endhöhe & $10668$ & $[m]$ \\ 
$\gamma_f$ & Endanstellwinkel & $0$ & $[Grad]$ \\
\hline
$\alpha$ &  & $1.247015$ & $[]$ \\ 
$\beta$ &  & $0.000104$ & $[]$ \\
$g$ & Erdbeschleunigung & $9.81$ & $[m/s^2]$ \\ 
$C_{D_0}$ & Nullluftwiderstandsbeiwert & $0.032$ & $[]$ \\ 
$AR$ & Streckung & $7.5$ & $[]$ \\ 
$e$ & Oswaldfaktor & $0.8$ & $[]$ \\ 
$F$ & wirksame Fläche & $845$ & $[m^2]$ \\ 
$m$ & Masse & $276800$ & $[kg]$ \\ 
$q_{\max}$ & maximaler Staudruck & $44154$ & $[N/m^2]$ \\
$T_{\min}$ & minimale Schubkraft & $0$ & $[N]$ \\  
$T_{\max}$ & maximale Schubkraft & $1260000$ & $[N]$ \\ 
$C_{L, \min}$ & minimaler Auftriebsbeiwert & $0$ & $[]$ \\ 
$C_{L, \max}$ & maximaler Auftriebsbeiwert & $1.48$ & $[]$ \\ 
\hline
\end{tabular} 
\end{center}

Es ergibt sich das Optimalsteuerungsproblem:
\[\begin{split}
\min_{T, C_L} F(x,v,h,\gamma,T,C_L) &:= -(x(t_f) - x_0) \\\
\text{unter}  \hspace{20mm} \dot{x}(t) &= v(t) \cos(\gamma(t)) \ \qquad \qquad \qquad  \ \ \text{(Dynamik)}\\\
\dot{h}(t) &= v(t) \sin(\gamma(t)) \\\
\dot{v}(t) &= \dfrac{1}{m} \left( T(t) - D(v(t),h(t),C_L(t)) - mg \sin(\gamma(t)) \right) \\\
\dot{\gamma}(t) &=  \dfrac{1}{mv(t)} \left( L(v(t),h(t),C_L(t)) - mg \cos(\gamma(t)) \right) \\\
%
(x,h,v,\gamma)(0) &= (x_0,h_0,v_0,\gamma_0) \qquad \qquad \qquad \ \ \ \ \text{(Anfangsbedingungen)}\\\
(h,\gamma)(t_f) &= (h_f,\gamma_f) \qquad \qquad \qquad \qquad \qquad  \  \text{(Endbedingungen)}\\\
%
q(v(t),h(t)) &\leq q_{\max} \ \ \ \ \forall t \in [t_0,t_f] \qquad \qquad  \ \ \ \ \ \ \text{(Steuer- und Zustandsbedingungen)}\\\
T(t) &\in [T_{\min},T_{\max}] \ \ \ \ \forall t \in [t_0,t_f] \\\
C_L(t) &\in [C_{L, \min},C_{L, \max}] \ \ \ \ \forall t \in [t_0,t_f] \\\
\end{split} \]
%Die Start- und Endbedingungen sind gegeben mit:
%\begin{equation}
%\begin{pmatrix}
%x_0 = 0 \\ 
%v_0 = 0 \\ 
%h_0 = 0 \\ 
%\gamma_0 = 0.27 \ [Grad]
%\end{pmatrix} \ \ \ \ \text{und} \ \ \ \ \begin{pmatrix}
%h_f = 10668 \ [m] \\ 
%\gamma_f = 0 \ [Grad]
%\end{pmatrix}  
%\end{equation}









%\section{Anwendung Minimumpinzip von Pontryagin}
%Sei nun $\dot{X}(t) = (\dot{x}(t), \dot{h}(t),\dot{v}(t),\dot{\gamma}(t))^T$, so ergibt sich das autonome Mayer-Problem mit \[\min_{T,C_L} F(X,T,C_L) := g(X(t_f)) =  -(x(t_f) - x_0)\] Für die Hamilton-Funktion gilt daher $f_0(X(t),T(t),C_L(t)) \equiv 0$ und $f(X(t),T(t),C_L(t)) = \dot{X}(t)$ mit
%\[\dot{X}(t) = \begin{pmatrix}
%v(t) \cos(\gamma(t)) \\ 
%v(t) \sin(\gamma(t)) \\ 
%\dfrac{1}{2m} \left(2T(t) + (-C_{D_0}-k C_L^2(t)) F  v^2(t) \alpha e^{-\beta h(t)} - 2 m g \sin(\gamma(t)) \right) \\ 
%\dfrac{1}{2mv(t)} \left(F C_L(t)v^2(t) \alpha e^{-\beta h(t)} - 2 m g \cos(\gamma(t)) \right)
%\end{pmatrix} \]
%Für die Hamilton-Funktion ergibt sich somit 
%\[\begin{split}
%H(X(t),T(t),C_L(t)) &= \lambda_0 f_0(t,X(t),T(t),C_L(t)) + \lambda^T f(t,X(t),T(t),C_L(t)) \\\
%&= \lambda^T f(t,X(t),T(t),C_L(t)) \\\
%&= \lambda_1 \dot{x}(t) + \lambda_2 \dot{h}(t) + \lambda_3 \dot{v}(t) + \lambda_4 \dot{\gamma}(t) \\\
%&= \lambda_1 v(t) \cos(\gamma(t)) \\\
%&\hspace{7mm} + \lambda_2 v(t) \sin(\gamma(t)) \\\
%&\hspace{7mm} + \dfrac{\lambda_3}{2m} \left(2T(t) + (-C_{D_0}-k C_L^2(t)) F  v^2(t) \alpha e^{-\beta h(t)} - 2 m g \sin(\gamma(t)) \right) \\\
%&\hspace{7mm} + \dfrac{\lambda_4}{2mv(t)} \left(F C_L(t)v^2(t) \alpha e^{-\beta h(t)} - 2 m g \cos(\gamma(t)) \right) 
%\end{split}\]
%Leitet man nun nach den Steuerfunktionen ab, so ergibt sich für die Minimumbedingung
%\[H_{T}(X^{\ast}(t),T^{\ast}(t),C_L(t),\lambda(t)) = \dfrac{\lambda_3}{m} \stackrel{!}{=} 0\]
%und 
%\[H_{C_L}(X^{\ast}(t),T(t),C_L^{\ast}(t),\lambda(t)) = - \dfrac{\lambda_3 k F v^2(t) \alpha e^{-\beta h(t)}}{m} C_L(t)  + \dfrac{\lambda_4 F v(t) \alpha e^{-\beta h(t)}}{2m} \stackrel{!}{=} 0 \]
%Leitet man nach dem Zustandsvektor $X(t)$ ab, also $H_{X}(X^{\ast}(t),T^{\ast}(t),C_L^{\ast}(t),\lambda(t))$ so erhält man 
%\[H_{X} = \begin{pmatrix}
%0 \\ 
%\dfrac{(C_{D_0}+k C_L^2(t)) v^2(t) \lambda_3 - C_L(t) v(t) \lambda_4}{2m} \alpha F \beta e^{-\beta h(t)}\\ 
%\lambda_1 \cos(\gamma(t)) + \lambda_2 \sin(\gamma(t)) + \dfrac{(-C_{D_0}-k C_L^2(t)) 2 v(t) \lambda_3 + C_L(t) \lambda_4}{2m}   F \alpha e^{-\beta h(t)} + \dfrac{g \cos(\gamma(t)) \lambda_4}{v^2(t)} \\ 
%\dfrac{- \lambda_1 v^2(t) + \lambda_4 g}{v(t)} \sin(\gamma(t)) + (\lambda_2 v(t) - \lambda_3 g) \cos(\gamma(t))
%\end{pmatrix} \]
%wobei gilt $\dot{\lambda}(t)^T = -H_{X}$. Im Endzeitpunkt $t_f$ gilt die Transversalitätsbedingung 
%\[\begin{split}
%\lambda(t_f)^T &= \lambda_0 g_X(X^{\ast}(t_f)) + \nu^T \psi_X(X^{\ast}(t_f)) \\\
%&= \lambda_0 \begin{pmatrix}
%-1 & 0 & 0 & 0
%\end{pmatrix}  + \nu^T \begin{pmatrix}
%0 & 0 & 0 & 0 \\
%0 & 1 & 0 & 0 \\ 
%0 & 0 & 0 & 0 \\
%0 & 0 & 0 & 1 
%\end{pmatrix}  \\\
%&= \begin{pmatrix}
%-\lambda_{0,x} & \nu_2 & 0 & \nu_4 
%\end{pmatrix}
%\end{split}\]
%(ist $\lambda_{0} =1$ da $x(t_f)$ frei ist ???) Außerdem gilt für autonome Systeme \[H(X^{\ast}(t),T^{\ast}(t),C_L^{\ast}(t), \lambda(t)) = const \ \in [0, T]\]

%\section{Aufstellen des Randwertproblems}
%Es ergibt sich somit das Randwertproblem, welches aus den beiden Funktionen $r(t,Z(t))$ und $r_0(Z(t_0),Z(t_f)) = 0$ besteht:
%\[\dot{Z}(t) = r(t,Z(t)) = \begin{pmatrix}
%\dot{x}(t) \\
%\dot{h}(t) \\
%\dot{v}(t) \\
%\dot{\gamma}(t) \\
%\dot{\lambda}_1(t) \\ 
%\dot{\lambda}_2(t) \\ 
%\dot{\lambda}_3(t) \\ 
%\dot{\lambda}_4(t)
%\end{pmatrix}  \ \ \ \ r_0(Z(t_0),Z(t_f)) = \begin{pmatrix}
%x(t_0) - x_0 \\ 
%v(t_0) - v_0 \\ 
%h(t_0) - h_0 \\ 
%\gamma(t_0) - \gamma_0 \\
%\lambda_1(t_f) + \lambda_{0,x} \\ 
%h(t_f) - h_f \\ 
%\lambda_3(t_f) - 0 \\ 
%\gamma(t_f) - \gamma_f
%\end{pmatrix}\]
%Für die einzelnen Einträge des Vektors $\dot{Z}(t)$ ergibt sich demnach:
%\[\dot{x}(t) = v(t) \cos(\gamma(t)) \]
%
%\[\dot{h}(t) = v(t) \sin(\gamma(t)) \]
%
%\[\dot{v}(t) = \dfrac{1}{2m} \left(2T(t) + (-C_{D_0}-k C_L^2(t)) F  v^2(t) \alpha e^{-\beta h(t)} - 2 m g \sin(\gamma(t)) \right) \]
%
%\[\dot{\gamma}(t) = \dfrac{1}{2mv(t)} \left(F C_L(t)v^2(t) \alpha e^{-\beta h(t)} - 2 m g \cos(\gamma(t)) \right)\]
%
%\[\dot{\lambda}_1(t) = 0\]
%
%\[\dot{\lambda}_2(t) = - \left( \dfrac{(C_{D_0}+k C_L^2(t)) v^2(t) \lambda_3 - C_L(t) v(t) \lambda_4}{2m} \alpha F \beta e^{-\beta h(t)} \right)\]
%
%\[\dot{\lambda}_3(t) = - \left( \lambda_1 \cos(\gamma(t)) + \lambda_2 \sin(\gamma(t)) + \dfrac{(-C_{D_0}-k C_L^2(t)) 2 v(t) \lambda_3 + C_L(t) \lambda_4}{2m}   F \alpha e^{-\beta h(t)} + \dfrac{g \cos(\gamma(t)) \lambda_4}{v^2(t)} \right)\]
%
%\[\dot{\lambda}_4(t) = - \left( \dfrac{- \lambda_1 v^2(t) + \lambda_4 g}{v(t)} \sin(\gamma(t)) + (\lambda_2 v(t) - \lambda_3 g) \cos(\gamma(t)) \right)\]


\newpage
\section{Anwendung Minimumpinzip von Pontryagin}
Aus dem Modellaufbau ergeben sich die Definitionen 
\[\begin{split}
g(X(t_f)) &=  -(x(t_f) - x_0) \hspace{34mm} (g : \R^n \to \R) \\\
f_0(X(t),U(t)) &\equiv 0 \hspace{54mm} (f_0 : \R^n \times \R^m \to \R^n)\\\
f(X(t),U(t)) &= \dot{X}(t) = (\dot{h}(t),\dot{\gamma}(t),\dot{x}(t),\dot{v}(t))^T \hspace{10.5mm} (f : \R^n \times \R^m \to \R^n) \\\
U(t) &= (T(t),C_L(t))^T \in \mathcal{U} \subset \R^2 \hspace{19mm} (U : [t_0,t_f] \to \R^m)
\end{split}\]
Des Weiteren sei $\psi : \R^4 \to \R^2$ eine $C^1$-Funktion (mit $0 \leq (r = 2) \leq (n = 4)$) \[\psi(X(t_f)) = \begin{pmatrix}
h(t_f) - h_f \\ 
\gamma(t_f) - \gamma_f
\end{pmatrix} = 0\]
So ergibt sich das autonome Mayer-Problem mit 
\[\begin{split}
\min F(X,U) &:= g(X(t_f)) =  -(x(t_f) - x_0) \\\
\text{unter}  \hspace{20mm} \dot{X}(t) &= f(X(t),U(t)) = (\dot{h}(t),\dot{\gamma}(t),\dot{x}(t),\dot{v}(t))^T \hspace{10mm} \forall t \in [t_0,t_f] \\\
%
X(t_0) &= X_0 = (h_0,\gamma_0,x_0,v_0)^T\\\
\psi(X(t_f)) &= 0\\\
%
q(X(t)) &\leq q_{\max} \hspace{10mm} \forall t \in [t_0,t_f] \\\
U(t) &= (T(t),C_L(t))^T \in \mathcal{U} \left( = \left[\begin{matrix}
[T_{\min},T_{\max}] \\ 
[C_{L, \min},C_{L, \max}]
\end{matrix} \right] \forall t \in [0,t_f] \right) \subset \R^2
\end{split} \]
Für die Funktion $f$ lässt sich konkret
\[f(X(t),U(t)) = \dot{X}(t) = \begin{pmatrix}
v(t) \sin(\gamma(t)) \\ 
\dfrac{1}{2mv(t)} \left(F C_L(t)v^2(t) \alpha e^{-\beta h(t)} - 2 m g \cos(\gamma(t)) \right) \\ 
v(t) \cos(\gamma(t)) \\ 
\dfrac{1}{2m} \left(2T(t) + (-C_{D_0}-k C_L^2(t)) F  v^2(t) \alpha e^{-\beta h(t)} - 2 m g \sin(\gamma(t)) \right)
\end{pmatrix} \]
Für die Hamilton-Funktion ergibt sich mit $\lambda_0 \in \R$ ($\lambda_0 \geq 0$) und $\lambda : [t_0,t_f] \to \R^n$
\[\begin{split}
H(X(t),U(t),\lambda(t)) &= \lambda_0 f_0(X(t),U(t)) + \lambda(t)^T f(X(t),U(t)) \\\
&= \lambda(t)^T f(X(t),U(t)) \\\
&= \lambda_1 \dot{h}(t) + \lambda_2 \dot{\gamma}(t)  + \lambda_3 \dot{x}(t) + \lambda_4 \dot{v}(t) \\\
&= \lambda_1 v(t) \sin(\gamma(t)) \\\
&\hspace{7mm} + \dfrac{\lambda_2}{2mv(t)} \left(F C_L(t)v^2(t) \alpha e^{-\beta h(t)} - 2 m g \cos(\gamma(t)) \right) \\\
&\hspace{7mm} + \lambda_3 v(t) \cos(\gamma(t)) \\\
&\hspace{7mm} + \dfrac{\lambda_4}{2m} \left(2T(t) + (-C_{D_0}-k C_L^2(t)) F  v^2(t) \alpha e^{-\beta h(t)} - 2 m g \sin(\gamma(t)) \right) 
\end{split}\]
Damit lassen sich nun die Optimalitätsbedingungen des Minimumprinzips von Pontryagin aufstellen:
\begin{enumerate}
\item \textbf{Minimumbedingung:} Es gilt an allen Stetigkeitsstellen $t \in [t_0,t_f]$ von $u^{\ast}(t)$ die Minimumbedingung \[H(X^{\ast}(t),U^{\ast}(t),\lambda(t)) = \min_{U(t) \in \mathcal{U}} H(X^{\ast}(t),U(t),\lambda(t))\] Leitet man nun nach der Steuerfunktion $U(t)$ ab, so ergibt sich für die Minimumbedingung
\[\dfrac{\partial}{\partial U} H(X^{\ast}(t),U^{\ast}(t),\lambda(t)) = \begin{pmatrix}
\dfrac{\lambda_4}{m} \\ 
- \dfrac{\lambda_4 k F v^2(t) \alpha e^{-\beta h(t)}}{m} C_L(t)  + \dfrac{\lambda_2 F v(t) \alpha e^{-\beta h(t)}}{2m}
\end{pmatrix}^T \stackrel{!}{=} 0\]
und 
\[\dfrac{\partial^2}{\partial U^2} H(X^{\ast}(t),U^{\ast}(t),\lambda(t)) = \begin{pmatrix}
0 & - \dfrac{\lambda_4 k F v^2(t) \alpha e^{-\beta h(t)}}{m} 
\end{pmatrix} \stackrel{!}{\geq} 0\] wobei \[\sigma(x(t),\lambda(t)) := H_u(X^{\ast}(t),U^{\ast}(t),\lambda(t))\] Schaltfunktion genannt wird.
%
\item \textbf{Adjungierte DGL:} Leitet man nach dem Zustandsvektor $X(t)$ ab, also $H_{X}(X^{\ast}(t),U^{\ast}(t),\lambda(t))$ so erhält man 
\[\dfrac{\partial}{\partial X} H = \begin{pmatrix}
\dfrac{(C_{D_0}+k C_L^2(t)) v^2(t) \lambda_4 - C_L(t) v(t) \lambda_2}{2m} \alpha F \beta e^{-\beta h(t)}\\ 
\dfrac{- \lambda_3 v^2(t) + \lambda_2 g}{v(t)} \sin(\gamma(t)) + (\lambda_1 v(t) - \lambda_4 g) \cos(\gamma(t)) \\
0 \\ 
\lambda_3 \cos(\gamma(t)) + \lambda_1 \sin(\gamma(t)) + \dfrac{(-C_{D_0}-k C_L^2(t)) 2 v(t) \lambda_4 + C_L(t) \lambda_2}{2m}   F \alpha e^{-\beta h(t)} + \dfrac{g \cos(\gamma(t)) \lambda_2}{v^2(t)} \\
\end{pmatrix}^T \]
wobei gilt $\dot{\lambda}(t)^T = -H_{X}$.
%
\item \textbf{Transversalitätsbedingung:} Im Endzeitpunkt $t_f$ gilt die Transversalitätsbedingung mit dem Vektor $\nu \in \R^r$ mit $(\lambda_0,\lambda(t),\nu) \neq 0$ für alle $t \in [t_0,t_f]$
\[\begin{split}
\lambda(t_f)^T &= \lambda_0 g_X(X^{\ast}(t_f)) + \nu^T \psi_X(X^{\ast}(t_f)) \\\
&= \lambda_0 \begin{pmatrix}
0 & 0 & -1 & 0
\end{pmatrix}  + \nu^T \begin{pmatrix}
1 & 0 & 0 & 0 \\
0 & 1 & 0 & 0 
\end{pmatrix}  \\\
&= \begin{pmatrix}
\nu_4 & \nu_2 & -\lambda_0 & 0 
\end{pmatrix}
\end{split}\]
(ist $\lambda_{0} =1$ da $x(t_f)$ frei ist und $\psi_X(X^{\ast}(t_f))$ hat vollen Zeilenrang???) 
%
\item \textbf{Konstanz:} Für autonome Systeme gilt \[H(X^{\ast}(t),U^{\ast}(t), \lambda(t)) = const \ \in [t_0,t_f]\]
\end{enumerate}






\section{Überprüfung der optimalen Steuerung}
Die Steuerung $T(t)$ (Schub) geht linear in die Hamilton-Funktion ein. Um die Hamilton-Funktion zu minimieren gilt für diese Bang-Bang Verhalten.

Die Steuerung $C_L(t)$ geht nichtlinear in die Hamilton-Funktion ein.






\section{Überprüfung der Hinreichenden Optimalitätsbedingungen}







\section{Aufstellen des Randwertproblems}
Muss also ein Mehrpunktrandwertproblem sein ???


Mit den Optimalitätsbedingungen des Minimumprinzips von Pontryagin lässt sich das Steuerungsproblem in ein Randwertproblem überführen, welches aus den beiden Funktionen $r(t,Z(t))$ und $r_0(Z(t_0),Z(t_f)) = 0$ besteht. Für $r(t,Z(t))$ ergibt sich \[r(t,Z(t)) = \dot{Z}(t) = \begin{pmatrix}
\dot{h}(t),\dot{\gamma}(t),\dot{x}(t),\dot{v}(t),\dot{\lambda}_1(t),\dot{\lambda}_2(t),\dot{\lambda}_3(t),\dot{\lambda}_4(t)
\end{pmatrix}^T\] Für $r_0(Z(t_0),Z(t_f)) = 0$ müssen zunächst die Endbedingungen mit gebildet aus 
\[\begin{split}
X_i(t_f) &= c_i \hspace{25mm} (i=1,...,r) \\\
\lambda_i(t_f) &= \lambda_0 g_{X_i}(X^{\ast}(t_f)) \hspace{5mm} (i=r+1,...,n)
\end{split}\] gebildet werden. Es ergibt sich dann \[r_0(Z(t_0),Z(t_f)) = \begin{pmatrix}
h(t_0) - h_0 \\ 
\gamma(t_0) - \gamma_0 \\
x(t_0) - x_0 \\ 
v(t_0) - v_0 \\ 
h(t_f) - h_f \\ 
\gamma(t_f) - \gamma_f \\
\lambda_3(t_f) + \lambda_0 \\ 
\lambda_4(t_f) - 0
\end{pmatrix}\]
