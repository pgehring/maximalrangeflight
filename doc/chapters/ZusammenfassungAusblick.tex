\chapter{Zusammenfassung und Ausblick}

In dieser Projektarbeit wurde der Steigflug des Airbus A380-800 von einem vorgegebenen Anfangspunkt zu einer Reiseflughöhe mithilfe der Optimalsteuerungstheorie behandelt.  Nach Herleitung des Differentialgleichungssystems  für die Dynamik des quasi-statischen Flugs wurde dieses diskretisiert um es mit einem Optimierungsverfahren zu lösen. Dazu wurden Voruntersuchungen mit verschiedenen numerischen Verfahren zur Lösung von Anfangswertproblemen und Optimierungsverfahren durchgeführt. Es zeigte sich, dass die Kombination von explizitem Euler- und SQP-Verfahren effizient ist und die Plausibilität der Lösung gegeben ist. Daneben wurde mithilfe des Minimumprinzips von Pontryagin das Problem in ein Zweipunkt-Randwertproblem umformuliert, um es mit Einfach- und Mehrfachschießverfahren zu lösen. Das Lösen mittels Einfach- und Mehrfachschießverfahren scheiterte an numerischen Schwierigkeiten (singuläre Matrix bzw. unterbestimmtes Gleichungssystem). Die numerische Untersuchung ergab, dass sich die direkten Verfahren eignen, das gegebene Optimalsteuerungsproblem zu lösen.  Nachteilig ist die lange Rechenzeit des direkten Verfahrens, welche im Durchschnitt eine Stunde betrug. 

Die Betrachtung des Optimalsteuerungsproblems mit anderen Methoden wie dem Homotopieverfahren wurden offen gelassen. Weiterreichende Untersuchungen können angestellt werden, um die Rechengeschwindigkeit zu erhöhen und um verschiedene alternative Lösungsverfahren hinsichtlich ihrer Eignung zu prüfen.