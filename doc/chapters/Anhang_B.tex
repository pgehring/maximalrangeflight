\chapter{Algorithmenvergleich für die adjungierten Differentialgleichungen}\label{appendix:methods}
Analog zu dem in \autoref{kap:TUNU} erstellten Vergleich der Lösungsalgorithmen für die Zustandsdifferentialgleichung, wurde der Vergleich für die adjungierten Differentialgleichungen durchgeführt.
Hierbei hat sich gezeigt, dass das implizite Euler Verfahren und das RADAU-2A Verfahren nicht konvergierte. Diese Verfahren werden daher nciht in dem Vergleich aufgeführt.

In Abbildung \autoref{fig:methods_indirect} ist die Lösung der drei Verfahren gezeigt. Es ist zu erkennen, dass das explizite Vefahren deutlich Abweichungen hat. Die Laufzeit ist jedoch deutlicher kürzer als bei den Verfahren \verb+ode45+ und \verb+ode23s+.

\begin{figure}[htbp]
    \centering 
    \subfloat[\label{fig:methods_i_h}]{\includegraphics{../code/methods/results/methods_plot_i_h.pdf}}
    \qquad
    \subfloat[\label{fig:methods_i_gamma}]{\includegraphics{../code/methods/results/methods_plot_i_gamma.pdf}} \\

    \subfloat[\label{fig:methods_i_x}]{\includegraphics{../code/methods/results/methods_plot_i_x.pdf}}
    \qquad
    \subfloat[\label{fig:methods_i_v}]{\includegraphics{../code/methods/results/methods_plot_i_v.pdf}} \\

    \subfloat[\label{fig:methods_i_l1}]{\includegraphics{../code/methods/results/methods_plot_i_l1.pdf}}
    \qquad
    \subfloat[\label{fig:methods_i_l2}]{\includegraphics{../code/methods/results/methods_plot_i_l2.pdf}} \\

    \subfloat[\label{fig:methods_i_l3}]{\includegraphics{../code/methods/results/methods_plot_i_l3.pdf}}
    \qquad
    \subfloat[\label{fig:methods_i_l4}]{\includegraphics{../code/methods/results/methods_plot_i_l4.pdf}} \\

    \caption{Lösung des Differentialgleichungsmodells. Die acht Zustandgrößen des Vektors wurden mit kontanten Steuerfunktionen \(T\,=\,1259999\,N\) und \(C_L\,=\,1,49\) gelöst.} \label{fig:methods_indirect}
    % \protect\subref{fig:methods_h} zeigt die gleichmäßig steigende Flughöhe mit kleiner Abweichung zwischend en Methoden. \protect\subref{fig:methods_gamma} zeigt den Anstellwinkel des Flugzeuges mit deutlicher Abweichung zwischen den Algorithmen. \protect\subref{fig:methods_x} zeigt die zurückgelegte Strecke des Flugzeuges. \protect\subref{fig:methods_v} zeigt die Geschwindigkeit des Flugzeuges.
\end{figure}

\begin{table}[htbp]
    \centering
    \caption{Untersuchte Einschrittalgorithmen zur Lösung adjungierten Differentialgleichungen}
    \begin{tabularx}{.9\textwidth}{Zccc}
        \toprule
        \textbf{Algorithmus}            & \textbf{Laufzeit} & \textbf{Laufzeitdifferenz } \\
                                        & \textbf{in \text{[$s$]}} & \textbf{in \text{[$s$]}} \\
        \midrule
        \textit{MATLAB} \verb+ode45+    &   0,499   &   0,490 \\
        \textit{MATLAB} \verb+ode23s+   &   0,217   &   0,209 \\
        explizites Euler Verfahren      &   0,00892 &   0     \\
        \bottomrule
    \end{tabularx}
\end{table}
