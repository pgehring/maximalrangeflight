\chapter{Technische Umsetzung}
Das in Gleichung \eqref{equ:mayer_problem} gezeigte Optimalsterungsproblem wird numerisch in MATLAB implementiert und gelöst. Im Folgenden Kapitel werden 
Methoden zur Lösung des Differentialgleichungssystems (vgl. Gleichung \eqref{equ:state_space}) verglichen wird die Implementierung der Lösung des 
Optimalsteuerungsproblems erläutert.

\section{Numerische Lösung von gewöhnlichen Differentialgleichungen}
Das in Kapitel \autoref{} gezeigte Modell des Flugzeugs wurde durch ein System von gewöhnlichen Differentialgleichungen abgebildet. Um dieses Modell
Lösen zu können, werden Methoden von MATLAB und eigens implementierte Methoden zur Integration verwendet.

In der Tabelle sind alle verwendeten Methoden aufgeführt.
Mit dem Vergleich der Methoden konnte die fehlerfreie Implementierung bestätigt werden und die für den Anwendungsfall effizienteste Methode 
ausgewählt werden.

\begin{table}[htbp]
    \caption{Untersuchte Einschrittalgorithmen}
    \begin{tabularx}{\textwidth}{Xccc}
        \toprule
        Algorithmus & Rechenzeit & Reichenzeitverbesserung & max. Fehler \\
        \midrule
        
        \bottomrule
    \end{tabularx}
\end{table}

\begin{figure}[!htbp]
    \centering 
    \subfloat[\label{fig:methods_h}]{\includegraphics[width=0.45\textwidth]{images/03_TechnischeUmsetzung/methods_plot_h}}
    \qquad
    \subfloat[\label{fig:methods_gamma}]{\includegraphics[width=0.45\textwidth]{images/03_TechnischeUmsetzung/methods_plot_gamma}} \\

    \subfloat[\label{fig:methods_x}]{\includegraphics[width=0.45\textwidth]{images/03_TechnischeUmsetzung/methods_plot_x}}
    \qquad
    \subfloat[\label{fig:methods_v}]{\includegraphics[width=0.45\textwidth]{images/03_TechnischeUmsetzung/methods_plot_v}}
    \caption{Lösung des Differentialgleichungsmodells. Die vier Zustandgrößen des Vektors wurden mit kontanten Steuerfunktionen \(T\) und \(C_L\) gelöst.} %
    % \subref{fig:methods_h} zeigt die gleichmäßig steigende Flughöhe mit kleiner Abweichung zwischend en Methoden. \subref{fig:methods_gamma} zeigt den %
    % Anstellwinkel des Flugzeuges mit deutlicher Abweichung zwischen den Algorithmen. \subref{fig:methods_x} zeigt die zurückgelegte Strecke des Flugzeuges.} %
    % \subref{fig:methods_v} zeigt die Geschwindigkeit des Flugzeuges.}
\end{figure}

\section{Klassenstruktur des Problems}

UML

